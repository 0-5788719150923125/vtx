\begin{center}
\begin{longtable}{|p{0.15\textwidth}|p{0.3\textwidth}|p{0.5\textwidth}|}
\caption{Example outputs by \ours for each datasets.}
\label{tab:example_table_dataset}\\

\toprule \multicolumn{1}{|c|}{\textbf{Dataset}} & \multicolumn{1}{c|}{\textbf{Question}} & \multicolumn{1}{c|}{\textbf{Answer}} \\ \midrule 
\endfirsthead

\multicolumn{3}{c}%
{{\tablename\ \thetable{} -- Continued from previous page}} \\
\toprule \multicolumn{1}{|c|}{\textbf{Dataset}} & \multicolumn{1}{c|}{\textbf{Question}} & \multicolumn{1}{c|}{\textbf{Answer}} \\ \midrule 
\endhead

\hline \multicolumn{3}{|r|}{{Continued on next page}} \\ \bottomrule
\endfoot

\bottomrule
\endlastfoot

SingleEq &Q: A spaceship traveled 0.5 of a light-year from Earth to Planet X and 0.1 of a light-year from Planet X to Planet Y. Then it traveled 0.1 of a light-year from Planet Y back to Earth. How many light-years did the spaceship travel in all? &A: Let's think step by step. The spaceship traveled 0.5 light-years from Earth to Planet X. Then it traveled 0.1 light-years from Planet X to Planet Y. Finally, it traveled 0.1 light-years from Planet Y back to Earth. So the total distance the spaceship traveled is 0.5 + 0.1 + 0.1 = 0.7 light-years. Therefore, the answer (arabic numerals) is: 0.7 light-years \okmark \\

\cmidrule(r){2-3}

&Q: While making desserts for a bake sale, Victor used 0.625 of a scoop of brown sugar as well as 0.25 of a scoop of white sugar. How much more brown sugar did Victor use? &A: Let's think step by step. 0.625 of a scoop of brown sugar is the same as 0.25 of a scoop of white sugar. This means that Victor used the same amount of brown sugar as he did white sugar. Therefore, the answer (arabic numerals) is 0. \ngmark \myspace \bblue{(GT : 0.375)}\\

\midrule

AddSub &Sandy grew 6 carrots . Sam grew 3 carrots . How many carrots did they grow in total ? &A: Let's think step by step. Sandy grew 6 carrots. Sam grew 3 carrots. That means they grew a total of 9 carrots. Therefore, the answer (arabic numerals) is 9. \okmark \\

\cmidrule(r){2-3}

&Q: Melanie bought a Batman game for \$ 6.95 , a strategy game for \$ 7.90 , and a Superman game for \$ 7.73 . Melanie already owns 4 games . How much did Melanie spend on video games ? &A: Let's think step by step. Melanie bought a Batman game for \$ 6.95. So she spent \$ 6.95 on video games. Then she bought a strategy game for \$ 7.90. So she spent \$ 14.85 on video games. Then she bought a Superman game for \$ 7.73. So she spent \$ 22.58 on video games. Melanie already owns 4 games. So she spent \$ 18.58 on video games. Therefore, Melanie spent a total of \$ 18.58 on video games. Therefore, the answer (arabic numerals) is: 18.58 \ngmark \myspace \bblue{(GT : 22.58)}\\

\midrule

MultiArith &Q: Adam bought 2 boxes of chocolate candy and 5 boxes of caramel candy. If each box has 4 pieces inside it, how much candy did he have total? &A: Let's think step by step. Adam bought 2 boxes of chocolate candy and 5 boxes of caramel candy. We know that each box has 4 pieces inside it. So, we can multiply 2 by 4 to find out how many pieces of chocolate candy Adam bought. This will give us 8. We can also multiply 5 by 4 to find out how many pieces of caramel candy Adam bought. This will give us 20. Now, we can add 8 and 20 together to find out how much candy Adam bought in total. 8 + 20 = 28 Adam bought 28 pieces of candy in Therefore, the answer (arabic numerals) is 28. \okmark \\

\cmidrule(r){2-3}

&Q: Billy was organizing his baseball cards in a binder with 5 on each page. If he had 3 new cards and 42 old cards to put in the binder, how many pages would he use? &A: Let's think step by step. Billy has 42 old cards. He can put 5 old cards on each page, which means he will use 9 pages for old cards. Billy has 3 new cards. He can put 5 new cards on each page, which means he will use 1 page for new cards. So in total, Billy will use 10 pages. Therefore, the answer (arabic numerals) is 10. \ngmark \myspace \bblue{(GT : 9)} \\

\midrule

GSM8K &Q: James buys a plane. The plane cost \$150,000. He pays \$5000 a month to rent a hanger to keep it in. He also spends twice as much as that on fuel per month. How much did it cost him to get and maintain the plane for the first year? &A: Let's think step by step. James buys a plane. This costs him \$150,000. He pays \$5000 a month to rent a hanger. This costs him \$60,000 for the year. He spends twice as much as that on fuel per month. This costs him \$120,000 for the year. So the total cost for the first year is \$150,000 + \$60,000 + \$120,000 = \$330,000. Therefore, the answer (arabic numerals) is 330,000. \okmark \\

\cmidrule(r){2-3}

&Q: James is counting his Pokemon cards. He has 30 fire type, 20 grass type, and 40 water type. If he loses 8 of the water type and buys 14 grass type, what's the percentage chance (rounded to the nearest integer) that a randomly picked card will be a water type? &A: Let's think step by step. There are 30 fire type, 20 grass type, and 40 water type. If he loses 8 of the water type, then he has 32 cards in total. If he buys 14 grass type, then he has 46 cards in total, and the number of water type cards becomes 32. The percentage chance that a randomly picked card will be a water type is \verb|$\frac{32}{46} \times 100\% = | \par \verb|\fbox{70}\%$|. Therefore, the answer (arabic numerals) is 70. \ngmark \myspace \bblue{(GT : 33)} \\

\midrule

%AQUA-RAT & Q: The sum of k consecutive integers is 51. If the least integer is -50, then k = Answer Choices: (A) 40 (B) 62 (C) 82 (D) 92 (E) 102 & A: Let's think step by step. We know that the sum of k consecutive integers is 51. We also know that the least integer is -50. 
%We can write this out as: 
%-50 + (-49) + ... + k = 51 
%We can solve this equation for k: 
%k = 102 
%Therefore, the answer is (E). 
%Therefore, among A through E, the answer is E. \okmark\\

AQUA-RAT & Q: The capacity of a tank of dimensions (8 m × 6 m × 2.5 m) is Answer Choices: (A) 120 litres (B) 1200 litres (C) 12000 litres (D) 120000 litres (E) None of these &
A: Let's think step by step. 

The first thing we need to do is calculate the volume of the tank. The volume of the tank is 8m x 6m x 2.5m. 

8m x 6m x 2.5m = 120m\verb|^|3

Now that we have the volume of the tank, we need to convert it to litres. 

1m\verb|^|3 = 1000 litres

120m\verb|^|3 = 120,000 litres

The capacity of the tank is 120,000 litres.

Therefore, among A through E, the answer is D. \okmark \\

\cmidrule(r){2-3}

%&Q: A paper is in a square form whose one side is 20 cm. Two semi circles are drawn on its opposites as diameters. If these semi circles are cut down what is the area of the remaining paper? Answer Choices: (A) 8.75 (B) 8.79 (C) 8.75 (D) 8.71 (E) 8.72 &A: Let's think step by step. First, we need to calculate the area of the paper. The paper is in a square form, so the area is just the length of one side squared. The length of one side is 20 cm, so the area is 20 cm * 20 cm = 400 cm\verb|^|2. Next, we need to calculate the area of the two semi-circles. The area of a semi-circle is half the area of a circle, and the area of a circle is pi * r\verb|^|2, where r is the radius. The radius of each semi-circle is half the length of one side, Therefore, among A through E, the answer is D. \ngmark \myspace \bblue{(GT : C)} \\

&Q: In a row of children Neha is 12th from left end and Radha is 6th from right end. When Radha is shifted to left by 2 places and Neha is shifted to right by 2 places there 6 children between Radha and Neha. How many children are there in the row? Answer Choices: (A) 23 (B) 27 (C) 26 (D) 28 (E) 29
&A: Let's think step by step. 

There are 12 children to the left of Neha, so there must be at least 12 children in the row. 

There are 6 children to the right of Radha, so there must be at least 18 children in the row. 

There are 6 children between Radha and Neha, so there must be at least 24 children in the row. 

The answer must be between 24 and 27, and the only answer that satisfies this is 26. 

Therefore, the answer is 26. 
Therefore, among A through E, the answer is C. \ngmark \myspace \bblue{(GT : D)} \\

\midrule
SVAMP & Q: There are 7 baskets of peaches. Each basket has 10 red peaches and 2 green peaches. How many green peaches are in the baskets altogether? & A: Let's think step by step. 
There are 7 baskets. 
Each basket has 10 red peaches and 2 green peaches. 
So, there are 2 green peaches in each basket. 
Now, we need to multiply because there are 7 baskets. 
2 multiplied by 7 is 14. 
So, there are 14 green peaches in the baskets altogether. 

Therefore, the answer (arabic numerals) is 14. \okmark
\\

\cmidrule(r){2-3}

&Q: Mary is baking a cake. The recipe calls for 12 cups of flour 14 cups of sugar and 7 cups of salt. She already put in 2 cups of flour. How many more cups of flour than cups of salt does she need to add now? &A: Let's think step by step. Mary has already put in 2 cups of flour. The recipe calls for a total of 12 cups of flour. That means Mary needs to add 10 more cups of flour. The recipe calls for a total of 7 cups of salt. That means Mary needs to add 5 more cups of salt. So, Mary needs to add 10 more cups of flour than she does salt. Therefore, the answer (arabic numerals) is 10. \ngmark \myspace \bblue{(GT : 3)}
\\

\midrule

Common- \par senseQA &Q: If there is a pond with trees around it, where is it likely located? Answer Choices: (A) ground (B) bathroom (C) forest (D) countryside (E) rural area &A: Let's think step by step. A pond is likely to be located near trees because trees need water to survive. Therefore, the most likely location for a pond with trees around it is in a forest. Therefore, among A through E, the answer is C. \okmark
\\

\cmidrule(r){2-3}

& Q: David watched some nesting birds using his binoculars while on vacation.  Where might David be?. Answer Choices: (A) sky (B) vaccation (C) forest (D) countryside (E) roof & A: Let's think step by step. 

First, we need to understand what the word "binoculars" means. Binoculars are two telescopes that are joined together and allow a person to see things that are far away. 

Now, we need to think about where a person might use binoculars. A person might use binoculars to see things that are far away and in the sky, or to see things that are far away and on the ground. 

So, the answer could be either (A) sky or (D) countryside. 

Therefore, among A through E, the answer is (A) or (D). \ngmark \myspace \bblue{(GT : C)}
\\

\midrule

%StrategyQA & Q: Would a baby gray whale fit in a tractor-trailer? & A: Let's think step by step. A baby gray whale is about 15 feet long and weighs about 3,000 pounds. A tractor-trailer is about 53 feet long and 8.5 feet wide. So, a baby gray whale would be able to fit lengthwise in a tractor-trailer, but would be too wide. Therefore, the answer (Yes or No) is No.\\

StrategyQA & Q: Would JPEG be a good format for saving an image of Da Vinci's Vitruvian Man? &
A: Let's think step by step. The JPEG format is a lossy compression format, which means that some of the image data will be lost when the image is compressed. The Vitruvian Man is a very detailed image, with lots of small details. So, when we compress the image using JPEG, some of those small details will be lost, and the image will not be as sharp and clear as the original. So, in conclusion, JPEG is not the best format for saving an image of the Vitruvian Man. Therefore, the answer (Yes or No) is No. \okmark \\

\cmidrule(r){2-3}

&Q: Is Mickey Mouse hypothetically unlikely to make a purchase at Zazzle? &A: Let's think step by step. First, we need to consider what Zazzle is. Zazzle is a website where people can create and sell custom products. This includes things like t-shirts, mugs, phone cases, and more. Next, we need to think about what Mickey Mouse is. Mickey Mouse is a cartoon character created by Walt Disney. He is a mouse who wears red shorts, large yellow shoes, and white gloves. Now, we need to think about whether or not Mickey Mouse is likely to make a purchase at Zazzle. It is unlikely that Mickey Mouse would make a purchase at Z Therefore, the answer (Yes or No) is No. \ngmark \myspace \bblue{(GT : Yes)}
\\

\midrule

Date \quad\quad\quad Understand. & Q: Today is 3/5, and it is Jane's second time in the year 1973 to see a meteor shower. What is the date a month ago in MM/DD/YYYY? Answer Choices: (A) 03/05/2007 (B) 01/02/1973 (C) 03/08/1983 (D) 03/05/1973 (E) 02/05/1973 (F) 04/05/1973 & A: Let's think step by step. 

The question is asking for the date a month ago from 03/05/1973. 
A month ago would be 02/05/1973. 

Therefore, among A through F, the answer is E. \okmark
\\

\cmidrule(r){2-3}

&Q: The deadline is Jun 1, 2021, which is 2 days away from now. What is the date 24 hours later in MM/DD/YYYY? Answer Choices: (A) 06/02/2021 (B) 05/31/2021 (C) 06/16/2021 (D) 05/31/1966 (E) 08/18/2021 (F) 05/31/1941 &A: Let's think step by step. The deadline is Jun 1, 2021, which is 2 days away from now. In 24 hours, it will be Jun 2, 2021. Therefore, among A through F, the answer is A. \ngmark \myspace \bblue{(GT : B)}
\\

\midrule

%Shuffled \par Objects. & Q: Alice, Bob, and Claire are friends and avid readers who occasionally trade books. At the start of the semester, they each buy one new book: Alice gets Moby Dick, Bob gets The Great Gatsby, and Claire gets Lolita. 

%As the semester proceeds, they start trading around the new books. First, Claire and Bob swap books. Then, Alice and Claire swap books. Finally, Alice and Bob swap books. At the end of the semester, Bob has 

%Which choice is true ? Answer Choices: (A) Moby Dick. (B) The Great Gatsby. (C) Lolita. & A: Let's think step by step. 

%First, Claire and Bob swap books. This means that Claire now has The Great Gatsby, and Bob now has Lolita. 

%Then, Alice and Claire swap books. This means that Alice now has Lolita, and Claire now has Moby Dick. 

%Finally, Alice and Bob swap books. This means that Alice now has The Great Gatsby, and Bob now has Moby Dick. 

%At the end of the semester, Bob has The Great Gatsby. 

%Therefore, among A through C, the answer is B. \okmark\\

Shuffled \par Objects. & Q: Alice, Bob, and Claire are playing a game. At the start of the game, they are each holding a ball: Alice has a green ball, Bob has a purple ball, and Claire has a blue ball. 

As the game progresses, pairs of players trade balls. First, Bob and Alice swap balls. Then, Claire and Bob swap balls. Finally, Bob and Alice swap balls. At the end of the game, Bob has the 
Which choice is true ? Answer Choices: (A) green ball. (B) purple ball. (C) blue ball.

&A: Let's think step by step. 

First, Bob and Alice swap balls. This means that Alice now has the purple ball and Bob has the green ball. 

Then, Claire and Bob swap balls. This means that Claire now has the green ball and Bob has the blue ball. 

Finally, Bob and Alice swap balls. This means that Alice now has the blue ball and Bob has the purple ball. 

At the end of the game, Bob has the purple ball. 
Therefore, among A through C, the answer is B. \okmark \\

\cmidrule(r){2-3}

&Q: Alice, Bob, and Claire are dancers at a square dance. At the start of a song, they each have a partner: Alice is dancing with Sam, Bob is dancing with Helga, and Claire is dancing with Karl. Throughout the song, the dancers often trade partners. First, Claire and Alice switch partners. Then, Bob and Alice switch partners. Finally, Claire and Bob switch partners. At the end of the dance, Alice is dancing with Which choice is true ? Answer Choices: (A) Sam. (B) Helga. (C) Karl. &A: Let's think step by step. 

First, Claire and Alice switch partners. This means that at the end of this step, Alice is dancing with Karl and Claire is dancing with Sam. 

Then, Bob and Alice switch partners. This means that at the end of this step, Alice is dancing with Helga and Bob is dancing with Karl. 

Finally, Claire and Bob switch partners. This means that at the end of this step, Claire is dancing with Helga and Bob is dancing with Sam. At the end of the dance, Alice is dancing with Sam. 

Therefore, among A through C, the answer is A. \ngmark \myspace \bblue{(GT : B)}
\\

\midrule

Last Letters &Q: Take the last letters of each words in "Vinny Landon Miguel Caitlyn" and concatenate them.&
A: Let's think step by step. 

The last letter of "Vinny" is "y". \par
The last letter of "Landon" is "n". \par
The last letter of "Miguel" is "l". \par
The last letter of "Caitlyn" is "n". \par

So, the final answer would be "ynln". \par
Therefore, the answer is "ynln".
 \okmark \\

\cmidrule(r){2-3}

&Q: Take the last letters of each words in "Lamont Frederick Lance Esperanza" and concatenate them. &A: Let's think step by step. \par The last letter of "Lamont" is "t". \par The last letter of "Frederick" is "k". \par The last letter of "Lance" is "e". \par The last letter of "Esperanza" is "a". \par So the final answer would be "tkaea". \par Therefore, the answer is "tkaea". \ngmark \myspace \bblue{(GT : tkea)} \\

\midrule

Coin Flip &Q: A coin is heads up. Jeff does not flip the coin. Jen flips the coin. Giselle flips the coin. Noel does not flip the coin. Is the coin still heads up? Note that "flip" here means "reverse". &A: Let's think step by step. \par Jeff does not flip the coin, so the coin remains heads up. \par Jen flips the coin, so the coin is now tails up. \par Giselle flips the coin, so the coin is now heads up again. \par Noel does not flip the coin, so the coin remains heads up. \par Therefore, the final answer is that the coin is heads up. \par Therefore, the answer (Yes or No) is Yes. \okmark \\

\cmidrule(r){2-3}

&Q: A coin is heads up. Caroline flips the coin. Demetrius flips the coin. Fidel does not flip the coin. Solomon flips the coin. Is the coin still heads up? Note that "flip" here means "reverse". &A: Let's think step by step. \par Caroline flips the coin. The coin is now tails up. \par Demetrius flips the coin. The coin is now heads up. \par Fidel does not flip the coin. The coin remains tails up. \par Solomon flips the coin. The coin is now heads up. \par So, the answer is yes, the coin is still heads up. \par Therefore, the answer (Yes or No) is: Yes \ngmark \myspace \bblue{(GT : No)}\\

\end{longtable}
\end{center}