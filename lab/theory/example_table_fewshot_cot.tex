\begin{table}[t]\centering
\caption{An example output by Few-Shot-CoT  (8 exemplars in context. These exemplars are cited from \citep{cot_wei} and randomly shuffled).}
\label{tab:example_table_fewshot_cot}
\begin{tabular}{|p{0.2\textwidth}|p{0.7\textwidth}|}
\toprule

Few-Shot-CoT &

\bred{***** Start In-Context Examples *****}

Q: Jason had 20 lollipops. He gave Denny some lollipops. Now Jason has 12 lollipops. How many lollipops did Jason give to Denny?

A: Jason started with 20 lollipops. Then he had 12 after giving some to Denny. So he gave Denny 20 - 12 = 8. The answer is 8.

Q: Michael had 58 golf balls. On tuesday, he lost 23 golf balls. On wednesday, he lost 2 more. How many golf balls did he have at the end of wednesday?

A: Michael started with 58 golf balls. After losing 23 on tuesday, he had 58 - 23 = 35. After losing 2 more, he had 35 - 2 = 33 golf balls. The answer is 33.

Q: If there are 3 cars in the parking lot and 2 more cars arrive, how many cars are in the parking lot?

A: There are originally 3 cars. 2 more cars arrive. 3 + 2 = 5. The answer is 5.

Q: There were nine computers in the server room. Five more computers were installed each day, from monday to thursday. How many computers are now in the server room?

A: There were originally 9 computers. For each of 4 days, 5 more computers were added. So 5 * 4 = 20 computers were added. 9 + 20 is 29. The answer is 29.

Q: Olivia has $23. She bought five bagels for $3 each. How much money does she have left?

A: Olivia had 23 dollars. 5 bagels for 3 dollars each will be 5 x 3 = 15 dollars. So she has 23 - 15 dollars left. 23 - 15 is 8. The answer is 8.

Q: There are 15 trees in the grove. Grove workers will plant trees in the grove today. After they are done, there will be 21 trees. How many trees did the grove workers plant today?

A: There are 15 trees originally. Then there were 21 trees after some more were planted. So there must have been 21 - 15 = 6. The answer is 6.

Q: Shawn has five toys. For Christmas, he got two toys each from his mom and dad. How many toys does he have now?

A: Shawn started with 5 toys. If he got 2 toys each from his mom and dad, then that is 4 more toys. 5 + 4 = 9. The answer is 9.

Q: Leah had 32 chocolates and her sister had 42. If they ate 35, how many pieces do they have left in total?

A: Originally, Leah had 32 chocolates. Her sister had 42. So in total they had 32 + 42 = 74. After eating 35, they had 74 - 35 = 39. The answer is 39.

\bred{***** End In-Context Examples *****}

Q: For Halloween Megan received 11 pieces of candy from neighbors and 5 pieces from her older sister. If she only ate 8 pieces a day, how long would the candy last her?

A: Megan received 11 pieces of candy from neighbors and 5 pieces from her older sister. So she had 11 + 5 = 16 pieces of candy. If she ate 8 pieces a day, then she would have 16 / 8 = 2 days worth of candy. The answer is 2. \okmark
\\
\bottomrule
\end{tabular}
\end{table}

%------------------------------------------------------

\begin{table}[t]\centering
\caption{An example output by Few-Shot-CoT with exemplars from entirely different task (7 exemplars in context. These exemplars are cited from \citep{cot_wei} and randomly shuffled) Exemplars are CommonsenseQA (Multi-Choice question), and a task is MultiArith (Number question).}
\label{tab:example_table_fewshot_cot_diff_task}
\begin{tabular}{|p{0.2\textwidth}|p{0.7\textwidth}|}
\toprule

Few-Shot-CoT \par with exemplars \par from a different task &

\bred{***** Start In-Context Examples *****}

Q: Sammy wanted to go to where the people were. Where might he go? Answer Choices: (A) populated areas (B) race track (C) desert (D) apartment (E) roadblock

A: The answer must be a place with a lot of people. Of the above choices, only populated areas have a lot of people. The answer is A.

Q: Before getting a divorce, what did the wife feel who was doing all the work? Answer Choices: (A) harder (B) anguish (C) bitterness (D) tears (E) sadness

A: The answer should be the feeling of someone getting divorced who was doing all the work. Of the above choices, the closest feeling is bitterness The answer is C.

Q: Google Maps and other highway and street GPS services have replaced what? Answer Choices: (A) united states (B) mexico (C) countryside (D) atlas

A: The answer must be something that used to do what Google Maps and GPS services do, which is to give directions. Of the above choices, only atlases are used to give directions. The answer is D.

Q: The fox walked from the city into the forest, what was it looking for? Answer Choices: (A) pretty flowers (B) hen house (C) natural habitat (D) storybook

A: The answer must be something in the forest. Of the above choices, only natural habitat is in the forest. The answer is B.

Q: What do people use to absorb extra ink from a fountainpen? Answer Choices: (A) shirt pocket (B) calligrapher's hand (C) inkwell (D) desk drawer (E) blotter

A: The answer must be an item that can absorb ink. Of the above choices, only blotters are used to absorb ink. The answer is E.

Q: Where do you put your grapes just before checking out? Answer Choices: (A) mouth (B) grocery cart (C)super market (D) fruit basket (E) fruit market

A: The answer should be the place where grocery items are placed before checking out. Of the above choices, grocery cart makes the most sense for holding grocery items.  The answer is B.

Q: What home entertainment equipment requires cable? Answer Choices: (A) radio shack (B) substation (C) television (D) cabinet

A: The answer must require cable. Of the above choices, only television requires cable. The answer is C.

\bred{***** End In-Context Examples *****}

Q: Paige and her friends were recycling paper for their class. For every 4 pounds they recycled they earned one point. If Paige recycled 14 pounds and her friends recycled 2 pounds, how many points did they earn?

A: Paige and her friends would have earned 3 points for recycling paper. \ngmark \myspace \bblue{(GT : 4)}\\

\bottomrule
\end{tabular}
\end{table}

%------------------------------------------------------

\begin{table}[t]\centering
\caption{An example output by Zero-Plus-Few-Shot-CoT (8 exemplars in context. These exemplars are cited from \citep{cot_wei} and randomly shuffled).}
\label{tab:example_table_zeroplusfewshot_cot}
\begin{tabular}{|p{0.2\textwidth}|p{0.7\textwidth}|}
\toprule

Zero-Plus-Few-Shot-CoT &

\bred{***** Start In-Context Examples *****}

Q: Jason had 20 lollipops. He gave Denny some lollipops. Now Jason has 12 lollipops. How many lollipops did Jason give to Denny?

A: \textbf{Let's think step by step.} Jason started with 20 lollipops. Then he had 12 after giving some to Denny. So he gave Denny 20 - 12 = 8. The answer is 8.

Q: Michael had 58 golf balls. On tuesday, he lost 23 golf balls. On wednesday, he lost 2 more. How many golf balls did he have at the end of wednesday?

A: \textbf{Let's think step by step.} Michael started with 58 golf balls. After losing 23 on tuesday, he had 58 - 23 = 35. After losing 2 more, he had 35 - 2 = 33 golf balls. The answer is 33.

Q: If there are 3 cars in the parking lot and 2 more cars arrive, how many cars are in the parking lot?

A: \textbf{Let's think step by step.} There are originally 3 cars. 2 more cars arrive. 3 + 2 = 5. The answer is 5.

Q: There were nine computers in the server room. Five more computers were installed each day, from monday to thursday. How many computers are now in the server room?

A: \textbf{Let's think step by step.} There were originally 9 computers. For each of 4 days, 5 more computers were added. So 5 * 4 = 20 computers were added. 9 + 20 is 29. The answer is 29.

Q: Olivia has $23. She bought five bagels for $3 each. How much money does she have left?

A: \textbf{Let's think step by step.} Olivia had 23 dollars. 5 bagels for 3 dollars each will be 5 x 3 = 15 dollars. So she has 23 - 15 dollars left. 23 - 15 is 8. The answer is 8.

Q: There are 15 trees in the grove. Grove workers will plant trees in the grove today. After they are done, there will be 21 trees. How many trees did the grove workers plant today?

A: \textbf{Let's think step by step.} There are 15 trees originally. Then there were 21 trees after some more were planted. So there must have been 21 - 15 = 6. The answer is 6.

Q: Shawn has five toys. For Christmas, he got two toys each from his mom and dad. How many toys does he have now?

A: \textbf{Let's think step by step.} Shawn started with 5 toys. If he got 2 toys each from his mom and dad, then that is 4 more toys. 5 + 4 = 9. The answer is 9.

Q: Leah had 32 chocolates and her sister had 42. If they ate 35, how many pieces do they have left in total?

A: \textbf{Let's think step by step.} Originally, Leah had 32 chocolates. Her sister had 42. So in total they had 32 + 42 = 74. After eating 35, they had 74 - 35 = 39. The answer is 39.

\bred{***** End In-Context Examples *****}

Q: For Halloween Megan received 11 pieces of candy from neighbors and 5 pieces from her older sister. If she only ate 8 pieces a day, how long would the candy last her?

A: \textbf{Let's think step by step.} Megan received 11 pieces of candy from neighbors and 5 pieces from her older sister. So in total she had 11 + 5 = 16 pieces of candy. If she ate 8 pieces a day, it would last her 16 / 8 = 2 days. The answer is 2. \okmark
\\
\bottomrule
\end{tabular}
\end{table}
