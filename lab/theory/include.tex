%% Let's put in these file all the definitions that we
%% require:

%% Setup language, encoding...
\RequirePackage{etex} 

\usepackage{standalone}
\usepackage{lmodern}
% Language
\usepackage[english]{babel}
% Input encoding
\usepackage[utf8]{inputenc}
% Output encoding https://tex.stackexchange.com/a/677. Important to copy accents
\usepackage[T1]{fontenc}
\usepackage{subfiles} % To quickly load this include file other files
\usepackage{microtype}
\usepackage[shortcuts,nospacearound]{extdash}    % Better endling of em dash (---) to cut sentences\===like 
%%% https://tex.stackexchange.com/questions/2773/how-do-i-make-latex-push-long-citations-to-a-new-line
\usepackage{breakcites}
\usepackage{booktabs}
% Makes sure floats don't fly accross sections
\usepackage{placeins}
\usepackage{adjustbox}
\usepackage{varwidth} % Like minipage, but with automatic width discovery.
\usepackage{complexity} % get \QMA{}...

\usepackage[normalem]{ulem}
%\usepackage[math-style=ISO]{unicode-math}
\usepackage{upgreek} % non italic greak letters

%%%% Bibliography
% I want to have separate references for appendix and main body
% https://tex.stackexchange.com/questions/98660/
\usepackage[style=trad-alpha,
  sortcites=true,
  doi=false,
  url=false,
  giveninits=true, % Bob Foo --> B. Foo
  isbn=false,
  url=false,
  eprint=false,
  sortcites=false, % \cite{B,A,C}: [A,B,C] --> [B,A,C]
]{biblatex}
\defbibfilter{appendixOnlyFilter}{
  segment=1 % Segment 1 will be chosen to be the one in appendix
  and not segment=0 % Default segment is 0
}
\renewcommand{\multicitedelim}{, } % [ABC96; DEF12] -> [ABC96, DEF12]
% [unknown_ref] => [??]
% https://tex.stackexchange.com/a/352573
\makeatletter
\protected\def\abx@missing#1{\textbf{??}}
\makeatother
\renewcommand*{\bibfont}{\normalfont\small} % BibLatex font seems bigger than Bibtex?

\usepackage{xcolor,cancel}
\usepackage{proof-at-the-end}
\NewDocumentEnvironment{theoremE}{O{}O{}+b}{%
  \begin{theoremEnd}[normal,#2]{theorem}[#1]%
    #3%
  \end{theoremEnd}
  %
}{}% Do not forget the second parameter or you might get Missing \begin{document} error
\NewDocumentEnvironment{lemmaE}{O{}O{}+b}{%
  \begin{theoremEnd}[normal,#2]{lemma}[#1]%
    #3%
  \end{theoremEnd}
  %
}{}% Do not forget the second parameter or you might get Missing \begin{document} error
\NewDocumentEnvironment{corollaryE}{O{}O{}+b}{%
  \begin{theoremEnd}[normal,#2]{corollary}[#1]%
    #3%
  \end{theoremEnd}
  %
}{}% Do not forget the second parameter or you might get Missing \begin{document} error
\NewDocumentEnvironment{proofE}{O{}+b}{%
  \begin{proofEnd}[#1]%
    #2
  \end{proofEnd}%
}{}%
\pgfkeys{/prAtEnd/global custom defaults/.style={
    text link={\noindent See \hyperref[proof:prAtEnd\pratendcountercurrent]{proof} in \cref{proofsection:prAtEnd\pratendcountercurrent}.}
  }
}

\ifdefined\displayArxivTexts%
  \pgfkeys{/prAtEnd/end if published/.style={no restate}}
  \pgfkeys{/prAtEnd/all end if published/.style={no restate}}
\else%
  \pgfkeys{/prAtEnd/end if published/.style={end}}
  \pgfkeys{/prAtEnd/all end if published/.style={all end}}
\fi

% https://tex.stackexchange.com/a/20613/116348
% https://tex.stackexchange.com/a/583244/116348
\usepackage{scalerel}
\newcommand\hcancel[2][black]{%
  \ThisStyle{%
    \setbox0=\hbox{$\SavedStyle #2$}%
    \rlap{\raisebox{.25\ht0}{\textcolor{#1}{\rule{\wd0}{1pt}}}}\SavedStyle #2}%
}

%% We provide 3 commands to add comments/text:
%% - \yournameAddText{your text to add}: proposition to add text
%% - \yournameRemoveText{text to remove}: proposition to remove text
%% - \yournameComment{your comments}: to add general comment to discuss. Should not contain proposition to add or remove text, but can be added before/after to explain the reason of the modification.
%% The interest it to provide some quick ways to see what is the final number of pages:
%% just add before the import a line:
%%   \def\removeCommentsAcceptPropositions{}
%% and the proposition will be "accepted" (i.e. the removed text is removed,
%% and the added text will be marked as black), and the comment will be removed, that way it is possible
%% to determine the final number of pages.
%% Example:
%% This text is normal\leoRemove{, this text should be removed}\leoAddText{, this text should be added}\leoComment{This is a comment, explaining why I made that changes.}.
\newcommand{\addEditor}[2]{%
  \expandafter\newcommand\csname #1AddText\endcsname[2][]{%
    \ifdefined\removeCommentsAcceptPropositions %
      ##2%
    \else%
      \textcolor{#2}{\sout{##1}}\textcolor{#2!60!black}{##2}%
    \fi%
  }%
  \expandafter\newcommand\csname #1Comment\endcsname[1]{%
    \ifdefined\removeCommentsAcceptPropositions %
    \else%
      \textcolor{#2!60!white}{\textbf{##1}}%
    \fi%
  }%
  \expandafter\newcommand\csname #1Remove\endcsname[1]{%
    \ifdefined\removeCommentsAcceptPropositions %
    \else%
      \textcolor{#2}{\sout{##1}}%
    \fi%
  }%
}
\definecolor{darkgreen}{rgb}{0., 0.7, 0}
\addEditor{leo}{darkgreen}
\addEditor{fred}{red}
\addEditor{elham}{blue}

%% We also provide an arxivOnly environment, in order to include text that are too long for the submitted version
%% but that would be good to have in the arxiv version.
% Use the \begin{arxivOnly} ... \end{arxivOnly} environment, by putting the opening and ending command
%% on there own line.
%% For shorter texts, you can use the inline version \arxivOnlyShort{your text}.
%% If you want to exclude text from the arxiv, use \begin{publishedOnly} ... \end{publishedOnly} and \publishedOnlyShort{your text}
%% "Comment" package is annoying, it does not allow block indentation, we use 'versions' instead
% \usepackage{versions} %% Do not work inside theoremE, don't know why
\ifdefined\displayArxivTexts%
  % \includeversion{arxivOnly}
  % \excludeversion{publishedOnly}
  \newcommand{\arxivOnly}[1]{#1}
  \newcommand{\arxivOnlyNoDiff}[1]{#1} % Like arxivOnly, but useful in diff mode (when it does not make sense to add colors, for instance inside a cite or when using arxivonly to create lists.
  \newcommand{\publishedOnly}[1]{}
  \newcommand{\publishedVsArxiv}[2]{#2}
  \newcommand{\inAppendixIfPublished}[1]{#1}
\else%
  \ifdefined\displayDiffTexts%
    \colorlet{arxivDiff}{green}
    \colorlet{publishedDiff}{red}
    \newcommand{\arxivOnly}[1]{{\color{arxivDiff}#1}}
    \newcommand{\arxivOnlyNoDiff}[1]{#1}
    \newcommand{\publishedOnly}[1]{{\color{publishedDiff}#1}}
    \newcommand{\publishedVsArxiv}[2]{{\color{publishedDiff}#1}{\color{arxivDiff}#2}}
    \newcommand{\inAppendixIfPublished}[1]{{\color{arxivDiff}\textbf{MOVED IN APPENDIX IN PUBLISHED}#1}\textEnd{{\color{publishedDiff}\textbf{MOVED IN MAIN BODY IN ARXIV}#1}}}
  \else
    % \excludeversion{arxivOnly}
    % \includeversion{publishedOnly}
    \newcommand{\arxivOnly}[1]{}
    \newcommand{\arxivOnlyNoDiff}[1]{}
    \newcommand{\publishedOnly}[1]{#1}
    \newcommand{\publishedVsArxiv}[2]{#1}
    \newcommand{\inAppendixIfPublished}[1]{\textEnd{#1}}
  \fi
\fi
% Alias, easier to remember  
\newcommand{\inAppendix}[1]{\textEnd{#1}}
  
% Symbol to display when an arxiv text is supposed to appear
\usepackage{todonotes}
\newcommand{\notifyArxiv}{\ifdefined\arxivDraftMode\todo{arXiv}\fi}

% Remove unwanted space before/after lists
\usepackage{enumitem}
\publishedOnly{\setlist[itemize]{noitemsep, topsep=0pt}}

% Usual packages:
\usepackage{amsmath}
\usepackage{amssymb}
% \usepackage{amsthm}
\usepackage{thmtools}
\usepackage{graphicx}
\graphicspath{{}{figures/}}
\usepackage{float}
\usepackage{subfig}
\usepackage{wrapfig}
\usepackage{array}
\usepackage{braket}
% Refer to footnotes:
% https://tex.stackexchange.com/a/167382/116348
\usepackage{footmisc}
% Useful to use the latest NewDocumentCommand:
%\usepackage{xparse}
%\usepackage[poster]{tcolorbox}

%% Space between pictures and text
%% https://mirrors.chevalier.io/CTAN/macros/latex/contrib/layouts/layman.pdf
%% \usepackage{layouts}
%% https://tex.stackexchange.com/questions/60477/remove-space-after-figure-and-before-text?rq=1
\setlength{\floatsep}{7pt plus 2pt minus 2pt}
\setlength{\textfloatsep}{7pt plus 2pt minus 2pt}
\setlength{\intextsep}{7pt plus 2pt minus 2pt}
% https://tex.stackexchange.com/questions/39017/how-to-influence-the-position-of-float-environments-like-figure-and-table-in-lat
\setcounter{topnumber}{8}
\setcounter{bottomnumber}{8}
\setcounter{totalnumber}{8}
%% https://tex.stackexchange.com/questions/45990/how-can-i-modify-vertical-space-between-figure-and-caption
%% plus/minus allows to stretch a bit around 15pt
% \setlength{\abovecaptionskip}{5pt}

% Remove number to subsubsection (in llncs it replaces \paragraph)
\setcounter{secnumdepth}{2}

%%% Some practical environment to define theorems, lemma...
% \declaretheorem[name=Theorem,numberwithin=section]{theorem}
% \declaretheorem[name=Definition,sibling=theorem]{definition}
% \declaretheorem[name=Lemma,sibling=theorem]{lemma}
% \declaretheorem[name=Corollary,sibling=theorem]{corollary}
% \declaretheorem[name=Remark,sibling=theorem]{remark}
%%% In LNCS we use
% https://tex.stackexchange.com/a/373125/116348
% \spnewtheorem{thm}{Theorem}[section]{\bfseries}{\itshape}

%%%
\usepackage[ruled,vlined]{algorithm2e}
\SetAlFnt{\normalsize}%
\setlength{\algomargin}{0em} % remove margin on left of algo
\SetAlgorithmName{Protocol}{protocol}{List of Protocols}
\newenvironment{protocol}[1][htb]{\begin{algorithm}[#1]}{\end{algorithm}}

%%% %%% Load algorithm package, and create a breakable version
%%% \usepackage{algorithm, algorithmicx, algpseudocode}
%%% % Decrease space between text:
%%% % https://tex.stackexchange.com/questions/25828/how-to-remove-change-the-vertical-spacing-before-and-after-an-algorithm-enviro/25832
%%% \setlength{\intextsep}{1\baselineskip}
%%% % https://tex.stackexchange.com/questions/387577/new-algorithm-environment-with-a-separate-counter
%%% \newcounter{protocol}
%%% \makeatletter
%%% \newenvironment{protocol}[1][htb]{%
%%%   \let\c@algorithm\c@protocol
%%%   \renewcommand{\ALG@name}{Protocol}% Update algorithm name
%%%   \begin{algorithm}[#1]%
%%%   }{\end{algorithm}
%%% }
%%% \makeatother
%%%
%%% \providecommand*\algorithmautorefname{Protocol}
%%% \makeatletter
%%% \newenvironment{breakablealgorithm}
%%%   {% \begin{breakablealgorithm}
%%%    % \begin{center}
%%%      \refstepcounter{algorithm}% New algorithm
%%%      \hrule height.8pt depth0pt \kern2pt% \@fs@pre for \@fs@ruled
%%%      \renewcommand{\caption}[2][\relax]{% Make a new \caption
%%%        % {\raggedright\textbf{\ALG@name~\thealgorithm} ##2\par}%
%%%        {\raggedright\textbf{Protocol~\thealgorithm} ##2\par}%
%%%        \ifx\relax##1\relax % #1 is \relax
%%%          \addcontentsline{loa}{algorithm}{\protect\numberline{\thealgorithm}##2}%
%%%        \else % #1 is not \relax
%%%          \addcontentsline{loa}{algorithm}{\protect\numberline{\thealgorithm}##1}%
%%%        \fi
%%%        \kern2pt\hrule\kern2pt
%%%        \small
%%%      }
%%%   }{% \end{breakablealgorithm}
%%%      \kern2pt\hrule\relax% \@fs@post for \@fs@ruled
%%%    % \end{center}
%%%   }
%%% \makeatother
%%%

\def\EQ#1{\begin{eqnarray}#1\end{eqnarray}} 


%%% Practical abreviations
\usepackage{calrsfs} % Replace \mathcal with different more "cursive" font
\DeclareMathAlphabet{\pazocal}{OMS}{zplm}{m}{n} % \pazocal is now the old \mathcal
\newcommand*{\cA}{\pazocal{A}}
\newcommand*{\cB}{\pazocal{B}}
\newcommand*{\cC}{\pazocal{C}}
\newcommand*{\cD}{\pazocal{D}}
\newcommand*{\cE}{\pazocal{E}}
\newcommand*{\cF}{\pazocal{F}}
\newcommand*{\cG}{\pazocal{G}}
\newcommand*{\cH}{\pazocal{H}}
\newcommand*{\cI}{\pazocal{I}}
\newcommand*{\cJ}{\pazocal{J}}
\newcommand*{\cK}{\pazocal{K}}
\renewcommand*{\cL}{\pazocal{L}} % defined in complexity
\newcommand*{\cM}{\pazocal{M}}
\newcommand*{\cN}{\pazocal{N}}
\newcommand*{\cO}{\pazocal{O}}
\renewcommand*{\cP}{\pazocal{P}} % defined in complexity
\newcommand*{\cQ}{\pazocal{Q}}
\newcommand*{\cR}{\pazocal{R}}
\renewcommand*{\cS}{\pazocal{S}} % defined in complexity
\newcommand*{\cT}{\pazocal{T}}
\newcommand*{\cU}{\pazocal{U}}
\newcommand*{\cV}{\pazocal{V}}
\newcommand*{\cW}{\pazocal{W}}
\newcommand*{\cX}{\pazocal{X}}
\newcommand*{\cY}{\pazocal{Y}}
\newcommand*{\cZ}{\pazocal{Z}}

% Notation for languages. Make sure not to collide with L(H) for linear operators on H
%\DeclareMathAlphabet{\pazocal}{OMS}{zplm}{m}{n}
\newcommand{\linearOp}{\mathcal{L}} % Defines the opearator L(H)
\renewcommand*{\lang}{\pazocal{L}} % Defines a new Turing/Quantum Language

%https://tex.stackexchange.com/questions/195025/
\newcommand*{\rightSend}{\rightsquigarrow} % snake "a --> b" to say "I send a to b".

% For some reasons, proof-at-the-end do not like # (problem with detokenize? To fix...). So we use a macro:
\newcommand*{\diese}{\#}

\newcommand*{\eps}[0]{\varepsilon}
\renewcommand*{\D}[0]{\mathbb{D}}
\renewcommand*{\E}{\mathbb{E}}
\newcommand*{\N}[0]{\mathbb{N}}
\renewcommand*{\R}[0]{\mathbb{R}}
\renewcommand*{\C}[0]{\mathbb{C}}
\newcommand*{\Z}[0]{\mathbb{Z}}
\newcommand*{\T}[0]{\mathbb{T}}

\newcommand*{\CPA}[0]{\textsf{CPA}}
\newcommand*{\LWE}[0]{\textsf{LWE}}
\newcommand*{\GapSVP}[0]{\textsf{GapSVP}}
\newcommand*{\SIVP}[0]{\textsf{SIVP}}

% first param is alpha, second is q (optional)
\newcommand{\contGauss}[1]{\cD_{#1}}
\newcommand{\disGauss}[2]{\cD_{\Z,#1 #2}}
\newcommand{\disGaussCont}[1]{\cD_{\Z,#1}}
\newcommand{\disGaussAQ}{\cD_{\Z,\alpha q}}
\newcommand{\rsafe}{r_{\text{safe}}}
% domain of the function.
\newcommand{\ccX}{\cX}
\newcommand{\ccXSphere}{\cX_\bullet}
\newcommand{\ccXCube}[1][]{\cX_{\scalebox{.4}{$\blacksquare$}#1}}

\newcommand{\BBHeightyFor}{\textsc{BB84}}

% "Correct" spacing around cdot as wildcard
% https://tex.stackexchange.com/questions/78165/spacing-around-cdot-when-used-as-a-wildcard
\newcommand\cdotwild{{\mkern 2mu\cdot\mkern 2mu}}

\newcommand*{\id}{\mathrm{id}}
\DeclareMathOperator{\Tr}{Tr}
\DeclareMathOperator{\supp}{supp}
\DeclareMathOperator{\Det}{Det}
\DeclareMathOperator{\ERR}{ERR}
\DeclareMathOperator{\Round}{Round}

\newcommand{\Ver}{ {\tt Ver} } %% Verify public key
\newcommand{\Gen}{ {\tt Gen} }
\newcommand{\GenLocal}{\ensuremath{ {\tt Gen}_{\tt Loc}} }
\newcommand{\Enc}{ {\tt Enc} }
\newcommand{\Dec}{ {\tt Invert} }
\newcommand{\Eval}{ {\tt Eval} }

\newcommand{\HGen}{ {\tt H.Gen} }
\newcommand{\HEnc}{ {\tt H.Enc} }
\newcommand{\HDec}{ {\tt H.Invert} }
\newcommand{\HEval}{ {\tt H.Eval} }
\newcommand{\Hh}{\ensuremath{{\tt H.}h} }
\newcommand{\HCheckHonestTrapdoor}{\ensuremath{{\tt H.CheckTrapdoor}}}

\newcommand{\MPGen}{ {\tt MP.Gen} }
\newcommand{\MPDec}{ {\tt MP.Invert} }
\newcommand{\InvertSmallGadget}{ \ensuremath{{\texttt{InvertSmallGadget}}} }
\newcommand{\InvertGadget}{ {\tt InvertGadget} }
% \newcommand{\MPEval}{ $g$ }


\DeclareMathOperator{\RoundMod}{RoundMod}


% https://tex.stackexchange.com/a/253746/116348
\usepackage{mleftright}
\newcommand{\BlockMatrix}[2]{%
  \mleft[%
  \begin{array}{@{}#1@{}}%
    #2
  \end{array}%
  \mright]%
}
\newcommand{\SmallBlockMatrix}[2]{
  \mleft[\begin{array}{c}
    #1\tabularnewline % https://tex.stackexchange.com/questions/328745/matrices-with-cryptocode-package
    \hline
    #2
  \end{array}\mright]
}
\newcommand{\HorizBlockMatrix}[2]{
  \mleft[\begin{array}{@{}c|c@{}}
    #1 & #2
  \end{array}\mright]
}

\newcommand*{\vect}[1]{\ensuremath{\mathbf{#1}}}


%% Already defined in *recent* cryptocode.
% \DeclareMathOperator*{\argmax}{arg \, max}
% \DeclareMathOperator*{\argmin}{arg \, min}

\usepackage{xspace}
\xspaceaddexceptions{]\}}
\newcommand{\RSP}{\ensuremath{\sf{RSP}}\xspace}%

\newcommand*{\RSPenc}{\mathrm{RSP}_{\mathrm{enc}}}
\newcommand*{\filter}{\vdash}
\newcommand*{\channelBB}{\cS_{\Z\frac{\pi}{2}}}
% To denote a protocol between two parties
% \newcommand*{\protocol}{\boldsymbol{\pi}}
\newcommand\compRSP{composable $\sf{RSP}_{CC}$}
\newcommand\RSPlike{$\sf{RSP}_{CC}$}
\newcommand\compUBQC{composable $\sf{UBQC}_{CC}$}
\newcommand\UBQClike{$\sf{UBQC}_{CC}$}
% GHZ
\newcommand*{\eqdef}{\coloneqq}
\newcommand*{\PERM}{\ensuremath{}{\sf PERM}}
\newcommand*{\PPT}{\ensuremath{}{\sf PPT}}
\newcommand*{\QPT}{\ensuremath{}{\sf QPT}}
\newcommand*{\Auth}{\ensuremath{}{\sf Auth}}
% Prover, verifier, extractor
\renewcommand*{\P}{\ensuremath{}{\sf P}}
\newcommand*{\V}{\ensuremath{}{\sf V}}
\renewcommand*{\K}{\ensuremath{}{\sf K}}


% Here are the different protocols we use
% Initial protocol
\newcommand{\blind}{\ensuremath{ {\tt BLIND} }}
\newcommand{\blindZK}{\ensuremath{ {\tt BLIND-ZK} }}
% Reveals to people if they are part of the support
\newcommand{\blindSup}{\ensuremath{ {\tt BLIND}^{\tt sup} }}
% Make sure that people in the support share a canonical GHZ
\newcommand{\blindCan}{\ensuremath{ {\tt BLIND}_{\tt can} }}
% Make sure that people in the support share a canonical GHZ and that they know they are supported
\newcommand{\blindCanSup}{\ensuremath{ {\tt BLIND}^{\tt sup}_{\tt can} }}
\newcommand{\authBlindCanDist}{\ensuremath{ {\tt AUTH-BLIND}^{\tt dist}_{\tt can} }}
\newcommand{\verifCanSup}{\ensuremath{ {\tt VERIF}^{\tt sup}_{\tt can} }}
\newcommand{\blindPoly}{\ensuremath{ {\tt BLIND-poly} }}
\newcommand{\INDFCT}{\ensuremath{ {\tt IND-D0} }}
\newcommand{\INDBLIND}{\refGame*{INDBLIND}}
\newcommand{\INDBLINDtxt}{\ensuremath{ {\tt IND-\blind{}} }}
\newcommand{\INDBLINDSUP}{\refGame*{INDBLINDSUP}}
\newcommand{\INDBLINDSUPtxt}{\ensuremath{ {\tt IND-\blindSup{}} }}
\newcommand{\INDBLINDCANSUP}{\ensuremath{ {\tt IND-\blindCanSup{}} }}
\newcommand{\INDBLINDCANAUTH}{\ensuremath{ {\tt IND-\authBlindCanDist{}} }}
\newcommand{\INDPARTIAL}{\ensuremath{ {\tt IND-PARTIAL} }}
\newcommand*\GHZ{\ensuremath{}{\sf GHZ}}
\newcommand*\GHZCan{\ensuremath{{\sf GHZ^{\sf can}}}}
% Generalized GGHZ
\newcommand*\GGHZ{\ensuremath{ {\sf GHZ^G} }}
\newcommand*\HGHZ{\ensuremath{ {\sf GHZ^H} }}
\newcommand{\AssumpFctNoDelta}{\HGHZ{} capable}
\newcommand{\AssumpFctCanNoDelta}{\GHZCan{} capable}
\newcommand{\AssumpFct}{$\delta$-\HGHZ{} capable}
\newcommand{\AssumpFctNegl}{$\negl[\lambda]$-\HGHZ{} capable}
\newcommand{\AssumpFctCan}{$\delta$-\GHZCan{} capable}
\newcommand{\AssumpFctCanPrime}{$\delta^\prime$-\GHZCan{} capable}
\newcommand{\AssumpFctDist}{distributable}
\newcommand{\partialInfo}{\ensuremath{{\tt PartInfo}}}
\newcommand{\partialInfoLocal}{\ensuremath{{\tt PartInfo_{\tt Loc}}}}
\newcommand{\partialAlpha}{\ensuremath{{\tt PartAlpha_{\tt Loc}}}}
\newcommand{\CombineAlpha}{\ensuremath{{\tt CombineAlpha}}}
\newcommand{\CombineRandom}{\ensuremath{{\tt CombineRandom}}}
\newcommand{\CheckHonestTrapdoor}{\ensuremath{{\tt CheckTrapdoor}}}
\newcommand{\ZKGenCheck}{\ensuremath{{\tt ZK-GenCheck}}}

\newcommand{\highlightChanges}[1]{\textcolor{green!50!black}{#1}}

\newcommand{\Real}{\ensuremath{\mathsf{REAL}}}
\newcommand{\Sim}{\ensuremath{\mathsf{Sim}}}
\newcommand{\Ideal}{\ensuremath{\mathsf{IDEAL}}}
\newcommand{\OUT}{\ensuremath{\mathsf{OUT}}}

\usepackage{pifont}% http://ctan.org/pkg/pifont
% To denote that a user is not in the final GHZ
\newcommand{\cross}{\ensuremath{\text{\ding{55}}}}

\usepackage{verbatim}
\newcommand{\leftI}{{\normalfont\ttfamily left}}
\newcommand{\rightI}{{\normalfont\ttfamily right}}
\newcommand*{\mybar}[1]{\overline{#1}}

\newcommand*{\quout}{\mathtt{out}}

\DeclareMathOperator{\Image}{Image}
\DeclareMathOperator{\Ima}{Im}
\DeclareMathOperator{\Dom}{Dom}

\newcommand*{\Pub}{\mathtt{Pub}}
\newcommand*{\accept}{\mathtt{accept}}
\newcommand*{\abort}{\mathtt{abort}}
\newcommand*{\gadxor}{\mathtt{Gad}_{\oplus}}
\newcommand*{\quin}{\mathtt{in}}

%%% Provide a "subproof" environment (pretty simple for now,
%%% just indent the sub-proofs)
% I'd love to have left border... https://tex.stackexchange.com/questions/532948/robustly-add-a-border-to-the-left-of-a-text-spanning-several-pages
\usepackage{changepage}% http://ctan.org/pkg/changepage
\newenvironment{subproof}{\begin{adjustwidth}{2em}{0pt}}{\end{adjustwidth}}


%%% To have nice probabilities
% https://tex.stackexchange.com/questions/198771/align-in-substack
\makeatletter
\newcommand{\subalign}[1]{%
  \vcenter{%
    \Let@ \restore@math@cr \default@tag
    \baselineskip\fontdimen10 \scriptfont\tw@
    \advance\baselineskip\fontdimen12 \scriptfont\tw@
    \lineskip\thr@@\fontdimen8 \scriptfont\thr@@
    \lineskiplimit\lineskip
    \ialign{\hfil$\m@th\scriptstyle##$&$\m@th\scriptstyle{}##$\hfil\crcr
      #1\crcr
    }%
  }%
}
\makeatother
\usepackage{xparse}
% https://tex.stackexchange.com/questions/431794/same-spacing-around-middle-as-around-mid
\let\originalmiddle=\middle
\def\middle#1{\mathrel{}\originalmiddle#1\mathrel{}}
\NewDocumentCommand{\pr}{som}{\Pr\IfValueT{#2}{_{\substack{#2}}}\left[\,#3\,\right]}
\NewDocumentCommand{\esp}{som}{\IfValueTF{#2}{\underset{\subalign{#2}}{\mathbb{E}}}{\mathbb{E}}[\,#3\,]}
\NewDocumentEnvironment{pral}{soO{}}
 {%
  \Pr\IfValueT{#2}{_{\IfBooleanTF{#1}{\substack{#2}}{#2}}}
\begin{alignedat}[t]{2}
  [\, } {\,]
 #3
 \end{alignedat}
 }
\NewDocumentEnvironment{espal}{soO{}}
 {%
   \IfValueTF{#2}{\underset{\subalign{#2}}{\mathbb{E}}}{\mathbb{E}}
   \begin{alignedat}[t]{2}[\,
 }
 {\,]#3\end{alignedat}
 }
% To draw |A><B|, the second argument is facultative and gives |A><A|
\NewDocumentCommand{\ketbra}{mg}{
  | #1 \rangle \langle \IfValueTF{#2}{#2}{#1} |
}
% https://tex.stackexchange.com/a/188364/116348
\DeclareRobustCommand{\minwidthbox}[2]{%
  \ifmmode
    \expandafter\mathmakebox
  \else
    \expandafter\makebox
  \fi
  [\ifdim#2<\width\width\else#2\fi]{#1}%f
}

\NewDocumentCommand{\constructs}{mg}{
  \xrightarrow[\IfValueT{#2}{#2}]{\minwidthbox{#1}{5mm}}
}

%%%%%%%%%% This part is specific to llncs-based papers
% To enable this part of the configuration, add
%  \def\enableLlncsCode{}
% before the %% Let's put in these file all the definitions that we
%% require:

%% Setup language, encoding...
\RequirePackage{etex} 

\usepackage{standalone}
\usepackage{lmodern}
% Language
\usepackage[english]{babel}
% Input encoding
\usepackage[utf8]{inputenc}
% Output encoding https://tex.stackexchange.com/a/677. Important to copy accents
\usepackage[T1]{fontenc}
\usepackage{subfiles} % To quickly load this include file other files
\usepackage{microtype}
\usepackage[shortcuts,nospacearound]{extdash}    % Better endling of em dash (---) to cut sentences\===like 
%%% https://tex.stackexchange.com/questions/2773/how-do-i-make-latex-push-long-citations-to-a-new-line
\usepackage{breakcites}
\usepackage{booktabs}
% Makes sure floats don't fly accross sections
\usepackage{placeins}
\usepackage{adjustbox}
\usepackage{varwidth} % Like minipage, but with automatic width discovery.
\usepackage{complexity} % get \QMA{}...

\usepackage[normalem]{ulem}
%\usepackage[math-style=ISO]{unicode-math}
\usepackage{upgreek} % non italic greak letters

%%%% Bibliography
% I want to have separate references for appendix and main body
% https://tex.stackexchange.com/questions/98660/
\usepackage[style=trad-alpha,
  sortcites=true,
  doi=false,
  url=false,
  giveninits=true, % Bob Foo --> B. Foo
  isbn=false,
  url=false,
  eprint=false,
  sortcites=false, % \cite{B,A,C}: [A,B,C] --> [B,A,C]
]{biblatex}
\defbibfilter{appendixOnlyFilter}{
  segment=1 % Segment 1 will be chosen to be the one in appendix
  and not segment=0 % Default segment is 0
}
\renewcommand{\multicitedelim}{, } % [ABC96; DEF12] -> [ABC96, DEF12]
% [unknown_ref] => [??]
% https://tex.stackexchange.com/a/352573
\makeatletter
\protected\def\abx@missing#1{\textbf{??}}
\makeatother
\renewcommand*{\bibfont}{\normalfont\small} % BibLatex font seems bigger than Bibtex?

\usepackage{xcolor,cancel}
\usepackage{proof-at-the-end}
\NewDocumentEnvironment{theoremE}{O{}O{}+b}{%
  \begin{theoremEnd}[normal,#2]{theorem}[#1]%
    #3%
  \end{theoremEnd}
  %
}{}% Do not forget the second parameter or you might get Missing \begin{document} error
\NewDocumentEnvironment{lemmaE}{O{}O{}+b}{%
  \begin{theoremEnd}[normal,#2]{lemma}[#1]%
    #3%
  \end{theoremEnd}
  %
}{}% Do not forget the second parameter or you might get Missing \begin{document} error
\NewDocumentEnvironment{corollaryE}{O{}O{}+b}{%
  \begin{theoremEnd}[normal,#2]{corollary}[#1]%
    #3%
  \end{theoremEnd}
  %
}{}% Do not forget the second parameter or you might get Missing \begin{document} error
\NewDocumentEnvironment{proofE}{O{}+b}{%
  \begin{proofEnd}[#1]%
    #2
  \end{proofEnd}%
}{}%
\pgfkeys{/prAtEnd/global custom defaults/.style={
    text link={\noindent See \hyperref[proof:prAtEnd\pratendcountercurrent]{proof} in \cref{proofsection:prAtEnd\pratendcountercurrent}.}
  }
}

\ifdefined\displayArxivTexts%
  \pgfkeys{/prAtEnd/end if published/.style={no restate}}
  \pgfkeys{/prAtEnd/all end if published/.style={no restate}}
\else%
  \pgfkeys{/prAtEnd/end if published/.style={end}}
  \pgfkeys{/prAtEnd/all end if published/.style={all end}}
\fi

% https://tex.stackexchange.com/a/20613/116348
% https://tex.stackexchange.com/a/583244/116348
\usepackage{scalerel}
\newcommand\hcancel[2][black]{%
  \ThisStyle{%
    \setbox0=\hbox{$\SavedStyle #2$}%
    \rlap{\raisebox{.25\ht0}{\textcolor{#1}{\rule{\wd0}{1pt}}}}\SavedStyle #2}%
}

%% We provide 3 commands to add comments/text:
%% - \yournameAddText{your text to add}: proposition to add text
%% - \yournameRemoveText{text to remove}: proposition to remove text
%% - \yournameComment{your comments}: to add general comment to discuss. Should not contain proposition to add or remove text, but can be added before/after to explain the reason of the modification.
%% The interest it to provide some quick ways to see what is the final number of pages:
%% just add before the import a line:
%%   \def\removeCommentsAcceptPropositions{}
%% and the proposition will be "accepted" (i.e. the removed text is removed,
%% and the added text will be marked as black), and the comment will be removed, that way it is possible
%% to determine the final number of pages.
%% Example:
%% This text is normal\leoRemove{, this text should be removed}\leoAddText{, this text should be added}\leoComment{This is a comment, explaining why I made that changes.}.
\newcommand{\addEditor}[2]{%
  \expandafter\newcommand\csname #1AddText\endcsname[2][]{%
    \ifdefined\removeCommentsAcceptPropositions %
      ##2%
    \else%
      \textcolor{#2}{\sout{##1}}\textcolor{#2!60!black}{##2}%
    \fi%
  }%
  \expandafter\newcommand\csname #1Comment\endcsname[1]{%
    \ifdefined\removeCommentsAcceptPropositions %
    \else%
      \textcolor{#2!60!white}{\textbf{##1}}%
    \fi%
  }%
  \expandafter\newcommand\csname #1Remove\endcsname[1]{%
    \ifdefined\removeCommentsAcceptPropositions %
    \else%
      \textcolor{#2}{\sout{##1}}%
    \fi%
  }%
}
\definecolor{darkgreen}{rgb}{0., 0.7, 0}
\addEditor{leo}{darkgreen}
\addEditor{fred}{red}
\addEditor{elham}{blue}

%% We also provide an arxivOnly environment, in order to include text that are too long for the submitted version
%% but that would be good to have in the arxiv version.
% Use the \begin{arxivOnly} ... \end{arxivOnly} environment, by putting the opening and ending command
%% on there own line.
%% For shorter texts, you can use the inline version \arxivOnlyShort{your text}.
%% If you want to exclude text from the arxiv, use \begin{publishedOnly} ... \end{publishedOnly} and \publishedOnlyShort{your text}
%% "Comment" package is annoying, it does not allow block indentation, we use 'versions' instead
% \usepackage{versions} %% Do not work inside theoremE, don't know why
\ifdefined\displayArxivTexts%
  % \includeversion{arxivOnly}
  % \excludeversion{publishedOnly}
  \newcommand{\arxivOnly}[1]{#1}
  \newcommand{\arxivOnlyNoDiff}[1]{#1} % Like arxivOnly, but useful in diff mode (when it does not make sense to add colors, for instance inside a cite or when using arxivonly to create lists.
  \newcommand{\publishedOnly}[1]{}
  \newcommand{\publishedVsArxiv}[2]{#2}
  \newcommand{\inAppendixIfPublished}[1]{#1}
\else%
  \ifdefined\displayDiffTexts%
    \colorlet{arxivDiff}{green}
    \colorlet{publishedDiff}{red}
    \newcommand{\arxivOnly}[1]{{\color{arxivDiff}#1}}
    \newcommand{\arxivOnlyNoDiff}[1]{#1}
    \newcommand{\publishedOnly}[1]{{\color{publishedDiff}#1}}
    \newcommand{\publishedVsArxiv}[2]{{\color{publishedDiff}#1}{\color{arxivDiff}#2}}
    \newcommand{\inAppendixIfPublished}[1]{{\color{arxivDiff}\textbf{MOVED IN APPENDIX IN PUBLISHED}#1}\textEnd{{\color{publishedDiff}\textbf{MOVED IN MAIN BODY IN ARXIV}#1}}}
  \else
    % \excludeversion{arxivOnly}
    % \includeversion{publishedOnly}
    \newcommand{\arxivOnly}[1]{}
    \newcommand{\arxivOnlyNoDiff}[1]{}
    \newcommand{\publishedOnly}[1]{#1}
    \newcommand{\publishedVsArxiv}[2]{#1}
    \newcommand{\inAppendixIfPublished}[1]{\textEnd{#1}}
  \fi
\fi
% Alias, easier to remember  
\newcommand{\inAppendix}[1]{\textEnd{#1}}
  
% Symbol to display when an arxiv text is supposed to appear
\usepackage{todonotes}
\newcommand{\notifyArxiv}{\ifdefined\arxivDraftMode\todo{arXiv}\fi}

% Remove unwanted space before/after lists
\usepackage{enumitem}
\publishedOnly{\setlist[itemize]{noitemsep, topsep=0pt}}

% Usual packages:
\usepackage{amsmath}
\usepackage{amssymb}
% \usepackage{amsthm}
\usepackage{thmtools}
\usepackage{graphicx}
\graphicspath{{}{figures/}}
\usepackage{float}
\usepackage{subfig}
\usepackage{wrapfig}
\usepackage{array}
\usepackage{braket}
% Refer to footnotes:
% https://tex.stackexchange.com/a/167382/116348
\usepackage{footmisc}
% Useful to use the latest NewDocumentCommand:
%\usepackage{xparse}
%\usepackage[poster]{tcolorbox}

%% Space between pictures and text
%% https://mirrors.chevalier.io/CTAN/macros/latex/contrib/layouts/layman.pdf
%% \usepackage{layouts}
%% https://tex.stackexchange.com/questions/60477/remove-space-after-figure-and-before-text?rq=1
\setlength{\floatsep}{7pt plus 2pt minus 2pt}
\setlength{\textfloatsep}{7pt plus 2pt minus 2pt}
\setlength{\intextsep}{7pt plus 2pt minus 2pt}
% https://tex.stackexchange.com/questions/39017/how-to-influence-the-position-of-float-environments-like-figure-and-table-in-lat
\setcounter{topnumber}{8}
\setcounter{bottomnumber}{8}
\setcounter{totalnumber}{8}
%% https://tex.stackexchange.com/questions/45990/how-can-i-modify-vertical-space-between-figure-and-caption
%% plus/minus allows to stretch a bit around 15pt
% \setlength{\abovecaptionskip}{5pt}

% Remove number to subsubsection (in llncs it replaces \paragraph)
\setcounter{secnumdepth}{2}

%%% Some practical environment to define theorems, lemma...
% \declaretheorem[name=Theorem,numberwithin=section]{theorem}
% \declaretheorem[name=Definition,sibling=theorem]{definition}
% \declaretheorem[name=Lemma,sibling=theorem]{lemma}
% \declaretheorem[name=Corollary,sibling=theorem]{corollary}
% \declaretheorem[name=Remark,sibling=theorem]{remark}
%%% In LNCS we use
% https://tex.stackexchange.com/a/373125/116348
% \spnewtheorem{thm}{Theorem}[section]{\bfseries}{\itshape}

%%%
\usepackage[ruled,vlined]{algorithm2e}
\SetAlFnt{\normalsize}%
\setlength{\algomargin}{0em} % remove margin on left of algo
\SetAlgorithmName{Protocol}{protocol}{List of Protocols}
\newenvironment{protocol}[1][htb]{\begin{algorithm}[#1]}{\end{algorithm}}

%%% %%% Load algorithm package, and create a breakable version
%%% \usepackage{algorithm, algorithmicx, algpseudocode}
%%% % Decrease space between text:
%%% % https://tex.stackexchange.com/questions/25828/how-to-remove-change-the-vertical-spacing-before-and-after-an-algorithm-enviro/25832
%%% \setlength{\intextsep}{1\baselineskip}
%%% % https://tex.stackexchange.com/questions/387577/new-algorithm-environment-with-a-separate-counter
%%% \newcounter{protocol}
%%% \makeatletter
%%% \newenvironment{protocol}[1][htb]{%
%%%   \let\c@algorithm\c@protocol
%%%   \renewcommand{\ALG@name}{Protocol}% Update algorithm name
%%%   \begin{algorithm}[#1]%
%%%   }{\end{algorithm}
%%% }
%%% \makeatother
%%%
%%% \providecommand*\algorithmautorefname{Protocol}
%%% \makeatletter
%%% \newenvironment{breakablealgorithm}
%%%   {% \begin{breakablealgorithm}
%%%    % \begin{center}
%%%      \refstepcounter{algorithm}% New algorithm
%%%      \hrule height.8pt depth0pt \kern2pt% \@fs@pre for \@fs@ruled
%%%      \renewcommand{\caption}[2][\relax]{% Make a new \caption
%%%        % {\raggedright\textbf{\ALG@name~\thealgorithm} ##2\par}%
%%%        {\raggedright\textbf{Protocol~\thealgorithm} ##2\par}%
%%%        \ifx\relax##1\relax % #1 is \relax
%%%          \addcontentsline{loa}{algorithm}{\protect\numberline{\thealgorithm}##2}%
%%%        \else % #1 is not \relax
%%%          \addcontentsline{loa}{algorithm}{\protect\numberline{\thealgorithm}##1}%
%%%        \fi
%%%        \kern2pt\hrule\kern2pt
%%%        \small
%%%      }
%%%   }{% \end{breakablealgorithm}
%%%      \kern2pt\hrule\relax% \@fs@post for \@fs@ruled
%%%    % \end{center}
%%%   }
%%% \makeatother
%%%

\def\EQ#1{\begin{eqnarray}#1\end{eqnarray}} 


%%% Practical abreviations
\usepackage{calrsfs} % Replace \mathcal with different more "cursive" font
\DeclareMathAlphabet{\pazocal}{OMS}{zplm}{m}{n} % \pazocal is now the old \mathcal
\newcommand*{\cA}{\pazocal{A}}
\newcommand*{\cB}{\pazocal{B}}
\newcommand*{\cC}{\pazocal{C}}
\newcommand*{\cD}{\pazocal{D}}
\newcommand*{\cE}{\pazocal{E}}
\newcommand*{\cF}{\pazocal{F}}
\newcommand*{\cG}{\pazocal{G}}
\newcommand*{\cH}{\pazocal{H}}
\newcommand*{\cI}{\pazocal{I}}
\newcommand*{\cJ}{\pazocal{J}}
\newcommand*{\cK}{\pazocal{K}}
\renewcommand*{\cL}{\pazocal{L}} % defined in complexity
\newcommand*{\cM}{\pazocal{M}}
\newcommand*{\cN}{\pazocal{N}}
\newcommand*{\cO}{\pazocal{O}}
\renewcommand*{\cP}{\pazocal{P}} % defined in complexity
\newcommand*{\cQ}{\pazocal{Q}}
\newcommand*{\cR}{\pazocal{R}}
\renewcommand*{\cS}{\pazocal{S}} % defined in complexity
\newcommand*{\cT}{\pazocal{T}}
\newcommand*{\cU}{\pazocal{U}}
\newcommand*{\cV}{\pazocal{V}}
\newcommand*{\cW}{\pazocal{W}}
\newcommand*{\cX}{\pazocal{X}}
\newcommand*{\cY}{\pazocal{Y}}
\newcommand*{\cZ}{\pazocal{Z}}

% Notation for languages. Make sure not to collide with L(H) for linear operators on H
%\DeclareMathAlphabet{\pazocal}{OMS}{zplm}{m}{n}
\newcommand{\linearOp}{\mathcal{L}} % Defines the opearator L(H)
\renewcommand*{\lang}{\pazocal{L}} % Defines a new Turing/Quantum Language

%https://tex.stackexchange.com/questions/195025/
\newcommand*{\rightSend}{\rightsquigarrow} % snake "a --> b" to say "I send a to b".

% For some reasons, proof-at-the-end do not like # (problem with detokenize? To fix...). So we use a macro:
\newcommand*{\diese}{\#}

\newcommand*{\eps}[0]{\varepsilon}
\renewcommand*{\D}[0]{\mathbb{D}}
\renewcommand*{\E}{\mathbb{E}}
\newcommand*{\N}[0]{\mathbb{N}}
\renewcommand*{\R}[0]{\mathbb{R}}
\renewcommand*{\C}[0]{\mathbb{C}}
\newcommand*{\Z}[0]{\mathbb{Z}}
\newcommand*{\T}[0]{\mathbb{T}}

\newcommand*{\CPA}[0]{\textsf{CPA}}
\newcommand*{\LWE}[0]{\textsf{LWE}}
\newcommand*{\GapSVP}[0]{\textsf{GapSVP}}
\newcommand*{\SIVP}[0]{\textsf{SIVP}}

% first param is alpha, second is q (optional)
\newcommand{\contGauss}[1]{\cD_{#1}}
\newcommand{\disGauss}[2]{\cD_{\Z,#1 #2}}
\newcommand{\disGaussCont}[1]{\cD_{\Z,#1}}
\newcommand{\disGaussAQ}{\cD_{\Z,\alpha q}}
\newcommand{\rsafe}{r_{\text{safe}}}
% domain of the function.
\newcommand{\ccX}{\cX}
\newcommand{\ccXSphere}{\cX_\bullet}
\newcommand{\ccXCube}[1][]{\cX_{\scalebox{.4}{$\blacksquare$}#1}}

\newcommand{\BBHeightyFor}{\textsc{BB84}}

% "Correct" spacing around cdot as wildcard
% https://tex.stackexchange.com/questions/78165/spacing-around-cdot-when-used-as-a-wildcard
\newcommand\cdotwild{{\mkern 2mu\cdot\mkern 2mu}}

\newcommand*{\id}{\mathrm{id}}
\DeclareMathOperator{\Tr}{Tr}
\DeclareMathOperator{\supp}{supp}
\DeclareMathOperator{\Det}{Det}
\DeclareMathOperator{\ERR}{ERR}
\DeclareMathOperator{\Round}{Round}

\newcommand{\Ver}{ {\tt Ver} } %% Verify public key
\newcommand{\Gen}{ {\tt Gen} }
\newcommand{\GenLocal}{\ensuremath{ {\tt Gen}_{\tt Loc}} }
\newcommand{\Enc}{ {\tt Enc} }
\newcommand{\Dec}{ {\tt Invert} }
\newcommand{\Eval}{ {\tt Eval} }

\newcommand{\HGen}{ {\tt H.Gen} }
\newcommand{\HEnc}{ {\tt H.Enc} }
\newcommand{\HDec}{ {\tt H.Invert} }
\newcommand{\HEval}{ {\tt H.Eval} }
\newcommand{\Hh}{\ensuremath{{\tt H.}h} }
\newcommand{\HCheckHonestTrapdoor}{\ensuremath{{\tt H.CheckTrapdoor}}}

\newcommand{\MPGen}{ {\tt MP.Gen} }
\newcommand{\MPDec}{ {\tt MP.Invert} }
\newcommand{\InvertSmallGadget}{ \ensuremath{{\texttt{InvertSmallGadget}}} }
\newcommand{\InvertGadget}{ {\tt InvertGadget} }
% \newcommand{\MPEval}{ $g$ }


\DeclareMathOperator{\RoundMod}{RoundMod}


% https://tex.stackexchange.com/a/253746/116348
\usepackage{mleftright}
\newcommand{\BlockMatrix}[2]{%
  \mleft[%
  \begin{array}{@{}#1@{}}%
    #2
  \end{array}%
  \mright]%
}
\newcommand{\SmallBlockMatrix}[2]{
  \mleft[\begin{array}{c}
    #1\tabularnewline % https://tex.stackexchange.com/questions/328745/matrices-with-cryptocode-package
    \hline
    #2
  \end{array}\mright]
}
\newcommand{\HorizBlockMatrix}[2]{
  \mleft[\begin{array}{@{}c|c@{}}
    #1 & #2
  \end{array}\mright]
}

\newcommand*{\vect}[1]{\ensuremath{\mathbf{#1}}}


%% Already defined in *recent* cryptocode.
% \DeclareMathOperator*{\argmax}{arg \, max}
% \DeclareMathOperator*{\argmin}{arg \, min}

\usepackage{xspace}
\xspaceaddexceptions{]\}}
\newcommand{\RSP}{\ensuremath{\sf{RSP}}\xspace}%

\newcommand*{\RSPenc}{\mathrm{RSP}_{\mathrm{enc}}}
\newcommand*{\filter}{\vdash}
\newcommand*{\channelBB}{\cS_{\Z\frac{\pi}{2}}}
% To denote a protocol between two parties
% \newcommand*{\protocol}{\boldsymbol{\pi}}
\newcommand\compRSP{composable $\sf{RSP}_{CC}$}
\newcommand\RSPlike{$\sf{RSP}_{CC}$}
\newcommand\compUBQC{composable $\sf{UBQC}_{CC}$}
\newcommand\UBQClike{$\sf{UBQC}_{CC}$}
% GHZ
\newcommand*{\eqdef}{\coloneqq}
\newcommand*{\PERM}{\ensuremath{}{\sf PERM}}
\newcommand*{\PPT}{\ensuremath{}{\sf PPT}}
\newcommand*{\QPT}{\ensuremath{}{\sf QPT}}
\newcommand*{\Auth}{\ensuremath{}{\sf Auth}}
% Prover, verifier, extractor
\renewcommand*{\P}{\ensuremath{}{\sf P}}
\newcommand*{\V}{\ensuremath{}{\sf V}}
\renewcommand*{\K}{\ensuremath{}{\sf K}}


% Here are the different protocols we use
% Initial protocol
\newcommand{\blind}{\ensuremath{ {\tt BLIND} }}
\newcommand{\blindZK}{\ensuremath{ {\tt BLIND-ZK} }}
% Reveals to people if they are part of the support
\newcommand{\blindSup}{\ensuremath{ {\tt BLIND}^{\tt sup} }}
% Make sure that people in the support share a canonical GHZ
\newcommand{\blindCan}{\ensuremath{ {\tt BLIND}_{\tt can} }}
% Make sure that people in the support share a canonical GHZ and that they know they are supported
\newcommand{\blindCanSup}{\ensuremath{ {\tt BLIND}^{\tt sup}_{\tt can} }}
\newcommand{\authBlindCanDist}{\ensuremath{ {\tt AUTH-BLIND}^{\tt dist}_{\tt can} }}
\newcommand{\verifCanSup}{\ensuremath{ {\tt VERIF}^{\tt sup}_{\tt can} }}
\newcommand{\blindPoly}{\ensuremath{ {\tt BLIND-poly} }}
\newcommand{\INDFCT}{\ensuremath{ {\tt IND-D0} }}
\newcommand{\INDBLIND}{\refGame*{INDBLIND}}
\newcommand{\INDBLINDtxt}{\ensuremath{ {\tt IND-\blind{}} }}
\newcommand{\INDBLINDSUP}{\refGame*{INDBLINDSUP}}
\newcommand{\INDBLINDSUPtxt}{\ensuremath{ {\tt IND-\blindSup{}} }}
\newcommand{\INDBLINDCANSUP}{\ensuremath{ {\tt IND-\blindCanSup{}} }}
\newcommand{\INDBLINDCANAUTH}{\ensuremath{ {\tt IND-\authBlindCanDist{}} }}
\newcommand{\INDPARTIAL}{\ensuremath{ {\tt IND-PARTIAL} }}
\newcommand*\GHZ{\ensuremath{}{\sf GHZ}}
\newcommand*\GHZCan{\ensuremath{{\sf GHZ^{\sf can}}}}
% Generalized GGHZ
\newcommand*\GGHZ{\ensuremath{ {\sf GHZ^G} }}
\newcommand*\HGHZ{\ensuremath{ {\sf GHZ^H} }}
\newcommand{\AssumpFctNoDelta}{\HGHZ{} capable}
\newcommand{\AssumpFctCanNoDelta}{\GHZCan{} capable}
\newcommand{\AssumpFct}{$\delta$-\HGHZ{} capable}
\newcommand{\AssumpFctNegl}{$\negl[\lambda]$-\HGHZ{} capable}
\newcommand{\AssumpFctCan}{$\delta$-\GHZCan{} capable}
\newcommand{\AssumpFctCanPrime}{$\delta^\prime$-\GHZCan{} capable}
\newcommand{\AssumpFctDist}{distributable}
\newcommand{\partialInfo}{\ensuremath{{\tt PartInfo}}}
\newcommand{\partialInfoLocal}{\ensuremath{{\tt PartInfo_{\tt Loc}}}}
\newcommand{\partialAlpha}{\ensuremath{{\tt PartAlpha_{\tt Loc}}}}
\newcommand{\CombineAlpha}{\ensuremath{{\tt CombineAlpha}}}
\newcommand{\CombineRandom}{\ensuremath{{\tt CombineRandom}}}
\newcommand{\CheckHonestTrapdoor}{\ensuremath{{\tt CheckTrapdoor}}}
\newcommand{\ZKGenCheck}{\ensuremath{{\tt ZK-GenCheck}}}

\newcommand{\highlightChanges}[1]{\textcolor{green!50!black}{#1}}

\newcommand{\Real}{\ensuremath{\mathsf{REAL}}}
\newcommand{\Sim}{\ensuremath{\mathsf{Sim}}}
\newcommand{\Ideal}{\ensuremath{\mathsf{IDEAL}}}
\newcommand{\OUT}{\ensuremath{\mathsf{OUT}}}

\usepackage{pifont}% http://ctan.org/pkg/pifont
% To denote that a user is not in the final GHZ
\newcommand{\cross}{\ensuremath{\text{\ding{55}}}}

\usepackage{verbatim}
\newcommand{\leftI}{{\normalfont\ttfamily left}}
\newcommand{\rightI}{{\normalfont\ttfamily right}}
\newcommand*{\mybar}[1]{\overline{#1}}

\newcommand*{\quout}{\mathtt{out}}

\DeclareMathOperator{\Image}{Image}
\DeclareMathOperator{\Ima}{Im}
\DeclareMathOperator{\Dom}{Dom}

\newcommand*{\Pub}{\mathtt{Pub}}
\newcommand*{\accept}{\mathtt{accept}}
\newcommand*{\abort}{\mathtt{abort}}
\newcommand*{\gadxor}{\mathtt{Gad}_{\oplus}}
\newcommand*{\quin}{\mathtt{in}}

%%% Provide a "subproof" environment (pretty simple for now,
%%% just indent the sub-proofs)
% I'd love to have left border... https://tex.stackexchange.com/questions/532948/robustly-add-a-border-to-the-left-of-a-text-spanning-several-pages
\usepackage{changepage}% http://ctan.org/pkg/changepage
\newenvironment{subproof}{\begin{adjustwidth}{2em}{0pt}}{\end{adjustwidth}}


%%% To have nice probabilities
% https://tex.stackexchange.com/questions/198771/align-in-substack
\makeatletter
\newcommand{\subalign}[1]{%
  \vcenter{%
    \Let@ \restore@math@cr \default@tag
    \baselineskip\fontdimen10 \scriptfont\tw@
    \advance\baselineskip\fontdimen12 \scriptfont\tw@
    \lineskip\thr@@\fontdimen8 \scriptfont\thr@@
    \lineskiplimit\lineskip
    \ialign{\hfil$\m@th\scriptstyle##$&$\m@th\scriptstyle{}##$\hfil\crcr
      #1\crcr
    }%
  }%
}
\makeatother
\usepackage{xparse}
% https://tex.stackexchange.com/questions/431794/same-spacing-around-middle-as-around-mid
\let\originalmiddle=\middle
\def\middle#1{\mathrel{}\originalmiddle#1\mathrel{}}
\NewDocumentCommand{\pr}{som}{\Pr\IfValueT{#2}{_{\substack{#2}}}\left[\,#3\,\right]}
\NewDocumentCommand{\esp}{som}{\IfValueTF{#2}{\underset{\subalign{#2}}{\mathbb{E}}}{\mathbb{E}}[\,#3\,]}
\NewDocumentEnvironment{pral}{soO{}}
 {%
  \Pr\IfValueT{#2}{_{\IfBooleanTF{#1}{\substack{#2}}{#2}}}
\begin{alignedat}[t]{2}
  [\, } {\,]
 #3
 \end{alignedat}
 }
\NewDocumentEnvironment{espal}{soO{}}
 {%
   \IfValueTF{#2}{\underset{\subalign{#2}}{\mathbb{E}}}{\mathbb{E}}
   \begin{alignedat}[t]{2}[\,
 }
 {\,]#3\end{alignedat}
 }
% To draw |A><B|, the second argument is facultative and gives |A><A|
\NewDocumentCommand{\ketbra}{mg}{
  | #1 \rangle \langle \IfValueTF{#2}{#2}{#1} |
}
% https://tex.stackexchange.com/a/188364/116348
\DeclareRobustCommand{\minwidthbox}[2]{%
  \ifmmode
    \expandafter\mathmakebox
  \else
    \expandafter\makebox
  \fi
  [\ifdim#2<\width\width\else#2\fi]{#1}%f
}

\NewDocumentCommand{\constructs}{mg}{
  \xrightarrow[\IfValueT{#2}{#2}]{\minwidthbox{#1}{5mm}}
}

%%%%%%%%%% This part is specific to llncs-based papers
% To enable this part of the configuration, add
%  \def\enableLlncsCode{}
% before the %% Let's put in these file all the definitions that we
%% require:

%% Setup language, encoding...
\RequirePackage{etex} 

\usepackage{standalone}
\usepackage{lmodern}
% Language
\usepackage[english]{babel}
% Input encoding
\usepackage[utf8]{inputenc}
% Output encoding https://tex.stackexchange.com/a/677. Important to copy accents
\usepackage[T1]{fontenc}
\usepackage{subfiles} % To quickly load this include file other files
\usepackage{microtype}
\usepackage[shortcuts,nospacearound]{extdash}    % Better endling of em dash (---) to cut sentences\===like 
%%% https://tex.stackexchange.com/questions/2773/how-do-i-make-latex-push-long-citations-to-a-new-line
\usepackage{breakcites}
\usepackage{booktabs}
% Makes sure floats don't fly accross sections
\usepackage{placeins}
\usepackage{adjustbox}
\usepackage{varwidth} % Like minipage, but with automatic width discovery.
\usepackage{complexity} % get \QMA{}...

\usepackage[normalem]{ulem}
%\usepackage[math-style=ISO]{unicode-math}
\usepackage{upgreek} % non italic greak letters

%%%% Bibliography
% I want to have separate references for appendix and main body
% https://tex.stackexchange.com/questions/98660/
\usepackage[style=trad-alpha,
  sortcites=true,
  doi=false,
  url=false,
  giveninits=true, % Bob Foo --> B. Foo
  isbn=false,
  url=false,
  eprint=false,
  sortcites=false, % \cite{B,A,C}: [A,B,C] --> [B,A,C]
]{biblatex}
\defbibfilter{appendixOnlyFilter}{
  segment=1 % Segment 1 will be chosen to be the one in appendix
  and not segment=0 % Default segment is 0
}
\renewcommand{\multicitedelim}{, } % [ABC96; DEF12] -> [ABC96, DEF12]
% [unknown_ref] => [??]
% https://tex.stackexchange.com/a/352573
\makeatletter
\protected\def\abx@missing#1{\textbf{??}}
\makeatother
\renewcommand*{\bibfont}{\normalfont\small} % BibLatex font seems bigger than Bibtex?

\usepackage{xcolor,cancel}
\usepackage{proof-at-the-end}
\NewDocumentEnvironment{theoremE}{O{}O{}+b}{%
  \begin{theoremEnd}[normal,#2]{theorem}[#1]%
    #3%
  \end{theoremEnd}
  %
}{}% Do not forget the second parameter or you might get Missing \begin{document} error
\NewDocumentEnvironment{lemmaE}{O{}O{}+b}{%
  \begin{theoremEnd}[normal,#2]{lemma}[#1]%
    #3%
  \end{theoremEnd}
  %
}{}% Do not forget the second parameter or you might get Missing \begin{document} error
\NewDocumentEnvironment{corollaryE}{O{}O{}+b}{%
  \begin{theoremEnd}[normal,#2]{corollary}[#1]%
    #3%
  \end{theoremEnd}
  %
}{}% Do not forget the second parameter or you might get Missing \begin{document} error
\NewDocumentEnvironment{proofE}{O{}+b}{%
  \begin{proofEnd}[#1]%
    #2
  \end{proofEnd}%
}{}%
\pgfkeys{/prAtEnd/global custom defaults/.style={
    text link={\noindent See \hyperref[proof:prAtEnd\pratendcountercurrent]{proof} in \cref{proofsection:prAtEnd\pratendcountercurrent}.}
  }
}

\ifdefined\displayArxivTexts%
  \pgfkeys{/prAtEnd/end if published/.style={no restate}}
  \pgfkeys{/prAtEnd/all end if published/.style={no restate}}
\else%
  \pgfkeys{/prAtEnd/end if published/.style={end}}
  \pgfkeys{/prAtEnd/all end if published/.style={all end}}
\fi

% https://tex.stackexchange.com/a/20613/116348
% https://tex.stackexchange.com/a/583244/116348
\usepackage{scalerel}
\newcommand\hcancel[2][black]{%
  \ThisStyle{%
    \setbox0=\hbox{$\SavedStyle #2$}%
    \rlap{\raisebox{.25\ht0}{\textcolor{#1}{\rule{\wd0}{1pt}}}}\SavedStyle #2}%
}

%% We provide 3 commands to add comments/text:
%% - \yournameAddText{your text to add}: proposition to add text
%% - \yournameRemoveText{text to remove}: proposition to remove text
%% - \yournameComment{your comments}: to add general comment to discuss. Should not contain proposition to add or remove text, but can be added before/after to explain the reason of the modification.
%% The interest it to provide some quick ways to see what is the final number of pages:
%% just add before the import a line:
%%   \def\removeCommentsAcceptPropositions{}
%% and the proposition will be "accepted" (i.e. the removed text is removed,
%% and the added text will be marked as black), and the comment will be removed, that way it is possible
%% to determine the final number of pages.
%% Example:
%% This text is normal\leoRemove{, this text should be removed}\leoAddText{, this text should be added}\leoComment{This is a comment, explaining why I made that changes.}.
\newcommand{\addEditor}[2]{%
  \expandafter\newcommand\csname #1AddText\endcsname[2][]{%
    \ifdefined\removeCommentsAcceptPropositions %
      ##2%
    \else%
      \textcolor{#2}{\sout{##1}}\textcolor{#2!60!black}{##2}%
    \fi%
  }%
  \expandafter\newcommand\csname #1Comment\endcsname[1]{%
    \ifdefined\removeCommentsAcceptPropositions %
    \else%
      \textcolor{#2!60!white}{\textbf{##1}}%
    \fi%
  }%
  \expandafter\newcommand\csname #1Remove\endcsname[1]{%
    \ifdefined\removeCommentsAcceptPropositions %
    \else%
      \textcolor{#2}{\sout{##1}}%
    \fi%
  }%
}
\definecolor{darkgreen}{rgb}{0., 0.7, 0}
\addEditor{leo}{darkgreen}
\addEditor{fred}{red}
\addEditor{elham}{blue}

%% We also provide an arxivOnly environment, in order to include text that are too long for the submitted version
%% but that would be good to have in the arxiv version.
% Use the \begin{arxivOnly} ... \end{arxivOnly} environment, by putting the opening and ending command
%% on there own line.
%% For shorter texts, you can use the inline version \arxivOnlyShort{your text}.
%% If you want to exclude text from the arxiv, use \begin{publishedOnly} ... \end{publishedOnly} and \publishedOnlyShort{your text}
%% "Comment" package is annoying, it does not allow block indentation, we use 'versions' instead
% \usepackage{versions} %% Do not work inside theoremE, don't know why
\ifdefined\displayArxivTexts%
  % \includeversion{arxivOnly}
  % \excludeversion{publishedOnly}
  \newcommand{\arxivOnly}[1]{#1}
  \newcommand{\arxivOnlyNoDiff}[1]{#1} % Like arxivOnly, but useful in diff mode (when it does not make sense to add colors, for instance inside a cite or when using arxivonly to create lists.
  \newcommand{\publishedOnly}[1]{}
  \newcommand{\publishedVsArxiv}[2]{#2}
  \newcommand{\inAppendixIfPublished}[1]{#1}
\else%
  \ifdefined\displayDiffTexts%
    \colorlet{arxivDiff}{green}
    \colorlet{publishedDiff}{red}
    \newcommand{\arxivOnly}[1]{{\color{arxivDiff}#1}}
    \newcommand{\arxivOnlyNoDiff}[1]{#1}
    \newcommand{\publishedOnly}[1]{{\color{publishedDiff}#1}}
    \newcommand{\publishedVsArxiv}[2]{{\color{publishedDiff}#1}{\color{arxivDiff}#2}}
    \newcommand{\inAppendixIfPublished}[1]{{\color{arxivDiff}\textbf{MOVED IN APPENDIX IN PUBLISHED}#1}\textEnd{{\color{publishedDiff}\textbf{MOVED IN MAIN BODY IN ARXIV}#1}}}
  \else
    % \excludeversion{arxivOnly}
    % \includeversion{publishedOnly}
    \newcommand{\arxivOnly}[1]{}
    \newcommand{\arxivOnlyNoDiff}[1]{}
    \newcommand{\publishedOnly}[1]{#1}
    \newcommand{\publishedVsArxiv}[2]{#1}
    \newcommand{\inAppendixIfPublished}[1]{\textEnd{#1}}
  \fi
\fi
% Alias, easier to remember  
\newcommand{\inAppendix}[1]{\textEnd{#1}}
  
% Symbol to display when an arxiv text is supposed to appear
\usepackage{todonotes}
\newcommand{\notifyArxiv}{\ifdefined\arxivDraftMode\todo{arXiv}\fi}

% Remove unwanted space before/after lists
\usepackage{enumitem}
\publishedOnly{\setlist[itemize]{noitemsep, topsep=0pt}}

% Usual packages:
\usepackage{amsmath}
\usepackage{amssymb}
% \usepackage{amsthm}
\usepackage{thmtools}
\usepackage{graphicx}
\graphicspath{{}{figures/}}
\usepackage{float}
\usepackage{subfig}
\usepackage{wrapfig}
\usepackage{array}
\usepackage{braket}
% Refer to footnotes:
% https://tex.stackexchange.com/a/167382/116348
\usepackage{footmisc}
% Useful to use the latest NewDocumentCommand:
%\usepackage{xparse}
%\usepackage[poster]{tcolorbox}

%% Space between pictures and text
%% https://mirrors.chevalier.io/CTAN/macros/latex/contrib/layouts/layman.pdf
%% \usepackage{layouts}
%% https://tex.stackexchange.com/questions/60477/remove-space-after-figure-and-before-text?rq=1
\setlength{\floatsep}{7pt plus 2pt minus 2pt}
\setlength{\textfloatsep}{7pt plus 2pt minus 2pt}
\setlength{\intextsep}{7pt plus 2pt minus 2pt}
% https://tex.stackexchange.com/questions/39017/how-to-influence-the-position-of-float-environments-like-figure-and-table-in-lat
\setcounter{topnumber}{8}
\setcounter{bottomnumber}{8}
\setcounter{totalnumber}{8}
%% https://tex.stackexchange.com/questions/45990/how-can-i-modify-vertical-space-between-figure-and-caption
%% plus/minus allows to stretch a bit around 15pt
% \setlength{\abovecaptionskip}{5pt}

% Remove number to subsubsection (in llncs it replaces \paragraph)
\setcounter{secnumdepth}{2}

%%% Some practical environment to define theorems, lemma...
% \declaretheorem[name=Theorem,numberwithin=section]{theorem}
% \declaretheorem[name=Definition,sibling=theorem]{definition}
% \declaretheorem[name=Lemma,sibling=theorem]{lemma}
% \declaretheorem[name=Corollary,sibling=theorem]{corollary}
% \declaretheorem[name=Remark,sibling=theorem]{remark}
%%% In LNCS we use
% https://tex.stackexchange.com/a/373125/116348
% \spnewtheorem{thm}{Theorem}[section]{\bfseries}{\itshape}

%%%
\usepackage[ruled,vlined]{algorithm2e}
\SetAlFnt{\normalsize}%
\setlength{\algomargin}{0em} % remove margin on left of algo
\SetAlgorithmName{Protocol}{protocol}{List of Protocols}
\newenvironment{protocol}[1][htb]{\begin{algorithm}[#1]}{\end{algorithm}}

%%% %%% Load algorithm package, and create a breakable version
%%% \usepackage{algorithm, algorithmicx, algpseudocode}
%%% % Decrease space between text:
%%% % https://tex.stackexchange.com/questions/25828/how-to-remove-change-the-vertical-spacing-before-and-after-an-algorithm-enviro/25832
%%% \setlength{\intextsep}{1\baselineskip}
%%% % https://tex.stackexchange.com/questions/387577/new-algorithm-environment-with-a-separate-counter
%%% \newcounter{protocol}
%%% \makeatletter
%%% \newenvironment{protocol}[1][htb]{%
%%%   \let\c@algorithm\c@protocol
%%%   \renewcommand{\ALG@name}{Protocol}% Update algorithm name
%%%   \begin{algorithm}[#1]%
%%%   }{\end{algorithm}
%%% }
%%% \makeatother
%%%
%%% \providecommand*\algorithmautorefname{Protocol}
%%% \makeatletter
%%% \newenvironment{breakablealgorithm}
%%%   {% \begin{breakablealgorithm}
%%%    % \begin{center}
%%%      \refstepcounter{algorithm}% New algorithm
%%%      \hrule height.8pt depth0pt \kern2pt% \@fs@pre for \@fs@ruled
%%%      \renewcommand{\caption}[2][\relax]{% Make a new \caption
%%%        % {\raggedright\textbf{\ALG@name~\thealgorithm} ##2\par}%
%%%        {\raggedright\textbf{Protocol~\thealgorithm} ##2\par}%
%%%        \ifx\relax##1\relax % #1 is \relax
%%%          \addcontentsline{loa}{algorithm}{\protect\numberline{\thealgorithm}##2}%
%%%        \else % #1 is not \relax
%%%          \addcontentsline{loa}{algorithm}{\protect\numberline{\thealgorithm}##1}%
%%%        \fi
%%%        \kern2pt\hrule\kern2pt
%%%        \small
%%%      }
%%%   }{% \end{breakablealgorithm}
%%%      \kern2pt\hrule\relax% \@fs@post for \@fs@ruled
%%%    % \end{center}
%%%   }
%%% \makeatother
%%%

\def\EQ#1{\begin{eqnarray}#1\end{eqnarray}} 


%%% Practical abreviations
\usepackage{calrsfs} % Replace \mathcal with different more "cursive" font
\DeclareMathAlphabet{\pazocal}{OMS}{zplm}{m}{n} % \pazocal is now the old \mathcal
\newcommand*{\cA}{\pazocal{A}}
\newcommand*{\cB}{\pazocal{B}}
\newcommand*{\cC}{\pazocal{C}}
\newcommand*{\cD}{\pazocal{D}}
\newcommand*{\cE}{\pazocal{E}}
\newcommand*{\cF}{\pazocal{F}}
\newcommand*{\cG}{\pazocal{G}}
\newcommand*{\cH}{\pazocal{H}}
\newcommand*{\cI}{\pazocal{I}}
\newcommand*{\cJ}{\pazocal{J}}
\newcommand*{\cK}{\pazocal{K}}
\renewcommand*{\cL}{\pazocal{L}} % defined in complexity
\newcommand*{\cM}{\pazocal{M}}
\newcommand*{\cN}{\pazocal{N}}
\newcommand*{\cO}{\pazocal{O}}
\renewcommand*{\cP}{\pazocal{P}} % defined in complexity
\newcommand*{\cQ}{\pazocal{Q}}
\newcommand*{\cR}{\pazocal{R}}
\renewcommand*{\cS}{\pazocal{S}} % defined in complexity
\newcommand*{\cT}{\pazocal{T}}
\newcommand*{\cU}{\pazocal{U}}
\newcommand*{\cV}{\pazocal{V}}
\newcommand*{\cW}{\pazocal{W}}
\newcommand*{\cX}{\pazocal{X}}
\newcommand*{\cY}{\pazocal{Y}}
\newcommand*{\cZ}{\pazocal{Z}}

% Notation for languages. Make sure not to collide with L(H) for linear operators on H
%\DeclareMathAlphabet{\pazocal}{OMS}{zplm}{m}{n}
\newcommand{\linearOp}{\mathcal{L}} % Defines the opearator L(H)
\renewcommand*{\lang}{\pazocal{L}} % Defines a new Turing/Quantum Language

%https://tex.stackexchange.com/questions/195025/
\newcommand*{\rightSend}{\rightsquigarrow} % snake "a --> b" to say "I send a to b".

% For some reasons, proof-at-the-end do not like # (problem with detokenize? To fix...). So we use a macro:
\newcommand*{\diese}{\#}

\newcommand*{\eps}[0]{\varepsilon}
\renewcommand*{\D}[0]{\mathbb{D}}
\renewcommand*{\E}{\mathbb{E}}
\newcommand*{\N}[0]{\mathbb{N}}
\renewcommand*{\R}[0]{\mathbb{R}}
\renewcommand*{\C}[0]{\mathbb{C}}
\newcommand*{\Z}[0]{\mathbb{Z}}
\newcommand*{\T}[0]{\mathbb{T}}

\newcommand*{\CPA}[0]{\textsf{CPA}}
\newcommand*{\LWE}[0]{\textsf{LWE}}
\newcommand*{\GapSVP}[0]{\textsf{GapSVP}}
\newcommand*{\SIVP}[0]{\textsf{SIVP}}

% first param is alpha, second is q (optional)
\newcommand{\contGauss}[1]{\cD_{#1}}
\newcommand{\disGauss}[2]{\cD_{\Z,#1 #2}}
\newcommand{\disGaussCont}[1]{\cD_{\Z,#1}}
\newcommand{\disGaussAQ}{\cD_{\Z,\alpha q}}
\newcommand{\rsafe}{r_{\text{safe}}}
% domain of the function.
\newcommand{\ccX}{\cX}
\newcommand{\ccXSphere}{\cX_\bullet}
\newcommand{\ccXCube}[1][]{\cX_{\scalebox{.4}{$\blacksquare$}#1}}

\newcommand{\BBHeightyFor}{\textsc{BB84}}

% "Correct" spacing around cdot as wildcard
% https://tex.stackexchange.com/questions/78165/spacing-around-cdot-when-used-as-a-wildcard
\newcommand\cdotwild{{\mkern 2mu\cdot\mkern 2mu}}

\newcommand*{\id}{\mathrm{id}}
\DeclareMathOperator{\Tr}{Tr}
\DeclareMathOperator{\supp}{supp}
\DeclareMathOperator{\Det}{Det}
\DeclareMathOperator{\ERR}{ERR}
\DeclareMathOperator{\Round}{Round}

\newcommand{\Ver}{ {\tt Ver} } %% Verify public key
\newcommand{\Gen}{ {\tt Gen} }
\newcommand{\GenLocal}{\ensuremath{ {\tt Gen}_{\tt Loc}} }
\newcommand{\Enc}{ {\tt Enc} }
\newcommand{\Dec}{ {\tt Invert} }
\newcommand{\Eval}{ {\tt Eval} }

\newcommand{\HGen}{ {\tt H.Gen} }
\newcommand{\HEnc}{ {\tt H.Enc} }
\newcommand{\HDec}{ {\tt H.Invert} }
\newcommand{\HEval}{ {\tt H.Eval} }
\newcommand{\Hh}{\ensuremath{{\tt H.}h} }
\newcommand{\HCheckHonestTrapdoor}{\ensuremath{{\tt H.CheckTrapdoor}}}

\newcommand{\MPGen}{ {\tt MP.Gen} }
\newcommand{\MPDec}{ {\tt MP.Invert} }
\newcommand{\InvertSmallGadget}{ \ensuremath{{\texttt{InvertSmallGadget}}} }
\newcommand{\InvertGadget}{ {\tt InvertGadget} }
% \newcommand{\MPEval}{ $g$ }


\DeclareMathOperator{\RoundMod}{RoundMod}


% https://tex.stackexchange.com/a/253746/116348
\usepackage{mleftright}
\newcommand{\BlockMatrix}[2]{%
  \mleft[%
  \begin{array}{@{}#1@{}}%
    #2
  \end{array}%
  \mright]%
}
\newcommand{\SmallBlockMatrix}[2]{
  \mleft[\begin{array}{c}
    #1\tabularnewline % https://tex.stackexchange.com/questions/328745/matrices-with-cryptocode-package
    \hline
    #2
  \end{array}\mright]
}
\newcommand{\HorizBlockMatrix}[2]{
  \mleft[\begin{array}{@{}c|c@{}}
    #1 & #2
  \end{array}\mright]
}

\newcommand*{\vect}[1]{\ensuremath{\mathbf{#1}}}


%% Already defined in *recent* cryptocode.
% \DeclareMathOperator*{\argmax}{arg \, max}
% \DeclareMathOperator*{\argmin}{arg \, min}

\usepackage{xspace}
\xspaceaddexceptions{]\}}
\newcommand{\RSP}{\ensuremath{\sf{RSP}}\xspace}%

\newcommand*{\RSPenc}{\mathrm{RSP}_{\mathrm{enc}}}
\newcommand*{\filter}{\vdash}
\newcommand*{\channelBB}{\cS_{\Z\frac{\pi}{2}}}
% To denote a protocol between two parties
% \newcommand*{\protocol}{\boldsymbol{\pi}}
\newcommand\compRSP{composable $\sf{RSP}_{CC}$}
\newcommand\RSPlike{$\sf{RSP}_{CC}$}
\newcommand\compUBQC{composable $\sf{UBQC}_{CC}$}
\newcommand\UBQClike{$\sf{UBQC}_{CC}$}
% GHZ
\newcommand*{\eqdef}{\coloneqq}
\newcommand*{\PERM}{\ensuremath{}{\sf PERM}}
\newcommand*{\PPT}{\ensuremath{}{\sf PPT}}
\newcommand*{\QPT}{\ensuremath{}{\sf QPT}}
\newcommand*{\Auth}{\ensuremath{}{\sf Auth}}
% Prover, verifier, extractor
\renewcommand*{\P}{\ensuremath{}{\sf P}}
\newcommand*{\V}{\ensuremath{}{\sf V}}
\renewcommand*{\K}{\ensuremath{}{\sf K}}


% Here are the different protocols we use
% Initial protocol
\newcommand{\blind}{\ensuremath{ {\tt BLIND} }}
\newcommand{\blindZK}{\ensuremath{ {\tt BLIND-ZK} }}
% Reveals to people if they are part of the support
\newcommand{\blindSup}{\ensuremath{ {\tt BLIND}^{\tt sup} }}
% Make sure that people in the support share a canonical GHZ
\newcommand{\blindCan}{\ensuremath{ {\tt BLIND}_{\tt can} }}
% Make sure that people in the support share a canonical GHZ and that they know they are supported
\newcommand{\blindCanSup}{\ensuremath{ {\tt BLIND}^{\tt sup}_{\tt can} }}
\newcommand{\authBlindCanDist}{\ensuremath{ {\tt AUTH-BLIND}^{\tt dist}_{\tt can} }}
\newcommand{\verifCanSup}{\ensuremath{ {\tt VERIF}^{\tt sup}_{\tt can} }}
\newcommand{\blindPoly}{\ensuremath{ {\tt BLIND-poly} }}
\newcommand{\INDFCT}{\ensuremath{ {\tt IND-D0} }}
\newcommand{\INDBLIND}{\refGame*{INDBLIND}}
\newcommand{\INDBLINDtxt}{\ensuremath{ {\tt IND-\blind{}} }}
\newcommand{\INDBLINDSUP}{\refGame*{INDBLINDSUP}}
\newcommand{\INDBLINDSUPtxt}{\ensuremath{ {\tt IND-\blindSup{}} }}
\newcommand{\INDBLINDCANSUP}{\ensuremath{ {\tt IND-\blindCanSup{}} }}
\newcommand{\INDBLINDCANAUTH}{\ensuremath{ {\tt IND-\authBlindCanDist{}} }}
\newcommand{\INDPARTIAL}{\ensuremath{ {\tt IND-PARTIAL} }}
\newcommand*\GHZ{\ensuremath{}{\sf GHZ}}
\newcommand*\GHZCan{\ensuremath{{\sf GHZ^{\sf can}}}}
% Generalized GGHZ
\newcommand*\GGHZ{\ensuremath{ {\sf GHZ^G} }}
\newcommand*\HGHZ{\ensuremath{ {\sf GHZ^H} }}
\newcommand{\AssumpFctNoDelta}{\HGHZ{} capable}
\newcommand{\AssumpFctCanNoDelta}{\GHZCan{} capable}
\newcommand{\AssumpFct}{$\delta$-\HGHZ{} capable}
\newcommand{\AssumpFctNegl}{$\negl[\lambda]$-\HGHZ{} capable}
\newcommand{\AssumpFctCan}{$\delta$-\GHZCan{} capable}
\newcommand{\AssumpFctCanPrime}{$\delta^\prime$-\GHZCan{} capable}
\newcommand{\AssumpFctDist}{distributable}
\newcommand{\partialInfo}{\ensuremath{{\tt PartInfo}}}
\newcommand{\partialInfoLocal}{\ensuremath{{\tt PartInfo_{\tt Loc}}}}
\newcommand{\partialAlpha}{\ensuremath{{\tt PartAlpha_{\tt Loc}}}}
\newcommand{\CombineAlpha}{\ensuremath{{\tt CombineAlpha}}}
\newcommand{\CombineRandom}{\ensuremath{{\tt CombineRandom}}}
\newcommand{\CheckHonestTrapdoor}{\ensuremath{{\tt CheckTrapdoor}}}
\newcommand{\ZKGenCheck}{\ensuremath{{\tt ZK-GenCheck}}}

\newcommand{\highlightChanges}[1]{\textcolor{green!50!black}{#1}}

\newcommand{\Real}{\ensuremath{\mathsf{REAL}}}
\newcommand{\Sim}{\ensuremath{\mathsf{Sim}}}
\newcommand{\Ideal}{\ensuremath{\mathsf{IDEAL}}}
\newcommand{\OUT}{\ensuremath{\mathsf{OUT}}}

\usepackage{pifont}% http://ctan.org/pkg/pifont
% To denote that a user is not in the final GHZ
\newcommand{\cross}{\ensuremath{\text{\ding{55}}}}

\usepackage{verbatim}
\newcommand{\leftI}{{\normalfont\ttfamily left}}
\newcommand{\rightI}{{\normalfont\ttfamily right}}
\newcommand*{\mybar}[1]{\overline{#1}}

\newcommand*{\quout}{\mathtt{out}}

\DeclareMathOperator{\Image}{Image}
\DeclareMathOperator{\Ima}{Im}
\DeclareMathOperator{\Dom}{Dom}

\newcommand*{\Pub}{\mathtt{Pub}}
\newcommand*{\accept}{\mathtt{accept}}
\newcommand*{\abort}{\mathtt{abort}}
\newcommand*{\gadxor}{\mathtt{Gad}_{\oplus}}
\newcommand*{\quin}{\mathtt{in}}

%%% Provide a "subproof" environment (pretty simple for now,
%%% just indent the sub-proofs)
% I'd love to have left border... https://tex.stackexchange.com/questions/532948/robustly-add-a-border-to-the-left-of-a-text-spanning-several-pages
\usepackage{changepage}% http://ctan.org/pkg/changepage
\newenvironment{subproof}{\begin{adjustwidth}{2em}{0pt}}{\end{adjustwidth}}


%%% To have nice probabilities
% https://tex.stackexchange.com/questions/198771/align-in-substack
\makeatletter
\newcommand{\subalign}[1]{%
  \vcenter{%
    \Let@ \restore@math@cr \default@tag
    \baselineskip\fontdimen10 \scriptfont\tw@
    \advance\baselineskip\fontdimen12 \scriptfont\tw@
    \lineskip\thr@@\fontdimen8 \scriptfont\thr@@
    \lineskiplimit\lineskip
    \ialign{\hfil$\m@th\scriptstyle##$&$\m@th\scriptstyle{}##$\hfil\crcr
      #1\crcr
    }%
  }%
}
\makeatother
\usepackage{xparse}
% https://tex.stackexchange.com/questions/431794/same-spacing-around-middle-as-around-mid
\let\originalmiddle=\middle
\def\middle#1{\mathrel{}\originalmiddle#1\mathrel{}}
\NewDocumentCommand{\pr}{som}{\Pr\IfValueT{#2}{_{\substack{#2}}}\left[\,#3\,\right]}
\NewDocumentCommand{\esp}{som}{\IfValueTF{#2}{\underset{\subalign{#2}}{\mathbb{E}}}{\mathbb{E}}[\,#3\,]}
\NewDocumentEnvironment{pral}{soO{}}
 {%
  \Pr\IfValueT{#2}{_{\IfBooleanTF{#1}{\substack{#2}}{#2}}}
\begin{alignedat}[t]{2}
  [\, } {\,]
 #3
 \end{alignedat}
 }
\NewDocumentEnvironment{espal}{soO{}}
 {%
   \IfValueTF{#2}{\underset{\subalign{#2}}{\mathbb{E}}}{\mathbb{E}}
   \begin{alignedat}[t]{2}[\,
 }
 {\,]#3\end{alignedat}
 }
% To draw |A><B|, the second argument is facultative and gives |A><A|
\NewDocumentCommand{\ketbra}{mg}{
  | #1 \rangle \langle \IfValueTF{#2}{#2}{#1} |
}
% https://tex.stackexchange.com/a/188364/116348
\DeclareRobustCommand{\minwidthbox}[2]{%
  \ifmmode
    \expandafter\mathmakebox
  \else
    \expandafter\makebox
  \fi
  [\ifdim#2<\width\width\else#2\fi]{#1}%f
}

\NewDocumentCommand{\constructs}{mg}{
  \xrightarrow[\IfValueT{#2}{#2}]{\minwidthbox{#1}{5mm}}
}

%%%%%%%%%% This part is specific to llncs-based papers
% To enable this part of the configuration, add
%  \def\enableLlncsCode{}
% before the %% Let's put in these file all the definitions that we
%% require:

%% Setup language, encoding...
\RequirePackage{etex} 

\usepackage{standalone}
\usepackage{lmodern}
% Language
\usepackage[english]{babel}
% Input encoding
\usepackage[utf8]{inputenc}
% Output encoding https://tex.stackexchange.com/a/677. Important to copy accents
\usepackage[T1]{fontenc}
\usepackage{subfiles} % To quickly load this include file other files
\usepackage{microtype}
\usepackage[shortcuts,nospacearound]{extdash}    % Better endling of em dash (---) to cut sentences\===like 
%%% https://tex.stackexchange.com/questions/2773/how-do-i-make-latex-push-long-citations-to-a-new-line
\usepackage{breakcites}
\usepackage{booktabs}
% Makes sure floats don't fly accross sections
\usepackage{placeins}
\usepackage{adjustbox}
\usepackage{varwidth} % Like minipage, but with automatic width discovery.
\usepackage{complexity} % get \QMA{}...

\usepackage[normalem]{ulem}
%\usepackage[math-style=ISO]{unicode-math}
\usepackage{upgreek} % non italic greak letters

%%%% Bibliography
% I want to have separate references for appendix and main body
% https://tex.stackexchange.com/questions/98660/
\usepackage[style=trad-alpha,
  sortcites=true,
  doi=false,
  url=false,
  giveninits=true, % Bob Foo --> B. Foo
  isbn=false,
  url=false,
  eprint=false,
  sortcites=false, % \cite{B,A,C}: [A,B,C] --> [B,A,C]
]{biblatex}
\defbibfilter{appendixOnlyFilter}{
  segment=1 % Segment 1 will be chosen to be the one in appendix
  and not segment=0 % Default segment is 0
}
\renewcommand{\multicitedelim}{, } % [ABC96; DEF12] -> [ABC96, DEF12]
% [unknown_ref] => [??]
% https://tex.stackexchange.com/a/352573
\makeatletter
\protected\def\abx@missing#1{\textbf{??}}
\makeatother
\renewcommand*{\bibfont}{\normalfont\small} % BibLatex font seems bigger than Bibtex?

\usepackage{xcolor,cancel}
\usepackage{proof-at-the-end}
\NewDocumentEnvironment{theoremE}{O{}O{}+b}{%
  \begin{theoremEnd}[normal,#2]{theorem}[#1]%
    #3%
  \end{theoremEnd}
  %
}{}% Do not forget the second parameter or you might get Missing \begin{document} error
\NewDocumentEnvironment{lemmaE}{O{}O{}+b}{%
  \begin{theoremEnd}[normal,#2]{lemma}[#1]%
    #3%
  \end{theoremEnd}
  %
}{}% Do not forget the second parameter or you might get Missing \begin{document} error
\NewDocumentEnvironment{corollaryE}{O{}O{}+b}{%
  \begin{theoremEnd}[normal,#2]{corollary}[#1]%
    #3%
  \end{theoremEnd}
  %
}{}% Do not forget the second parameter or you might get Missing \begin{document} error
\NewDocumentEnvironment{proofE}{O{}+b}{%
  \begin{proofEnd}[#1]%
    #2
  \end{proofEnd}%
}{}%
\pgfkeys{/prAtEnd/global custom defaults/.style={
    text link={\noindent See \hyperref[proof:prAtEnd\pratendcountercurrent]{proof} in \cref{proofsection:prAtEnd\pratendcountercurrent}.}
  }
}

\ifdefined\displayArxivTexts%
  \pgfkeys{/prAtEnd/end if published/.style={no restate}}
  \pgfkeys{/prAtEnd/all end if published/.style={no restate}}
\else%
  \pgfkeys{/prAtEnd/end if published/.style={end}}
  \pgfkeys{/prAtEnd/all end if published/.style={all end}}
\fi

% https://tex.stackexchange.com/a/20613/116348
% https://tex.stackexchange.com/a/583244/116348
\usepackage{scalerel}
\newcommand\hcancel[2][black]{%
  \ThisStyle{%
    \setbox0=\hbox{$\SavedStyle #2$}%
    \rlap{\raisebox{.25\ht0}{\textcolor{#1}{\rule{\wd0}{1pt}}}}\SavedStyle #2}%
}

%% We provide 3 commands to add comments/text:
%% - \yournameAddText{your text to add}: proposition to add text
%% - \yournameRemoveText{text to remove}: proposition to remove text
%% - \yournameComment{your comments}: to add general comment to discuss. Should not contain proposition to add or remove text, but can be added before/after to explain the reason of the modification.
%% The interest it to provide some quick ways to see what is the final number of pages:
%% just add before the import a line:
%%   \def\removeCommentsAcceptPropositions{}
%% and the proposition will be "accepted" (i.e. the removed text is removed,
%% and the added text will be marked as black), and the comment will be removed, that way it is possible
%% to determine the final number of pages.
%% Example:
%% This text is normal\leoRemove{, this text should be removed}\leoAddText{, this text should be added}\leoComment{This is a comment, explaining why I made that changes.}.
\newcommand{\addEditor}[2]{%
  \expandafter\newcommand\csname #1AddText\endcsname[2][]{%
    \ifdefined\removeCommentsAcceptPropositions %
      ##2%
    \else%
      \textcolor{#2}{\sout{##1}}\textcolor{#2!60!black}{##2}%
    \fi%
  }%
  \expandafter\newcommand\csname #1Comment\endcsname[1]{%
    \ifdefined\removeCommentsAcceptPropositions %
    \else%
      \textcolor{#2!60!white}{\textbf{##1}}%
    \fi%
  }%
  \expandafter\newcommand\csname #1Remove\endcsname[1]{%
    \ifdefined\removeCommentsAcceptPropositions %
    \else%
      \textcolor{#2}{\sout{##1}}%
    \fi%
  }%
}
\definecolor{darkgreen}{rgb}{0., 0.7, 0}
\addEditor{leo}{darkgreen}
\addEditor{fred}{red}
\addEditor{elham}{blue}

%% We also provide an arxivOnly environment, in order to include text that are too long for the submitted version
%% but that would be good to have in the arxiv version.
% Use the \begin{arxivOnly} ... \end{arxivOnly} environment, by putting the opening and ending command
%% on there own line.
%% For shorter texts, you can use the inline version \arxivOnlyShort{your text}.
%% If you want to exclude text from the arxiv, use \begin{publishedOnly} ... \end{publishedOnly} and \publishedOnlyShort{your text}
%% "Comment" package is annoying, it does not allow block indentation, we use 'versions' instead
% \usepackage{versions} %% Do not work inside theoremE, don't know why
\ifdefined\displayArxivTexts%
  % \includeversion{arxivOnly}
  % \excludeversion{publishedOnly}
  \newcommand{\arxivOnly}[1]{#1}
  \newcommand{\arxivOnlyNoDiff}[1]{#1} % Like arxivOnly, but useful in diff mode (when it does not make sense to add colors, for instance inside a cite or when using arxivonly to create lists.
  \newcommand{\publishedOnly}[1]{}
  \newcommand{\publishedVsArxiv}[2]{#2}
  \newcommand{\inAppendixIfPublished}[1]{#1}
\else%
  \ifdefined\displayDiffTexts%
    \colorlet{arxivDiff}{green}
    \colorlet{publishedDiff}{red}
    \newcommand{\arxivOnly}[1]{{\color{arxivDiff}#1}}
    \newcommand{\arxivOnlyNoDiff}[1]{#1}
    \newcommand{\publishedOnly}[1]{{\color{publishedDiff}#1}}
    \newcommand{\publishedVsArxiv}[2]{{\color{publishedDiff}#1}{\color{arxivDiff}#2}}
    \newcommand{\inAppendixIfPublished}[1]{{\color{arxivDiff}\textbf{MOVED IN APPENDIX IN PUBLISHED}#1}\textEnd{{\color{publishedDiff}\textbf{MOVED IN MAIN BODY IN ARXIV}#1}}}
  \else
    % \excludeversion{arxivOnly}
    % \includeversion{publishedOnly}
    \newcommand{\arxivOnly}[1]{}
    \newcommand{\arxivOnlyNoDiff}[1]{}
    \newcommand{\publishedOnly}[1]{#1}
    \newcommand{\publishedVsArxiv}[2]{#1}
    \newcommand{\inAppendixIfPublished}[1]{\textEnd{#1}}
  \fi
\fi
% Alias, easier to remember  
\newcommand{\inAppendix}[1]{\textEnd{#1}}
  
% Symbol to display when an arxiv text is supposed to appear
\usepackage{todonotes}
\newcommand{\notifyArxiv}{\ifdefined\arxivDraftMode\todo{arXiv}\fi}

% Remove unwanted space before/after lists
\usepackage{enumitem}
\publishedOnly{\setlist[itemize]{noitemsep, topsep=0pt}}

% Usual packages:
\usepackage{amsmath}
\usepackage{amssymb}
% \usepackage{amsthm}
\usepackage{thmtools}
\usepackage{graphicx}
\graphicspath{{}{figures/}}
\usepackage{float}
\usepackage{subfig}
\usepackage{wrapfig}
\usepackage{array}
\usepackage{braket}
% Refer to footnotes:
% https://tex.stackexchange.com/a/167382/116348
\usepackage{footmisc}
% Useful to use the latest NewDocumentCommand:
%\usepackage{xparse}
%\usepackage[poster]{tcolorbox}

%% Space between pictures and text
%% https://mirrors.chevalier.io/CTAN/macros/latex/contrib/layouts/layman.pdf
%% \usepackage{layouts}
%% https://tex.stackexchange.com/questions/60477/remove-space-after-figure-and-before-text?rq=1
\setlength{\floatsep}{7pt plus 2pt minus 2pt}
\setlength{\textfloatsep}{7pt plus 2pt minus 2pt}
\setlength{\intextsep}{7pt plus 2pt minus 2pt}
% https://tex.stackexchange.com/questions/39017/how-to-influence-the-position-of-float-environments-like-figure-and-table-in-lat
\setcounter{topnumber}{8}
\setcounter{bottomnumber}{8}
\setcounter{totalnumber}{8}
%% https://tex.stackexchange.com/questions/45990/how-can-i-modify-vertical-space-between-figure-and-caption
%% plus/minus allows to stretch a bit around 15pt
% \setlength{\abovecaptionskip}{5pt}

% Remove number to subsubsection (in llncs it replaces \paragraph)
\setcounter{secnumdepth}{2}

%%% Some practical environment to define theorems, lemma...
% \declaretheorem[name=Theorem,numberwithin=section]{theorem}
% \declaretheorem[name=Definition,sibling=theorem]{definition}
% \declaretheorem[name=Lemma,sibling=theorem]{lemma}
% \declaretheorem[name=Corollary,sibling=theorem]{corollary}
% \declaretheorem[name=Remark,sibling=theorem]{remark}
%%% In LNCS we use
% https://tex.stackexchange.com/a/373125/116348
% \spnewtheorem{thm}{Theorem}[section]{\bfseries}{\itshape}

%%%
\usepackage[ruled,vlined]{algorithm2e}
\SetAlFnt{\normalsize}%
\setlength{\algomargin}{0em} % remove margin on left of algo
\SetAlgorithmName{Protocol}{protocol}{List of Protocols}
\newenvironment{protocol}[1][htb]{\begin{algorithm}[#1]}{\end{algorithm}}

%%% %%% Load algorithm package, and create a breakable version
%%% \usepackage{algorithm, algorithmicx, algpseudocode}
%%% % Decrease space between text:
%%% % https://tex.stackexchange.com/questions/25828/how-to-remove-change-the-vertical-spacing-before-and-after-an-algorithm-enviro/25832
%%% \setlength{\intextsep}{1\baselineskip}
%%% % https://tex.stackexchange.com/questions/387577/new-algorithm-environment-with-a-separate-counter
%%% \newcounter{protocol}
%%% \makeatletter
%%% \newenvironment{protocol}[1][htb]{%
%%%   \let\c@algorithm\c@protocol
%%%   \renewcommand{\ALG@name}{Protocol}% Update algorithm name
%%%   \begin{algorithm}[#1]%
%%%   }{\end{algorithm}
%%% }
%%% \makeatother
%%%
%%% \providecommand*\algorithmautorefname{Protocol}
%%% \makeatletter
%%% \newenvironment{breakablealgorithm}
%%%   {% \begin{breakablealgorithm}
%%%    % \begin{center}
%%%      \refstepcounter{algorithm}% New algorithm
%%%      \hrule height.8pt depth0pt \kern2pt% \@fs@pre for \@fs@ruled
%%%      \renewcommand{\caption}[2][\relax]{% Make a new \caption
%%%        % {\raggedright\textbf{\ALG@name~\thealgorithm} ##2\par}%
%%%        {\raggedright\textbf{Protocol~\thealgorithm} ##2\par}%
%%%        \ifx\relax##1\relax % #1 is \relax
%%%          \addcontentsline{loa}{algorithm}{\protect\numberline{\thealgorithm}##2}%
%%%        \else % #1 is not \relax
%%%          \addcontentsline{loa}{algorithm}{\protect\numberline{\thealgorithm}##1}%
%%%        \fi
%%%        \kern2pt\hrule\kern2pt
%%%        \small
%%%      }
%%%   }{% \end{breakablealgorithm}
%%%      \kern2pt\hrule\relax% \@fs@post for \@fs@ruled
%%%    % \end{center}
%%%   }
%%% \makeatother
%%%

\def\EQ#1{\begin{eqnarray}#1\end{eqnarray}} 


%%% Practical abreviations
\usepackage{calrsfs} % Replace \mathcal with different more "cursive" font
\DeclareMathAlphabet{\pazocal}{OMS}{zplm}{m}{n} % \pazocal is now the old \mathcal
\newcommand*{\cA}{\pazocal{A}}
\newcommand*{\cB}{\pazocal{B}}
\newcommand*{\cC}{\pazocal{C}}
\newcommand*{\cD}{\pazocal{D}}
\newcommand*{\cE}{\pazocal{E}}
\newcommand*{\cF}{\pazocal{F}}
\newcommand*{\cG}{\pazocal{G}}
\newcommand*{\cH}{\pazocal{H}}
\newcommand*{\cI}{\pazocal{I}}
\newcommand*{\cJ}{\pazocal{J}}
\newcommand*{\cK}{\pazocal{K}}
\renewcommand*{\cL}{\pazocal{L}} % defined in complexity
\newcommand*{\cM}{\pazocal{M}}
\newcommand*{\cN}{\pazocal{N}}
\newcommand*{\cO}{\pazocal{O}}
\renewcommand*{\cP}{\pazocal{P}} % defined in complexity
\newcommand*{\cQ}{\pazocal{Q}}
\newcommand*{\cR}{\pazocal{R}}
\renewcommand*{\cS}{\pazocal{S}} % defined in complexity
\newcommand*{\cT}{\pazocal{T}}
\newcommand*{\cU}{\pazocal{U}}
\newcommand*{\cV}{\pazocal{V}}
\newcommand*{\cW}{\pazocal{W}}
\newcommand*{\cX}{\pazocal{X}}
\newcommand*{\cY}{\pazocal{Y}}
\newcommand*{\cZ}{\pazocal{Z}}

% Notation for languages. Make sure not to collide with L(H) for linear operators on H
%\DeclareMathAlphabet{\pazocal}{OMS}{zplm}{m}{n}
\newcommand{\linearOp}{\mathcal{L}} % Defines the opearator L(H)
\renewcommand*{\lang}{\pazocal{L}} % Defines a new Turing/Quantum Language

%https://tex.stackexchange.com/questions/195025/
\newcommand*{\rightSend}{\rightsquigarrow} % snake "a --> b" to say "I send a to b".

% For some reasons, proof-at-the-end do not like # (problem with detokenize? To fix...). So we use a macro:
\newcommand*{\diese}{\#}

\newcommand*{\eps}[0]{\varepsilon}
\renewcommand*{\D}[0]{\mathbb{D}}
\renewcommand*{\E}{\mathbb{E}}
\newcommand*{\N}[0]{\mathbb{N}}
\renewcommand*{\R}[0]{\mathbb{R}}
\renewcommand*{\C}[0]{\mathbb{C}}
\newcommand*{\Z}[0]{\mathbb{Z}}
\newcommand*{\T}[0]{\mathbb{T}}

\newcommand*{\CPA}[0]{\textsf{CPA}}
\newcommand*{\LWE}[0]{\textsf{LWE}}
\newcommand*{\GapSVP}[0]{\textsf{GapSVP}}
\newcommand*{\SIVP}[0]{\textsf{SIVP}}

% first param is alpha, second is q (optional)
\newcommand{\contGauss}[1]{\cD_{#1}}
\newcommand{\disGauss}[2]{\cD_{\Z,#1 #2}}
\newcommand{\disGaussCont}[1]{\cD_{\Z,#1}}
\newcommand{\disGaussAQ}{\cD_{\Z,\alpha q}}
\newcommand{\rsafe}{r_{\text{safe}}}
% domain of the function.
\newcommand{\ccX}{\cX}
\newcommand{\ccXSphere}{\cX_\bullet}
\newcommand{\ccXCube}[1][]{\cX_{\scalebox{.4}{$\blacksquare$}#1}}

\newcommand{\BBHeightyFor}{\textsc{BB84}}

% "Correct" spacing around cdot as wildcard
% https://tex.stackexchange.com/questions/78165/spacing-around-cdot-when-used-as-a-wildcard
\newcommand\cdotwild{{\mkern 2mu\cdot\mkern 2mu}}

\newcommand*{\id}{\mathrm{id}}
\DeclareMathOperator{\Tr}{Tr}
\DeclareMathOperator{\supp}{supp}
\DeclareMathOperator{\Det}{Det}
\DeclareMathOperator{\ERR}{ERR}
\DeclareMathOperator{\Round}{Round}

\newcommand{\Ver}{ {\tt Ver} } %% Verify public key
\newcommand{\Gen}{ {\tt Gen} }
\newcommand{\GenLocal}{\ensuremath{ {\tt Gen}_{\tt Loc}} }
\newcommand{\Enc}{ {\tt Enc} }
\newcommand{\Dec}{ {\tt Invert} }
\newcommand{\Eval}{ {\tt Eval} }

\newcommand{\HGen}{ {\tt H.Gen} }
\newcommand{\HEnc}{ {\tt H.Enc} }
\newcommand{\HDec}{ {\tt H.Invert} }
\newcommand{\HEval}{ {\tt H.Eval} }
\newcommand{\Hh}{\ensuremath{{\tt H.}h} }
\newcommand{\HCheckHonestTrapdoor}{\ensuremath{{\tt H.CheckTrapdoor}}}

\newcommand{\MPGen}{ {\tt MP.Gen} }
\newcommand{\MPDec}{ {\tt MP.Invert} }
\newcommand{\InvertSmallGadget}{ \ensuremath{{\texttt{InvertSmallGadget}}} }
\newcommand{\InvertGadget}{ {\tt InvertGadget} }
% \newcommand{\MPEval}{ $g$ }


\DeclareMathOperator{\RoundMod}{RoundMod}


% https://tex.stackexchange.com/a/253746/116348
\usepackage{mleftright}
\newcommand{\BlockMatrix}[2]{%
  \mleft[%
  \begin{array}{@{}#1@{}}%
    #2
  \end{array}%
  \mright]%
}
\newcommand{\SmallBlockMatrix}[2]{
  \mleft[\begin{array}{c}
    #1\tabularnewline % https://tex.stackexchange.com/questions/328745/matrices-with-cryptocode-package
    \hline
    #2
  \end{array}\mright]
}
\newcommand{\HorizBlockMatrix}[2]{
  \mleft[\begin{array}{@{}c|c@{}}
    #1 & #2
  \end{array}\mright]
}

\newcommand*{\vect}[1]{\ensuremath{\mathbf{#1}}}


%% Already defined in *recent* cryptocode.
% \DeclareMathOperator*{\argmax}{arg \, max}
% \DeclareMathOperator*{\argmin}{arg \, min}

\usepackage{xspace}
\xspaceaddexceptions{]\}}
\newcommand{\RSP}{\ensuremath{\sf{RSP}}\xspace}%

\newcommand*{\RSPenc}{\mathrm{RSP}_{\mathrm{enc}}}
\newcommand*{\filter}{\vdash}
\newcommand*{\channelBB}{\cS_{\Z\frac{\pi}{2}}}
% To denote a protocol between two parties
% \newcommand*{\protocol}{\boldsymbol{\pi}}
\newcommand\compRSP{composable $\sf{RSP}_{CC}$}
\newcommand\RSPlike{$\sf{RSP}_{CC}$}
\newcommand\compUBQC{composable $\sf{UBQC}_{CC}$}
\newcommand\UBQClike{$\sf{UBQC}_{CC}$}
% GHZ
\newcommand*{\eqdef}{\coloneqq}
\newcommand*{\PERM}{\ensuremath{}{\sf PERM}}
\newcommand*{\PPT}{\ensuremath{}{\sf PPT}}
\newcommand*{\QPT}{\ensuremath{}{\sf QPT}}
\newcommand*{\Auth}{\ensuremath{}{\sf Auth}}
% Prover, verifier, extractor
\renewcommand*{\P}{\ensuremath{}{\sf P}}
\newcommand*{\V}{\ensuremath{}{\sf V}}
\renewcommand*{\K}{\ensuremath{}{\sf K}}


% Here are the different protocols we use
% Initial protocol
\newcommand{\blind}{\ensuremath{ {\tt BLIND} }}
\newcommand{\blindZK}{\ensuremath{ {\tt BLIND-ZK} }}
% Reveals to people if they are part of the support
\newcommand{\blindSup}{\ensuremath{ {\tt BLIND}^{\tt sup} }}
% Make sure that people in the support share a canonical GHZ
\newcommand{\blindCan}{\ensuremath{ {\tt BLIND}_{\tt can} }}
% Make sure that people in the support share a canonical GHZ and that they know they are supported
\newcommand{\blindCanSup}{\ensuremath{ {\tt BLIND}^{\tt sup}_{\tt can} }}
\newcommand{\authBlindCanDist}{\ensuremath{ {\tt AUTH-BLIND}^{\tt dist}_{\tt can} }}
\newcommand{\verifCanSup}{\ensuremath{ {\tt VERIF}^{\tt sup}_{\tt can} }}
\newcommand{\blindPoly}{\ensuremath{ {\tt BLIND-poly} }}
\newcommand{\INDFCT}{\ensuremath{ {\tt IND-D0} }}
\newcommand{\INDBLIND}{\refGame*{INDBLIND}}
\newcommand{\INDBLINDtxt}{\ensuremath{ {\tt IND-\blind{}} }}
\newcommand{\INDBLINDSUP}{\refGame*{INDBLINDSUP}}
\newcommand{\INDBLINDSUPtxt}{\ensuremath{ {\tt IND-\blindSup{}} }}
\newcommand{\INDBLINDCANSUP}{\ensuremath{ {\tt IND-\blindCanSup{}} }}
\newcommand{\INDBLINDCANAUTH}{\ensuremath{ {\tt IND-\authBlindCanDist{}} }}
\newcommand{\INDPARTIAL}{\ensuremath{ {\tt IND-PARTIAL} }}
\newcommand*\GHZ{\ensuremath{}{\sf GHZ}}
\newcommand*\GHZCan{\ensuremath{{\sf GHZ^{\sf can}}}}
% Generalized GGHZ
\newcommand*\GGHZ{\ensuremath{ {\sf GHZ^G} }}
\newcommand*\HGHZ{\ensuremath{ {\sf GHZ^H} }}
\newcommand{\AssumpFctNoDelta}{\HGHZ{} capable}
\newcommand{\AssumpFctCanNoDelta}{\GHZCan{} capable}
\newcommand{\AssumpFct}{$\delta$-\HGHZ{} capable}
\newcommand{\AssumpFctNegl}{$\negl[\lambda]$-\HGHZ{} capable}
\newcommand{\AssumpFctCan}{$\delta$-\GHZCan{} capable}
\newcommand{\AssumpFctCanPrime}{$\delta^\prime$-\GHZCan{} capable}
\newcommand{\AssumpFctDist}{distributable}
\newcommand{\partialInfo}{\ensuremath{{\tt PartInfo}}}
\newcommand{\partialInfoLocal}{\ensuremath{{\tt PartInfo_{\tt Loc}}}}
\newcommand{\partialAlpha}{\ensuremath{{\tt PartAlpha_{\tt Loc}}}}
\newcommand{\CombineAlpha}{\ensuremath{{\tt CombineAlpha}}}
\newcommand{\CombineRandom}{\ensuremath{{\tt CombineRandom}}}
\newcommand{\CheckHonestTrapdoor}{\ensuremath{{\tt CheckTrapdoor}}}
\newcommand{\ZKGenCheck}{\ensuremath{{\tt ZK-GenCheck}}}

\newcommand{\highlightChanges}[1]{\textcolor{green!50!black}{#1}}

\newcommand{\Real}{\ensuremath{\mathsf{REAL}}}
\newcommand{\Sim}{\ensuremath{\mathsf{Sim}}}
\newcommand{\Ideal}{\ensuremath{\mathsf{IDEAL}}}
\newcommand{\OUT}{\ensuremath{\mathsf{OUT}}}

\usepackage{pifont}% http://ctan.org/pkg/pifont
% To denote that a user is not in the final GHZ
\newcommand{\cross}{\ensuremath{\text{\ding{55}}}}

\usepackage{verbatim}
\newcommand{\leftI}{{\normalfont\ttfamily left}}
\newcommand{\rightI}{{\normalfont\ttfamily right}}
\newcommand*{\mybar}[1]{\overline{#1}}

\newcommand*{\quout}{\mathtt{out}}

\DeclareMathOperator{\Image}{Image}
\DeclareMathOperator{\Ima}{Im}
\DeclareMathOperator{\Dom}{Dom}

\newcommand*{\Pub}{\mathtt{Pub}}
\newcommand*{\accept}{\mathtt{accept}}
\newcommand*{\abort}{\mathtt{abort}}
\newcommand*{\gadxor}{\mathtt{Gad}_{\oplus}}
\newcommand*{\quin}{\mathtt{in}}

%%% Provide a "subproof" environment (pretty simple for now,
%%% just indent the sub-proofs)
% I'd love to have left border... https://tex.stackexchange.com/questions/532948/robustly-add-a-border-to-the-left-of-a-text-spanning-several-pages
\usepackage{changepage}% http://ctan.org/pkg/changepage
\newenvironment{subproof}{\begin{adjustwidth}{2em}{0pt}}{\end{adjustwidth}}


%%% To have nice probabilities
% https://tex.stackexchange.com/questions/198771/align-in-substack
\makeatletter
\newcommand{\subalign}[1]{%
  \vcenter{%
    \Let@ \restore@math@cr \default@tag
    \baselineskip\fontdimen10 \scriptfont\tw@
    \advance\baselineskip\fontdimen12 \scriptfont\tw@
    \lineskip\thr@@\fontdimen8 \scriptfont\thr@@
    \lineskiplimit\lineskip
    \ialign{\hfil$\m@th\scriptstyle##$&$\m@th\scriptstyle{}##$\hfil\crcr
      #1\crcr
    }%
  }%
}
\makeatother
\usepackage{xparse}
% https://tex.stackexchange.com/questions/431794/same-spacing-around-middle-as-around-mid
\let\originalmiddle=\middle
\def\middle#1{\mathrel{}\originalmiddle#1\mathrel{}}
\NewDocumentCommand{\pr}{som}{\Pr\IfValueT{#2}{_{\substack{#2}}}\left[\,#3\,\right]}
\NewDocumentCommand{\esp}{som}{\IfValueTF{#2}{\underset{\subalign{#2}}{\mathbb{E}}}{\mathbb{E}}[\,#3\,]}
\NewDocumentEnvironment{pral}{soO{}}
 {%
  \Pr\IfValueT{#2}{_{\IfBooleanTF{#1}{\substack{#2}}{#2}}}
\begin{alignedat}[t]{2}
  [\, } {\,]
 #3
 \end{alignedat}
 }
\NewDocumentEnvironment{espal}{soO{}}
 {%
   \IfValueTF{#2}{\underset{\subalign{#2}}{\mathbb{E}}}{\mathbb{E}}
   \begin{alignedat}[t]{2}[\,
 }
 {\,]#3\end{alignedat}
 }
% To draw |A><B|, the second argument is facultative and gives |A><A|
\NewDocumentCommand{\ketbra}{mg}{
  | #1 \rangle \langle \IfValueTF{#2}{#2}{#1} |
}
% https://tex.stackexchange.com/a/188364/116348
\DeclareRobustCommand{\minwidthbox}[2]{%
  \ifmmode
    \expandafter\mathmakebox
  \else
    \expandafter\makebox
  \fi
  [\ifdim#2<\width\width\else#2\fi]{#1}%f
}

\NewDocumentCommand{\constructs}{mg}{
  \xrightarrow[\IfValueT{#2}{#2}]{\minwidthbox{#1}{5mm}}
}

%%%%%%%%%% This part is specific to llncs-based papers
% To enable this part of the configuration, add
%  \def\enableLlncsCode{}
% before the \input{include}
\ifdefined\enableLlncsCode%
% go back to standard amsthm proof environments with Qed at the end
\let\proof\relax\let\endproof\relax
  \usepackage{amsthm}
  % To debug, it's a nightmare without page number... To disable
  % in the final version
  \pagestyle{plain}
\fi

% Define environments when loaded in standalone...
\ifdefined\enableNonLlncsCode%
\usepackage{thmtools}
\declaretheorem[name=Theorem,numberwithin=chapter]{theorem}
\declaretheorem[name=Definition,sibling=theorem]{definition}
\declaretheorem[name=Lemma,sibling=theorem]{lemma}
\declaretheorem[name=Corollary,sibling=theorem]{corollary}
\declaretheorem[name=Remark,sibling=theorem]{remark}
\fi

\usepackage{enumitem}
\newlist{compressedList}{itemize}{10}
\setlist[compressedList]{label=--,topsep=0pt,parsep=0pt,partopsep=0pt}
\usepackage{mathtools}
\DeclarePairedDelimiter\ceil{\lceil}{\rceil}
\DeclarePairedDelimiter\floor{\lfloor}{\rfloor}

% https://tex.stackexchange.com/questions/134278/overriding-the-paragraph-style-in-the-lncs-document-class
\usepackage{etoolbox}
\patchcmd{\paragraph}{\itshape}{\bfseries\boldmath}{}{}


% This part is specific to llncs-based papers with nicer output
% (page number, theorem labelled with sections...). This should
% be the version in the arxiv.
% To enable this part of the configuration, add
%  \def\nicerThanLlncs{}
% before the \input{include}
\ifdefined\nicerThanLlncs%
  % change margins, page layout
  \usepackage[a4paper, top=1.4in, bottom=1.4in, right=1in, left=1in]{geometry}
  % enable page numbering, suppressed by default by llncs
  \pagestyle{plain}
  \setcounter{tocdepth}{2}
  \renewcommand\small{\fontsize{10pt}{12pt}\selectfont}
\fi
\ifdefined\nicerThanLlncsAbstract%
  % change margins, page layout
  \usepackage[a4paper, margin=2.5cm]{geometry}
  % enable page numbering, suppressed by default by llncs
  \pagestyle{plain}
  \setcounter{tocdepth}{2}
\fi
\ifdefined\enableLlncsCodeSlightlyImproved%
  \pagestyle{plain}
  \setcounter{tocdepth}{2}
\fi


\ifdefined\disableHyperref%
\else%
% https://rcweb.dartmouth.edu/doc/texmf-dist/doc/latex/cryptocode/cryptocode.pdf (page 55)
%%%%%%%%%% This part should be loaded last
% Usually hyperref needs to be at the end
\definecolor{secondaryColor}{RGB}{206,149,0} %% darker orange, looks like gold. <3
\usepackage[colorlinks, allcolors=secondaryColor]{hyperref}
\fi
%% Hyperref should be loaded before cryptocode! Otherwise, errors with labels of lines for example.
%%% Cryptocode
\usepackage [
n,
advantage,
operators,
sets,
adversary,
landau,
probability,
notions,
logic,
ff,
mm,
primitives,
events,
complexity,
asymptotics,
keys
] {cryptocode}
% I don't like the way "samples" is defined in cryptocode with a big dollar in front.
% \renewcommand\sample{\mathrel{\overset{\vbox to.7ex{\hbox{\fontsize{6}{0}\selectfont\vspace{.5ex} \$}}}{\leftarrow}}}
%% https://tex.stackexchange.com/a/584533/116348
\makeatletter
\renewcommand{\sample}{\mathrel{\mathpalette\sample@\relax}}
\newcommand{\sample@}[2]{%
  \ooalign{%
    \hspace{\stretch{2}}\raisebox{0.7\height}{$\m@th\demotestyle{#1}\mathdollar$}\hspace{\stretch{1}}\cr
    $\m@th#1\leftarrow$\cr
  }%
}
\newcommand{\demotestyle}[1]{%
  \ifx#1\displaystyle\scriptstyle\else
  \ifx#1\textstyle\scriptstyle\else
  \scriptscriptstyle\fi\fi
}
\makeatother
% Create a new command to create games using \game[linenumbering]{Name of game}{Game definition}
\createprocedureblock{game}{center,boxed}{}{}{}
% Create a new environment (following what I did in the thesis) to provide named games
\newcommand{\gameFont}[1]{\texttt{#1}}
\renewcommand{\Game}[1]{\ensuremath{\gameFont{Game#1}}} % By default \Game produces a nice G (symetric), but it's not very standard.

\makeatletter
\createprocedureblock{gameProc}{center,boxed}{}{}{}
\createprocedureblock{interactGame}{center,boxed}{}{}{}
%%% Nicer way to refer to games.
%%% See also https://tex.stackexchange.com/questions/623163 which provides a cleaner way to make it work with cleverref.
% Usage: \begin{namedGame}*[label][short title]{title}*[game options like skipfirstln,lnstart=1] content \end{namedGame}
% first * = disable floating. WARNING: do not put labels with non-letter chars.
% second * = do not include the game in the TOC.
\NewDocumentEnvironment{namedGame}{soomsO{}b}{%
  \IfBooleanTF{#1}{}{%
    \par% without par, the figure will be put after the line arriving after the picture if there is no paragraph.
    \setlength{\intextsep}{5.0pt plus 2.0pt minus 2.0pt}% reduce the space when image is inserted inline.
    \begin{figure}[htbp]}% 
    \begin{pcimage}%
      %% Add to toc see command listofgames
      \phantomsection% Create a fake label, useful to ensure the link point to this part.
      \IfBooleanTF{#5}{}{% If no star is given, add to "table of contents" for games
        \IfValueTF{#3}{% If a short title is provided
          \addcontentsline{game}{section}{\ensuremath{#3}}%
        }{%
          \addcontentsline{game}{section}{\ensuremath{#4}}%
        }%
      }%
      {\normalfont\gameProc[linenumbering,#6]{\ensuremath{#4}}{\IfValueTF{#2}{%
            %% Add an anchor if a label is present
            \raisebox{1em}{\hypertarget{#2}{}}%
            %% Create a macro "mygametitle@nameoflabel" to store the title
            \IfValueTF{#3}{% If a short title is provided
              \expandafter\gdef\csname mygametitle@#2 \endcsname{\ensuremath{#3}}%%
              \write\@auxout{\gdef\string\mygametitle@#2{\ensuremath{#3}}}%
            }{%
              \expandafter\gdef\csname mygametitle@#2 \endcsname{\ensuremath{#4}}%%
              \write\@auxout{\gdef\string\mygametitle@#2{\ensuremath{#4}}}%
            }%
          }{} #7 }}%
    \end{pcimage}
    \IfBooleanTF{#1}{}{\end{figure}}%
}{}

% Usage: \refGame{label}. Use \refGame*{label} if you don't want to color the link (useful in proofs)
\NewDocumentCommand{\refGame}{sm}{%
  \IfBooleanTF{#1}{%
    \hyperlink{#2}{\textcolor{black}{\csname mygametitle@#2\endcsname}}% Do not put a white space after #1!
  }{%
    \hyperlink{#2}{\csname mygametitle@#2\endcsname}% Do not put a white space after #1!
  }%
}
\makeatother

\ifdefined\disableHyperref%
\else%
% Well... not as much as cleveref
\usepackage[capitalise,noabbrev]{cleveref}
% Don't abbrev general names... but still abbrev equations
\crefname{equation}{Eq.}{Eqs.}
\crefname{figure}{Fig.}{Figs.}
\crefname{algorithm}{Protocol}{Protocols}
\Crefname{equation}{Equation}{Equations}
\Crefname{figure}{Figure}{Figures}
\Crefname{algorithm}{Protocol}{Protocols}
\fi

% Put that package at the end of the file,
% or you will have some strange errors like
% "Only one # is allowed per tab"
\usetikzlibrary{quantikz}
% Bug in llncs https://tex.stackexchange.com/a/372108/116348
\def\theHlemma{\theHtheorem}
\def\theHdefinition{\theHtheorem}
\def\theHcorollary{\theHtheorem}

%%%%%% PLEASE DO NOT WRITE ANYTHING AFTER THIS LINE %%%%%%
%%%%%% Indeed, hyperref, cleveref and quantikz need %%%%%%
%%%%%% to be loaded at the very end.                %%%%%%

\ifdefined\enableLlncsCode%
% go back to standard amsthm proof environments with Qed at the end
\let\proof\relax\let\endproof\relax
  \usepackage{amsthm}
  % To debug, it's a nightmare without page number... To disable
  % in the final version
  \pagestyle{plain}
\fi

% Define environments when loaded in standalone...
\ifdefined\enableNonLlncsCode%
\usepackage{thmtools}
\declaretheorem[name=Theorem,numberwithin=chapter]{theorem}
\declaretheorem[name=Definition,sibling=theorem]{definition}
\declaretheorem[name=Lemma,sibling=theorem]{lemma}
\declaretheorem[name=Corollary,sibling=theorem]{corollary}
\declaretheorem[name=Remark,sibling=theorem]{remark}
\fi

\usepackage{enumitem}
\newlist{compressedList}{itemize}{10}
\setlist[compressedList]{label=--,topsep=0pt,parsep=0pt,partopsep=0pt}
\usepackage{mathtools}
\DeclarePairedDelimiter\ceil{\lceil}{\rceil}
\DeclarePairedDelimiter\floor{\lfloor}{\rfloor}

% https://tex.stackexchange.com/questions/134278/overriding-the-paragraph-style-in-the-lncs-document-class
\usepackage{etoolbox}
\patchcmd{\paragraph}{\itshape}{\bfseries\boldmath}{}{}


% This part is specific to llncs-based papers with nicer output
% (page number, theorem labelled with sections...). This should
% be the version in the arxiv.
% To enable this part of the configuration, add
%  \def\nicerThanLlncs{}
% before the %% Let's put in these file all the definitions that we
%% require:

%% Setup language, encoding...
\RequirePackage{etex} 

\usepackage{standalone}
\usepackage{lmodern}
% Language
\usepackage[english]{babel}
% Input encoding
\usepackage[utf8]{inputenc}
% Output encoding https://tex.stackexchange.com/a/677. Important to copy accents
\usepackage[T1]{fontenc}
\usepackage{subfiles} % To quickly load this include file other files
\usepackage{microtype}
\usepackage[shortcuts,nospacearound]{extdash}    % Better endling of em dash (---) to cut sentences\===like 
%%% https://tex.stackexchange.com/questions/2773/how-do-i-make-latex-push-long-citations-to-a-new-line
\usepackage{breakcites}
\usepackage{booktabs}
% Makes sure floats don't fly accross sections
\usepackage{placeins}
\usepackage{adjustbox}
\usepackage{varwidth} % Like minipage, but with automatic width discovery.
\usepackage{complexity} % get \QMA{}...

\usepackage[normalem]{ulem}
%\usepackage[math-style=ISO]{unicode-math}
\usepackage{upgreek} % non italic greak letters

%%%% Bibliography
% I want to have separate references for appendix and main body
% https://tex.stackexchange.com/questions/98660/
\usepackage[style=trad-alpha,
  sortcites=true,
  doi=false,
  url=false,
  giveninits=true, % Bob Foo --> B. Foo
  isbn=false,
  url=false,
  eprint=false,
  sortcites=false, % \cite{B,A,C}: [A,B,C] --> [B,A,C]
]{biblatex}
\defbibfilter{appendixOnlyFilter}{
  segment=1 % Segment 1 will be chosen to be the one in appendix
  and not segment=0 % Default segment is 0
}
\renewcommand{\multicitedelim}{, } % [ABC96; DEF12] -> [ABC96, DEF12]
% [unknown_ref] => [??]
% https://tex.stackexchange.com/a/352573
\makeatletter
\protected\def\abx@missing#1{\textbf{??}}
\makeatother
\renewcommand*{\bibfont}{\normalfont\small} % BibLatex font seems bigger than Bibtex?

\usepackage{xcolor,cancel}
\usepackage{proof-at-the-end}
\NewDocumentEnvironment{theoremE}{O{}O{}+b}{%
  \begin{theoremEnd}[normal,#2]{theorem}[#1]%
    #3%
  \end{theoremEnd}
  %
}{}% Do not forget the second parameter or you might get Missing \begin{document} error
\NewDocumentEnvironment{lemmaE}{O{}O{}+b}{%
  \begin{theoremEnd}[normal,#2]{lemma}[#1]%
    #3%
  \end{theoremEnd}
  %
}{}% Do not forget the second parameter or you might get Missing \begin{document} error
\NewDocumentEnvironment{corollaryE}{O{}O{}+b}{%
  \begin{theoremEnd}[normal,#2]{corollary}[#1]%
    #3%
  \end{theoremEnd}
  %
}{}% Do not forget the second parameter or you might get Missing \begin{document} error
\NewDocumentEnvironment{proofE}{O{}+b}{%
  \begin{proofEnd}[#1]%
    #2
  \end{proofEnd}%
}{}%
\pgfkeys{/prAtEnd/global custom defaults/.style={
    text link={\noindent See \hyperref[proof:prAtEnd\pratendcountercurrent]{proof} in \cref{proofsection:prAtEnd\pratendcountercurrent}.}
  }
}

\ifdefined\displayArxivTexts%
  \pgfkeys{/prAtEnd/end if published/.style={no restate}}
  \pgfkeys{/prAtEnd/all end if published/.style={no restate}}
\else%
  \pgfkeys{/prAtEnd/end if published/.style={end}}
  \pgfkeys{/prAtEnd/all end if published/.style={all end}}
\fi

% https://tex.stackexchange.com/a/20613/116348
% https://tex.stackexchange.com/a/583244/116348
\usepackage{scalerel}
\newcommand\hcancel[2][black]{%
  \ThisStyle{%
    \setbox0=\hbox{$\SavedStyle #2$}%
    \rlap{\raisebox{.25\ht0}{\textcolor{#1}{\rule{\wd0}{1pt}}}}\SavedStyle #2}%
}

%% We provide 3 commands to add comments/text:
%% - \yournameAddText{your text to add}: proposition to add text
%% - \yournameRemoveText{text to remove}: proposition to remove text
%% - \yournameComment{your comments}: to add general comment to discuss. Should not contain proposition to add or remove text, but can be added before/after to explain the reason of the modification.
%% The interest it to provide some quick ways to see what is the final number of pages:
%% just add before the import a line:
%%   \def\removeCommentsAcceptPropositions{}
%% and the proposition will be "accepted" (i.e. the removed text is removed,
%% and the added text will be marked as black), and the comment will be removed, that way it is possible
%% to determine the final number of pages.
%% Example:
%% This text is normal\leoRemove{, this text should be removed}\leoAddText{, this text should be added}\leoComment{This is a comment, explaining why I made that changes.}.
\newcommand{\addEditor}[2]{%
  \expandafter\newcommand\csname #1AddText\endcsname[2][]{%
    \ifdefined\removeCommentsAcceptPropositions %
      ##2%
    \else%
      \textcolor{#2}{\sout{##1}}\textcolor{#2!60!black}{##2}%
    \fi%
  }%
  \expandafter\newcommand\csname #1Comment\endcsname[1]{%
    \ifdefined\removeCommentsAcceptPropositions %
    \else%
      \textcolor{#2!60!white}{\textbf{##1}}%
    \fi%
  }%
  \expandafter\newcommand\csname #1Remove\endcsname[1]{%
    \ifdefined\removeCommentsAcceptPropositions %
    \else%
      \textcolor{#2}{\sout{##1}}%
    \fi%
  }%
}
\definecolor{darkgreen}{rgb}{0., 0.7, 0}
\addEditor{leo}{darkgreen}
\addEditor{fred}{red}
\addEditor{elham}{blue}

%% We also provide an arxivOnly environment, in order to include text that are too long for the submitted version
%% but that would be good to have in the arxiv version.
% Use the \begin{arxivOnly} ... \end{arxivOnly} environment, by putting the opening and ending command
%% on there own line.
%% For shorter texts, you can use the inline version \arxivOnlyShort{your text}.
%% If you want to exclude text from the arxiv, use \begin{publishedOnly} ... \end{publishedOnly} and \publishedOnlyShort{your text}
%% "Comment" package is annoying, it does not allow block indentation, we use 'versions' instead
% \usepackage{versions} %% Do not work inside theoremE, don't know why
\ifdefined\displayArxivTexts%
  % \includeversion{arxivOnly}
  % \excludeversion{publishedOnly}
  \newcommand{\arxivOnly}[1]{#1}
  \newcommand{\arxivOnlyNoDiff}[1]{#1} % Like arxivOnly, but useful in diff mode (when it does not make sense to add colors, for instance inside a cite or when using arxivonly to create lists.
  \newcommand{\publishedOnly}[1]{}
  \newcommand{\publishedVsArxiv}[2]{#2}
  \newcommand{\inAppendixIfPublished}[1]{#1}
\else%
  \ifdefined\displayDiffTexts%
    \colorlet{arxivDiff}{green}
    \colorlet{publishedDiff}{red}
    \newcommand{\arxivOnly}[1]{{\color{arxivDiff}#1}}
    \newcommand{\arxivOnlyNoDiff}[1]{#1}
    \newcommand{\publishedOnly}[1]{{\color{publishedDiff}#1}}
    \newcommand{\publishedVsArxiv}[2]{{\color{publishedDiff}#1}{\color{arxivDiff}#2}}
    \newcommand{\inAppendixIfPublished}[1]{{\color{arxivDiff}\textbf{MOVED IN APPENDIX IN PUBLISHED}#1}\textEnd{{\color{publishedDiff}\textbf{MOVED IN MAIN BODY IN ARXIV}#1}}}
  \else
    % \excludeversion{arxivOnly}
    % \includeversion{publishedOnly}
    \newcommand{\arxivOnly}[1]{}
    \newcommand{\arxivOnlyNoDiff}[1]{}
    \newcommand{\publishedOnly}[1]{#1}
    \newcommand{\publishedVsArxiv}[2]{#1}
    \newcommand{\inAppendixIfPublished}[1]{\textEnd{#1}}
  \fi
\fi
% Alias, easier to remember  
\newcommand{\inAppendix}[1]{\textEnd{#1}}
  
% Symbol to display when an arxiv text is supposed to appear
\usepackage{todonotes}
\newcommand{\notifyArxiv}{\ifdefined\arxivDraftMode\todo{arXiv}\fi}

% Remove unwanted space before/after lists
\usepackage{enumitem}
\publishedOnly{\setlist[itemize]{noitemsep, topsep=0pt}}

% Usual packages:
\usepackage{amsmath}
\usepackage{amssymb}
% \usepackage{amsthm}
\usepackage{thmtools}
\usepackage{graphicx}
\graphicspath{{}{figures/}}
\usepackage{float}
\usepackage{subfig}
\usepackage{wrapfig}
\usepackage{array}
\usepackage{braket}
% Refer to footnotes:
% https://tex.stackexchange.com/a/167382/116348
\usepackage{footmisc}
% Useful to use the latest NewDocumentCommand:
%\usepackage{xparse}
%\usepackage[poster]{tcolorbox}

%% Space between pictures and text
%% https://mirrors.chevalier.io/CTAN/macros/latex/contrib/layouts/layman.pdf
%% \usepackage{layouts}
%% https://tex.stackexchange.com/questions/60477/remove-space-after-figure-and-before-text?rq=1
\setlength{\floatsep}{7pt plus 2pt minus 2pt}
\setlength{\textfloatsep}{7pt plus 2pt minus 2pt}
\setlength{\intextsep}{7pt plus 2pt minus 2pt}
% https://tex.stackexchange.com/questions/39017/how-to-influence-the-position-of-float-environments-like-figure-and-table-in-lat
\setcounter{topnumber}{8}
\setcounter{bottomnumber}{8}
\setcounter{totalnumber}{8}
%% https://tex.stackexchange.com/questions/45990/how-can-i-modify-vertical-space-between-figure-and-caption
%% plus/minus allows to stretch a bit around 15pt
% \setlength{\abovecaptionskip}{5pt}

% Remove number to subsubsection (in llncs it replaces \paragraph)
\setcounter{secnumdepth}{2}

%%% Some practical environment to define theorems, lemma...
% \declaretheorem[name=Theorem,numberwithin=section]{theorem}
% \declaretheorem[name=Definition,sibling=theorem]{definition}
% \declaretheorem[name=Lemma,sibling=theorem]{lemma}
% \declaretheorem[name=Corollary,sibling=theorem]{corollary}
% \declaretheorem[name=Remark,sibling=theorem]{remark}
%%% In LNCS we use
% https://tex.stackexchange.com/a/373125/116348
% \spnewtheorem{thm}{Theorem}[section]{\bfseries}{\itshape}

%%%
\usepackage[ruled,vlined]{algorithm2e}
\SetAlFnt{\normalsize}%
\setlength{\algomargin}{0em} % remove margin on left of algo
\SetAlgorithmName{Protocol}{protocol}{List of Protocols}
\newenvironment{protocol}[1][htb]{\begin{algorithm}[#1]}{\end{algorithm}}

%%% %%% Load algorithm package, and create a breakable version
%%% \usepackage{algorithm, algorithmicx, algpseudocode}
%%% % Decrease space between text:
%%% % https://tex.stackexchange.com/questions/25828/how-to-remove-change-the-vertical-spacing-before-and-after-an-algorithm-enviro/25832
%%% \setlength{\intextsep}{1\baselineskip}
%%% % https://tex.stackexchange.com/questions/387577/new-algorithm-environment-with-a-separate-counter
%%% \newcounter{protocol}
%%% \makeatletter
%%% \newenvironment{protocol}[1][htb]{%
%%%   \let\c@algorithm\c@protocol
%%%   \renewcommand{\ALG@name}{Protocol}% Update algorithm name
%%%   \begin{algorithm}[#1]%
%%%   }{\end{algorithm}
%%% }
%%% \makeatother
%%%
%%% \providecommand*\algorithmautorefname{Protocol}
%%% \makeatletter
%%% \newenvironment{breakablealgorithm}
%%%   {% \begin{breakablealgorithm}
%%%    % \begin{center}
%%%      \refstepcounter{algorithm}% New algorithm
%%%      \hrule height.8pt depth0pt \kern2pt% \@fs@pre for \@fs@ruled
%%%      \renewcommand{\caption}[2][\relax]{% Make a new \caption
%%%        % {\raggedright\textbf{\ALG@name~\thealgorithm} ##2\par}%
%%%        {\raggedright\textbf{Protocol~\thealgorithm} ##2\par}%
%%%        \ifx\relax##1\relax % #1 is \relax
%%%          \addcontentsline{loa}{algorithm}{\protect\numberline{\thealgorithm}##2}%
%%%        \else % #1 is not \relax
%%%          \addcontentsline{loa}{algorithm}{\protect\numberline{\thealgorithm}##1}%
%%%        \fi
%%%        \kern2pt\hrule\kern2pt
%%%        \small
%%%      }
%%%   }{% \end{breakablealgorithm}
%%%      \kern2pt\hrule\relax% \@fs@post for \@fs@ruled
%%%    % \end{center}
%%%   }
%%% \makeatother
%%%

\def\EQ#1{\begin{eqnarray}#1\end{eqnarray}} 


%%% Practical abreviations
\usepackage{calrsfs} % Replace \mathcal with different more "cursive" font
\DeclareMathAlphabet{\pazocal}{OMS}{zplm}{m}{n} % \pazocal is now the old \mathcal
\newcommand*{\cA}{\pazocal{A}}
\newcommand*{\cB}{\pazocal{B}}
\newcommand*{\cC}{\pazocal{C}}
\newcommand*{\cD}{\pazocal{D}}
\newcommand*{\cE}{\pazocal{E}}
\newcommand*{\cF}{\pazocal{F}}
\newcommand*{\cG}{\pazocal{G}}
\newcommand*{\cH}{\pazocal{H}}
\newcommand*{\cI}{\pazocal{I}}
\newcommand*{\cJ}{\pazocal{J}}
\newcommand*{\cK}{\pazocal{K}}
\renewcommand*{\cL}{\pazocal{L}} % defined in complexity
\newcommand*{\cM}{\pazocal{M}}
\newcommand*{\cN}{\pazocal{N}}
\newcommand*{\cO}{\pazocal{O}}
\renewcommand*{\cP}{\pazocal{P}} % defined in complexity
\newcommand*{\cQ}{\pazocal{Q}}
\newcommand*{\cR}{\pazocal{R}}
\renewcommand*{\cS}{\pazocal{S}} % defined in complexity
\newcommand*{\cT}{\pazocal{T}}
\newcommand*{\cU}{\pazocal{U}}
\newcommand*{\cV}{\pazocal{V}}
\newcommand*{\cW}{\pazocal{W}}
\newcommand*{\cX}{\pazocal{X}}
\newcommand*{\cY}{\pazocal{Y}}
\newcommand*{\cZ}{\pazocal{Z}}

% Notation for languages. Make sure not to collide with L(H) for linear operators on H
%\DeclareMathAlphabet{\pazocal}{OMS}{zplm}{m}{n}
\newcommand{\linearOp}{\mathcal{L}} % Defines the opearator L(H)
\renewcommand*{\lang}{\pazocal{L}} % Defines a new Turing/Quantum Language

%https://tex.stackexchange.com/questions/195025/
\newcommand*{\rightSend}{\rightsquigarrow} % snake "a --> b" to say "I send a to b".

% For some reasons, proof-at-the-end do not like # (problem with detokenize? To fix...). So we use a macro:
\newcommand*{\diese}{\#}

\newcommand*{\eps}[0]{\varepsilon}
\renewcommand*{\D}[0]{\mathbb{D}}
\renewcommand*{\E}{\mathbb{E}}
\newcommand*{\N}[0]{\mathbb{N}}
\renewcommand*{\R}[0]{\mathbb{R}}
\renewcommand*{\C}[0]{\mathbb{C}}
\newcommand*{\Z}[0]{\mathbb{Z}}
\newcommand*{\T}[0]{\mathbb{T}}

\newcommand*{\CPA}[0]{\textsf{CPA}}
\newcommand*{\LWE}[0]{\textsf{LWE}}
\newcommand*{\GapSVP}[0]{\textsf{GapSVP}}
\newcommand*{\SIVP}[0]{\textsf{SIVP}}

% first param is alpha, second is q (optional)
\newcommand{\contGauss}[1]{\cD_{#1}}
\newcommand{\disGauss}[2]{\cD_{\Z,#1 #2}}
\newcommand{\disGaussCont}[1]{\cD_{\Z,#1}}
\newcommand{\disGaussAQ}{\cD_{\Z,\alpha q}}
\newcommand{\rsafe}{r_{\text{safe}}}
% domain of the function.
\newcommand{\ccX}{\cX}
\newcommand{\ccXSphere}{\cX_\bullet}
\newcommand{\ccXCube}[1][]{\cX_{\scalebox{.4}{$\blacksquare$}#1}}

\newcommand{\BBHeightyFor}{\textsc{BB84}}

% "Correct" spacing around cdot as wildcard
% https://tex.stackexchange.com/questions/78165/spacing-around-cdot-when-used-as-a-wildcard
\newcommand\cdotwild{{\mkern 2mu\cdot\mkern 2mu}}

\newcommand*{\id}{\mathrm{id}}
\DeclareMathOperator{\Tr}{Tr}
\DeclareMathOperator{\supp}{supp}
\DeclareMathOperator{\Det}{Det}
\DeclareMathOperator{\ERR}{ERR}
\DeclareMathOperator{\Round}{Round}

\newcommand{\Ver}{ {\tt Ver} } %% Verify public key
\newcommand{\Gen}{ {\tt Gen} }
\newcommand{\GenLocal}{\ensuremath{ {\tt Gen}_{\tt Loc}} }
\newcommand{\Enc}{ {\tt Enc} }
\newcommand{\Dec}{ {\tt Invert} }
\newcommand{\Eval}{ {\tt Eval} }

\newcommand{\HGen}{ {\tt H.Gen} }
\newcommand{\HEnc}{ {\tt H.Enc} }
\newcommand{\HDec}{ {\tt H.Invert} }
\newcommand{\HEval}{ {\tt H.Eval} }
\newcommand{\Hh}{\ensuremath{{\tt H.}h} }
\newcommand{\HCheckHonestTrapdoor}{\ensuremath{{\tt H.CheckTrapdoor}}}

\newcommand{\MPGen}{ {\tt MP.Gen} }
\newcommand{\MPDec}{ {\tt MP.Invert} }
\newcommand{\InvertSmallGadget}{ \ensuremath{{\texttt{InvertSmallGadget}}} }
\newcommand{\InvertGadget}{ {\tt InvertGadget} }
% \newcommand{\MPEval}{ $g$ }


\DeclareMathOperator{\RoundMod}{RoundMod}


% https://tex.stackexchange.com/a/253746/116348
\usepackage{mleftright}
\newcommand{\BlockMatrix}[2]{%
  \mleft[%
  \begin{array}{@{}#1@{}}%
    #2
  \end{array}%
  \mright]%
}
\newcommand{\SmallBlockMatrix}[2]{
  \mleft[\begin{array}{c}
    #1\tabularnewline % https://tex.stackexchange.com/questions/328745/matrices-with-cryptocode-package
    \hline
    #2
  \end{array}\mright]
}
\newcommand{\HorizBlockMatrix}[2]{
  \mleft[\begin{array}{@{}c|c@{}}
    #1 & #2
  \end{array}\mright]
}

\newcommand*{\vect}[1]{\ensuremath{\mathbf{#1}}}


%% Already defined in *recent* cryptocode.
% \DeclareMathOperator*{\argmax}{arg \, max}
% \DeclareMathOperator*{\argmin}{arg \, min}

\usepackage{xspace}
\xspaceaddexceptions{]\}}
\newcommand{\RSP}{\ensuremath{\sf{RSP}}\xspace}%

\newcommand*{\RSPenc}{\mathrm{RSP}_{\mathrm{enc}}}
\newcommand*{\filter}{\vdash}
\newcommand*{\channelBB}{\cS_{\Z\frac{\pi}{2}}}
% To denote a protocol between two parties
% \newcommand*{\protocol}{\boldsymbol{\pi}}
\newcommand\compRSP{composable $\sf{RSP}_{CC}$}
\newcommand\RSPlike{$\sf{RSP}_{CC}$}
\newcommand\compUBQC{composable $\sf{UBQC}_{CC}$}
\newcommand\UBQClike{$\sf{UBQC}_{CC}$}
% GHZ
\newcommand*{\eqdef}{\coloneqq}
\newcommand*{\PERM}{\ensuremath{}{\sf PERM}}
\newcommand*{\PPT}{\ensuremath{}{\sf PPT}}
\newcommand*{\QPT}{\ensuremath{}{\sf QPT}}
\newcommand*{\Auth}{\ensuremath{}{\sf Auth}}
% Prover, verifier, extractor
\renewcommand*{\P}{\ensuremath{}{\sf P}}
\newcommand*{\V}{\ensuremath{}{\sf V}}
\renewcommand*{\K}{\ensuremath{}{\sf K}}


% Here are the different protocols we use
% Initial protocol
\newcommand{\blind}{\ensuremath{ {\tt BLIND} }}
\newcommand{\blindZK}{\ensuremath{ {\tt BLIND-ZK} }}
% Reveals to people if they are part of the support
\newcommand{\blindSup}{\ensuremath{ {\tt BLIND}^{\tt sup} }}
% Make sure that people in the support share a canonical GHZ
\newcommand{\blindCan}{\ensuremath{ {\tt BLIND}_{\tt can} }}
% Make sure that people in the support share a canonical GHZ and that they know they are supported
\newcommand{\blindCanSup}{\ensuremath{ {\tt BLIND}^{\tt sup}_{\tt can} }}
\newcommand{\authBlindCanDist}{\ensuremath{ {\tt AUTH-BLIND}^{\tt dist}_{\tt can} }}
\newcommand{\verifCanSup}{\ensuremath{ {\tt VERIF}^{\tt sup}_{\tt can} }}
\newcommand{\blindPoly}{\ensuremath{ {\tt BLIND-poly} }}
\newcommand{\INDFCT}{\ensuremath{ {\tt IND-D0} }}
\newcommand{\INDBLIND}{\refGame*{INDBLIND}}
\newcommand{\INDBLINDtxt}{\ensuremath{ {\tt IND-\blind{}} }}
\newcommand{\INDBLINDSUP}{\refGame*{INDBLINDSUP}}
\newcommand{\INDBLINDSUPtxt}{\ensuremath{ {\tt IND-\blindSup{}} }}
\newcommand{\INDBLINDCANSUP}{\ensuremath{ {\tt IND-\blindCanSup{}} }}
\newcommand{\INDBLINDCANAUTH}{\ensuremath{ {\tt IND-\authBlindCanDist{}} }}
\newcommand{\INDPARTIAL}{\ensuremath{ {\tt IND-PARTIAL} }}
\newcommand*\GHZ{\ensuremath{}{\sf GHZ}}
\newcommand*\GHZCan{\ensuremath{{\sf GHZ^{\sf can}}}}
% Generalized GGHZ
\newcommand*\GGHZ{\ensuremath{ {\sf GHZ^G} }}
\newcommand*\HGHZ{\ensuremath{ {\sf GHZ^H} }}
\newcommand{\AssumpFctNoDelta}{\HGHZ{} capable}
\newcommand{\AssumpFctCanNoDelta}{\GHZCan{} capable}
\newcommand{\AssumpFct}{$\delta$-\HGHZ{} capable}
\newcommand{\AssumpFctNegl}{$\negl[\lambda]$-\HGHZ{} capable}
\newcommand{\AssumpFctCan}{$\delta$-\GHZCan{} capable}
\newcommand{\AssumpFctCanPrime}{$\delta^\prime$-\GHZCan{} capable}
\newcommand{\AssumpFctDist}{distributable}
\newcommand{\partialInfo}{\ensuremath{{\tt PartInfo}}}
\newcommand{\partialInfoLocal}{\ensuremath{{\tt PartInfo_{\tt Loc}}}}
\newcommand{\partialAlpha}{\ensuremath{{\tt PartAlpha_{\tt Loc}}}}
\newcommand{\CombineAlpha}{\ensuremath{{\tt CombineAlpha}}}
\newcommand{\CombineRandom}{\ensuremath{{\tt CombineRandom}}}
\newcommand{\CheckHonestTrapdoor}{\ensuremath{{\tt CheckTrapdoor}}}
\newcommand{\ZKGenCheck}{\ensuremath{{\tt ZK-GenCheck}}}

\newcommand{\highlightChanges}[1]{\textcolor{green!50!black}{#1}}

\newcommand{\Real}{\ensuremath{\mathsf{REAL}}}
\newcommand{\Sim}{\ensuremath{\mathsf{Sim}}}
\newcommand{\Ideal}{\ensuremath{\mathsf{IDEAL}}}
\newcommand{\OUT}{\ensuremath{\mathsf{OUT}}}

\usepackage{pifont}% http://ctan.org/pkg/pifont
% To denote that a user is not in the final GHZ
\newcommand{\cross}{\ensuremath{\text{\ding{55}}}}

\usepackage{verbatim}
\newcommand{\leftI}{{\normalfont\ttfamily left}}
\newcommand{\rightI}{{\normalfont\ttfamily right}}
\newcommand*{\mybar}[1]{\overline{#1}}

\newcommand*{\quout}{\mathtt{out}}

\DeclareMathOperator{\Image}{Image}
\DeclareMathOperator{\Ima}{Im}
\DeclareMathOperator{\Dom}{Dom}

\newcommand*{\Pub}{\mathtt{Pub}}
\newcommand*{\accept}{\mathtt{accept}}
\newcommand*{\abort}{\mathtt{abort}}
\newcommand*{\gadxor}{\mathtt{Gad}_{\oplus}}
\newcommand*{\quin}{\mathtt{in}}

%%% Provide a "subproof" environment (pretty simple for now,
%%% just indent the sub-proofs)
% I'd love to have left border... https://tex.stackexchange.com/questions/532948/robustly-add-a-border-to-the-left-of-a-text-spanning-several-pages
\usepackage{changepage}% http://ctan.org/pkg/changepage
\newenvironment{subproof}{\begin{adjustwidth}{2em}{0pt}}{\end{adjustwidth}}


%%% To have nice probabilities
% https://tex.stackexchange.com/questions/198771/align-in-substack
\makeatletter
\newcommand{\subalign}[1]{%
  \vcenter{%
    \Let@ \restore@math@cr \default@tag
    \baselineskip\fontdimen10 \scriptfont\tw@
    \advance\baselineskip\fontdimen12 \scriptfont\tw@
    \lineskip\thr@@\fontdimen8 \scriptfont\thr@@
    \lineskiplimit\lineskip
    \ialign{\hfil$\m@th\scriptstyle##$&$\m@th\scriptstyle{}##$\hfil\crcr
      #1\crcr
    }%
  }%
}
\makeatother
\usepackage{xparse}
% https://tex.stackexchange.com/questions/431794/same-spacing-around-middle-as-around-mid
\let\originalmiddle=\middle
\def\middle#1{\mathrel{}\originalmiddle#1\mathrel{}}
\NewDocumentCommand{\pr}{som}{\Pr\IfValueT{#2}{_{\substack{#2}}}\left[\,#3\,\right]}
\NewDocumentCommand{\esp}{som}{\IfValueTF{#2}{\underset{\subalign{#2}}{\mathbb{E}}}{\mathbb{E}}[\,#3\,]}
\NewDocumentEnvironment{pral}{soO{}}
 {%
  \Pr\IfValueT{#2}{_{\IfBooleanTF{#1}{\substack{#2}}{#2}}}
\begin{alignedat}[t]{2}
  [\, } {\,]
 #3
 \end{alignedat}
 }
\NewDocumentEnvironment{espal}{soO{}}
 {%
   \IfValueTF{#2}{\underset{\subalign{#2}}{\mathbb{E}}}{\mathbb{E}}
   \begin{alignedat}[t]{2}[\,
 }
 {\,]#3\end{alignedat}
 }
% To draw |A><B|, the second argument is facultative and gives |A><A|
\NewDocumentCommand{\ketbra}{mg}{
  | #1 \rangle \langle \IfValueTF{#2}{#2}{#1} |
}
% https://tex.stackexchange.com/a/188364/116348
\DeclareRobustCommand{\minwidthbox}[2]{%
  \ifmmode
    \expandafter\mathmakebox
  \else
    \expandafter\makebox
  \fi
  [\ifdim#2<\width\width\else#2\fi]{#1}%f
}

\NewDocumentCommand{\constructs}{mg}{
  \xrightarrow[\IfValueT{#2}{#2}]{\minwidthbox{#1}{5mm}}
}

%%%%%%%%%% This part is specific to llncs-based papers
% To enable this part of the configuration, add
%  \def\enableLlncsCode{}
% before the \input{include}
\ifdefined\enableLlncsCode%
% go back to standard amsthm proof environments with Qed at the end
\let\proof\relax\let\endproof\relax
  \usepackage{amsthm}
  % To debug, it's a nightmare without page number... To disable
  % in the final version
  \pagestyle{plain}
\fi

% Define environments when loaded in standalone...
\ifdefined\enableNonLlncsCode%
\usepackage{thmtools}
\declaretheorem[name=Theorem,numberwithin=chapter]{theorem}
\declaretheorem[name=Definition,sibling=theorem]{definition}
\declaretheorem[name=Lemma,sibling=theorem]{lemma}
\declaretheorem[name=Corollary,sibling=theorem]{corollary}
\declaretheorem[name=Remark,sibling=theorem]{remark}
\fi

\usepackage{enumitem}
\newlist{compressedList}{itemize}{10}
\setlist[compressedList]{label=--,topsep=0pt,parsep=0pt,partopsep=0pt}
\usepackage{mathtools}
\DeclarePairedDelimiter\ceil{\lceil}{\rceil}
\DeclarePairedDelimiter\floor{\lfloor}{\rfloor}

% https://tex.stackexchange.com/questions/134278/overriding-the-paragraph-style-in-the-lncs-document-class
\usepackage{etoolbox}
\patchcmd{\paragraph}{\itshape}{\bfseries\boldmath}{}{}


% This part is specific to llncs-based papers with nicer output
% (page number, theorem labelled with sections...). This should
% be the version in the arxiv.
% To enable this part of the configuration, add
%  \def\nicerThanLlncs{}
% before the \input{include}
\ifdefined\nicerThanLlncs%
  % change margins, page layout
  \usepackage[a4paper, top=1.4in, bottom=1.4in, right=1in, left=1in]{geometry}
  % enable page numbering, suppressed by default by llncs
  \pagestyle{plain}
  \setcounter{tocdepth}{2}
  \renewcommand\small{\fontsize{10pt}{12pt}\selectfont}
\fi
\ifdefined\nicerThanLlncsAbstract%
  % change margins, page layout
  \usepackage[a4paper, margin=2.5cm]{geometry}
  % enable page numbering, suppressed by default by llncs
  \pagestyle{plain}
  \setcounter{tocdepth}{2}
\fi
\ifdefined\enableLlncsCodeSlightlyImproved%
  \pagestyle{plain}
  \setcounter{tocdepth}{2}
\fi


\ifdefined\disableHyperref%
\else%
% https://rcweb.dartmouth.edu/doc/texmf-dist/doc/latex/cryptocode/cryptocode.pdf (page 55)
%%%%%%%%%% This part should be loaded last
% Usually hyperref needs to be at the end
\definecolor{secondaryColor}{RGB}{206,149,0} %% darker orange, looks like gold. <3
\usepackage[colorlinks, allcolors=secondaryColor]{hyperref}
\fi
%% Hyperref should be loaded before cryptocode! Otherwise, errors with labels of lines for example.
%%% Cryptocode
\usepackage [
n,
advantage,
operators,
sets,
adversary,
landau,
probability,
notions,
logic,
ff,
mm,
primitives,
events,
complexity,
asymptotics,
keys
] {cryptocode}
% I don't like the way "samples" is defined in cryptocode with a big dollar in front.
% \renewcommand\sample{\mathrel{\overset{\vbox to.7ex{\hbox{\fontsize{6}{0}\selectfont\vspace{.5ex} \$}}}{\leftarrow}}}
%% https://tex.stackexchange.com/a/584533/116348
\makeatletter
\renewcommand{\sample}{\mathrel{\mathpalette\sample@\relax}}
\newcommand{\sample@}[2]{%
  \ooalign{%
    \hspace{\stretch{2}}\raisebox{0.7\height}{$\m@th\demotestyle{#1}\mathdollar$}\hspace{\stretch{1}}\cr
    $\m@th#1\leftarrow$\cr
  }%
}
\newcommand{\demotestyle}[1]{%
  \ifx#1\displaystyle\scriptstyle\else
  \ifx#1\textstyle\scriptstyle\else
  \scriptscriptstyle\fi\fi
}
\makeatother
% Create a new command to create games using \game[linenumbering]{Name of game}{Game definition}
\createprocedureblock{game}{center,boxed}{}{}{}
% Create a new environment (following what I did in the thesis) to provide named games
\newcommand{\gameFont}[1]{\texttt{#1}}
\renewcommand{\Game}[1]{\ensuremath{\gameFont{Game#1}}} % By default \Game produces a nice G (symetric), but it's not very standard.

\makeatletter
\createprocedureblock{gameProc}{center,boxed}{}{}{}
\createprocedureblock{interactGame}{center,boxed}{}{}{}
%%% Nicer way to refer to games.
%%% See also https://tex.stackexchange.com/questions/623163 which provides a cleaner way to make it work with cleverref.
% Usage: \begin{namedGame}*[label][short title]{title}*[game options like skipfirstln,lnstart=1] content \end{namedGame}
% first * = disable floating. WARNING: do not put labels with non-letter chars.
% second * = do not include the game in the TOC.
\NewDocumentEnvironment{namedGame}{soomsO{}b}{%
  \IfBooleanTF{#1}{}{%
    \par% without par, the figure will be put after the line arriving after the picture if there is no paragraph.
    \setlength{\intextsep}{5.0pt plus 2.0pt minus 2.0pt}% reduce the space when image is inserted inline.
    \begin{figure}[htbp]}% 
    \begin{pcimage}%
      %% Add to toc see command listofgames
      \phantomsection% Create a fake label, useful to ensure the link point to this part.
      \IfBooleanTF{#5}{}{% If no star is given, add to "table of contents" for games
        \IfValueTF{#3}{% If a short title is provided
          \addcontentsline{game}{section}{\ensuremath{#3}}%
        }{%
          \addcontentsline{game}{section}{\ensuremath{#4}}%
        }%
      }%
      {\normalfont\gameProc[linenumbering,#6]{\ensuremath{#4}}{\IfValueTF{#2}{%
            %% Add an anchor if a label is present
            \raisebox{1em}{\hypertarget{#2}{}}%
            %% Create a macro "mygametitle@nameoflabel" to store the title
            \IfValueTF{#3}{% If a short title is provided
              \expandafter\gdef\csname mygametitle@#2 \endcsname{\ensuremath{#3}}%%
              \write\@auxout{\gdef\string\mygametitle@#2{\ensuremath{#3}}}%
            }{%
              \expandafter\gdef\csname mygametitle@#2 \endcsname{\ensuremath{#4}}%%
              \write\@auxout{\gdef\string\mygametitle@#2{\ensuremath{#4}}}%
            }%
          }{} #7 }}%
    \end{pcimage}
    \IfBooleanTF{#1}{}{\end{figure}}%
}{}

% Usage: \refGame{label}. Use \refGame*{label} if you don't want to color the link (useful in proofs)
\NewDocumentCommand{\refGame}{sm}{%
  \IfBooleanTF{#1}{%
    \hyperlink{#2}{\textcolor{black}{\csname mygametitle@#2\endcsname}}% Do not put a white space after #1!
  }{%
    \hyperlink{#2}{\csname mygametitle@#2\endcsname}% Do not put a white space after #1!
  }%
}
\makeatother

\ifdefined\disableHyperref%
\else%
% Well... not as much as cleveref
\usepackage[capitalise,noabbrev]{cleveref}
% Don't abbrev general names... but still abbrev equations
\crefname{equation}{Eq.}{Eqs.}
\crefname{figure}{Fig.}{Figs.}
\crefname{algorithm}{Protocol}{Protocols}
\Crefname{equation}{Equation}{Equations}
\Crefname{figure}{Figure}{Figures}
\Crefname{algorithm}{Protocol}{Protocols}
\fi

% Put that package at the end of the file,
% or you will have some strange errors like
% "Only one # is allowed per tab"
\usetikzlibrary{quantikz}
% Bug in llncs https://tex.stackexchange.com/a/372108/116348
\def\theHlemma{\theHtheorem}
\def\theHdefinition{\theHtheorem}
\def\theHcorollary{\theHtheorem}

%%%%%% PLEASE DO NOT WRITE ANYTHING AFTER THIS LINE %%%%%%
%%%%%% Indeed, hyperref, cleveref and quantikz need %%%%%%
%%%%%% to be loaded at the very end.                %%%%%%

\ifdefined\nicerThanLlncs%
  % change margins, page layout
  \usepackage[a4paper, top=1.4in, bottom=1.4in, right=1in, left=1in]{geometry}
  % enable page numbering, suppressed by default by llncs
  \pagestyle{plain}
  \setcounter{tocdepth}{2}
  \renewcommand\small{\fontsize{10pt}{12pt}\selectfont}
\fi
\ifdefined\nicerThanLlncsAbstract%
  % change margins, page layout
  \usepackage[a4paper, margin=2.5cm]{geometry}
  % enable page numbering, suppressed by default by llncs
  \pagestyle{plain}
  \setcounter{tocdepth}{2}
\fi
\ifdefined\enableLlncsCodeSlightlyImproved%
  \pagestyle{plain}
  \setcounter{tocdepth}{2}
\fi


\ifdefined\disableHyperref%
\else%
% https://rcweb.dartmouth.edu/doc/texmf-dist/doc/latex/cryptocode/cryptocode.pdf (page 55)
%%%%%%%%%% This part should be loaded last
% Usually hyperref needs to be at the end
\definecolor{secondaryColor}{RGB}{206,149,0} %% darker orange, looks like gold. <3
\usepackage[colorlinks, allcolors=secondaryColor]{hyperref}
\fi
%% Hyperref should be loaded before cryptocode! Otherwise, errors with labels of lines for example.
%%% Cryptocode
\usepackage [
n,
advantage,
operators,
sets,
adversary,
landau,
probability,
notions,
logic,
ff,
mm,
primitives,
events,
complexity,
asymptotics,
keys
] {cryptocode}
% I don't like the way "samples" is defined in cryptocode with a big dollar in front.
% \renewcommand\sample{\mathrel{\overset{\vbox to.7ex{\hbox{\fontsize{6}{0}\selectfont\vspace{.5ex} \$}}}{\leftarrow}}}
%% https://tex.stackexchange.com/a/584533/116348
\makeatletter
\renewcommand{\sample}{\mathrel{\mathpalette\sample@\relax}}
\newcommand{\sample@}[2]{%
  \ooalign{%
    \hspace{\stretch{2}}\raisebox{0.7\height}{$\m@th\demotestyle{#1}\mathdollar$}\hspace{\stretch{1}}\cr
    $\m@th#1\leftarrow$\cr
  }%
}
\newcommand{\demotestyle}[1]{%
  \ifx#1\displaystyle\scriptstyle\else
  \ifx#1\textstyle\scriptstyle\else
  \scriptscriptstyle\fi\fi
}
\makeatother
% Create a new command to create games using \game[linenumbering]{Name of game}{Game definition}
\createprocedureblock{game}{center,boxed}{}{}{}
% Create a new environment (following what I did in the thesis) to provide named games
\newcommand{\gameFont}[1]{\texttt{#1}}
\renewcommand{\Game}[1]{\ensuremath{\gameFont{Game#1}}} % By default \Game produces a nice G (symetric), but it's not very standard.

\makeatletter
\createprocedureblock{gameProc}{center,boxed}{}{}{}
\createprocedureblock{interactGame}{center,boxed}{}{}{}
%%% Nicer way to refer to games.
%%% See also https://tex.stackexchange.com/questions/623163 which provides a cleaner way to make it work with cleverref.
% Usage: \begin{namedGame}*[label][short title]{title}*[game options like skipfirstln,lnstart=1] content \end{namedGame}
% first * = disable floating. WARNING: do not put labels with non-letter chars.
% second * = do not include the game in the TOC.
\NewDocumentEnvironment{namedGame}{soomsO{}b}{%
  \IfBooleanTF{#1}{}{%
    \par% without par, the figure will be put after the line arriving after the picture if there is no paragraph.
    \setlength{\intextsep}{5.0pt plus 2.0pt minus 2.0pt}% reduce the space when image is inserted inline.
    \begin{figure}[htbp]}% 
    \begin{pcimage}%
      %% Add to toc see command listofgames
      \phantomsection% Create a fake label, useful to ensure the link point to this part.
      \IfBooleanTF{#5}{}{% If no star is given, add to "table of contents" for games
        \IfValueTF{#3}{% If a short title is provided
          \addcontentsline{game}{section}{\ensuremath{#3}}%
        }{%
          \addcontentsline{game}{section}{\ensuremath{#4}}%
        }%
      }%
      {\normalfont\gameProc[linenumbering,#6]{\ensuremath{#4}}{\IfValueTF{#2}{%
            %% Add an anchor if a label is present
            \raisebox{1em}{\hypertarget{#2}{}}%
            %% Create a macro "mygametitle@nameoflabel" to store the title
            \IfValueTF{#3}{% If a short title is provided
              \expandafter\gdef\csname mygametitle@#2 \endcsname{\ensuremath{#3}}%%
              \write\@auxout{\gdef\string\mygametitle@#2{\ensuremath{#3}}}%
            }{%
              \expandafter\gdef\csname mygametitle@#2 \endcsname{\ensuremath{#4}}%%
              \write\@auxout{\gdef\string\mygametitle@#2{\ensuremath{#4}}}%
            }%
          }{} #7 }}%
    \end{pcimage}
    \IfBooleanTF{#1}{}{\end{figure}}%
}{}

% Usage: \refGame{label}. Use \refGame*{label} if you don't want to color the link (useful in proofs)
\NewDocumentCommand{\refGame}{sm}{%
  \IfBooleanTF{#1}{%
    \hyperlink{#2}{\textcolor{black}{\csname mygametitle@#2\endcsname}}% Do not put a white space after #1!
  }{%
    \hyperlink{#2}{\csname mygametitle@#2\endcsname}% Do not put a white space after #1!
  }%
}
\makeatother

\ifdefined\disableHyperref%
\else%
% Well... not as much as cleveref
\usepackage[capitalise,noabbrev]{cleveref}
% Don't abbrev general names... but still abbrev equations
\crefname{equation}{Eq.}{Eqs.}
\crefname{figure}{Fig.}{Figs.}
\crefname{algorithm}{Protocol}{Protocols}
\Crefname{equation}{Equation}{Equations}
\Crefname{figure}{Figure}{Figures}
\Crefname{algorithm}{Protocol}{Protocols}
\fi

% Put that package at the end of the file,
% or you will have some strange errors like
% "Only one # is allowed per tab"
\usetikzlibrary{quantikz}
% Bug in llncs https://tex.stackexchange.com/a/372108/116348
\def\theHlemma{\theHtheorem}
\def\theHdefinition{\theHtheorem}
\def\theHcorollary{\theHtheorem}

%%%%%% PLEASE DO NOT WRITE ANYTHING AFTER THIS LINE %%%%%%
%%%%%% Indeed, hyperref, cleveref and quantikz need %%%%%%
%%%%%% to be loaded at the very end.                %%%%%%

\ifdefined\enableLlncsCode%
% go back to standard amsthm proof environments with Qed at the end
\let\proof\relax\let\endproof\relax
  \usepackage{amsthm}
  % To debug, it's a nightmare without page number... To disable
  % in the final version
  \pagestyle{plain}
\fi

% Define environments when loaded in standalone...
\ifdefined\enableNonLlncsCode%
\usepackage{thmtools}
\declaretheorem[name=Theorem,numberwithin=chapter]{theorem}
\declaretheorem[name=Definition,sibling=theorem]{definition}
\declaretheorem[name=Lemma,sibling=theorem]{lemma}
\declaretheorem[name=Corollary,sibling=theorem]{corollary}
\declaretheorem[name=Remark,sibling=theorem]{remark}
\fi

\usepackage{enumitem}
\newlist{compressedList}{itemize}{10}
\setlist[compressedList]{label=--,topsep=0pt,parsep=0pt,partopsep=0pt}
\usepackage{mathtools}
\DeclarePairedDelimiter\ceil{\lceil}{\rceil}
\DeclarePairedDelimiter\floor{\lfloor}{\rfloor}

% https://tex.stackexchange.com/questions/134278/overriding-the-paragraph-style-in-the-lncs-document-class
\usepackage{etoolbox}
\patchcmd{\paragraph}{\itshape}{\bfseries\boldmath}{}{}


% This part is specific to llncs-based papers with nicer output
% (page number, theorem labelled with sections...). This should
% be the version in the arxiv.
% To enable this part of the configuration, add
%  \def\nicerThanLlncs{}
% before the %% Let's put in these file all the definitions that we
%% require:

%% Setup language, encoding...
\RequirePackage{etex} 

\usepackage{standalone}
\usepackage{lmodern}
% Language
\usepackage[english]{babel}
% Input encoding
\usepackage[utf8]{inputenc}
% Output encoding https://tex.stackexchange.com/a/677. Important to copy accents
\usepackage[T1]{fontenc}
\usepackage{subfiles} % To quickly load this include file other files
\usepackage{microtype}
\usepackage[shortcuts,nospacearound]{extdash}    % Better endling of em dash (---) to cut sentences\===like 
%%% https://tex.stackexchange.com/questions/2773/how-do-i-make-latex-push-long-citations-to-a-new-line
\usepackage{breakcites}
\usepackage{booktabs}
% Makes sure floats don't fly accross sections
\usepackage{placeins}
\usepackage{adjustbox}
\usepackage{varwidth} % Like minipage, but with automatic width discovery.
\usepackage{complexity} % get \QMA{}...

\usepackage[normalem]{ulem}
%\usepackage[math-style=ISO]{unicode-math}
\usepackage{upgreek} % non italic greak letters

%%%% Bibliography
% I want to have separate references for appendix and main body
% https://tex.stackexchange.com/questions/98660/
\usepackage[style=trad-alpha,
  sortcites=true,
  doi=false,
  url=false,
  giveninits=true, % Bob Foo --> B. Foo
  isbn=false,
  url=false,
  eprint=false,
  sortcites=false, % \cite{B,A,C}: [A,B,C] --> [B,A,C]
]{biblatex}
\defbibfilter{appendixOnlyFilter}{
  segment=1 % Segment 1 will be chosen to be the one in appendix
  and not segment=0 % Default segment is 0
}
\renewcommand{\multicitedelim}{, } % [ABC96; DEF12] -> [ABC96, DEF12]
% [unknown_ref] => [??]
% https://tex.stackexchange.com/a/352573
\makeatletter
\protected\def\abx@missing#1{\textbf{??}}
\makeatother
\renewcommand*{\bibfont}{\normalfont\small} % BibLatex font seems bigger than Bibtex?

\usepackage{xcolor,cancel}
\usepackage{proof-at-the-end}
\NewDocumentEnvironment{theoremE}{O{}O{}+b}{%
  \begin{theoremEnd}[normal,#2]{theorem}[#1]%
    #3%
  \end{theoremEnd}
  %
}{}% Do not forget the second parameter or you might get Missing \begin{document} error
\NewDocumentEnvironment{lemmaE}{O{}O{}+b}{%
  \begin{theoremEnd}[normal,#2]{lemma}[#1]%
    #3%
  \end{theoremEnd}
  %
}{}% Do not forget the second parameter or you might get Missing \begin{document} error
\NewDocumentEnvironment{corollaryE}{O{}O{}+b}{%
  \begin{theoremEnd}[normal,#2]{corollary}[#1]%
    #3%
  \end{theoremEnd}
  %
}{}% Do not forget the second parameter or you might get Missing \begin{document} error
\NewDocumentEnvironment{proofE}{O{}+b}{%
  \begin{proofEnd}[#1]%
    #2
  \end{proofEnd}%
}{}%
\pgfkeys{/prAtEnd/global custom defaults/.style={
    text link={\noindent See \hyperref[proof:prAtEnd\pratendcountercurrent]{proof} in \cref{proofsection:prAtEnd\pratendcountercurrent}.}
  }
}

\ifdefined\displayArxivTexts%
  \pgfkeys{/prAtEnd/end if published/.style={no restate}}
  \pgfkeys{/prAtEnd/all end if published/.style={no restate}}
\else%
  \pgfkeys{/prAtEnd/end if published/.style={end}}
  \pgfkeys{/prAtEnd/all end if published/.style={all end}}
\fi

% https://tex.stackexchange.com/a/20613/116348
% https://tex.stackexchange.com/a/583244/116348
\usepackage{scalerel}
\newcommand\hcancel[2][black]{%
  \ThisStyle{%
    \setbox0=\hbox{$\SavedStyle #2$}%
    \rlap{\raisebox{.25\ht0}{\textcolor{#1}{\rule{\wd0}{1pt}}}}\SavedStyle #2}%
}

%% We provide 3 commands to add comments/text:
%% - \yournameAddText{your text to add}: proposition to add text
%% - \yournameRemoveText{text to remove}: proposition to remove text
%% - \yournameComment{your comments}: to add general comment to discuss. Should not contain proposition to add or remove text, but can be added before/after to explain the reason of the modification.
%% The interest it to provide some quick ways to see what is the final number of pages:
%% just add before the import a line:
%%   \def\removeCommentsAcceptPropositions{}
%% and the proposition will be "accepted" (i.e. the removed text is removed,
%% and the added text will be marked as black), and the comment will be removed, that way it is possible
%% to determine the final number of pages.
%% Example:
%% This text is normal\leoRemove{, this text should be removed}\leoAddText{, this text should be added}\leoComment{This is a comment, explaining why I made that changes.}.
\newcommand{\addEditor}[2]{%
  \expandafter\newcommand\csname #1AddText\endcsname[2][]{%
    \ifdefined\removeCommentsAcceptPropositions %
      ##2%
    \else%
      \textcolor{#2}{\sout{##1}}\textcolor{#2!60!black}{##2}%
    \fi%
  }%
  \expandafter\newcommand\csname #1Comment\endcsname[1]{%
    \ifdefined\removeCommentsAcceptPropositions %
    \else%
      \textcolor{#2!60!white}{\textbf{##1}}%
    \fi%
  }%
  \expandafter\newcommand\csname #1Remove\endcsname[1]{%
    \ifdefined\removeCommentsAcceptPropositions %
    \else%
      \textcolor{#2}{\sout{##1}}%
    \fi%
  }%
}
\definecolor{darkgreen}{rgb}{0., 0.7, 0}
\addEditor{leo}{darkgreen}
\addEditor{fred}{red}
\addEditor{elham}{blue}

%% We also provide an arxivOnly environment, in order to include text that are too long for the submitted version
%% but that would be good to have in the arxiv version.
% Use the \begin{arxivOnly} ... \end{arxivOnly} environment, by putting the opening and ending command
%% on there own line.
%% For shorter texts, you can use the inline version \arxivOnlyShort{your text}.
%% If you want to exclude text from the arxiv, use \begin{publishedOnly} ... \end{publishedOnly} and \publishedOnlyShort{your text}
%% "Comment" package is annoying, it does not allow block indentation, we use 'versions' instead
% \usepackage{versions} %% Do not work inside theoremE, don't know why
\ifdefined\displayArxivTexts%
  % \includeversion{arxivOnly}
  % \excludeversion{publishedOnly}
  \newcommand{\arxivOnly}[1]{#1}
  \newcommand{\arxivOnlyNoDiff}[1]{#1} % Like arxivOnly, but useful in diff mode (when it does not make sense to add colors, for instance inside a cite or when using arxivonly to create lists.
  \newcommand{\publishedOnly}[1]{}
  \newcommand{\publishedVsArxiv}[2]{#2}
  \newcommand{\inAppendixIfPublished}[1]{#1}
\else%
  \ifdefined\displayDiffTexts%
    \colorlet{arxivDiff}{green}
    \colorlet{publishedDiff}{red}
    \newcommand{\arxivOnly}[1]{{\color{arxivDiff}#1}}
    \newcommand{\arxivOnlyNoDiff}[1]{#1}
    \newcommand{\publishedOnly}[1]{{\color{publishedDiff}#1}}
    \newcommand{\publishedVsArxiv}[2]{{\color{publishedDiff}#1}{\color{arxivDiff}#2}}
    \newcommand{\inAppendixIfPublished}[1]{{\color{arxivDiff}\textbf{MOVED IN APPENDIX IN PUBLISHED}#1}\textEnd{{\color{publishedDiff}\textbf{MOVED IN MAIN BODY IN ARXIV}#1}}}
  \else
    % \excludeversion{arxivOnly}
    % \includeversion{publishedOnly}
    \newcommand{\arxivOnly}[1]{}
    \newcommand{\arxivOnlyNoDiff}[1]{}
    \newcommand{\publishedOnly}[1]{#1}
    \newcommand{\publishedVsArxiv}[2]{#1}
    \newcommand{\inAppendixIfPublished}[1]{\textEnd{#1}}
  \fi
\fi
% Alias, easier to remember  
\newcommand{\inAppendix}[1]{\textEnd{#1}}
  
% Symbol to display when an arxiv text is supposed to appear
\usepackage{todonotes}
\newcommand{\notifyArxiv}{\ifdefined\arxivDraftMode\todo{arXiv}\fi}

% Remove unwanted space before/after lists
\usepackage{enumitem}
\publishedOnly{\setlist[itemize]{noitemsep, topsep=0pt}}

% Usual packages:
\usepackage{amsmath}
\usepackage{amssymb}
% \usepackage{amsthm}
\usepackage{thmtools}
\usepackage{graphicx}
\graphicspath{{}{figures/}}
\usepackage{float}
\usepackage{subfig}
\usepackage{wrapfig}
\usepackage{array}
\usepackage{braket}
% Refer to footnotes:
% https://tex.stackexchange.com/a/167382/116348
\usepackage{footmisc}
% Useful to use the latest NewDocumentCommand:
%\usepackage{xparse}
%\usepackage[poster]{tcolorbox}

%% Space between pictures and text
%% https://mirrors.chevalier.io/CTAN/macros/latex/contrib/layouts/layman.pdf
%% \usepackage{layouts}
%% https://tex.stackexchange.com/questions/60477/remove-space-after-figure-and-before-text?rq=1
\setlength{\floatsep}{7pt plus 2pt minus 2pt}
\setlength{\textfloatsep}{7pt plus 2pt minus 2pt}
\setlength{\intextsep}{7pt plus 2pt minus 2pt}
% https://tex.stackexchange.com/questions/39017/how-to-influence-the-position-of-float-environments-like-figure-and-table-in-lat
\setcounter{topnumber}{8}
\setcounter{bottomnumber}{8}
\setcounter{totalnumber}{8}
%% https://tex.stackexchange.com/questions/45990/how-can-i-modify-vertical-space-between-figure-and-caption
%% plus/minus allows to stretch a bit around 15pt
% \setlength{\abovecaptionskip}{5pt}

% Remove number to subsubsection (in llncs it replaces \paragraph)
\setcounter{secnumdepth}{2}

%%% Some practical environment to define theorems, lemma...
% \declaretheorem[name=Theorem,numberwithin=section]{theorem}
% \declaretheorem[name=Definition,sibling=theorem]{definition}
% \declaretheorem[name=Lemma,sibling=theorem]{lemma}
% \declaretheorem[name=Corollary,sibling=theorem]{corollary}
% \declaretheorem[name=Remark,sibling=theorem]{remark}
%%% In LNCS we use
% https://tex.stackexchange.com/a/373125/116348
% \spnewtheorem{thm}{Theorem}[section]{\bfseries}{\itshape}

%%%
\usepackage[ruled,vlined]{algorithm2e}
\SetAlFnt{\normalsize}%
\setlength{\algomargin}{0em} % remove margin on left of algo
\SetAlgorithmName{Protocol}{protocol}{List of Protocols}
\newenvironment{protocol}[1][htb]{\begin{algorithm}[#1]}{\end{algorithm}}

%%% %%% Load algorithm package, and create a breakable version
%%% \usepackage{algorithm, algorithmicx, algpseudocode}
%%% % Decrease space between text:
%%% % https://tex.stackexchange.com/questions/25828/how-to-remove-change-the-vertical-spacing-before-and-after-an-algorithm-enviro/25832
%%% \setlength{\intextsep}{1\baselineskip}
%%% % https://tex.stackexchange.com/questions/387577/new-algorithm-environment-with-a-separate-counter
%%% \newcounter{protocol}
%%% \makeatletter
%%% \newenvironment{protocol}[1][htb]{%
%%%   \let\c@algorithm\c@protocol
%%%   \renewcommand{\ALG@name}{Protocol}% Update algorithm name
%%%   \begin{algorithm}[#1]%
%%%   }{\end{algorithm}
%%% }
%%% \makeatother
%%%
%%% \providecommand*\algorithmautorefname{Protocol}
%%% \makeatletter
%%% \newenvironment{breakablealgorithm}
%%%   {% \begin{breakablealgorithm}
%%%    % \begin{center}
%%%      \refstepcounter{algorithm}% New algorithm
%%%      \hrule height.8pt depth0pt \kern2pt% \@fs@pre for \@fs@ruled
%%%      \renewcommand{\caption}[2][\relax]{% Make a new \caption
%%%        % {\raggedright\textbf{\ALG@name~\thealgorithm} ##2\par}%
%%%        {\raggedright\textbf{Protocol~\thealgorithm} ##2\par}%
%%%        \ifx\relax##1\relax % #1 is \relax
%%%          \addcontentsline{loa}{algorithm}{\protect\numberline{\thealgorithm}##2}%
%%%        \else % #1 is not \relax
%%%          \addcontentsline{loa}{algorithm}{\protect\numberline{\thealgorithm}##1}%
%%%        \fi
%%%        \kern2pt\hrule\kern2pt
%%%        \small
%%%      }
%%%   }{% \end{breakablealgorithm}
%%%      \kern2pt\hrule\relax% \@fs@post for \@fs@ruled
%%%    % \end{center}
%%%   }
%%% \makeatother
%%%

\def\EQ#1{\begin{eqnarray}#1\end{eqnarray}} 


%%% Practical abreviations
\usepackage{calrsfs} % Replace \mathcal with different more "cursive" font
\DeclareMathAlphabet{\pazocal}{OMS}{zplm}{m}{n} % \pazocal is now the old \mathcal
\newcommand*{\cA}{\pazocal{A}}
\newcommand*{\cB}{\pazocal{B}}
\newcommand*{\cC}{\pazocal{C}}
\newcommand*{\cD}{\pazocal{D}}
\newcommand*{\cE}{\pazocal{E}}
\newcommand*{\cF}{\pazocal{F}}
\newcommand*{\cG}{\pazocal{G}}
\newcommand*{\cH}{\pazocal{H}}
\newcommand*{\cI}{\pazocal{I}}
\newcommand*{\cJ}{\pazocal{J}}
\newcommand*{\cK}{\pazocal{K}}
\renewcommand*{\cL}{\pazocal{L}} % defined in complexity
\newcommand*{\cM}{\pazocal{M}}
\newcommand*{\cN}{\pazocal{N}}
\newcommand*{\cO}{\pazocal{O}}
\renewcommand*{\cP}{\pazocal{P}} % defined in complexity
\newcommand*{\cQ}{\pazocal{Q}}
\newcommand*{\cR}{\pazocal{R}}
\renewcommand*{\cS}{\pazocal{S}} % defined in complexity
\newcommand*{\cT}{\pazocal{T}}
\newcommand*{\cU}{\pazocal{U}}
\newcommand*{\cV}{\pazocal{V}}
\newcommand*{\cW}{\pazocal{W}}
\newcommand*{\cX}{\pazocal{X}}
\newcommand*{\cY}{\pazocal{Y}}
\newcommand*{\cZ}{\pazocal{Z}}

% Notation for languages. Make sure not to collide with L(H) for linear operators on H
%\DeclareMathAlphabet{\pazocal}{OMS}{zplm}{m}{n}
\newcommand{\linearOp}{\mathcal{L}} % Defines the opearator L(H)
\renewcommand*{\lang}{\pazocal{L}} % Defines a new Turing/Quantum Language

%https://tex.stackexchange.com/questions/195025/
\newcommand*{\rightSend}{\rightsquigarrow} % snake "a --> b" to say "I send a to b".

% For some reasons, proof-at-the-end do not like # (problem with detokenize? To fix...). So we use a macro:
\newcommand*{\diese}{\#}

\newcommand*{\eps}[0]{\varepsilon}
\renewcommand*{\D}[0]{\mathbb{D}}
\renewcommand*{\E}{\mathbb{E}}
\newcommand*{\N}[0]{\mathbb{N}}
\renewcommand*{\R}[0]{\mathbb{R}}
\renewcommand*{\C}[0]{\mathbb{C}}
\newcommand*{\Z}[0]{\mathbb{Z}}
\newcommand*{\T}[0]{\mathbb{T}}

\newcommand*{\CPA}[0]{\textsf{CPA}}
\newcommand*{\LWE}[0]{\textsf{LWE}}
\newcommand*{\GapSVP}[0]{\textsf{GapSVP}}
\newcommand*{\SIVP}[0]{\textsf{SIVP}}

% first param is alpha, second is q (optional)
\newcommand{\contGauss}[1]{\cD_{#1}}
\newcommand{\disGauss}[2]{\cD_{\Z,#1 #2}}
\newcommand{\disGaussCont}[1]{\cD_{\Z,#1}}
\newcommand{\disGaussAQ}{\cD_{\Z,\alpha q}}
\newcommand{\rsafe}{r_{\text{safe}}}
% domain of the function.
\newcommand{\ccX}{\cX}
\newcommand{\ccXSphere}{\cX_\bullet}
\newcommand{\ccXCube}[1][]{\cX_{\scalebox{.4}{$\blacksquare$}#1}}

\newcommand{\BBHeightyFor}{\textsc{BB84}}

% "Correct" spacing around cdot as wildcard
% https://tex.stackexchange.com/questions/78165/spacing-around-cdot-when-used-as-a-wildcard
\newcommand\cdotwild{{\mkern 2mu\cdot\mkern 2mu}}

\newcommand*{\id}{\mathrm{id}}
\DeclareMathOperator{\Tr}{Tr}
\DeclareMathOperator{\supp}{supp}
\DeclareMathOperator{\Det}{Det}
\DeclareMathOperator{\ERR}{ERR}
\DeclareMathOperator{\Round}{Round}

\newcommand{\Ver}{ {\tt Ver} } %% Verify public key
\newcommand{\Gen}{ {\tt Gen} }
\newcommand{\GenLocal}{\ensuremath{ {\tt Gen}_{\tt Loc}} }
\newcommand{\Enc}{ {\tt Enc} }
\newcommand{\Dec}{ {\tt Invert} }
\newcommand{\Eval}{ {\tt Eval} }

\newcommand{\HGen}{ {\tt H.Gen} }
\newcommand{\HEnc}{ {\tt H.Enc} }
\newcommand{\HDec}{ {\tt H.Invert} }
\newcommand{\HEval}{ {\tt H.Eval} }
\newcommand{\Hh}{\ensuremath{{\tt H.}h} }
\newcommand{\HCheckHonestTrapdoor}{\ensuremath{{\tt H.CheckTrapdoor}}}

\newcommand{\MPGen}{ {\tt MP.Gen} }
\newcommand{\MPDec}{ {\tt MP.Invert} }
\newcommand{\InvertSmallGadget}{ \ensuremath{{\texttt{InvertSmallGadget}}} }
\newcommand{\InvertGadget}{ {\tt InvertGadget} }
% \newcommand{\MPEval}{ $g$ }


\DeclareMathOperator{\RoundMod}{RoundMod}


% https://tex.stackexchange.com/a/253746/116348
\usepackage{mleftright}
\newcommand{\BlockMatrix}[2]{%
  \mleft[%
  \begin{array}{@{}#1@{}}%
    #2
  \end{array}%
  \mright]%
}
\newcommand{\SmallBlockMatrix}[2]{
  \mleft[\begin{array}{c}
    #1\tabularnewline % https://tex.stackexchange.com/questions/328745/matrices-with-cryptocode-package
    \hline
    #2
  \end{array}\mright]
}
\newcommand{\HorizBlockMatrix}[2]{
  \mleft[\begin{array}{@{}c|c@{}}
    #1 & #2
  \end{array}\mright]
}

\newcommand*{\vect}[1]{\ensuremath{\mathbf{#1}}}


%% Already defined in *recent* cryptocode.
% \DeclareMathOperator*{\argmax}{arg \, max}
% \DeclareMathOperator*{\argmin}{arg \, min}

\usepackage{xspace}
\xspaceaddexceptions{]\}}
\newcommand{\RSP}{\ensuremath{\sf{RSP}}\xspace}%

\newcommand*{\RSPenc}{\mathrm{RSP}_{\mathrm{enc}}}
\newcommand*{\filter}{\vdash}
\newcommand*{\channelBB}{\cS_{\Z\frac{\pi}{2}}}
% To denote a protocol between two parties
% \newcommand*{\protocol}{\boldsymbol{\pi}}
\newcommand\compRSP{composable $\sf{RSP}_{CC}$}
\newcommand\RSPlike{$\sf{RSP}_{CC}$}
\newcommand\compUBQC{composable $\sf{UBQC}_{CC}$}
\newcommand\UBQClike{$\sf{UBQC}_{CC}$}
% GHZ
\newcommand*{\eqdef}{\coloneqq}
\newcommand*{\PERM}{\ensuremath{}{\sf PERM}}
\newcommand*{\PPT}{\ensuremath{}{\sf PPT}}
\newcommand*{\QPT}{\ensuremath{}{\sf QPT}}
\newcommand*{\Auth}{\ensuremath{}{\sf Auth}}
% Prover, verifier, extractor
\renewcommand*{\P}{\ensuremath{}{\sf P}}
\newcommand*{\V}{\ensuremath{}{\sf V}}
\renewcommand*{\K}{\ensuremath{}{\sf K}}


% Here are the different protocols we use
% Initial protocol
\newcommand{\blind}{\ensuremath{ {\tt BLIND} }}
\newcommand{\blindZK}{\ensuremath{ {\tt BLIND-ZK} }}
% Reveals to people if they are part of the support
\newcommand{\blindSup}{\ensuremath{ {\tt BLIND}^{\tt sup} }}
% Make sure that people in the support share a canonical GHZ
\newcommand{\blindCan}{\ensuremath{ {\tt BLIND}_{\tt can} }}
% Make sure that people in the support share a canonical GHZ and that they know they are supported
\newcommand{\blindCanSup}{\ensuremath{ {\tt BLIND}^{\tt sup}_{\tt can} }}
\newcommand{\authBlindCanDist}{\ensuremath{ {\tt AUTH-BLIND}^{\tt dist}_{\tt can} }}
\newcommand{\verifCanSup}{\ensuremath{ {\tt VERIF}^{\tt sup}_{\tt can} }}
\newcommand{\blindPoly}{\ensuremath{ {\tt BLIND-poly} }}
\newcommand{\INDFCT}{\ensuremath{ {\tt IND-D0} }}
\newcommand{\INDBLIND}{\refGame*{INDBLIND}}
\newcommand{\INDBLINDtxt}{\ensuremath{ {\tt IND-\blind{}} }}
\newcommand{\INDBLINDSUP}{\refGame*{INDBLINDSUP}}
\newcommand{\INDBLINDSUPtxt}{\ensuremath{ {\tt IND-\blindSup{}} }}
\newcommand{\INDBLINDCANSUP}{\ensuremath{ {\tt IND-\blindCanSup{}} }}
\newcommand{\INDBLINDCANAUTH}{\ensuremath{ {\tt IND-\authBlindCanDist{}} }}
\newcommand{\INDPARTIAL}{\ensuremath{ {\tt IND-PARTIAL} }}
\newcommand*\GHZ{\ensuremath{}{\sf GHZ}}
\newcommand*\GHZCan{\ensuremath{{\sf GHZ^{\sf can}}}}
% Generalized GGHZ
\newcommand*\GGHZ{\ensuremath{ {\sf GHZ^G} }}
\newcommand*\HGHZ{\ensuremath{ {\sf GHZ^H} }}
\newcommand{\AssumpFctNoDelta}{\HGHZ{} capable}
\newcommand{\AssumpFctCanNoDelta}{\GHZCan{} capable}
\newcommand{\AssumpFct}{$\delta$-\HGHZ{} capable}
\newcommand{\AssumpFctNegl}{$\negl[\lambda]$-\HGHZ{} capable}
\newcommand{\AssumpFctCan}{$\delta$-\GHZCan{} capable}
\newcommand{\AssumpFctCanPrime}{$\delta^\prime$-\GHZCan{} capable}
\newcommand{\AssumpFctDist}{distributable}
\newcommand{\partialInfo}{\ensuremath{{\tt PartInfo}}}
\newcommand{\partialInfoLocal}{\ensuremath{{\tt PartInfo_{\tt Loc}}}}
\newcommand{\partialAlpha}{\ensuremath{{\tt PartAlpha_{\tt Loc}}}}
\newcommand{\CombineAlpha}{\ensuremath{{\tt CombineAlpha}}}
\newcommand{\CombineRandom}{\ensuremath{{\tt CombineRandom}}}
\newcommand{\CheckHonestTrapdoor}{\ensuremath{{\tt CheckTrapdoor}}}
\newcommand{\ZKGenCheck}{\ensuremath{{\tt ZK-GenCheck}}}

\newcommand{\highlightChanges}[1]{\textcolor{green!50!black}{#1}}

\newcommand{\Real}{\ensuremath{\mathsf{REAL}}}
\newcommand{\Sim}{\ensuremath{\mathsf{Sim}}}
\newcommand{\Ideal}{\ensuremath{\mathsf{IDEAL}}}
\newcommand{\OUT}{\ensuremath{\mathsf{OUT}}}

\usepackage{pifont}% http://ctan.org/pkg/pifont
% To denote that a user is not in the final GHZ
\newcommand{\cross}{\ensuremath{\text{\ding{55}}}}

\usepackage{verbatim}
\newcommand{\leftI}{{\normalfont\ttfamily left}}
\newcommand{\rightI}{{\normalfont\ttfamily right}}
\newcommand*{\mybar}[1]{\overline{#1}}

\newcommand*{\quout}{\mathtt{out}}

\DeclareMathOperator{\Image}{Image}
\DeclareMathOperator{\Ima}{Im}
\DeclareMathOperator{\Dom}{Dom}

\newcommand*{\Pub}{\mathtt{Pub}}
\newcommand*{\accept}{\mathtt{accept}}
\newcommand*{\abort}{\mathtt{abort}}
\newcommand*{\gadxor}{\mathtt{Gad}_{\oplus}}
\newcommand*{\quin}{\mathtt{in}}

%%% Provide a "subproof" environment (pretty simple for now,
%%% just indent the sub-proofs)
% I'd love to have left border... https://tex.stackexchange.com/questions/532948/robustly-add-a-border-to-the-left-of-a-text-spanning-several-pages
\usepackage{changepage}% http://ctan.org/pkg/changepage
\newenvironment{subproof}{\begin{adjustwidth}{2em}{0pt}}{\end{adjustwidth}}


%%% To have nice probabilities
% https://tex.stackexchange.com/questions/198771/align-in-substack
\makeatletter
\newcommand{\subalign}[1]{%
  \vcenter{%
    \Let@ \restore@math@cr \default@tag
    \baselineskip\fontdimen10 \scriptfont\tw@
    \advance\baselineskip\fontdimen12 \scriptfont\tw@
    \lineskip\thr@@\fontdimen8 \scriptfont\thr@@
    \lineskiplimit\lineskip
    \ialign{\hfil$\m@th\scriptstyle##$&$\m@th\scriptstyle{}##$\hfil\crcr
      #1\crcr
    }%
  }%
}
\makeatother
\usepackage{xparse}
% https://tex.stackexchange.com/questions/431794/same-spacing-around-middle-as-around-mid
\let\originalmiddle=\middle
\def\middle#1{\mathrel{}\originalmiddle#1\mathrel{}}
\NewDocumentCommand{\pr}{som}{\Pr\IfValueT{#2}{_{\substack{#2}}}\left[\,#3\,\right]}
\NewDocumentCommand{\esp}{som}{\IfValueTF{#2}{\underset{\subalign{#2}}{\mathbb{E}}}{\mathbb{E}}[\,#3\,]}
\NewDocumentEnvironment{pral}{soO{}}
 {%
  \Pr\IfValueT{#2}{_{\IfBooleanTF{#1}{\substack{#2}}{#2}}}
\begin{alignedat}[t]{2}
  [\, } {\,]
 #3
 \end{alignedat}
 }
\NewDocumentEnvironment{espal}{soO{}}
 {%
   \IfValueTF{#2}{\underset{\subalign{#2}}{\mathbb{E}}}{\mathbb{E}}
   \begin{alignedat}[t]{2}[\,
 }
 {\,]#3\end{alignedat}
 }
% To draw |A><B|, the second argument is facultative and gives |A><A|
\NewDocumentCommand{\ketbra}{mg}{
  | #1 \rangle \langle \IfValueTF{#2}{#2}{#1} |
}
% https://tex.stackexchange.com/a/188364/116348
\DeclareRobustCommand{\minwidthbox}[2]{%
  \ifmmode
    \expandafter\mathmakebox
  \else
    \expandafter\makebox
  \fi
  [\ifdim#2<\width\width\else#2\fi]{#1}%f
}

\NewDocumentCommand{\constructs}{mg}{
  \xrightarrow[\IfValueT{#2}{#2}]{\minwidthbox{#1}{5mm}}
}

%%%%%%%%%% This part is specific to llncs-based papers
% To enable this part of the configuration, add
%  \def\enableLlncsCode{}
% before the %% Let's put in these file all the definitions that we
%% require:

%% Setup language, encoding...
\RequirePackage{etex} 

\usepackage{standalone}
\usepackage{lmodern}
% Language
\usepackage[english]{babel}
% Input encoding
\usepackage[utf8]{inputenc}
% Output encoding https://tex.stackexchange.com/a/677. Important to copy accents
\usepackage[T1]{fontenc}
\usepackage{subfiles} % To quickly load this include file other files
\usepackage{microtype}
\usepackage[shortcuts,nospacearound]{extdash}    % Better endling of em dash (---) to cut sentences\===like 
%%% https://tex.stackexchange.com/questions/2773/how-do-i-make-latex-push-long-citations-to-a-new-line
\usepackage{breakcites}
\usepackage{booktabs}
% Makes sure floats don't fly accross sections
\usepackage{placeins}
\usepackage{adjustbox}
\usepackage{varwidth} % Like minipage, but with automatic width discovery.
\usepackage{complexity} % get \QMA{}...

\usepackage[normalem]{ulem}
%\usepackage[math-style=ISO]{unicode-math}
\usepackage{upgreek} % non italic greak letters

%%%% Bibliography
% I want to have separate references for appendix and main body
% https://tex.stackexchange.com/questions/98660/
\usepackage[style=trad-alpha,
  sortcites=true,
  doi=false,
  url=false,
  giveninits=true, % Bob Foo --> B. Foo
  isbn=false,
  url=false,
  eprint=false,
  sortcites=false, % \cite{B,A,C}: [A,B,C] --> [B,A,C]
]{biblatex}
\defbibfilter{appendixOnlyFilter}{
  segment=1 % Segment 1 will be chosen to be the one in appendix
  and not segment=0 % Default segment is 0
}
\renewcommand{\multicitedelim}{, } % [ABC96; DEF12] -> [ABC96, DEF12]
% [unknown_ref] => [??]
% https://tex.stackexchange.com/a/352573
\makeatletter
\protected\def\abx@missing#1{\textbf{??}}
\makeatother
\renewcommand*{\bibfont}{\normalfont\small} % BibLatex font seems bigger than Bibtex?

\usepackage{xcolor,cancel}
\usepackage{proof-at-the-end}
\NewDocumentEnvironment{theoremE}{O{}O{}+b}{%
  \begin{theoremEnd}[normal,#2]{theorem}[#1]%
    #3%
  \end{theoremEnd}
  %
}{}% Do not forget the second parameter or you might get Missing \begin{document} error
\NewDocumentEnvironment{lemmaE}{O{}O{}+b}{%
  \begin{theoremEnd}[normal,#2]{lemma}[#1]%
    #3%
  \end{theoremEnd}
  %
}{}% Do not forget the second parameter or you might get Missing \begin{document} error
\NewDocumentEnvironment{corollaryE}{O{}O{}+b}{%
  \begin{theoremEnd}[normal,#2]{corollary}[#1]%
    #3%
  \end{theoremEnd}
  %
}{}% Do not forget the second parameter or you might get Missing \begin{document} error
\NewDocumentEnvironment{proofE}{O{}+b}{%
  \begin{proofEnd}[#1]%
    #2
  \end{proofEnd}%
}{}%
\pgfkeys{/prAtEnd/global custom defaults/.style={
    text link={\noindent See \hyperref[proof:prAtEnd\pratendcountercurrent]{proof} in \cref{proofsection:prAtEnd\pratendcountercurrent}.}
  }
}

\ifdefined\displayArxivTexts%
  \pgfkeys{/prAtEnd/end if published/.style={no restate}}
  \pgfkeys{/prAtEnd/all end if published/.style={no restate}}
\else%
  \pgfkeys{/prAtEnd/end if published/.style={end}}
  \pgfkeys{/prAtEnd/all end if published/.style={all end}}
\fi

% https://tex.stackexchange.com/a/20613/116348
% https://tex.stackexchange.com/a/583244/116348
\usepackage{scalerel}
\newcommand\hcancel[2][black]{%
  \ThisStyle{%
    \setbox0=\hbox{$\SavedStyle #2$}%
    \rlap{\raisebox{.25\ht0}{\textcolor{#1}{\rule{\wd0}{1pt}}}}\SavedStyle #2}%
}

%% We provide 3 commands to add comments/text:
%% - \yournameAddText{your text to add}: proposition to add text
%% - \yournameRemoveText{text to remove}: proposition to remove text
%% - \yournameComment{your comments}: to add general comment to discuss. Should not contain proposition to add or remove text, but can be added before/after to explain the reason of the modification.
%% The interest it to provide some quick ways to see what is the final number of pages:
%% just add before the import a line:
%%   \def\removeCommentsAcceptPropositions{}
%% and the proposition will be "accepted" (i.e. the removed text is removed,
%% and the added text will be marked as black), and the comment will be removed, that way it is possible
%% to determine the final number of pages.
%% Example:
%% This text is normal\leoRemove{, this text should be removed}\leoAddText{, this text should be added}\leoComment{This is a comment, explaining why I made that changes.}.
\newcommand{\addEditor}[2]{%
  \expandafter\newcommand\csname #1AddText\endcsname[2][]{%
    \ifdefined\removeCommentsAcceptPropositions %
      ##2%
    \else%
      \textcolor{#2}{\sout{##1}}\textcolor{#2!60!black}{##2}%
    \fi%
  }%
  \expandafter\newcommand\csname #1Comment\endcsname[1]{%
    \ifdefined\removeCommentsAcceptPropositions %
    \else%
      \textcolor{#2!60!white}{\textbf{##1}}%
    \fi%
  }%
  \expandafter\newcommand\csname #1Remove\endcsname[1]{%
    \ifdefined\removeCommentsAcceptPropositions %
    \else%
      \textcolor{#2}{\sout{##1}}%
    \fi%
  }%
}
\definecolor{darkgreen}{rgb}{0., 0.7, 0}
\addEditor{leo}{darkgreen}
\addEditor{fred}{red}
\addEditor{elham}{blue}

%% We also provide an arxivOnly environment, in order to include text that are too long for the submitted version
%% but that would be good to have in the arxiv version.
% Use the \begin{arxivOnly} ... \end{arxivOnly} environment, by putting the opening and ending command
%% on there own line.
%% For shorter texts, you can use the inline version \arxivOnlyShort{your text}.
%% If you want to exclude text from the arxiv, use \begin{publishedOnly} ... \end{publishedOnly} and \publishedOnlyShort{your text}
%% "Comment" package is annoying, it does not allow block indentation, we use 'versions' instead
% \usepackage{versions} %% Do not work inside theoremE, don't know why
\ifdefined\displayArxivTexts%
  % \includeversion{arxivOnly}
  % \excludeversion{publishedOnly}
  \newcommand{\arxivOnly}[1]{#1}
  \newcommand{\arxivOnlyNoDiff}[1]{#1} % Like arxivOnly, but useful in diff mode (when it does not make sense to add colors, for instance inside a cite or when using arxivonly to create lists.
  \newcommand{\publishedOnly}[1]{}
  \newcommand{\publishedVsArxiv}[2]{#2}
  \newcommand{\inAppendixIfPublished}[1]{#1}
\else%
  \ifdefined\displayDiffTexts%
    \colorlet{arxivDiff}{green}
    \colorlet{publishedDiff}{red}
    \newcommand{\arxivOnly}[1]{{\color{arxivDiff}#1}}
    \newcommand{\arxivOnlyNoDiff}[1]{#1}
    \newcommand{\publishedOnly}[1]{{\color{publishedDiff}#1}}
    \newcommand{\publishedVsArxiv}[2]{{\color{publishedDiff}#1}{\color{arxivDiff}#2}}
    \newcommand{\inAppendixIfPublished}[1]{{\color{arxivDiff}\textbf{MOVED IN APPENDIX IN PUBLISHED}#1}\textEnd{{\color{publishedDiff}\textbf{MOVED IN MAIN BODY IN ARXIV}#1}}}
  \else
    % \excludeversion{arxivOnly}
    % \includeversion{publishedOnly}
    \newcommand{\arxivOnly}[1]{}
    \newcommand{\arxivOnlyNoDiff}[1]{}
    \newcommand{\publishedOnly}[1]{#1}
    \newcommand{\publishedVsArxiv}[2]{#1}
    \newcommand{\inAppendixIfPublished}[1]{\textEnd{#1}}
  \fi
\fi
% Alias, easier to remember  
\newcommand{\inAppendix}[1]{\textEnd{#1}}
  
% Symbol to display when an arxiv text is supposed to appear
\usepackage{todonotes}
\newcommand{\notifyArxiv}{\ifdefined\arxivDraftMode\todo{arXiv}\fi}

% Remove unwanted space before/after lists
\usepackage{enumitem}
\publishedOnly{\setlist[itemize]{noitemsep, topsep=0pt}}

% Usual packages:
\usepackage{amsmath}
\usepackage{amssymb}
% \usepackage{amsthm}
\usepackage{thmtools}
\usepackage{graphicx}
\graphicspath{{}{figures/}}
\usepackage{float}
\usepackage{subfig}
\usepackage{wrapfig}
\usepackage{array}
\usepackage{braket}
% Refer to footnotes:
% https://tex.stackexchange.com/a/167382/116348
\usepackage{footmisc}
% Useful to use the latest NewDocumentCommand:
%\usepackage{xparse}
%\usepackage[poster]{tcolorbox}

%% Space between pictures and text
%% https://mirrors.chevalier.io/CTAN/macros/latex/contrib/layouts/layman.pdf
%% \usepackage{layouts}
%% https://tex.stackexchange.com/questions/60477/remove-space-after-figure-and-before-text?rq=1
\setlength{\floatsep}{7pt plus 2pt minus 2pt}
\setlength{\textfloatsep}{7pt plus 2pt minus 2pt}
\setlength{\intextsep}{7pt plus 2pt minus 2pt}
% https://tex.stackexchange.com/questions/39017/how-to-influence-the-position-of-float-environments-like-figure-and-table-in-lat
\setcounter{topnumber}{8}
\setcounter{bottomnumber}{8}
\setcounter{totalnumber}{8}
%% https://tex.stackexchange.com/questions/45990/how-can-i-modify-vertical-space-between-figure-and-caption
%% plus/minus allows to stretch a bit around 15pt
% \setlength{\abovecaptionskip}{5pt}

% Remove number to subsubsection (in llncs it replaces \paragraph)
\setcounter{secnumdepth}{2}

%%% Some practical environment to define theorems, lemma...
% \declaretheorem[name=Theorem,numberwithin=section]{theorem}
% \declaretheorem[name=Definition,sibling=theorem]{definition}
% \declaretheorem[name=Lemma,sibling=theorem]{lemma}
% \declaretheorem[name=Corollary,sibling=theorem]{corollary}
% \declaretheorem[name=Remark,sibling=theorem]{remark}
%%% In LNCS we use
% https://tex.stackexchange.com/a/373125/116348
% \spnewtheorem{thm}{Theorem}[section]{\bfseries}{\itshape}

%%%
\usepackage[ruled,vlined]{algorithm2e}
\SetAlFnt{\normalsize}%
\setlength{\algomargin}{0em} % remove margin on left of algo
\SetAlgorithmName{Protocol}{protocol}{List of Protocols}
\newenvironment{protocol}[1][htb]{\begin{algorithm}[#1]}{\end{algorithm}}

%%% %%% Load algorithm package, and create a breakable version
%%% \usepackage{algorithm, algorithmicx, algpseudocode}
%%% % Decrease space between text:
%%% % https://tex.stackexchange.com/questions/25828/how-to-remove-change-the-vertical-spacing-before-and-after-an-algorithm-enviro/25832
%%% \setlength{\intextsep}{1\baselineskip}
%%% % https://tex.stackexchange.com/questions/387577/new-algorithm-environment-with-a-separate-counter
%%% \newcounter{protocol}
%%% \makeatletter
%%% \newenvironment{protocol}[1][htb]{%
%%%   \let\c@algorithm\c@protocol
%%%   \renewcommand{\ALG@name}{Protocol}% Update algorithm name
%%%   \begin{algorithm}[#1]%
%%%   }{\end{algorithm}
%%% }
%%% \makeatother
%%%
%%% \providecommand*\algorithmautorefname{Protocol}
%%% \makeatletter
%%% \newenvironment{breakablealgorithm}
%%%   {% \begin{breakablealgorithm}
%%%    % \begin{center}
%%%      \refstepcounter{algorithm}% New algorithm
%%%      \hrule height.8pt depth0pt \kern2pt% \@fs@pre for \@fs@ruled
%%%      \renewcommand{\caption}[2][\relax]{% Make a new \caption
%%%        % {\raggedright\textbf{\ALG@name~\thealgorithm} ##2\par}%
%%%        {\raggedright\textbf{Protocol~\thealgorithm} ##2\par}%
%%%        \ifx\relax##1\relax % #1 is \relax
%%%          \addcontentsline{loa}{algorithm}{\protect\numberline{\thealgorithm}##2}%
%%%        \else % #1 is not \relax
%%%          \addcontentsline{loa}{algorithm}{\protect\numberline{\thealgorithm}##1}%
%%%        \fi
%%%        \kern2pt\hrule\kern2pt
%%%        \small
%%%      }
%%%   }{% \end{breakablealgorithm}
%%%      \kern2pt\hrule\relax% \@fs@post for \@fs@ruled
%%%    % \end{center}
%%%   }
%%% \makeatother
%%%

\def\EQ#1{\begin{eqnarray}#1\end{eqnarray}} 


%%% Practical abreviations
\usepackage{calrsfs} % Replace \mathcal with different more "cursive" font
\DeclareMathAlphabet{\pazocal}{OMS}{zplm}{m}{n} % \pazocal is now the old \mathcal
\newcommand*{\cA}{\pazocal{A}}
\newcommand*{\cB}{\pazocal{B}}
\newcommand*{\cC}{\pazocal{C}}
\newcommand*{\cD}{\pazocal{D}}
\newcommand*{\cE}{\pazocal{E}}
\newcommand*{\cF}{\pazocal{F}}
\newcommand*{\cG}{\pazocal{G}}
\newcommand*{\cH}{\pazocal{H}}
\newcommand*{\cI}{\pazocal{I}}
\newcommand*{\cJ}{\pazocal{J}}
\newcommand*{\cK}{\pazocal{K}}
\renewcommand*{\cL}{\pazocal{L}} % defined in complexity
\newcommand*{\cM}{\pazocal{M}}
\newcommand*{\cN}{\pazocal{N}}
\newcommand*{\cO}{\pazocal{O}}
\renewcommand*{\cP}{\pazocal{P}} % defined in complexity
\newcommand*{\cQ}{\pazocal{Q}}
\newcommand*{\cR}{\pazocal{R}}
\renewcommand*{\cS}{\pazocal{S}} % defined in complexity
\newcommand*{\cT}{\pazocal{T}}
\newcommand*{\cU}{\pazocal{U}}
\newcommand*{\cV}{\pazocal{V}}
\newcommand*{\cW}{\pazocal{W}}
\newcommand*{\cX}{\pazocal{X}}
\newcommand*{\cY}{\pazocal{Y}}
\newcommand*{\cZ}{\pazocal{Z}}

% Notation for languages. Make sure not to collide with L(H) for linear operators on H
%\DeclareMathAlphabet{\pazocal}{OMS}{zplm}{m}{n}
\newcommand{\linearOp}{\mathcal{L}} % Defines the opearator L(H)
\renewcommand*{\lang}{\pazocal{L}} % Defines a new Turing/Quantum Language

%https://tex.stackexchange.com/questions/195025/
\newcommand*{\rightSend}{\rightsquigarrow} % snake "a --> b" to say "I send a to b".

% For some reasons, proof-at-the-end do not like # (problem with detokenize? To fix...). So we use a macro:
\newcommand*{\diese}{\#}

\newcommand*{\eps}[0]{\varepsilon}
\renewcommand*{\D}[0]{\mathbb{D}}
\renewcommand*{\E}{\mathbb{E}}
\newcommand*{\N}[0]{\mathbb{N}}
\renewcommand*{\R}[0]{\mathbb{R}}
\renewcommand*{\C}[0]{\mathbb{C}}
\newcommand*{\Z}[0]{\mathbb{Z}}
\newcommand*{\T}[0]{\mathbb{T}}

\newcommand*{\CPA}[0]{\textsf{CPA}}
\newcommand*{\LWE}[0]{\textsf{LWE}}
\newcommand*{\GapSVP}[0]{\textsf{GapSVP}}
\newcommand*{\SIVP}[0]{\textsf{SIVP}}

% first param is alpha, second is q (optional)
\newcommand{\contGauss}[1]{\cD_{#1}}
\newcommand{\disGauss}[2]{\cD_{\Z,#1 #2}}
\newcommand{\disGaussCont}[1]{\cD_{\Z,#1}}
\newcommand{\disGaussAQ}{\cD_{\Z,\alpha q}}
\newcommand{\rsafe}{r_{\text{safe}}}
% domain of the function.
\newcommand{\ccX}{\cX}
\newcommand{\ccXSphere}{\cX_\bullet}
\newcommand{\ccXCube}[1][]{\cX_{\scalebox{.4}{$\blacksquare$}#1}}

\newcommand{\BBHeightyFor}{\textsc{BB84}}

% "Correct" spacing around cdot as wildcard
% https://tex.stackexchange.com/questions/78165/spacing-around-cdot-when-used-as-a-wildcard
\newcommand\cdotwild{{\mkern 2mu\cdot\mkern 2mu}}

\newcommand*{\id}{\mathrm{id}}
\DeclareMathOperator{\Tr}{Tr}
\DeclareMathOperator{\supp}{supp}
\DeclareMathOperator{\Det}{Det}
\DeclareMathOperator{\ERR}{ERR}
\DeclareMathOperator{\Round}{Round}

\newcommand{\Ver}{ {\tt Ver} } %% Verify public key
\newcommand{\Gen}{ {\tt Gen} }
\newcommand{\GenLocal}{\ensuremath{ {\tt Gen}_{\tt Loc}} }
\newcommand{\Enc}{ {\tt Enc} }
\newcommand{\Dec}{ {\tt Invert} }
\newcommand{\Eval}{ {\tt Eval} }

\newcommand{\HGen}{ {\tt H.Gen} }
\newcommand{\HEnc}{ {\tt H.Enc} }
\newcommand{\HDec}{ {\tt H.Invert} }
\newcommand{\HEval}{ {\tt H.Eval} }
\newcommand{\Hh}{\ensuremath{{\tt H.}h} }
\newcommand{\HCheckHonestTrapdoor}{\ensuremath{{\tt H.CheckTrapdoor}}}

\newcommand{\MPGen}{ {\tt MP.Gen} }
\newcommand{\MPDec}{ {\tt MP.Invert} }
\newcommand{\InvertSmallGadget}{ \ensuremath{{\texttt{InvertSmallGadget}}} }
\newcommand{\InvertGadget}{ {\tt InvertGadget} }
% \newcommand{\MPEval}{ $g$ }


\DeclareMathOperator{\RoundMod}{RoundMod}


% https://tex.stackexchange.com/a/253746/116348
\usepackage{mleftright}
\newcommand{\BlockMatrix}[2]{%
  \mleft[%
  \begin{array}{@{}#1@{}}%
    #2
  \end{array}%
  \mright]%
}
\newcommand{\SmallBlockMatrix}[2]{
  \mleft[\begin{array}{c}
    #1\tabularnewline % https://tex.stackexchange.com/questions/328745/matrices-with-cryptocode-package
    \hline
    #2
  \end{array}\mright]
}
\newcommand{\HorizBlockMatrix}[2]{
  \mleft[\begin{array}{@{}c|c@{}}
    #1 & #2
  \end{array}\mright]
}

\newcommand*{\vect}[1]{\ensuremath{\mathbf{#1}}}


%% Already defined in *recent* cryptocode.
% \DeclareMathOperator*{\argmax}{arg \, max}
% \DeclareMathOperator*{\argmin}{arg \, min}

\usepackage{xspace}
\xspaceaddexceptions{]\}}
\newcommand{\RSP}{\ensuremath{\sf{RSP}}\xspace}%

\newcommand*{\RSPenc}{\mathrm{RSP}_{\mathrm{enc}}}
\newcommand*{\filter}{\vdash}
\newcommand*{\channelBB}{\cS_{\Z\frac{\pi}{2}}}
% To denote a protocol between two parties
% \newcommand*{\protocol}{\boldsymbol{\pi}}
\newcommand\compRSP{composable $\sf{RSP}_{CC}$}
\newcommand\RSPlike{$\sf{RSP}_{CC}$}
\newcommand\compUBQC{composable $\sf{UBQC}_{CC}$}
\newcommand\UBQClike{$\sf{UBQC}_{CC}$}
% GHZ
\newcommand*{\eqdef}{\coloneqq}
\newcommand*{\PERM}{\ensuremath{}{\sf PERM}}
\newcommand*{\PPT}{\ensuremath{}{\sf PPT}}
\newcommand*{\QPT}{\ensuremath{}{\sf QPT}}
\newcommand*{\Auth}{\ensuremath{}{\sf Auth}}
% Prover, verifier, extractor
\renewcommand*{\P}{\ensuremath{}{\sf P}}
\newcommand*{\V}{\ensuremath{}{\sf V}}
\renewcommand*{\K}{\ensuremath{}{\sf K}}


% Here are the different protocols we use
% Initial protocol
\newcommand{\blind}{\ensuremath{ {\tt BLIND} }}
\newcommand{\blindZK}{\ensuremath{ {\tt BLIND-ZK} }}
% Reveals to people if they are part of the support
\newcommand{\blindSup}{\ensuremath{ {\tt BLIND}^{\tt sup} }}
% Make sure that people in the support share a canonical GHZ
\newcommand{\blindCan}{\ensuremath{ {\tt BLIND}_{\tt can} }}
% Make sure that people in the support share a canonical GHZ and that they know they are supported
\newcommand{\blindCanSup}{\ensuremath{ {\tt BLIND}^{\tt sup}_{\tt can} }}
\newcommand{\authBlindCanDist}{\ensuremath{ {\tt AUTH-BLIND}^{\tt dist}_{\tt can} }}
\newcommand{\verifCanSup}{\ensuremath{ {\tt VERIF}^{\tt sup}_{\tt can} }}
\newcommand{\blindPoly}{\ensuremath{ {\tt BLIND-poly} }}
\newcommand{\INDFCT}{\ensuremath{ {\tt IND-D0} }}
\newcommand{\INDBLIND}{\refGame*{INDBLIND}}
\newcommand{\INDBLINDtxt}{\ensuremath{ {\tt IND-\blind{}} }}
\newcommand{\INDBLINDSUP}{\refGame*{INDBLINDSUP}}
\newcommand{\INDBLINDSUPtxt}{\ensuremath{ {\tt IND-\blindSup{}} }}
\newcommand{\INDBLINDCANSUP}{\ensuremath{ {\tt IND-\blindCanSup{}} }}
\newcommand{\INDBLINDCANAUTH}{\ensuremath{ {\tt IND-\authBlindCanDist{}} }}
\newcommand{\INDPARTIAL}{\ensuremath{ {\tt IND-PARTIAL} }}
\newcommand*\GHZ{\ensuremath{}{\sf GHZ}}
\newcommand*\GHZCan{\ensuremath{{\sf GHZ^{\sf can}}}}
% Generalized GGHZ
\newcommand*\GGHZ{\ensuremath{ {\sf GHZ^G} }}
\newcommand*\HGHZ{\ensuremath{ {\sf GHZ^H} }}
\newcommand{\AssumpFctNoDelta}{\HGHZ{} capable}
\newcommand{\AssumpFctCanNoDelta}{\GHZCan{} capable}
\newcommand{\AssumpFct}{$\delta$-\HGHZ{} capable}
\newcommand{\AssumpFctNegl}{$\negl[\lambda]$-\HGHZ{} capable}
\newcommand{\AssumpFctCan}{$\delta$-\GHZCan{} capable}
\newcommand{\AssumpFctCanPrime}{$\delta^\prime$-\GHZCan{} capable}
\newcommand{\AssumpFctDist}{distributable}
\newcommand{\partialInfo}{\ensuremath{{\tt PartInfo}}}
\newcommand{\partialInfoLocal}{\ensuremath{{\tt PartInfo_{\tt Loc}}}}
\newcommand{\partialAlpha}{\ensuremath{{\tt PartAlpha_{\tt Loc}}}}
\newcommand{\CombineAlpha}{\ensuremath{{\tt CombineAlpha}}}
\newcommand{\CombineRandom}{\ensuremath{{\tt CombineRandom}}}
\newcommand{\CheckHonestTrapdoor}{\ensuremath{{\tt CheckTrapdoor}}}
\newcommand{\ZKGenCheck}{\ensuremath{{\tt ZK-GenCheck}}}

\newcommand{\highlightChanges}[1]{\textcolor{green!50!black}{#1}}

\newcommand{\Real}{\ensuremath{\mathsf{REAL}}}
\newcommand{\Sim}{\ensuremath{\mathsf{Sim}}}
\newcommand{\Ideal}{\ensuremath{\mathsf{IDEAL}}}
\newcommand{\OUT}{\ensuremath{\mathsf{OUT}}}

\usepackage{pifont}% http://ctan.org/pkg/pifont
% To denote that a user is not in the final GHZ
\newcommand{\cross}{\ensuremath{\text{\ding{55}}}}

\usepackage{verbatim}
\newcommand{\leftI}{{\normalfont\ttfamily left}}
\newcommand{\rightI}{{\normalfont\ttfamily right}}
\newcommand*{\mybar}[1]{\overline{#1}}

\newcommand*{\quout}{\mathtt{out}}

\DeclareMathOperator{\Image}{Image}
\DeclareMathOperator{\Ima}{Im}
\DeclareMathOperator{\Dom}{Dom}

\newcommand*{\Pub}{\mathtt{Pub}}
\newcommand*{\accept}{\mathtt{accept}}
\newcommand*{\abort}{\mathtt{abort}}
\newcommand*{\gadxor}{\mathtt{Gad}_{\oplus}}
\newcommand*{\quin}{\mathtt{in}}

%%% Provide a "subproof" environment (pretty simple for now,
%%% just indent the sub-proofs)
% I'd love to have left border... https://tex.stackexchange.com/questions/532948/robustly-add-a-border-to-the-left-of-a-text-spanning-several-pages
\usepackage{changepage}% http://ctan.org/pkg/changepage
\newenvironment{subproof}{\begin{adjustwidth}{2em}{0pt}}{\end{adjustwidth}}


%%% To have nice probabilities
% https://tex.stackexchange.com/questions/198771/align-in-substack
\makeatletter
\newcommand{\subalign}[1]{%
  \vcenter{%
    \Let@ \restore@math@cr \default@tag
    \baselineskip\fontdimen10 \scriptfont\tw@
    \advance\baselineskip\fontdimen12 \scriptfont\tw@
    \lineskip\thr@@\fontdimen8 \scriptfont\thr@@
    \lineskiplimit\lineskip
    \ialign{\hfil$\m@th\scriptstyle##$&$\m@th\scriptstyle{}##$\hfil\crcr
      #1\crcr
    }%
  }%
}
\makeatother
\usepackage{xparse}
% https://tex.stackexchange.com/questions/431794/same-spacing-around-middle-as-around-mid
\let\originalmiddle=\middle
\def\middle#1{\mathrel{}\originalmiddle#1\mathrel{}}
\NewDocumentCommand{\pr}{som}{\Pr\IfValueT{#2}{_{\substack{#2}}}\left[\,#3\,\right]}
\NewDocumentCommand{\esp}{som}{\IfValueTF{#2}{\underset{\subalign{#2}}{\mathbb{E}}}{\mathbb{E}}[\,#3\,]}
\NewDocumentEnvironment{pral}{soO{}}
 {%
  \Pr\IfValueT{#2}{_{\IfBooleanTF{#1}{\substack{#2}}{#2}}}
\begin{alignedat}[t]{2}
  [\, } {\,]
 #3
 \end{alignedat}
 }
\NewDocumentEnvironment{espal}{soO{}}
 {%
   \IfValueTF{#2}{\underset{\subalign{#2}}{\mathbb{E}}}{\mathbb{E}}
   \begin{alignedat}[t]{2}[\,
 }
 {\,]#3\end{alignedat}
 }
% To draw |A><B|, the second argument is facultative and gives |A><A|
\NewDocumentCommand{\ketbra}{mg}{
  | #1 \rangle \langle \IfValueTF{#2}{#2}{#1} |
}
% https://tex.stackexchange.com/a/188364/116348
\DeclareRobustCommand{\minwidthbox}[2]{%
  \ifmmode
    \expandafter\mathmakebox
  \else
    \expandafter\makebox
  \fi
  [\ifdim#2<\width\width\else#2\fi]{#1}%f
}

\NewDocumentCommand{\constructs}{mg}{
  \xrightarrow[\IfValueT{#2}{#2}]{\minwidthbox{#1}{5mm}}
}

%%%%%%%%%% This part is specific to llncs-based papers
% To enable this part of the configuration, add
%  \def\enableLlncsCode{}
% before the \input{include}
\ifdefined\enableLlncsCode%
% go back to standard amsthm proof environments with Qed at the end
\let\proof\relax\let\endproof\relax
  \usepackage{amsthm}
  % To debug, it's a nightmare without page number... To disable
  % in the final version
  \pagestyle{plain}
\fi

% Define environments when loaded in standalone...
\ifdefined\enableNonLlncsCode%
\usepackage{thmtools}
\declaretheorem[name=Theorem,numberwithin=chapter]{theorem}
\declaretheorem[name=Definition,sibling=theorem]{definition}
\declaretheorem[name=Lemma,sibling=theorem]{lemma}
\declaretheorem[name=Corollary,sibling=theorem]{corollary}
\declaretheorem[name=Remark,sibling=theorem]{remark}
\fi

\usepackage{enumitem}
\newlist{compressedList}{itemize}{10}
\setlist[compressedList]{label=--,topsep=0pt,parsep=0pt,partopsep=0pt}
\usepackage{mathtools}
\DeclarePairedDelimiter\ceil{\lceil}{\rceil}
\DeclarePairedDelimiter\floor{\lfloor}{\rfloor}

% https://tex.stackexchange.com/questions/134278/overriding-the-paragraph-style-in-the-lncs-document-class
\usepackage{etoolbox}
\patchcmd{\paragraph}{\itshape}{\bfseries\boldmath}{}{}


% This part is specific to llncs-based papers with nicer output
% (page number, theorem labelled with sections...). This should
% be the version in the arxiv.
% To enable this part of the configuration, add
%  \def\nicerThanLlncs{}
% before the \input{include}
\ifdefined\nicerThanLlncs%
  % change margins, page layout
  \usepackage[a4paper, top=1.4in, bottom=1.4in, right=1in, left=1in]{geometry}
  % enable page numbering, suppressed by default by llncs
  \pagestyle{plain}
  \setcounter{tocdepth}{2}
  \renewcommand\small{\fontsize{10pt}{12pt}\selectfont}
\fi
\ifdefined\nicerThanLlncsAbstract%
  % change margins, page layout
  \usepackage[a4paper, margin=2.5cm]{geometry}
  % enable page numbering, suppressed by default by llncs
  \pagestyle{plain}
  \setcounter{tocdepth}{2}
\fi
\ifdefined\enableLlncsCodeSlightlyImproved%
  \pagestyle{plain}
  \setcounter{tocdepth}{2}
\fi


\ifdefined\disableHyperref%
\else%
% https://rcweb.dartmouth.edu/doc/texmf-dist/doc/latex/cryptocode/cryptocode.pdf (page 55)
%%%%%%%%%% This part should be loaded last
% Usually hyperref needs to be at the end
\definecolor{secondaryColor}{RGB}{206,149,0} %% darker orange, looks like gold. <3
\usepackage[colorlinks, allcolors=secondaryColor]{hyperref}
\fi
%% Hyperref should be loaded before cryptocode! Otherwise, errors with labels of lines for example.
%%% Cryptocode
\usepackage [
n,
advantage,
operators,
sets,
adversary,
landau,
probability,
notions,
logic,
ff,
mm,
primitives,
events,
complexity,
asymptotics,
keys
] {cryptocode}
% I don't like the way "samples" is defined in cryptocode with a big dollar in front.
% \renewcommand\sample{\mathrel{\overset{\vbox to.7ex{\hbox{\fontsize{6}{0}\selectfont\vspace{.5ex} \$}}}{\leftarrow}}}
%% https://tex.stackexchange.com/a/584533/116348
\makeatletter
\renewcommand{\sample}{\mathrel{\mathpalette\sample@\relax}}
\newcommand{\sample@}[2]{%
  \ooalign{%
    \hspace{\stretch{2}}\raisebox{0.7\height}{$\m@th\demotestyle{#1}\mathdollar$}\hspace{\stretch{1}}\cr
    $\m@th#1\leftarrow$\cr
  }%
}
\newcommand{\demotestyle}[1]{%
  \ifx#1\displaystyle\scriptstyle\else
  \ifx#1\textstyle\scriptstyle\else
  \scriptscriptstyle\fi\fi
}
\makeatother
% Create a new command to create games using \game[linenumbering]{Name of game}{Game definition}
\createprocedureblock{game}{center,boxed}{}{}{}
% Create a new environment (following what I did in the thesis) to provide named games
\newcommand{\gameFont}[1]{\texttt{#1}}
\renewcommand{\Game}[1]{\ensuremath{\gameFont{Game#1}}} % By default \Game produces a nice G (symetric), but it's not very standard.

\makeatletter
\createprocedureblock{gameProc}{center,boxed}{}{}{}
\createprocedureblock{interactGame}{center,boxed}{}{}{}
%%% Nicer way to refer to games.
%%% See also https://tex.stackexchange.com/questions/623163 which provides a cleaner way to make it work with cleverref.
% Usage: \begin{namedGame}*[label][short title]{title}*[game options like skipfirstln,lnstart=1] content \end{namedGame}
% first * = disable floating. WARNING: do not put labels with non-letter chars.
% second * = do not include the game in the TOC.
\NewDocumentEnvironment{namedGame}{soomsO{}b}{%
  \IfBooleanTF{#1}{}{%
    \par% without par, the figure will be put after the line arriving after the picture if there is no paragraph.
    \setlength{\intextsep}{5.0pt plus 2.0pt minus 2.0pt}% reduce the space when image is inserted inline.
    \begin{figure}[htbp]}% 
    \begin{pcimage}%
      %% Add to toc see command listofgames
      \phantomsection% Create a fake label, useful to ensure the link point to this part.
      \IfBooleanTF{#5}{}{% If no star is given, add to "table of contents" for games
        \IfValueTF{#3}{% If a short title is provided
          \addcontentsline{game}{section}{\ensuremath{#3}}%
        }{%
          \addcontentsline{game}{section}{\ensuremath{#4}}%
        }%
      }%
      {\normalfont\gameProc[linenumbering,#6]{\ensuremath{#4}}{\IfValueTF{#2}{%
            %% Add an anchor if a label is present
            \raisebox{1em}{\hypertarget{#2}{}}%
            %% Create a macro "mygametitle@nameoflabel" to store the title
            \IfValueTF{#3}{% If a short title is provided
              \expandafter\gdef\csname mygametitle@#2 \endcsname{\ensuremath{#3}}%%
              \write\@auxout{\gdef\string\mygametitle@#2{\ensuremath{#3}}}%
            }{%
              \expandafter\gdef\csname mygametitle@#2 \endcsname{\ensuremath{#4}}%%
              \write\@auxout{\gdef\string\mygametitle@#2{\ensuremath{#4}}}%
            }%
          }{} #7 }}%
    \end{pcimage}
    \IfBooleanTF{#1}{}{\end{figure}}%
}{}

% Usage: \refGame{label}. Use \refGame*{label} if you don't want to color the link (useful in proofs)
\NewDocumentCommand{\refGame}{sm}{%
  \IfBooleanTF{#1}{%
    \hyperlink{#2}{\textcolor{black}{\csname mygametitle@#2\endcsname}}% Do not put a white space after #1!
  }{%
    \hyperlink{#2}{\csname mygametitle@#2\endcsname}% Do not put a white space after #1!
  }%
}
\makeatother

\ifdefined\disableHyperref%
\else%
% Well... not as much as cleveref
\usepackage[capitalise,noabbrev]{cleveref}
% Don't abbrev general names... but still abbrev equations
\crefname{equation}{Eq.}{Eqs.}
\crefname{figure}{Fig.}{Figs.}
\crefname{algorithm}{Protocol}{Protocols}
\Crefname{equation}{Equation}{Equations}
\Crefname{figure}{Figure}{Figures}
\Crefname{algorithm}{Protocol}{Protocols}
\fi

% Put that package at the end of the file,
% or you will have some strange errors like
% "Only one # is allowed per tab"
\usetikzlibrary{quantikz}
% Bug in llncs https://tex.stackexchange.com/a/372108/116348
\def\theHlemma{\theHtheorem}
\def\theHdefinition{\theHtheorem}
\def\theHcorollary{\theHtheorem}

%%%%%% PLEASE DO NOT WRITE ANYTHING AFTER THIS LINE %%%%%%
%%%%%% Indeed, hyperref, cleveref and quantikz need %%%%%%
%%%%%% to be loaded at the very end.                %%%%%%

\ifdefined\enableLlncsCode%
% go back to standard amsthm proof environments with Qed at the end
\let\proof\relax\let\endproof\relax
  \usepackage{amsthm}
  % To debug, it's a nightmare without page number... To disable
  % in the final version
  \pagestyle{plain}
\fi

% Define environments when loaded in standalone...
\ifdefined\enableNonLlncsCode%
\usepackage{thmtools}
\declaretheorem[name=Theorem,numberwithin=chapter]{theorem}
\declaretheorem[name=Definition,sibling=theorem]{definition}
\declaretheorem[name=Lemma,sibling=theorem]{lemma}
\declaretheorem[name=Corollary,sibling=theorem]{corollary}
\declaretheorem[name=Remark,sibling=theorem]{remark}
\fi

\usepackage{enumitem}
\newlist{compressedList}{itemize}{10}
\setlist[compressedList]{label=--,topsep=0pt,parsep=0pt,partopsep=0pt}
\usepackage{mathtools}
\DeclarePairedDelimiter\ceil{\lceil}{\rceil}
\DeclarePairedDelimiter\floor{\lfloor}{\rfloor}

% https://tex.stackexchange.com/questions/134278/overriding-the-paragraph-style-in-the-lncs-document-class
\usepackage{etoolbox}
\patchcmd{\paragraph}{\itshape}{\bfseries\boldmath}{}{}


% This part is specific to llncs-based papers with nicer output
% (page number, theorem labelled with sections...). This should
% be the version in the arxiv.
% To enable this part of the configuration, add
%  \def\nicerThanLlncs{}
% before the %% Let's put in these file all the definitions that we
%% require:

%% Setup language, encoding...
\RequirePackage{etex} 

\usepackage{standalone}
\usepackage{lmodern}
% Language
\usepackage[english]{babel}
% Input encoding
\usepackage[utf8]{inputenc}
% Output encoding https://tex.stackexchange.com/a/677. Important to copy accents
\usepackage[T1]{fontenc}
\usepackage{subfiles} % To quickly load this include file other files
\usepackage{microtype}
\usepackage[shortcuts,nospacearound]{extdash}    % Better endling of em dash (---) to cut sentences\===like 
%%% https://tex.stackexchange.com/questions/2773/how-do-i-make-latex-push-long-citations-to-a-new-line
\usepackage{breakcites}
\usepackage{booktabs}
% Makes sure floats don't fly accross sections
\usepackage{placeins}
\usepackage{adjustbox}
\usepackage{varwidth} % Like minipage, but with automatic width discovery.
\usepackage{complexity} % get \QMA{}...

\usepackage[normalem]{ulem}
%\usepackage[math-style=ISO]{unicode-math}
\usepackage{upgreek} % non italic greak letters

%%%% Bibliography
% I want to have separate references for appendix and main body
% https://tex.stackexchange.com/questions/98660/
\usepackage[style=trad-alpha,
  sortcites=true,
  doi=false,
  url=false,
  giveninits=true, % Bob Foo --> B. Foo
  isbn=false,
  url=false,
  eprint=false,
  sortcites=false, % \cite{B,A,C}: [A,B,C] --> [B,A,C]
]{biblatex}
\defbibfilter{appendixOnlyFilter}{
  segment=1 % Segment 1 will be chosen to be the one in appendix
  and not segment=0 % Default segment is 0
}
\renewcommand{\multicitedelim}{, } % [ABC96; DEF12] -> [ABC96, DEF12]
% [unknown_ref] => [??]
% https://tex.stackexchange.com/a/352573
\makeatletter
\protected\def\abx@missing#1{\textbf{??}}
\makeatother
\renewcommand*{\bibfont}{\normalfont\small} % BibLatex font seems bigger than Bibtex?

\usepackage{xcolor,cancel}
\usepackage{proof-at-the-end}
\NewDocumentEnvironment{theoremE}{O{}O{}+b}{%
  \begin{theoremEnd}[normal,#2]{theorem}[#1]%
    #3%
  \end{theoremEnd}
  %
}{}% Do not forget the second parameter or you might get Missing \begin{document} error
\NewDocumentEnvironment{lemmaE}{O{}O{}+b}{%
  \begin{theoremEnd}[normal,#2]{lemma}[#1]%
    #3%
  \end{theoremEnd}
  %
}{}% Do not forget the second parameter or you might get Missing \begin{document} error
\NewDocumentEnvironment{corollaryE}{O{}O{}+b}{%
  \begin{theoremEnd}[normal,#2]{corollary}[#1]%
    #3%
  \end{theoremEnd}
  %
}{}% Do not forget the second parameter or you might get Missing \begin{document} error
\NewDocumentEnvironment{proofE}{O{}+b}{%
  \begin{proofEnd}[#1]%
    #2
  \end{proofEnd}%
}{}%
\pgfkeys{/prAtEnd/global custom defaults/.style={
    text link={\noindent See \hyperref[proof:prAtEnd\pratendcountercurrent]{proof} in \cref{proofsection:prAtEnd\pratendcountercurrent}.}
  }
}

\ifdefined\displayArxivTexts%
  \pgfkeys{/prAtEnd/end if published/.style={no restate}}
  \pgfkeys{/prAtEnd/all end if published/.style={no restate}}
\else%
  \pgfkeys{/prAtEnd/end if published/.style={end}}
  \pgfkeys{/prAtEnd/all end if published/.style={all end}}
\fi

% https://tex.stackexchange.com/a/20613/116348
% https://tex.stackexchange.com/a/583244/116348
\usepackage{scalerel}
\newcommand\hcancel[2][black]{%
  \ThisStyle{%
    \setbox0=\hbox{$\SavedStyle #2$}%
    \rlap{\raisebox{.25\ht0}{\textcolor{#1}{\rule{\wd0}{1pt}}}}\SavedStyle #2}%
}

%% We provide 3 commands to add comments/text:
%% - \yournameAddText{your text to add}: proposition to add text
%% - \yournameRemoveText{text to remove}: proposition to remove text
%% - \yournameComment{your comments}: to add general comment to discuss. Should not contain proposition to add or remove text, but can be added before/after to explain the reason of the modification.
%% The interest it to provide some quick ways to see what is the final number of pages:
%% just add before the import a line:
%%   \def\removeCommentsAcceptPropositions{}
%% and the proposition will be "accepted" (i.e. the removed text is removed,
%% and the added text will be marked as black), and the comment will be removed, that way it is possible
%% to determine the final number of pages.
%% Example:
%% This text is normal\leoRemove{, this text should be removed}\leoAddText{, this text should be added}\leoComment{This is a comment, explaining why I made that changes.}.
\newcommand{\addEditor}[2]{%
  \expandafter\newcommand\csname #1AddText\endcsname[2][]{%
    \ifdefined\removeCommentsAcceptPropositions %
      ##2%
    \else%
      \textcolor{#2}{\sout{##1}}\textcolor{#2!60!black}{##2}%
    \fi%
  }%
  \expandafter\newcommand\csname #1Comment\endcsname[1]{%
    \ifdefined\removeCommentsAcceptPropositions %
    \else%
      \textcolor{#2!60!white}{\textbf{##1}}%
    \fi%
  }%
  \expandafter\newcommand\csname #1Remove\endcsname[1]{%
    \ifdefined\removeCommentsAcceptPropositions %
    \else%
      \textcolor{#2}{\sout{##1}}%
    \fi%
  }%
}
\definecolor{darkgreen}{rgb}{0., 0.7, 0}
\addEditor{leo}{darkgreen}
\addEditor{fred}{red}
\addEditor{elham}{blue}

%% We also provide an arxivOnly environment, in order to include text that are too long for the submitted version
%% but that would be good to have in the arxiv version.
% Use the \begin{arxivOnly} ... \end{arxivOnly} environment, by putting the opening and ending command
%% on there own line.
%% For shorter texts, you can use the inline version \arxivOnlyShort{your text}.
%% If you want to exclude text from the arxiv, use \begin{publishedOnly} ... \end{publishedOnly} and \publishedOnlyShort{your text}
%% "Comment" package is annoying, it does not allow block indentation, we use 'versions' instead
% \usepackage{versions} %% Do not work inside theoremE, don't know why
\ifdefined\displayArxivTexts%
  % \includeversion{arxivOnly}
  % \excludeversion{publishedOnly}
  \newcommand{\arxivOnly}[1]{#1}
  \newcommand{\arxivOnlyNoDiff}[1]{#1} % Like arxivOnly, but useful in diff mode (when it does not make sense to add colors, for instance inside a cite or when using arxivonly to create lists.
  \newcommand{\publishedOnly}[1]{}
  \newcommand{\publishedVsArxiv}[2]{#2}
  \newcommand{\inAppendixIfPublished}[1]{#1}
\else%
  \ifdefined\displayDiffTexts%
    \colorlet{arxivDiff}{green}
    \colorlet{publishedDiff}{red}
    \newcommand{\arxivOnly}[1]{{\color{arxivDiff}#1}}
    \newcommand{\arxivOnlyNoDiff}[1]{#1}
    \newcommand{\publishedOnly}[1]{{\color{publishedDiff}#1}}
    \newcommand{\publishedVsArxiv}[2]{{\color{publishedDiff}#1}{\color{arxivDiff}#2}}
    \newcommand{\inAppendixIfPublished}[1]{{\color{arxivDiff}\textbf{MOVED IN APPENDIX IN PUBLISHED}#1}\textEnd{{\color{publishedDiff}\textbf{MOVED IN MAIN BODY IN ARXIV}#1}}}
  \else
    % \excludeversion{arxivOnly}
    % \includeversion{publishedOnly}
    \newcommand{\arxivOnly}[1]{}
    \newcommand{\arxivOnlyNoDiff}[1]{}
    \newcommand{\publishedOnly}[1]{#1}
    \newcommand{\publishedVsArxiv}[2]{#1}
    \newcommand{\inAppendixIfPublished}[1]{\textEnd{#1}}
  \fi
\fi
% Alias, easier to remember  
\newcommand{\inAppendix}[1]{\textEnd{#1}}
  
% Symbol to display when an arxiv text is supposed to appear
\usepackage{todonotes}
\newcommand{\notifyArxiv}{\ifdefined\arxivDraftMode\todo{arXiv}\fi}

% Remove unwanted space before/after lists
\usepackage{enumitem}
\publishedOnly{\setlist[itemize]{noitemsep, topsep=0pt}}

% Usual packages:
\usepackage{amsmath}
\usepackage{amssymb}
% \usepackage{amsthm}
\usepackage{thmtools}
\usepackage{graphicx}
\graphicspath{{}{figures/}}
\usepackage{float}
\usepackage{subfig}
\usepackage{wrapfig}
\usepackage{array}
\usepackage{braket}
% Refer to footnotes:
% https://tex.stackexchange.com/a/167382/116348
\usepackage{footmisc}
% Useful to use the latest NewDocumentCommand:
%\usepackage{xparse}
%\usepackage[poster]{tcolorbox}

%% Space between pictures and text
%% https://mirrors.chevalier.io/CTAN/macros/latex/contrib/layouts/layman.pdf
%% \usepackage{layouts}
%% https://tex.stackexchange.com/questions/60477/remove-space-after-figure-and-before-text?rq=1
\setlength{\floatsep}{7pt plus 2pt minus 2pt}
\setlength{\textfloatsep}{7pt plus 2pt minus 2pt}
\setlength{\intextsep}{7pt plus 2pt minus 2pt}
% https://tex.stackexchange.com/questions/39017/how-to-influence-the-position-of-float-environments-like-figure-and-table-in-lat
\setcounter{topnumber}{8}
\setcounter{bottomnumber}{8}
\setcounter{totalnumber}{8}
%% https://tex.stackexchange.com/questions/45990/how-can-i-modify-vertical-space-between-figure-and-caption
%% plus/minus allows to stretch a bit around 15pt
% \setlength{\abovecaptionskip}{5pt}

% Remove number to subsubsection (in llncs it replaces \paragraph)
\setcounter{secnumdepth}{2}

%%% Some practical environment to define theorems, lemma...
% \declaretheorem[name=Theorem,numberwithin=section]{theorem}
% \declaretheorem[name=Definition,sibling=theorem]{definition}
% \declaretheorem[name=Lemma,sibling=theorem]{lemma}
% \declaretheorem[name=Corollary,sibling=theorem]{corollary}
% \declaretheorem[name=Remark,sibling=theorem]{remark}
%%% In LNCS we use
% https://tex.stackexchange.com/a/373125/116348
% \spnewtheorem{thm}{Theorem}[section]{\bfseries}{\itshape}

%%%
\usepackage[ruled,vlined]{algorithm2e}
\SetAlFnt{\normalsize}%
\setlength{\algomargin}{0em} % remove margin on left of algo
\SetAlgorithmName{Protocol}{protocol}{List of Protocols}
\newenvironment{protocol}[1][htb]{\begin{algorithm}[#1]}{\end{algorithm}}

%%% %%% Load algorithm package, and create a breakable version
%%% \usepackage{algorithm, algorithmicx, algpseudocode}
%%% % Decrease space between text:
%%% % https://tex.stackexchange.com/questions/25828/how-to-remove-change-the-vertical-spacing-before-and-after-an-algorithm-enviro/25832
%%% \setlength{\intextsep}{1\baselineskip}
%%% % https://tex.stackexchange.com/questions/387577/new-algorithm-environment-with-a-separate-counter
%%% \newcounter{protocol}
%%% \makeatletter
%%% \newenvironment{protocol}[1][htb]{%
%%%   \let\c@algorithm\c@protocol
%%%   \renewcommand{\ALG@name}{Protocol}% Update algorithm name
%%%   \begin{algorithm}[#1]%
%%%   }{\end{algorithm}
%%% }
%%% \makeatother
%%%
%%% \providecommand*\algorithmautorefname{Protocol}
%%% \makeatletter
%%% \newenvironment{breakablealgorithm}
%%%   {% \begin{breakablealgorithm}
%%%    % \begin{center}
%%%      \refstepcounter{algorithm}% New algorithm
%%%      \hrule height.8pt depth0pt \kern2pt% \@fs@pre for \@fs@ruled
%%%      \renewcommand{\caption}[2][\relax]{% Make a new \caption
%%%        % {\raggedright\textbf{\ALG@name~\thealgorithm} ##2\par}%
%%%        {\raggedright\textbf{Protocol~\thealgorithm} ##2\par}%
%%%        \ifx\relax##1\relax % #1 is \relax
%%%          \addcontentsline{loa}{algorithm}{\protect\numberline{\thealgorithm}##2}%
%%%        \else % #1 is not \relax
%%%          \addcontentsline{loa}{algorithm}{\protect\numberline{\thealgorithm}##1}%
%%%        \fi
%%%        \kern2pt\hrule\kern2pt
%%%        \small
%%%      }
%%%   }{% \end{breakablealgorithm}
%%%      \kern2pt\hrule\relax% \@fs@post for \@fs@ruled
%%%    % \end{center}
%%%   }
%%% \makeatother
%%%

\def\EQ#1{\begin{eqnarray}#1\end{eqnarray}} 


%%% Practical abreviations
\usepackage{calrsfs} % Replace \mathcal with different more "cursive" font
\DeclareMathAlphabet{\pazocal}{OMS}{zplm}{m}{n} % \pazocal is now the old \mathcal
\newcommand*{\cA}{\pazocal{A}}
\newcommand*{\cB}{\pazocal{B}}
\newcommand*{\cC}{\pazocal{C}}
\newcommand*{\cD}{\pazocal{D}}
\newcommand*{\cE}{\pazocal{E}}
\newcommand*{\cF}{\pazocal{F}}
\newcommand*{\cG}{\pazocal{G}}
\newcommand*{\cH}{\pazocal{H}}
\newcommand*{\cI}{\pazocal{I}}
\newcommand*{\cJ}{\pazocal{J}}
\newcommand*{\cK}{\pazocal{K}}
\renewcommand*{\cL}{\pazocal{L}} % defined in complexity
\newcommand*{\cM}{\pazocal{M}}
\newcommand*{\cN}{\pazocal{N}}
\newcommand*{\cO}{\pazocal{O}}
\renewcommand*{\cP}{\pazocal{P}} % defined in complexity
\newcommand*{\cQ}{\pazocal{Q}}
\newcommand*{\cR}{\pazocal{R}}
\renewcommand*{\cS}{\pazocal{S}} % defined in complexity
\newcommand*{\cT}{\pazocal{T}}
\newcommand*{\cU}{\pazocal{U}}
\newcommand*{\cV}{\pazocal{V}}
\newcommand*{\cW}{\pazocal{W}}
\newcommand*{\cX}{\pazocal{X}}
\newcommand*{\cY}{\pazocal{Y}}
\newcommand*{\cZ}{\pazocal{Z}}

% Notation for languages. Make sure not to collide with L(H) for linear operators on H
%\DeclareMathAlphabet{\pazocal}{OMS}{zplm}{m}{n}
\newcommand{\linearOp}{\mathcal{L}} % Defines the opearator L(H)
\renewcommand*{\lang}{\pazocal{L}} % Defines a new Turing/Quantum Language

%https://tex.stackexchange.com/questions/195025/
\newcommand*{\rightSend}{\rightsquigarrow} % snake "a --> b" to say "I send a to b".

% For some reasons, proof-at-the-end do not like # (problem with detokenize? To fix...). So we use a macro:
\newcommand*{\diese}{\#}

\newcommand*{\eps}[0]{\varepsilon}
\renewcommand*{\D}[0]{\mathbb{D}}
\renewcommand*{\E}{\mathbb{E}}
\newcommand*{\N}[0]{\mathbb{N}}
\renewcommand*{\R}[0]{\mathbb{R}}
\renewcommand*{\C}[0]{\mathbb{C}}
\newcommand*{\Z}[0]{\mathbb{Z}}
\newcommand*{\T}[0]{\mathbb{T}}

\newcommand*{\CPA}[0]{\textsf{CPA}}
\newcommand*{\LWE}[0]{\textsf{LWE}}
\newcommand*{\GapSVP}[0]{\textsf{GapSVP}}
\newcommand*{\SIVP}[0]{\textsf{SIVP}}

% first param is alpha, second is q (optional)
\newcommand{\contGauss}[1]{\cD_{#1}}
\newcommand{\disGauss}[2]{\cD_{\Z,#1 #2}}
\newcommand{\disGaussCont}[1]{\cD_{\Z,#1}}
\newcommand{\disGaussAQ}{\cD_{\Z,\alpha q}}
\newcommand{\rsafe}{r_{\text{safe}}}
% domain of the function.
\newcommand{\ccX}{\cX}
\newcommand{\ccXSphere}{\cX_\bullet}
\newcommand{\ccXCube}[1][]{\cX_{\scalebox{.4}{$\blacksquare$}#1}}

\newcommand{\BBHeightyFor}{\textsc{BB84}}

% "Correct" spacing around cdot as wildcard
% https://tex.stackexchange.com/questions/78165/spacing-around-cdot-when-used-as-a-wildcard
\newcommand\cdotwild{{\mkern 2mu\cdot\mkern 2mu}}

\newcommand*{\id}{\mathrm{id}}
\DeclareMathOperator{\Tr}{Tr}
\DeclareMathOperator{\supp}{supp}
\DeclareMathOperator{\Det}{Det}
\DeclareMathOperator{\ERR}{ERR}
\DeclareMathOperator{\Round}{Round}

\newcommand{\Ver}{ {\tt Ver} } %% Verify public key
\newcommand{\Gen}{ {\tt Gen} }
\newcommand{\GenLocal}{\ensuremath{ {\tt Gen}_{\tt Loc}} }
\newcommand{\Enc}{ {\tt Enc} }
\newcommand{\Dec}{ {\tt Invert} }
\newcommand{\Eval}{ {\tt Eval} }

\newcommand{\HGen}{ {\tt H.Gen} }
\newcommand{\HEnc}{ {\tt H.Enc} }
\newcommand{\HDec}{ {\tt H.Invert} }
\newcommand{\HEval}{ {\tt H.Eval} }
\newcommand{\Hh}{\ensuremath{{\tt H.}h} }
\newcommand{\HCheckHonestTrapdoor}{\ensuremath{{\tt H.CheckTrapdoor}}}

\newcommand{\MPGen}{ {\tt MP.Gen} }
\newcommand{\MPDec}{ {\tt MP.Invert} }
\newcommand{\InvertSmallGadget}{ \ensuremath{{\texttt{InvertSmallGadget}}} }
\newcommand{\InvertGadget}{ {\tt InvertGadget} }
% \newcommand{\MPEval}{ $g$ }


\DeclareMathOperator{\RoundMod}{RoundMod}


% https://tex.stackexchange.com/a/253746/116348
\usepackage{mleftright}
\newcommand{\BlockMatrix}[2]{%
  \mleft[%
  \begin{array}{@{}#1@{}}%
    #2
  \end{array}%
  \mright]%
}
\newcommand{\SmallBlockMatrix}[2]{
  \mleft[\begin{array}{c}
    #1\tabularnewline % https://tex.stackexchange.com/questions/328745/matrices-with-cryptocode-package
    \hline
    #2
  \end{array}\mright]
}
\newcommand{\HorizBlockMatrix}[2]{
  \mleft[\begin{array}{@{}c|c@{}}
    #1 & #2
  \end{array}\mright]
}

\newcommand*{\vect}[1]{\ensuremath{\mathbf{#1}}}


%% Already defined in *recent* cryptocode.
% \DeclareMathOperator*{\argmax}{arg \, max}
% \DeclareMathOperator*{\argmin}{arg \, min}

\usepackage{xspace}
\xspaceaddexceptions{]\}}
\newcommand{\RSP}{\ensuremath{\sf{RSP}}\xspace}%

\newcommand*{\RSPenc}{\mathrm{RSP}_{\mathrm{enc}}}
\newcommand*{\filter}{\vdash}
\newcommand*{\channelBB}{\cS_{\Z\frac{\pi}{2}}}
% To denote a protocol between two parties
% \newcommand*{\protocol}{\boldsymbol{\pi}}
\newcommand\compRSP{composable $\sf{RSP}_{CC}$}
\newcommand\RSPlike{$\sf{RSP}_{CC}$}
\newcommand\compUBQC{composable $\sf{UBQC}_{CC}$}
\newcommand\UBQClike{$\sf{UBQC}_{CC}$}
% GHZ
\newcommand*{\eqdef}{\coloneqq}
\newcommand*{\PERM}{\ensuremath{}{\sf PERM}}
\newcommand*{\PPT}{\ensuremath{}{\sf PPT}}
\newcommand*{\QPT}{\ensuremath{}{\sf QPT}}
\newcommand*{\Auth}{\ensuremath{}{\sf Auth}}
% Prover, verifier, extractor
\renewcommand*{\P}{\ensuremath{}{\sf P}}
\newcommand*{\V}{\ensuremath{}{\sf V}}
\renewcommand*{\K}{\ensuremath{}{\sf K}}


% Here are the different protocols we use
% Initial protocol
\newcommand{\blind}{\ensuremath{ {\tt BLIND} }}
\newcommand{\blindZK}{\ensuremath{ {\tt BLIND-ZK} }}
% Reveals to people if they are part of the support
\newcommand{\blindSup}{\ensuremath{ {\tt BLIND}^{\tt sup} }}
% Make sure that people in the support share a canonical GHZ
\newcommand{\blindCan}{\ensuremath{ {\tt BLIND}_{\tt can} }}
% Make sure that people in the support share a canonical GHZ and that they know they are supported
\newcommand{\blindCanSup}{\ensuremath{ {\tt BLIND}^{\tt sup}_{\tt can} }}
\newcommand{\authBlindCanDist}{\ensuremath{ {\tt AUTH-BLIND}^{\tt dist}_{\tt can} }}
\newcommand{\verifCanSup}{\ensuremath{ {\tt VERIF}^{\tt sup}_{\tt can} }}
\newcommand{\blindPoly}{\ensuremath{ {\tt BLIND-poly} }}
\newcommand{\INDFCT}{\ensuremath{ {\tt IND-D0} }}
\newcommand{\INDBLIND}{\refGame*{INDBLIND}}
\newcommand{\INDBLINDtxt}{\ensuremath{ {\tt IND-\blind{}} }}
\newcommand{\INDBLINDSUP}{\refGame*{INDBLINDSUP}}
\newcommand{\INDBLINDSUPtxt}{\ensuremath{ {\tt IND-\blindSup{}} }}
\newcommand{\INDBLINDCANSUP}{\ensuremath{ {\tt IND-\blindCanSup{}} }}
\newcommand{\INDBLINDCANAUTH}{\ensuremath{ {\tt IND-\authBlindCanDist{}} }}
\newcommand{\INDPARTIAL}{\ensuremath{ {\tt IND-PARTIAL} }}
\newcommand*\GHZ{\ensuremath{}{\sf GHZ}}
\newcommand*\GHZCan{\ensuremath{{\sf GHZ^{\sf can}}}}
% Generalized GGHZ
\newcommand*\GGHZ{\ensuremath{ {\sf GHZ^G} }}
\newcommand*\HGHZ{\ensuremath{ {\sf GHZ^H} }}
\newcommand{\AssumpFctNoDelta}{\HGHZ{} capable}
\newcommand{\AssumpFctCanNoDelta}{\GHZCan{} capable}
\newcommand{\AssumpFct}{$\delta$-\HGHZ{} capable}
\newcommand{\AssumpFctNegl}{$\negl[\lambda]$-\HGHZ{} capable}
\newcommand{\AssumpFctCan}{$\delta$-\GHZCan{} capable}
\newcommand{\AssumpFctCanPrime}{$\delta^\prime$-\GHZCan{} capable}
\newcommand{\AssumpFctDist}{distributable}
\newcommand{\partialInfo}{\ensuremath{{\tt PartInfo}}}
\newcommand{\partialInfoLocal}{\ensuremath{{\tt PartInfo_{\tt Loc}}}}
\newcommand{\partialAlpha}{\ensuremath{{\tt PartAlpha_{\tt Loc}}}}
\newcommand{\CombineAlpha}{\ensuremath{{\tt CombineAlpha}}}
\newcommand{\CombineRandom}{\ensuremath{{\tt CombineRandom}}}
\newcommand{\CheckHonestTrapdoor}{\ensuremath{{\tt CheckTrapdoor}}}
\newcommand{\ZKGenCheck}{\ensuremath{{\tt ZK-GenCheck}}}

\newcommand{\highlightChanges}[1]{\textcolor{green!50!black}{#1}}

\newcommand{\Real}{\ensuremath{\mathsf{REAL}}}
\newcommand{\Sim}{\ensuremath{\mathsf{Sim}}}
\newcommand{\Ideal}{\ensuremath{\mathsf{IDEAL}}}
\newcommand{\OUT}{\ensuremath{\mathsf{OUT}}}

\usepackage{pifont}% http://ctan.org/pkg/pifont
% To denote that a user is not in the final GHZ
\newcommand{\cross}{\ensuremath{\text{\ding{55}}}}

\usepackage{verbatim}
\newcommand{\leftI}{{\normalfont\ttfamily left}}
\newcommand{\rightI}{{\normalfont\ttfamily right}}
\newcommand*{\mybar}[1]{\overline{#1}}

\newcommand*{\quout}{\mathtt{out}}

\DeclareMathOperator{\Image}{Image}
\DeclareMathOperator{\Ima}{Im}
\DeclareMathOperator{\Dom}{Dom}

\newcommand*{\Pub}{\mathtt{Pub}}
\newcommand*{\accept}{\mathtt{accept}}
\newcommand*{\abort}{\mathtt{abort}}
\newcommand*{\gadxor}{\mathtt{Gad}_{\oplus}}
\newcommand*{\quin}{\mathtt{in}}

%%% Provide a "subproof" environment (pretty simple for now,
%%% just indent the sub-proofs)
% I'd love to have left border... https://tex.stackexchange.com/questions/532948/robustly-add-a-border-to-the-left-of-a-text-spanning-several-pages
\usepackage{changepage}% http://ctan.org/pkg/changepage
\newenvironment{subproof}{\begin{adjustwidth}{2em}{0pt}}{\end{adjustwidth}}


%%% To have nice probabilities
% https://tex.stackexchange.com/questions/198771/align-in-substack
\makeatletter
\newcommand{\subalign}[1]{%
  \vcenter{%
    \Let@ \restore@math@cr \default@tag
    \baselineskip\fontdimen10 \scriptfont\tw@
    \advance\baselineskip\fontdimen12 \scriptfont\tw@
    \lineskip\thr@@\fontdimen8 \scriptfont\thr@@
    \lineskiplimit\lineskip
    \ialign{\hfil$\m@th\scriptstyle##$&$\m@th\scriptstyle{}##$\hfil\crcr
      #1\crcr
    }%
  }%
}
\makeatother
\usepackage{xparse}
% https://tex.stackexchange.com/questions/431794/same-spacing-around-middle-as-around-mid
\let\originalmiddle=\middle
\def\middle#1{\mathrel{}\originalmiddle#1\mathrel{}}
\NewDocumentCommand{\pr}{som}{\Pr\IfValueT{#2}{_{\substack{#2}}}\left[\,#3\,\right]}
\NewDocumentCommand{\esp}{som}{\IfValueTF{#2}{\underset{\subalign{#2}}{\mathbb{E}}}{\mathbb{E}}[\,#3\,]}
\NewDocumentEnvironment{pral}{soO{}}
 {%
  \Pr\IfValueT{#2}{_{\IfBooleanTF{#1}{\substack{#2}}{#2}}}
\begin{alignedat}[t]{2}
  [\, } {\,]
 #3
 \end{alignedat}
 }
\NewDocumentEnvironment{espal}{soO{}}
 {%
   \IfValueTF{#2}{\underset{\subalign{#2}}{\mathbb{E}}}{\mathbb{E}}
   \begin{alignedat}[t]{2}[\,
 }
 {\,]#3\end{alignedat}
 }
% To draw |A><B|, the second argument is facultative and gives |A><A|
\NewDocumentCommand{\ketbra}{mg}{
  | #1 \rangle \langle \IfValueTF{#2}{#2}{#1} |
}
% https://tex.stackexchange.com/a/188364/116348
\DeclareRobustCommand{\minwidthbox}[2]{%
  \ifmmode
    \expandafter\mathmakebox
  \else
    \expandafter\makebox
  \fi
  [\ifdim#2<\width\width\else#2\fi]{#1}%f
}

\NewDocumentCommand{\constructs}{mg}{
  \xrightarrow[\IfValueT{#2}{#2}]{\minwidthbox{#1}{5mm}}
}

%%%%%%%%%% This part is specific to llncs-based papers
% To enable this part of the configuration, add
%  \def\enableLlncsCode{}
% before the \input{include}
\ifdefined\enableLlncsCode%
% go back to standard amsthm proof environments with Qed at the end
\let\proof\relax\let\endproof\relax
  \usepackage{amsthm}
  % To debug, it's a nightmare without page number... To disable
  % in the final version
  \pagestyle{plain}
\fi

% Define environments when loaded in standalone...
\ifdefined\enableNonLlncsCode%
\usepackage{thmtools}
\declaretheorem[name=Theorem,numberwithin=chapter]{theorem}
\declaretheorem[name=Definition,sibling=theorem]{definition}
\declaretheorem[name=Lemma,sibling=theorem]{lemma}
\declaretheorem[name=Corollary,sibling=theorem]{corollary}
\declaretheorem[name=Remark,sibling=theorem]{remark}
\fi

\usepackage{enumitem}
\newlist{compressedList}{itemize}{10}
\setlist[compressedList]{label=--,topsep=0pt,parsep=0pt,partopsep=0pt}
\usepackage{mathtools}
\DeclarePairedDelimiter\ceil{\lceil}{\rceil}
\DeclarePairedDelimiter\floor{\lfloor}{\rfloor}

% https://tex.stackexchange.com/questions/134278/overriding-the-paragraph-style-in-the-lncs-document-class
\usepackage{etoolbox}
\patchcmd{\paragraph}{\itshape}{\bfseries\boldmath}{}{}


% This part is specific to llncs-based papers with nicer output
% (page number, theorem labelled with sections...). This should
% be the version in the arxiv.
% To enable this part of the configuration, add
%  \def\nicerThanLlncs{}
% before the \input{include}
\ifdefined\nicerThanLlncs%
  % change margins, page layout
  \usepackage[a4paper, top=1.4in, bottom=1.4in, right=1in, left=1in]{geometry}
  % enable page numbering, suppressed by default by llncs
  \pagestyle{plain}
  \setcounter{tocdepth}{2}
  \renewcommand\small{\fontsize{10pt}{12pt}\selectfont}
\fi
\ifdefined\nicerThanLlncsAbstract%
  % change margins, page layout
  \usepackage[a4paper, margin=2.5cm]{geometry}
  % enable page numbering, suppressed by default by llncs
  \pagestyle{plain}
  \setcounter{tocdepth}{2}
\fi
\ifdefined\enableLlncsCodeSlightlyImproved%
  \pagestyle{plain}
  \setcounter{tocdepth}{2}
\fi


\ifdefined\disableHyperref%
\else%
% https://rcweb.dartmouth.edu/doc/texmf-dist/doc/latex/cryptocode/cryptocode.pdf (page 55)
%%%%%%%%%% This part should be loaded last
% Usually hyperref needs to be at the end
\definecolor{secondaryColor}{RGB}{206,149,0} %% darker orange, looks like gold. <3
\usepackage[colorlinks, allcolors=secondaryColor]{hyperref}
\fi
%% Hyperref should be loaded before cryptocode! Otherwise, errors with labels of lines for example.
%%% Cryptocode
\usepackage [
n,
advantage,
operators,
sets,
adversary,
landau,
probability,
notions,
logic,
ff,
mm,
primitives,
events,
complexity,
asymptotics,
keys
] {cryptocode}
% I don't like the way "samples" is defined in cryptocode with a big dollar in front.
% \renewcommand\sample{\mathrel{\overset{\vbox to.7ex{\hbox{\fontsize{6}{0}\selectfont\vspace{.5ex} \$}}}{\leftarrow}}}
%% https://tex.stackexchange.com/a/584533/116348
\makeatletter
\renewcommand{\sample}{\mathrel{\mathpalette\sample@\relax}}
\newcommand{\sample@}[2]{%
  \ooalign{%
    \hspace{\stretch{2}}\raisebox{0.7\height}{$\m@th\demotestyle{#1}\mathdollar$}\hspace{\stretch{1}}\cr
    $\m@th#1\leftarrow$\cr
  }%
}
\newcommand{\demotestyle}[1]{%
  \ifx#1\displaystyle\scriptstyle\else
  \ifx#1\textstyle\scriptstyle\else
  \scriptscriptstyle\fi\fi
}
\makeatother
% Create a new command to create games using \game[linenumbering]{Name of game}{Game definition}
\createprocedureblock{game}{center,boxed}{}{}{}
% Create a new environment (following what I did in the thesis) to provide named games
\newcommand{\gameFont}[1]{\texttt{#1}}
\renewcommand{\Game}[1]{\ensuremath{\gameFont{Game#1}}} % By default \Game produces a nice G (symetric), but it's not very standard.

\makeatletter
\createprocedureblock{gameProc}{center,boxed}{}{}{}
\createprocedureblock{interactGame}{center,boxed}{}{}{}
%%% Nicer way to refer to games.
%%% See also https://tex.stackexchange.com/questions/623163 which provides a cleaner way to make it work with cleverref.
% Usage: \begin{namedGame}*[label][short title]{title}*[game options like skipfirstln,lnstart=1] content \end{namedGame}
% first * = disable floating. WARNING: do not put labels with non-letter chars.
% second * = do not include the game in the TOC.
\NewDocumentEnvironment{namedGame}{soomsO{}b}{%
  \IfBooleanTF{#1}{}{%
    \par% without par, the figure will be put after the line arriving after the picture if there is no paragraph.
    \setlength{\intextsep}{5.0pt plus 2.0pt minus 2.0pt}% reduce the space when image is inserted inline.
    \begin{figure}[htbp]}% 
    \begin{pcimage}%
      %% Add to toc see command listofgames
      \phantomsection% Create a fake label, useful to ensure the link point to this part.
      \IfBooleanTF{#5}{}{% If no star is given, add to "table of contents" for games
        \IfValueTF{#3}{% If a short title is provided
          \addcontentsline{game}{section}{\ensuremath{#3}}%
        }{%
          \addcontentsline{game}{section}{\ensuremath{#4}}%
        }%
      }%
      {\normalfont\gameProc[linenumbering,#6]{\ensuremath{#4}}{\IfValueTF{#2}{%
            %% Add an anchor if a label is present
            \raisebox{1em}{\hypertarget{#2}{}}%
            %% Create a macro "mygametitle@nameoflabel" to store the title
            \IfValueTF{#3}{% If a short title is provided
              \expandafter\gdef\csname mygametitle@#2 \endcsname{\ensuremath{#3}}%%
              \write\@auxout{\gdef\string\mygametitle@#2{\ensuremath{#3}}}%
            }{%
              \expandafter\gdef\csname mygametitle@#2 \endcsname{\ensuremath{#4}}%%
              \write\@auxout{\gdef\string\mygametitle@#2{\ensuremath{#4}}}%
            }%
          }{} #7 }}%
    \end{pcimage}
    \IfBooleanTF{#1}{}{\end{figure}}%
}{}

% Usage: \refGame{label}. Use \refGame*{label} if you don't want to color the link (useful in proofs)
\NewDocumentCommand{\refGame}{sm}{%
  \IfBooleanTF{#1}{%
    \hyperlink{#2}{\textcolor{black}{\csname mygametitle@#2\endcsname}}% Do not put a white space after #1!
  }{%
    \hyperlink{#2}{\csname mygametitle@#2\endcsname}% Do not put a white space after #1!
  }%
}
\makeatother

\ifdefined\disableHyperref%
\else%
% Well... not as much as cleveref
\usepackage[capitalise,noabbrev]{cleveref}
% Don't abbrev general names... but still abbrev equations
\crefname{equation}{Eq.}{Eqs.}
\crefname{figure}{Fig.}{Figs.}
\crefname{algorithm}{Protocol}{Protocols}
\Crefname{equation}{Equation}{Equations}
\Crefname{figure}{Figure}{Figures}
\Crefname{algorithm}{Protocol}{Protocols}
\fi

% Put that package at the end of the file,
% or you will have some strange errors like
% "Only one # is allowed per tab"
\usetikzlibrary{quantikz}
% Bug in llncs https://tex.stackexchange.com/a/372108/116348
\def\theHlemma{\theHtheorem}
\def\theHdefinition{\theHtheorem}
\def\theHcorollary{\theHtheorem}

%%%%%% PLEASE DO NOT WRITE ANYTHING AFTER THIS LINE %%%%%%
%%%%%% Indeed, hyperref, cleveref and quantikz need %%%%%%
%%%%%% to be loaded at the very end.                %%%%%%

\ifdefined\nicerThanLlncs%
  % change margins, page layout
  \usepackage[a4paper, top=1.4in, bottom=1.4in, right=1in, left=1in]{geometry}
  % enable page numbering, suppressed by default by llncs
  \pagestyle{plain}
  \setcounter{tocdepth}{2}
  \renewcommand\small{\fontsize{10pt}{12pt}\selectfont}
\fi
\ifdefined\nicerThanLlncsAbstract%
  % change margins, page layout
  \usepackage[a4paper, margin=2.5cm]{geometry}
  % enable page numbering, suppressed by default by llncs
  \pagestyle{plain}
  \setcounter{tocdepth}{2}
\fi
\ifdefined\enableLlncsCodeSlightlyImproved%
  \pagestyle{plain}
  \setcounter{tocdepth}{2}
\fi


\ifdefined\disableHyperref%
\else%
% https://rcweb.dartmouth.edu/doc/texmf-dist/doc/latex/cryptocode/cryptocode.pdf (page 55)
%%%%%%%%%% This part should be loaded last
% Usually hyperref needs to be at the end
\definecolor{secondaryColor}{RGB}{206,149,0} %% darker orange, looks like gold. <3
\usepackage[colorlinks, allcolors=secondaryColor]{hyperref}
\fi
%% Hyperref should be loaded before cryptocode! Otherwise, errors with labels of lines for example.
%%% Cryptocode
\usepackage [
n,
advantage,
operators,
sets,
adversary,
landau,
probability,
notions,
logic,
ff,
mm,
primitives,
events,
complexity,
asymptotics,
keys
] {cryptocode}
% I don't like the way "samples" is defined in cryptocode with a big dollar in front.
% \renewcommand\sample{\mathrel{\overset{\vbox to.7ex{\hbox{\fontsize{6}{0}\selectfont\vspace{.5ex} \$}}}{\leftarrow}}}
%% https://tex.stackexchange.com/a/584533/116348
\makeatletter
\renewcommand{\sample}{\mathrel{\mathpalette\sample@\relax}}
\newcommand{\sample@}[2]{%
  \ooalign{%
    \hspace{\stretch{2}}\raisebox{0.7\height}{$\m@th\demotestyle{#1}\mathdollar$}\hspace{\stretch{1}}\cr
    $\m@th#1\leftarrow$\cr
  }%
}
\newcommand{\demotestyle}[1]{%
  \ifx#1\displaystyle\scriptstyle\else
  \ifx#1\textstyle\scriptstyle\else
  \scriptscriptstyle\fi\fi
}
\makeatother
% Create a new command to create games using \game[linenumbering]{Name of game}{Game definition}
\createprocedureblock{game}{center,boxed}{}{}{}
% Create a new environment (following what I did in the thesis) to provide named games
\newcommand{\gameFont}[1]{\texttt{#1}}
\renewcommand{\Game}[1]{\ensuremath{\gameFont{Game#1}}} % By default \Game produces a nice G (symetric), but it's not very standard.

\makeatletter
\createprocedureblock{gameProc}{center,boxed}{}{}{}
\createprocedureblock{interactGame}{center,boxed}{}{}{}
%%% Nicer way to refer to games.
%%% See also https://tex.stackexchange.com/questions/623163 which provides a cleaner way to make it work with cleverref.
% Usage: \begin{namedGame}*[label][short title]{title}*[game options like skipfirstln,lnstart=1] content \end{namedGame}
% first * = disable floating. WARNING: do not put labels with non-letter chars.
% second * = do not include the game in the TOC.
\NewDocumentEnvironment{namedGame}{soomsO{}b}{%
  \IfBooleanTF{#1}{}{%
    \par% without par, the figure will be put after the line arriving after the picture if there is no paragraph.
    \setlength{\intextsep}{5.0pt plus 2.0pt minus 2.0pt}% reduce the space when image is inserted inline.
    \begin{figure}[htbp]}% 
    \begin{pcimage}%
      %% Add to toc see command listofgames
      \phantomsection% Create a fake label, useful to ensure the link point to this part.
      \IfBooleanTF{#5}{}{% If no star is given, add to "table of contents" for games
        \IfValueTF{#3}{% If a short title is provided
          \addcontentsline{game}{section}{\ensuremath{#3}}%
        }{%
          \addcontentsline{game}{section}{\ensuremath{#4}}%
        }%
      }%
      {\normalfont\gameProc[linenumbering,#6]{\ensuremath{#4}}{\IfValueTF{#2}{%
            %% Add an anchor if a label is present
            \raisebox{1em}{\hypertarget{#2}{}}%
            %% Create a macro "mygametitle@nameoflabel" to store the title
            \IfValueTF{#3}{% If a short title is provided
              \expandafter\gdef\csname mygametitle@#2 \endcsname{\ensuremath{#3}}%%
              \write\@auxout{\gdef\string\mygametitle@#2{\ensuremath{#3}}}%
            }{%
              \expandafter\gdef\csname mygametitle@#2 \endcsname{\ensuremath{#4}}%%
              \write\@auxout{\gdef\string\mygametitle@#2{\ensuremath{#4}}}%
            }%
          }{} #7 }}%
    \end{pcimage}
    \IfBooleanTF{#1}{}{\end{figure}}%
}{}

% Usage: \refGame{label}. Use \refGame*{label} if you don't want to color the link (useful in proofs)
\NewDocumentCommand{\refGame}{sm}{%
  \IfBooleanTF{#1}{%
    \hyperlink{#2}{\textcolor{black}{\csname mygametitle@#2\endcsname}}% Do not put a white space after #1!
  }{%
    \hyperlink{#2}{\csname mygametitle@#2\endcsname}% Do not put a white space after #1!
  }%
}
\makeatother

\ifdefined\disableHyperref%
\else%
% Well... not as much as cleveref
\usepackage[capitalise,noabbrev]{cleveref}
% Don't abbrev general names... but still abbrev equations
\crefname{equation}{Eq.}{Eqs.}
\crefname{figure}{Fig.}{Figs.}
\crefname{algorithm}{Protocol}{Protocols}
\Crefname{equation}{Equation}{Equations}
\Crefname{figure}{Figure}{Figures}
\Crefname{algorithm}{Protocol}{Protocols}
\fi

% Put that package at the end of the file,
% or you will have some strange errors like
% "Only one # is allowed per tab"
\usetikzlibrary{quantikz}
% Bug in llncs https://tex.stackexchange.com/a/372108/116348
\def\theHlemma{\theHtheorem}
\def\theHdefinition{\theHtheorem}
\def\theHcorollary{\theHtheorem}

%%%%%% PLEASE DO NOT WRITE ANYTHING AFTER THIS LINE %%%%%%
%%%%%% Indeed, hyperref, cleveref and quantikz need %%%%%%
%%%%%% to be loaded at the very end.                %%%%%%

\ifdefined\nicerThanLlncs%
  % change margins, page layout
  \usepackage[a4paper, top=1.4in, bottom=1.4in, right=1in, left=1in]{geometry}
  % enable page numbering, suppressed by default by llncs
  \pagestyle{plain}
  \setcounter{tocdepth}{2}
  \renewcommand\small{\fontsize{10pt}{12pt}\selectfont}
\fi
\ifdefined\nicerThanLlncsAbstract%
  % change margins, page layout
  \usepackage[a4paper, margin=2.5cm]{geometry}
  % enable page numbering, suppressed by default by llncs
  \pagestyle{plain}
  \setcounter{tocdepth}{2}
\fi
\ifdefined\enableLlncsCodeSlightlyImproved%
  \pagestyle{plain}
  \setcounter{tocdepth}{2}
\fi


\ifdefined\disableHyperref%
\else%
% https://rcweb.dartmouth.edu/doc/texmf-dist/doc/latex/cryptocode/cryptocode.pdf (page 55)
%%%%%%%%%% This part should be loaded last
% Usually hyperref needs to be at the end
\definecolor{secondaryColor}{RGB}{206,149,0} %% darker orange, looks like gold. <3
\usepackage[colorlinks, allcolors=secondaryColor]{hyperref}
\fi
%% Hyperref should be loaded before cryptocode! Otherwise, errors with labels of lines for example.
%%% Cryptocode
\usepackage [
n,
advantage,
operators,
sets,
adversary,
landau,
probability,
notions,
logic,
ff,
mm,
primitives,
events,
complexity,
asymptotics,
keys
] {cryptocode}
% I don't like the way "samples" is defined in cryptocode with a big dollar in front.
% \renewcommand\sample{\mathrel{\overset{\vbox to.7ex{\hbox{\fontsize{6}{0}\selectfont\vspace{.5ex} \$}}}{\leftarrow}}}
%% https://tex.stackexchange.com/a/584533/116348
\makeatletter
\renewcommand{\sample}{\mathrel{\mathpalette\sample@\relax}}
\newcommand{\sample@}[2]{%
  \ooalign{%
    \hspace{\stretch{2}}\raisebox{0.7\height}{$\m@th\demotestyle{#1}\mathdollar$}\hspace{\stretch{1}}\cr
    $\m@th#1\leftarrow$\cr
  }%
}
\newcommand{\demotestyle}[1]{%
  \ifx#1\displaystyle\scriptstyle\else
  \ifx#1\textstyle\scriptstyle\else
  \scriptscriptstyle\fi\fi
}
\makeatother
% Create a new command to create games using \game[linenumbering]{Name of game}{Game definition}
\createprocedureblock{game}{center,boxed}{}{}{}
% Create a new environment (following what I did in the thesis) to provide named games
\newcommand{\gameFont}[1]{\texttt{#1}}
\renewcommand{\Game}[1]{\ensuremath{\gameFont{Game#1}}} % By default \Game produces a nice G (symetric), but it's not very standard.

\makeatletter
\createprocedureblock{gameProc}{center,boxed}{}{}{}
\createprocedureblock{interactGame}{center,boxed}{}{}{}
%%% Nicer way to refer to games.
%%% See also https://tex.stackexchange.com/questions/623163 which provides a cleaner way to make it work with cleverref.
% Usage: \begin{namedGame}*[label][short title]{title}*[game options like skipfirstln,lnstart=1] content \end{namedGame}
% first * = disable floating. WARNING: do not put labels with non-letter chars.
% second * = do not include the game in the TOC.
\NewDocumentEnvironment{namedGame}{soomsO{}b}{%
  \IfBooleanTF{#1}{}{%
    \par% without par, the figure will be put after the line arriving after the picture if there is no paragraph.
    \setlength{\intextsep}{5.0pt plus 2.0pt minus 2.0pt}% reduce the space when image is inserted inline.
    \begin{figure}[htbp]}% 
    \begin{pcimage}%
      %% Add to toc see command listofgames
      \phantomsection% Create a fake label, useful to ensure the link point to this part.
      \IfBooleanTF{#5}{}{% If no star is given, add to "table of contents" for games
        \IfValueTF{#3}{% If a short title is provided
          \addcontentsline{game}{section}{\ensuremath{#3}}%
        }{%
          \addcontentsline{game}{section}{\ensuremath{#4}}%
        }%
      }%
      {\normalfont\gameProc[linenumbering,#6]{\ensuremath{#4}}{\IfValueTF{#2}{%
            %% Add an anchor if a label is present
            \raisebox{1em}{\hypertarget{#2}{}}%
            %% Create a macro "mygametitle@nameoflabel" to store the title
            \IfValueTF{#3}{% If a short title is provided
              \expandafter\gdef\csname mygametitle@#2 \endcsname{\ensuremath{#3}}%%
              \write\@auxout{\gdef\string\mygametitle@#2{\ensuremath{#3}}}%
            }{%
              \expandafter\gdef\csname mygametitle@#2 \endcsname{\ensuremath{#4}}%%
              \write\@auxout{\gdef\string\mygametitle@#2{\ensuremath{#4}}}%
            }%
          }{} #7 }}%
    \end{pcimage}
    \IfBooleanTF{#1}{}{\end{figure}}%
}{}

% Usage: \refGame{label}. Use \refGame*{label} if you don't want to color the link (useful in proofs)
\NewDocumentCommand{\refGame}{sm}{%
  \IfBooleanTF{#1}{%
    \hyperlink{#2}{\textcolor{black}{\csname mygametitle@#2\endcsname}}% Do not put a white space after #1!
  }{%
    \hyperlink{#2}{\csname mygametitle@#2\endcsname}% Do not put a white space after #1!
  }%
}
\makeatother

\ifdefined\disableHyperref%
\else%
% Well... not as much as cleveref
\usepackage[capitalise,noabbrev]{cleveref}
% Don't abbrev general names... but still abbrev equations
\crefname{equation}{Eq.}{Eqs.}
\crefname{figure}{Fig.}{Figs.}
\crefname{algorithm}{Protocol}{Protocols}
\Crefname{equation}{Equation}{Equations}
\Crefname{figure}{Figure}{Figures}
\Crefname{algorithm}{Protocol}{Protocols}
\fi

% Put that package at the end of the file,
% or you will have some strange errors like
% "Only one # is allowed per tab"
\usetikzlibrary{quantikz}
% Bug in llncs https://tex.stackexchange.com/a/372108/116348
\def\theHlemma{\theHtheorem}
\def\theHdefinition{\theHtheorem}
\def\theHcorollary{\theHtheorem}

%%%%%% PLEASE DO NOT WRITE ANYTHING AFTER THIS LINE %%%%%%
%%%%%% Indeed, hyperref, cleveref and quantikz need %%%%%%
%%%%%% to be loaded at the very end.                %%%%%%

\ifdefined\enableLlncsCode%
% go back to standard amsthm proof environments with Qed at the end
\let\proof\relax\let\endproof\relax
  \usepackage{amsthm}
  % To debug, it's a nightmare without page number... To disable
  % in the final version
  \pagestyle{plain}
\fi

% Define environments when loaded in standalone...
\ifdefined\enableNonLlncsCode%
\usepackage{thmtools}
\declaretheorem[name=Theorem,numberwithin=chapter]{theorem}
\declaretheorem[name=Definition,sibling=theorem]{definition}
\declaretheorem[name=Lemma,sibling=theorem]{lemma}
\declaretheorem[name=Corollary,sibling=theorem]{corollary}
\declaretheorem[name=Remark,sibling=theorem]{remark}
\fi

\usepackage{enumitem}
\newlist{compressedList}{itemize}{10}
\setlist[compressedList]{label=--,topsep=0pt,parsep=0pt,partopsep=0pt}
\usepackage{mathtools}
\DeclarePairedDelimiter\ceil{\lceil}{\rceil}
\DeclarePairedDelimiter\floor{\lfloor}{\rfloor}

% https://tex.stackexchange.com/questions/134278/overriding-the-paragraph-style-in-the-lncs-document-class
\usepackage{etoolbox}
\patchcmd{\paragraph}{\itshape}{\bfseries\boldmath}{}{}


% This part is specific to llncs-based papers with nicer output
% (page number, theorem labelled with sections...). This should
% be the version in the arxiv.
% To enable this part of the configuration, add
%  \def\nicerThanLlncs{}
% before the %% Let's put in these file all the definitions that we
%% require:

%% Setup language, encoding...
\RequirePackage{etex} 

\usepackage{standalone}
\usepackage{lmodern}
% Language
\usepackage[english]{babel}
% Input encoding
\usepackage[utf8]{inputenc}
% Output encoding https://tex.stackexchange.com/a/677. Important to copy accents
\usepackage[T1]{fontenc}
\usepackage{subfiles} % To quickly load this include file other files
\usepackage{microtype}
\usepackage[shortcuts,nospacearound]{extdash}    % Better endling of em dash (---) to cut sentences\===like 
%%% https://tex.stackexchange.com/questions/2773/how-do-i-make-latex-push-long-citations-to-a-new-line
\usepackage{breakcites}
\usepackage{booktabs}
% Makes sure floats don't fly accross sections
\usepackage{placeins}
\usepackage{adjustbox}
\usepackage{varwidth} % Like minipage, but with automatic width discovery.
\usepackage{complexity} % get \QMA{}...

\usepackage[normalem]{ulem}
%\usepackage[math-style=ISO]{unicode-math}
\usepackage{upgreek} % non italic greak letters

%%%% Bibliography
% I want to have separate references for appendix and main body
% https://tex.stackexchange.com/questions/98660/
\usepackage[style=trad-alpha,
  sortcites=true,
  doi=false,
  url=false,
  giveninits=true, % Bob Foo --> B. Foo
  isbn=false,
  url=false,
  eprint=false,
  sortcites=false, % \cite{B,A,C}: [A,B,C] --> [B,A,C]
]{biblatex}
\defbibfilter{appendixOnlyFilter}{
  segment=1 % Segment 1 will be chosen to be the one in appendix
  and not segment=0 % Default segment is 0
}
\renewcommand{\multicitedelim}{, } % [ABC96; DEF12] -> [ABC96, DEF12]
% [unknown_ref] => [??]
% https://tex.stackexchange.com/a/352573
\makeatletter
\protected\def\abx@missing#1{\textbf{??}}
\makeatother
\renewcommand*{\bibfont}{\normalfont\small} % BibLatex font seems bigger than Bibtex?

\usepackage{xcolor,cancel}
\usepackage{proof-at-the-end}
\NewDocumentEnvironment{theoremE}{O{}O{}+b}{%
  \begin{theoremEnd}[normal,#2]{theorem}[#1]%
    #3%
  \end{theoremEnd}
  %
}{}% Do not forget the second parameter or you might get Missing \begin{document} error
\NewDocumentEnvironment{lemmaE}{O{}O{}+b}{%
  \begin{theoremEnd}[normal,#2]{lemma}[#1]%
    #3%
  \end{theoremEnd}
  %
}{}% Do not forget the second parameter or you might get Missing \begin{document} error
\NewDocumentEnvironment{corollaryE}{O{}O{}+b}{%
  \begin{theoremEnd}[normal,#2]{corollary}[#1]%
    #3%
  \end{theoremEnd}
  %
}{}% Do not forget the second parameter or you might get Missing \begin{document} error
\NewDocumentEnvironment{proofE}{O{}+b}{%
  \begin{proofEnd}[#1]%
    #2
  \end{proofEnd}%
}{}%
\pgfkeys{/prAtEnd/global custom defaults/.style={
    text link={\noindent See \hyperref[proof:prAtEnd\pratendcountercurrent]{proof} in \cref{proofsection:prAtEnd\pratendcountercurrent}.}
  }
}

\ifdefined\displayArxivTexts%
  \pgfkeys{/prAtEnd/end if published/.style={no restate}}
  \pgfkeys{/prAtEnd/all end if published/.style={no restate}}
\else%
  \pgfkeys{/prAtEnd/end if published/.style={end}}
  \pgfkeys{/prAtEnd/all end if published/.style={all end}}
\fi

% https://tex.stackexchange.com/a/20613/116348
% https://tex.stackexchange.com/a/583244/116348
\usepackage{scalerel}
\newcommand\hcancel[2][black]{%
  \ThisStyle{%
    \setbox0=\hbox{$\SavedStyle #2$}%
    \rlap{\raisebox{.25\ht0}{\textcolor{#1}{\rule{\wd0}{1pt}}}}\SavedStyle #2}%
}

%% We provide 3 commands to add comments/text:
%% - \yournameAddText{your text to add}: proposition to add text
%% - \yournameRemoveText{text to remove}: proposition to remove text
%% - \yournameComment{your comments}: to add general comment to discuss. Should not contain proposition to add or remove text, but can be added before/after to explain the reason of the modification.
%% The interest it to provide some quick ways to see what is the final number of pages:
%% just add before the import a line:
%%   \def\removeCommentsAcceptPropositions{}
%% and the proposition will be "accepted" (i.e. the removed text is removed,
%% and the added text will be marked as black), and the comment will be removed, that way it is possible
%% to determine the final number of pages.
%% Example:
%% This text is normal\leoRemove{, this text should be removed}\leoAddText{, this text should be added}\leoComment{This is a comment, explaining why I made that changes.}.
\newcommand{\addEditor}[2]{%
  \expandafter\newcommand\csname #1AddText\endcsname[2][]{%
    \ifdefined\removeCommentsAcceptPropositions %
      ##2%
    \else%
      \textcolor{#2}{\sout{##1}}\textcolor{#2!60!black}{##2}%
    \fi%
  }%
  \expandafter\newcommand\csname #1Comment\endcsname[1]{%
    \ifdefined\removeCommentsAcceptPropositions %
    \else%
      \textcolor{#2!60!white}{\textbf{##1}}%
    \fi%
  }%
  \expandafter\newcommand\csname #1Remove\endcsname[1]{%
    \ifdefined\removeCommentsAcceptPropositions %
    \else%
      \textcolor{#2}{\sout{##1}}%
    \fi%
  }%
}
\definecolor{darkgreen}{rgb}{0., 0.7, 0}
\addEditor{leo}{darkgreen}
\addEditor{fred}{red}
\addEditor{elham}{blue}

%% We also provide an arxivOnly environment, in order to include text that are too long for the submitted version
%% but that would be good to have in the arxiv version.
% Use the \begin{arxivOnly} ... \end{arxivOnly} environment, by putting the opening and ending command
%% on there own line.
%% For shorter texts, you can use the inline version \arxivOnlyShort{your text}.
%% If you want to exclude text from the arxiv, use \begin{publishedOnly} ... \end{publishedOnly} and \publishedOnlyShort{your text}
%% "Comment" package is annoying, it does not allow block indentation, we use 'versions' instead
% \usepackage{versions} %% Do not work inside theoremE, don't know why
\ifdefined\displayArxivTexts%
  % \includeversion{arxivOnly}
  % \excludeversion{publishedOnly}
  \newcommand{\arxivOnly}[1]{#1}
  \newcommand{\arxivOnlyNoDiff}[1]{#1} % Like arxivOnly, but useful in diff mode (when it does not make sense to add colors, for instance inside a cite or when using arxivonly to create lists.
  \newcommand{\publishedOnly}[1]{}
  \newcommand{\publishedVsArxiv}[2]{#2}
  \newcommand{\inAppendixIfPublished}[1]{#1}
\else%
  \ifdefined\displayDiffTexts%
    \colorlet{arxivDiff}{green}
    \colorlet{publishedDiff}{red}
    \newcommand{\arxivOnly}[1]{{\color{arxivDiff}#1}}
    \newcommand{\arxivOnlyNoDiff}[1]{#1}
    \newcommand{\publishedOnly}[1]{{\color{publishedDiff}#1}}
    \newcommand{\publishedVsArxiv}[2]{{\color{publishedDiff}#1}{\color{arxivDiff}#2}}
    \newcommand{\inAppendixIfPublished}[1]{{\color{arxivDiff}\textbf{MOVED IN APPENDIX IN PUBLISHED}#1}\textEnd{{\color{publishedDiff}\textbf{MOVED IN MAIN BODY IN ARXIV}#1}}}
  \else
    % \excludeversion{arxivOnly}
    % \includeversion{publishedOnly}
    \newcommand{\arxivOnly}[1]{}
    \newcommand{\arxivOnlyNoDiff}[1]{}
    \newcommand{\publishedOnly}[1]{#1}
    \newcommand{\publishedVsArxiv}[2]{#1}
    \newcommand{\inAppendixIfPublished}[1]{\textEnd{#1}}
  \fi
\fi
% Alias, easier to remember  
\newcommand{\inAppendix}[1]{\textEnd{#1}}
  
% Symbol to display when an arxiv text is supposed to appear
\usepackage{todonotes}
\newcommand{\notifyArxiv}{\ifdefined\arxivDraftMode\todo{arXiv}\fi}

% Remove unwanted space before/after lists
\usepackage{enumitem}
\publishedOnly{\setlist[itemize]{noitemsep, topsep=0pt}}

% Usual packages:
\usepackage{amsmath}
\usepackage{amssymb}
% \usepackage{amsthm}
\usepackage{thmtools}
\usepackage{graphicx}
\graphicspath{{}{figures/}}
\usepackage{float}
\usepackage{subfig}
\usepackage{wrapfig}
\usepackage{array}
\usepackage{braket}
% Refer to footnotes:
% https://tex.stackexchange.com/a/167382/116348
\usepackage{footmisc}
% Useful to use the latest NewDocumentCommand:
%\usepackage{xparse}
%\usepackage[poster]{tcolorbox}

%% Space between pictures and text
%% https://mirrors.chevalier.io/CTAN/macros/latex/contrib/layouts/layman.pdf
%% \usepackage{layouts}
%% https://tex.stackexchange.com/questions/60477/remove-space-after-figure-and-before-text?rq=1
\setlength{\floatsep}{7pt plus 2pt minus 2pt}
\setlength{\textfloatsep}{7pt plus 2pt minus 2pt}
\setlength{\intextsep}{7pt plus 2pt minus 2pt}
% https://tex.stackexchange.com/questions/39017/how-to-influence-the-position-of-float-environments-like-figure-and-table-in-lat
\setcounter{topnumber}{8}
\setcounter{bottomnumber}{8}
\setcounter{totalnumber}{8}
%% https://tex.stackexchange.com/questions/45990/how-can-i-modify-vertical-space-between-figure-and-caption
%% plus/minus allows to stretch a bit around 15pt
% \setlength{\abovecaptionskip}{5pt}

% Remove number to subsubsection (in llncs it replaces \paragraph)
\setcounter{secnumdepth}{2}

%%% Some practical environment to define theorems, lemma...
% \declaretheorem[name=Theorem,numberwithin=section]{theorem}
% \declaretheorem[name=Definition,sibling=theorem]{definition}
% \declaretheorem[name=Lemma,sibling=theorem]{lemma}
% \declaretheorem[name=Corollary,sibling=theorem]{corollary}
% \declaretheorem[name=Remark,sibling=theorem]{remark}
%%% In LNCS we use
% https://tex.stackexchange.com/a/373125/116348
% \spnewtheorem{thm}{Theorem}[section]{\bfseries}{\itshape}

%%%
\usepackage[ruled,vlined]{algorithm2e}
\SetAlFnt{\normalsize}%
\setlength{\algomargin}{0em} % remove margin on left of algo
\SetAlgorithmName{Protocol}{protocol}{List of Protocols}
\newenvironment{protocol}[1][htb]{\begin{algorithm}[#1]}{\end{algorithm}}

%%% %%% Load algorithm package, and create a breakable version
%%% \usepackage{algorithm, algorithmicx, algpseudocode}
%%% % Decrease space between text:
%%% % https://tex.stackexchange.com/questions/25828/how-to-remove-change-the-vertical-spacing-before-and-after-an-algorithm-enviro/25832
%%% \setlength{\intextsep}{1\baselineskip}
%%% % https://tex.stackexchange.com/questions/387577/new-algorithm-environment-with-a-separate-counter
%%% \newcounter{protocol}
%%% \makeatletter
%%% \newenvironment{protocol}[1][htb]{%
%%%   \let\c@algorithm\c@protocol
%%%   \renewcommand{\ALG@name}{Protocol}% Update algorithm name
%%%   \begin{algorithm}[#1]%
%%%   }{\end{algorithm}
%%% }
%%% \makeatother
%%%
%%% \providecommand*\algorithmautorefname{Protocol}
%%% \makeatletter
%%% \newenvironment{breakablealgorithm}
%%%   {% \begin{breakablealgorithm}
%%%    % \begin{center}
%%%      \refstepcounter{algorithm}% New algorithm
%%%      \hrule height.8pt depth0pt \kern2pt% \@fs@pre for \@fs@ruled
%%%      \renewcommand{\caption}[2][\relax]{% Make a new \caption
%%%        % {\raggedright\textbf{\ALG@name~\thealgorithm} ##2\par}%
%%%        {\raggedright\textbf{Protocol~\thealgorithm} ##2\par}%
%%%        \ifx\relax##1\relax % #1 is \relax
%%%          \addcontentsline{loa}{algorithm}{\protect\numberline{\thealgorithm}##2}%
%%%        \else % #1 is not \relax
%%%          \addcontentsline{loa}{algorithm}{\protect\numberline{\thealgorithm}##1}%
%%%        \fi
%%%        \kern2pt\hrule\kern2pt
%%%        \small
%%%      }
%%%   }{% \end{breakablealgorithm}
%%%      \kern2pt\hrule\relax% \@fs@post for \@fs@ruled
%%%    % \end{center}
%%%   }
%%% \makeatother
%%%

\def\EQ#1{\begin{eqnarray}#1\end{eqnarray}} 


%%% Practical abreviations
\usepackage{calrsfs} % Replace \mathcal with different more "cursive" font
\DeclareMathAlphabet{\pazocal}{OMS}{zplm}{m}{n} % \pazocal is now the old \mathcal
\newcommand*{\cA}{\pazocal{A}}
\newcommand*{\cB}{\pazocal{B}}
\newcommand*{\cC}{\pazocal{C}}
\newcommand*{\cD}{\pazocal{D}}
\newcommand*{\cE}{\pazocal{E}}
\newcommand*{\cF}{\pazocal{F}}
\newcommand*{\cG}{\pazocal{G}}
\newcommand*{\cH}{\pazocal{H}}
\newcommand*{\cI}{\pazocal{I}}
\newcommand*{\cJ}{\pazocal{J}}
\newcommand*{\cK}{\pazocal{K}}
\renewcommand*{\cL}{\pazocal{L}} % defined in complexity
\newcommand*{\cM}{\pazocal{M}}
\newcommand*{\cN}{\pazocal{N}}
\newcommand*{\cO}{\pazocal{O}}
\renewcommand*{\cP}{\pazocal{P}} % defined in complexity
\newcommand*{\cQ}{\pazocal{Q}}
\newcommand*{\cR}{\pazocal{R}}
\renewcommand*{\cS}{\pazocal{S}} % defined in complexity
\newcommand*{\cT}{\pazocal{T}}
\newcommand*{\cU}{\pazocal{U}}
\newcommand*{\cV}{\pazocal{V}}
\newcommand*{\cW}{\pazocal{W}}
\newcommand*{\cX}{\pazocal{X}}
\newcommand*{\cY}{\pazocal{Y}}
\newcommand*{\cZ}{\pazocal{Z}}

% Notation for languages. Make sure not to collide with L(H) for linear operators on H
%\DeclareMathAlphabet{\pazocal}{OMS}{zplm}{m}{n}
\newcommand{\linearOp}{\mathcal{L}} % Defines the opearator L(H)
\renewcommand*{\lang}{\pazocal{L}} % Defines a new Turing/Quantum Language

%https://tex.stackexchange.com/questions/195025/
\newcommand*{\rightSend}{\rightsquigarrow} % snake "a --> b" to say "I send a to b".

% For some reasons, proof-at-the-end do not like # (problem with detokenize? To fix...). So we use a macro:
\newcommand*{\diese}{\#}

\newcommand*{\eps}[0]{\varepsilon}
\renewcommand*{\D}[0]{\mathbb{D}}
\renewcommand*{\E}{\mathbb{E}}
\newcommand*{\N}[0]{\mathbb{N}}
\renewcommand*{\R}[0]{\mathbb{R}}
\renewcommand*{\C}[0]{\mathbb{C}}
\newcommand*{\Z}[0]{\mathbb{Z}}
\newcommand*{\T}[0]{\mathbb{T}}

\newcommand*{\CPA}[0]{\textsf{CPA}}
\newcommand*{\LWE}[0]{\textsf{LWE}}
\newcommand*{\GapSVP}[0]{\textsf{GapSVP}}
\newcommand*{\SIVP}[0]{\textsf{SIVP}}

% first param is alpha, second is q (optional)
\newcommand{\contGauss}[1]{\cD_{#1}}
\newcommand{\disGauss}[2]{\cD_{\Z,#1 #2}}
\newcommand{\disGaussCont}[1]{\cD_{\Z,#1}}
\newcommand{\disGaussAQ}{\cD_{\Z,\alpha q}}
\newcommand{\rsafe}{r_{\text{safe}}}
% domain of the function.
\newcommand{\ccX}{\cX}
\newcommand{\ccXSphere}{\cX_\bullet}
\newcommand{\ccXCube}[1][]{\cX_{\scalebox{.4}{$\blacksquare$}#1}}

\newcommand{\BBHeightyFor}{\textsc{BB84}}

% "Correct" spacing around cdot as wildcard
% https://tex.stackexchange.com/questions/78165/spacing-around-cdot-when-used-as-a-wildcard
\newcommand\cdotwild{{\mkern 2mu\cdot\mkern 2mu}}

\newcommand*{\id}{\mathrm{id}}
\DeclareMathOperator{\Tr}{Tr}
\DeclareMathOperator{\supp}{supp}
\DeclareMathOperator{\Det}{Det}
\DeclareMathOperator{\ERR}{ERR}
\DeclareMathOperator{\Round}{Round}

\newcommand{\Ver}{ {\tt Ver} } %% Verify public key
\newcommand{\Gen}{ {\tt Gen} }
\newcommand{\GenLocal}{\ensuremath{ {\tt Gen}_{\tt Loc}} }
\newcommand{\Enc}{ {\tt Enc} }
\newcommand{\Dec}{ {\tt Invert} }
\newcommand{\Eval}{ {\tt Eval} }

\newcommand{\HGen}{ {\tt H.Gen} }
\newcommand{\HEnc}{ {\tt H.Enc} }
\newcommand{\HDec}{ {\tt H.Invert} }
\newcommand{\HEval}{ {\tt H.Eval} }
\newcommand{\Hh}{\ensuremath{{\tt H.}h} }
\newcommand{\HCheckHonestTrapdoor}{\ensuremath{{\tt H.CheckTrapdoor}}}

\newcommand{\MPGen}{ {\tt MP.Gen} }
\newcommand{\MPDec}{ {\tt MP.Invert} }
\newcommand{\InvertSmallGadget}{ \ensuremath{{\texttt{InvertSmallGadget}}} }
\newcommand{\InvertGadget}{ {\tt InvertGadget} }
% \newcommand{\MPEval}{ $g$ }


\DeclareMathOperator{\RoundMod}{RoundMod}


% https://tex.stackexchange.com/a/253746/116348
\usepackage{mleftright}
\newcommand{\BlockMatrix}[2]{%
  \mleft[%
  \begin{array}{@{}#1@{}}%
    #2
  \end{array}%
  \mright]%
}
\newcommand{\SmallBlockMatrix}[2]{
  \mleft[\begin{array}{c}
    #1\tabularnewline % https://tex.stackexchange.com/questions/328745/matrices-with-cryptocode-package
    \hline
    #2
  \end{array}\mright]
}
\newcommand{\HorizBlockMatrix}[2]{
  \mleft[\begin{array}{@{}c|c@{}}
    #1 & #2
  \end{array}\mright]
}

\newcommand*{\vect}[1]{\ensuremath{\mathbf{#1}}}


%% Already defined in *recent* cryptocode.
% \DeclareMathOperator*{\argmax}{arg \, max}
% \DeclareMathOperator*{\argmin}{arg \, min}

\usepackage{xspace}
\xspaceaddexceptions{]\}}
\newcommand{\RSP}{\ensuremath{\sf{RSP}}\xspace}%

\newcommand*{\RSPenc}{\mathrm{RSP}_{\mathrm{enc}}}
\newcommand*{\filter}{\vdash}
\newcommand*{\channelBB}{\cS_{\Z\frac{\pi}{2}}}
% To denote a protocol between two parties
% \newcommand*{\protocol}{\boldsymbol{\pi}}
\newcommand\compRSP{composable $\sf{RSP}_{CC}$}
\newcommand\RSPlike{$\sf{RSP}_{CC}$}
\newcommand\compUBQC{composable $\sf{UBQC}_{CC}$}
\newcommand\UBQClike{$\sf{UBQC}_{CC}$}
% GHZ
\newcommand*{\eqdef}{\coloneqq}
\newcommand*{\PERM}{\ensuremath{}{\sf PERM}}
\newcommand*{\PPT}{\ensuremath{}{\sf PPT}}
\newcommand*{\QPT}{\ensuremath{}{\sf QPT}}
\newcommand*{\Auth}{\ensuremath{}{\sf Auth}}
% Prover, verifier, extractor
\renewcommand*{\P}{\ensuremath{}{\sf P}}
\newcommand*{\V}{\ensuremath{}{\sf V}}
\renewcommand*{\K}{\ensuremath{}{\sf K}}


% Here are the different protocols we use
% Initial protocol
\newcommand{\blind}{\ensuremath{ {\tt BLIND} }}
\newcommand{\blindZK}{\ensuremath{ {\tt BLIND-ZK} }}
% Reveals to people if they are part of the support
\newcommand{\blindSup}{\ensuremath{ {\tt BLIND}^{\tt sup} }}
% Make sure that people in the support share a canonical GHZ
\newcommand{\blindCan}{\ensuremath{ {\tt BLIND}_{\tt can} }}
% Make sure that people in the support share a canonical GHZ and that they know they are supported
\newcommand{\blindCanSup}{\ensuremath{ {\tt BLIND}^{\tt sup}_{\tt can} }}
\newcommand{\authBlindCanDist}{\ensuremath{ {\tt AUTH-BLIND}^{\tt dist}_{\tt can} }}
\newcommand{\verifCanSup}{\ensuremath{ {\tt VERIF}^{\tt sup}_{\tt can} }}
\newcommand{\blindPoly}{\ensuremath{ {\tt BLIND-poly} }}
\newcommand{\INDFCT}{\ensuremath{ {\tt IND-D0} }}
\newcommand{\INDBLIND}{\refGame*{INDBLIND}}
\newcommand{\INDBLINDtxt}{\ensuremath{ {\tt IND-\blind{}} }}
\newcommand{\INDBLINDSUP}{\refGame*{INDBLINDSUP}}
\newcommand{\INDBLINDSUPtxt}{\ensuremath{ {\tt IND-\blindSup{}} }}
\newcommand{\INDBLINDCANSUP}{\ensuremath{ {\tt IND-\blindCanSup{}} }}
\newcommand{\INDBLINDCANAUTH}{\ensuremath{ {\tt IND-\authBlindCanDist{}} }}
\newcommand{\INDPARTIAL}{\ensuremath{ {\tt IND-PARTIAL} }}
\newcommand*\GHZ{\ensuremath{}{\sf GHZ}}
\newcommand*\GHZCan{\ensuremath{{\sf GHZ^{\sf can}}}}
% Generalized GGHZ
\newcommand*\GGHZ{\ensuremath{ {\sf GHZ^G} }}
\newcommand*\HGHZ{\ensuremath{ {\sf GHZ^H} }}
\newcommand{\AssumpFctNoDelta}{\HGHZ{} capable}
\newcommand{\AssumpFctCanNoDelta}{\GHZCan{} capable}
\newcommand{\AssumpFct}{$\delta$-\HGHZ{} capable}
\newcommand{\AssumpFctNegl}{$\negl[\lambda]$-\HGHZ{} capable}
\newcommand{\AssumpFctCan}{$\delta$-\GHZCan{} capable}
\newcommand{\AssumpFctCanPrime}{$\delta^\prime$-\GHZCan{} capable}
\newcommand{\AssumpFctDist}{distributable}
\newcommand{\partialInfo}{\ensuremath{{\tt PartInfo}}}
\newcommand{\partialInfoLocal}{\ensuremath{{\tt PartInfo_{\tt Loc}}}}
\newcommand{\partialAlpha}{\ensuremath{{\tt PartAlpha_{\tt Loc}}}}
\newcommand{\CombineAlpha}{\ensuremath{{\tt CombineAlpha}}}
\newcommand{\CombineRandom}{\ensuremath{{\tt CombineRandom}}}
\newcommand{\CheckHonestTrapdoor}{\ensuremath{{\tt CheckTrapdoor}}}
\newcommand{\ZKGenCheck}{\ensuremath{{\tt ZK-GenCheck}}}

\newcommand{\highlightChanges}[1]{\textcolor{green!50!black}{#1}}

\newcommand{\Real}{\ensuremath{\mathsf{REAL}}}
\newcommand{\Sim}{\ensuremath{\mathsf{Sim}}}
\newcommand{\Ideal}{\ensuremath{\mathsf{IDEAL}}}
\newcommand{\OUT}{\ensuremath{\mathsf{OUT}}}

\usepackage{pifont}% http://ctan.org/pkg/pifont
% To denote that a user is not in the final GHZ
\newcommand{\cross}{\ensuremath{\text{\ding{55}}}}

\usepackage{verbatim}
\newcommand{\leftI}{{\normalfont\ttfamily left}}
\newcommand{\rightI}{{\normalfont\ttfamily right}}
\newcommand*{\mybar}[1]{\overline{#1}}

\newcommand*{\quout}{\mathtt{out}}

\DeclareMathOperator{\Image}{Image}
\DeclareMathOperator{\Ima}{Im}
\DeclareMathOperator{\Dom}{Dom}

\newcommand*{\Pub}{\mathtt{Pub}}
\newcommand*{\accept}{\mathtt{accept}}
\newcommand*{\abort}{\mathtt{abort}}
\newcommand*{\gadxor}{\mathtt{Gad}_{\oplus}}
\newcommand*{\quin}{\mathtt{in}}

%%% Provide a "subproof" environment (pretty simple for now,
%%% just indent the sub-proofs)
% I'd love to have left border... https://tex.stackexchange.com/questions/532948/robustly-add-a-border-to-the-left-of-a-text-spanning-several-pages
\usepackage{changepage}% http://ctan.org/pkg/changepage
\newenvironment{subproof}{\begin{adjustwidth}{2em}{0pt}}{\end{adjustwidth}}


%%% To have nice probabilities
% https://tex.stackexchange.com/questions/198771/align-in-substack
\makeatletter
\newcommand{\subalign}[1]{%
  \vcenter{%
    \Let@ \restore@math@cr \default@tag
    \baselineskip\fontdimen10 \scriptfont\tw@
    \advance\baselineskip\fontdimen12 \scriptfont\tw@
    \lineskip\thr@@\fontdimen8 \scriptfont\thr@@
    \lineskiplimit\lineskip
    \ialign{\hfil$\m@th\scriptstyle##$&$\m@th\scriptstyle{}##$\hfil\crcr
      #1\crcr
    }%
  }%
}
\makeatother
\usepackage{xparse}
% https://tex.stackexchange.com/questions/431794/same-spacing-around-middle-as-around-mid
\let\originalmiddle=\middle
\def\middle#1{\mathrel{}\originalmiddle#1\mathrel{}}
\NewDocumentCommand{\pr}{som}{\Pr\IfValueT{#2}{_{\substack{#2}}}\left[\,#3\,\right]}
\NewDocumentCommand{\esp}{som}{\IfValueTF{#2}{\underset{\subalign{#2}}{\mathbb{E}}}{\mathbb{E}}[\,#3\,]}
\NewDocumentEnvironment{pral}{soO{}}
 {%
  \Pr\IfValueT{#2}{_{\IfBooleanTF{#1}{\substack{#2}}{#2}}}
\begin{alignedat}[t]{2}
  [\, } {\,]
 #3
 \end{alignedat}
 }
\NewDocumentEnvironment{espal}{soO{}}
 {%
   \IfValueTF{#2}{\underset{\subalign{#2}}{\mathbb{E}}}{\mathbb{E}}
   \begin{alignedat}[t]{2}[\,
 }
 {\,]#3\end{alignedat}
 }
% To draw |A><B|, the second argument is facultative and gives |A><A|
\NewDocumentCommand{\ketbra}{mg}{
  | #1 \rangle \langle \IfValueTF{#2}{#2}{#1} |
}
% https://tex.stackexchange.com/a/188364/116348
\DeclareRobustCommand{\minwidthbox}[2]{%
  \ifmmode
    \expandafter\mathmakebox
  \else
    \expandafter\makebox
  \fi
  [\ifdim#2<\width\width\else#2\fi]{#1}%f
}

\NewDocumentCommand{\constructs}{mg}{
  \xrightarrow[\IfValueT{#2}{#2}]{\minwidthbox{#1}{5mm}}
}

%%%%%%%%%% This part is specific to llncs-based papers
% To enable this part of the configuration, add
%  \def\enableLlncsCode{}
% before the %% Let's put in these file all the definitions that we
%% require:

%% Setup language, encoding...
\RequirePackage{etex} 

\usepackage{standalone}
\usepackage{lmodern}
% Language
\usepackage[english]{babel}
% Input encoding
\usepackage[utf8]{inputenc}
% Output encoding https://tex.stackexchange.com/a/677. Important to copy accents
\usepackage[T1]{fontenc}
\usepackage{subfiles} % To quickly load this include file other files
\usepackage{microtype}
\usepackage[shortcuts,nospacearound]{extdash}    % Better endling of em dash (---) to cut sentences\===like 
%%% https://tex.stackexchange.com/questions/2773/how-do-i-make-latex-push-long-citations-to-a-new-line
\usepackage{breakcites}
\usepackage{booktabs}
% Makes sure floats don't fly accross sections
\usepackage{placeins}
\usepackage{adjustbox}
\usepackage{varwidth} % Like minipage, but with automatic width discovery.
\usepackage{complexity} % get \QMA{}...

\usepackage[normalem]{ulem}
%\usepackage[math-style=ISO]{unicode-math}
\usepackage{upgreek} % non italic greak letters

%%%% Bibliography
% I want to have separate references for appendix and main body
% https://tex.stackexchange.com/questions/98660/
\usepackage[style=trad-alpha,
  sortcites=true,
  doi=false,
  url=false,
  giveninits=true, % Bob Foo --> B. Foo
  isbn=false,
  url=false,
  eprint=false,
  sortcites=false, % \cite{B,A,C}: [A,B,C] --> [B,A,C]
]{biblatex}
\defbibfilter{appendixOnlyFilter}{
  segment=1 % Segment 1 will be chosen to be the one in appendix
  and not segment=0 % Default segment is 0
}
\renewcommand{\multicitedelim}{, } % [ABC96; DEF12] -> [ABC96, DEF12]
% [unknown_ref] => [??]
% https://tex.stackexchange.com/a/352573
\makeatletter
\protected\def\abx@missing#1{\textbf{??}}
\makeatother
\renewcommand*{\bibfont}{\normalfont\small} % BibLatex font seems bigger than Bibtex?

\usepackage{xcolor,cancel}
\usepackage{proof-at-the-end}
\NewDocumentEnvironment{theoremE}{O{}O{}+b}{%
  \begin{theoremEnd}[normal,#2]{theorem}[#1]%
    #3%
  \end{theoremEnd}
  %
}{}% Do not forget the second parameter or you might get Missing \begin{document} error
\NewDocumentEnvironment{lemmaE}{O{}O{}+b}{%
  \begin{theoremEnd}[normal,#2]{lemma}[#1]%
    #3%
  \end{theoremEnd}
  %
}{}% Do not forget the second parameter or you might get Missing \begin{document} error
\NewDocumentEnvironment{corollaryE}{O{}O{}+b}{%
  \begin{theoremEnd}[normal,#2]{corollary}[#1]%
    #3%
  \end{theoremEnd}
  %
}{}% Do not forget the second parameter or you might get Missing \begin{document} error
\NewDocumentEnvironment{proofE}{O{}+b}{%
  \begin{proofEnd}[#1]%
    #2
  \end{proofEnd}%
}{}%
\pgfkeys{/prAtEnd/global custom defaults/.style={
    text link={\noindent See \hyperref[proof:prAtEnd\pratendcountercurrent]{proof} in \cref{proofsection:prAtEnd\pratendcountercurrent}.}
  }
}

\ifdefined\displayArxivTexts%
  \pgfkeys{/prAtEnd/end if published/.style={no restate}}
  \pgfkeys{/prAtEnd/all end if published/.style={no restate}}
\else%
  \pgfkeys{/prAtEnd/end if published/.style={end}}
  \pgfkeys{/prAtEnd/all end if published/.style={all end}}
\fi

% https://tex.stackexchange.com/a/20613/116348
% https://tex.stackexchange.com/a/583244/116348
\usepackage{scalerel}
\newcommand\hcancel[2][black]{%
  \ThisStyle{%
    \setbox0=\hbox{$\SavedStyle #2$}%
    \rlap{\raisebox{.25\ht0}{\textcolor{#1}{\rule{\wd0}{1pt}}}}\SavedStyle #2}%
}

%% We provide 3 commands to add comments/text:
%% - \yournameAddText{your text to add}: proposition to add text
%% - \yournameRemoveText{text to remove}: proposition to remove text
%% - \yournameComment{your comments}: to add general comment to discuss. Should not contain proposition to add or remove text, but can be added before/after to explain the reason of the modification.
%% The interest it to provide some quick ways to see what is the final number of pages:
%% just add before the import a line:
%%   \def\removeCommentsAcceptPropositions{}
%% and the proposition will be "accepted" (i.e. the removed text is removed,
%% and the added text will be marked as black), and the comment will be removed, that way it is possible
%% to determine the final number of pages.
%% Example:
%% This text is normal\leoRemove{, this text should be removed}\leoAddText{, this text should be added}\leoComment{This is a comment, explaining why I made that changes.}.
\newcommand{\addEditor}[2]{%
  \expandafter\newcommand\csname #1AddText\endcsname[2][]{%
    \ifdefined\removeCommentsAcceptPropositions %
      ##2%
    \else%
      \textcolor{#2}{\sout{##1}}\textcolor{#2!60!black}{##2}%
    \fi%
  }%
  \expandafter\newcommand\csname #1Comment\endcsname[1]{%
    \ifdefined\removeCommentsAcceptPropositions %
    \else%
      \textcolor{#2!60!white}{\textbf{##1}}%
    \fi%
  }%
  \expandafter\newcommand\csname #1Remove\endcsname[1]{%
    \ifdefined\removeCommentsAcceptPropositions %
    \else%
      \textcolor{#2}{\sout{##1}}%
    \fi%
  }%
}
\definecolor{darkgreen}{rgb}{0., 0.7, 0}
\addEditor{leo}{darkgreen}
\addEditor{fred}{red}
\addEditor{elham}{blue}

%% We also provide an arxivOnly environment, in order to include text that are too long for the submitted version
%% but that would be good to have in the arxiv version.
% Use the \begin{arxivOnly} ... \end{arxivOnly} environment, by putting the opening and ending command
%% on there own line.
%% For shorter texts, you can use the inline version \arxivOnlyShort{your text}.
%% If you want to exclude text from the arxiv, use \begin{publishedOnly} ... \end{publishedOnly} and \publishedOnlyShort{your text}
%% "Comment" package is annoying, it does not allow block indentation, we use 'versions' instead
% \usepackage{versions} %% Do not work inside theoremE, don't know why
\ifdefined\displayArxivTexts%
  % \includeversion{arxivOnly}
  % \excludeversion{publishedOnly}
  \newcommand{\arxivOnly}[1]{#1}
  \newcommand{\arxivOnlyNoDiff}[1]{#1} % Like arxivOnly, but useful in diff mode (when it does not make sense to add colors, for instance inside a cite or when using arxivonly to create lists.
  \newcommand{\publishedOnly}[1]{}
  \newcommand{\publishedVsArxiv}[2]{#2}
  \newcommand{\inAppendixIfPublished}[1]{#1}
\else%
  \ifdefined\displayDiffTexts%
    \colorlet{arxivDiff}{green}
    \colorlet{publishedDiff}{red}
    \newcommand{\arxivOnly}[1]{{\color{arxivDiff}#1}}
    \newcommand{\arxivOnlyNoDiff}[1]{#1}
    \newcommand{\publishedOnly}[1]{{\color{publishedDiff}#1}}
    \newcommand{\publishedVsArxiv}[2]{{\color{publishedDiff}#1}{\color{arxivDiff}#2}}
    \newcommand{\inAppendixIfPublished}[1]{{\color{arxivDiff}\textbf{MOVED IN APPENDIX IN PUBLISHED}#1}\textEnd{{\color{publishedDiff}\textbf{MOVED IN MAIN BODY IN ARXIV}#1}}}
  \else
    % \excludeversion{arxivOnly}
    % \includeversion{publishedOnly}
    \newcommand{\arxivOnly}[1]{}
    \newcommand{\arxivOnlyNoDiff}[1]{}
    \newcommand{\publishedOnly}[1]{#1}
    \newcommand{\publishedVsArxiv}[2]{#1}
    \newcommand{\inAppendixIfPublished}[1]{\textEnd{#1}}
  \fi
\fi
% Alias, easier to remember  
\newcommand{\inAppendix}[1]{\textEnd{#1}}
  
% Symbol to display when an arxiv text is supposed to appear
\usepackage{todonotes}
\newcommand{\notifyArxiv}{\ifdefined\arxivDraftMode\todo{arXiv}\fi}

% Remove unwanted space before/after lists
\usepackage{enumitem}
\publishedOnly{\setlist[itemize]{noitemsep, topsep=0pt}}

% Usual packages:
\usepackage{amsmath}
\usepackage{amssymb}
% \usepackage{amsthm}
\usepackage{thmtools}
\usepackage{graphicx}
\graphicspath{{}{figures/}}
\usepackage{float}
\usepackage{subfig}
\usepackage{wrapfig}
\usepackage{array}
\usepackage{braket}
% Refer to footnotes:
% https://tex.stackexchange.com/a/167382/116348
\usepackage{footmisc}
% Useful to use the latest NewDocumentCommand:
%\usepackage{xparse}
%\usepackage[poster]{tcolorbox}

%% Space between pictures and text
%% https://mirrors.chevalier.io/CTAN/macros/latex/contrib/layouts/layman.pdf
%% \usepackage{layouts}
%% https://tex.stackexchange.com/questions/60477/remove-space-after-figure-and-before-text?rq=1
\setlength{\floatsep}{7pt plus 2pt minus 2pt}
\setlength{\textfloatsep}{7pt plus 2pt minus 2pt}
\setlength{\intextsep}{7pt plus 2pt minus 2pt}
% https://tex.stackexchange.com/questions/39017/how-to-influence-the-position-of-float-environments-like-figure-and-table-in-lat
\setcounter{topnumber}{8}
\setcounter{bottomnumber}{8}
\setcounter{totalnumber}{8}
%% https://tex.stackexchange.com/questions/45990/how-can-i-modify-vertical-space-between-figure-and-caption
%% plus/minus allows to stretch a bit around 15pt
% \setlength{\abovecaptionskip}{5pt}

% Remove number to subsubsection (in llncs it replaces \paragraph)
\setcounter{secnumdepth}{2}

%%% Some practical environment to define theorems, lemma...
% \declaretheorem[name=Theorem,numberwithin=section]{theorem}
% \declaretheorem[name=Definition,sibling=theorem]{definition}
% \declaretheorem[name=Lemma,sibling=theorem]{lemma}
% \declaretheorem[name=Corollary,sibling=theorem]{corollary}
% \declaretheorem[name=Remark,sibling=theorem]{remark}
%%% In LNCS we use
% https://tex.stackexchange.com/a/373125/116348
% \spnewtheorem{thm}{Theorem}[section]{\bfseries}{\itshape}

%%%
\usepackage[ruled,vlined]{algorithm2e}
\SetAlFnt{\normalsize}%
\setlength{\algomargin}{0em} % remove margin on left of algo
\SetAlgorithmName{Protocol}{protocol}{List of Protocols}
\newenvironment{protocol}[1][htb]{\begin{algorithm}[#1]}{\end{algorithm}}

%%% %%% Load algorithm package, and create a breakable version
%%% \usepackage{algorithm, algorithmicx, algpseudocode}
%%% % Decrease space between text:
%%% % https://tex.stackexchange.com/questions/25828/how-to-remove-change-the-vertical-spacing-before-and-after-an-algorithm-enviro/25832
%%% \setlength{\intextsep}{1\baselineskip}
%%% % https://tex.stackexchange.com/questions/387577/new-algorithm-environment-with-a-separate-counter
%%% \newcounter{protocol}
%%% \makeatletter
%%% \newenvironment{protocol}[1][htb]{%
%%%   \let\c@algorithm\c@protocol
%%%   \renewcommand{\ALG@name}{Protocol}% Update algorithm name
%%%   \begin{algorithm}[#1]%
%%%   }{\end{algorithm}
%%% }
%%% \makeatother
%%%
%%% \providecommand*\algorithmautorefname{Protocol}
%%% \makeatletter
%%% \newenvironment{breakablealgorithm}
%%%   {% \begin{breakablealgorithm}
%%%    % \begin{center}
%%%      \refstepcounter{algorithm}% New algorithm
%%%      \hrule height.8pt depth0pt \kern2pt% \@fs@pre for \@fs@ruled
%%%      \renewcommand{\caption}[2][\relax]{% Make a new \caption
%%%        % {\raggedright\textbf{\ALG@name~\thealgorithm} ##2\par}%
%%%        {\raggedright\textbf{Protocol~\thealgorithm} ##2\par}%
%%%        \ifx\relax##1\relax % #1 is \relax
%%%          \addcontentsline{loa}{algorithm}{\protect\numberline{\thealgorithm}##2}%
%%%        \else % #1 is not \relax
%%%          \addcontentsline{loa}{algorithm}{\protect\numberline{\thealgorithm}##1}%
%%%        \fi
%%%        \kern2pt\hrule\kern2pt
%%%        \small
%%%      }
%%%   }{% \end{breakablealgorithm}
%%%      \kern2pt\hrule\relax% \@fs@post for \@fs@ruled
%%%    % \end{center}
%%%   }
%%% \makeatother
%%%

\def\EQ#1{\begin{eqnarray}#1\end{eqnarray}} 


%%% Practical abreviations
\usepackage{calrsfs} % Replace \mathcal with different more "cursive" font
\DeclareMathAlphabet{\pazocal}{OMS}{zplm}{m}{n} % \pazocal is now the old \mathcal
\newcommand*{\cA}{\pazocal{A}}
\newcommand*{\cB}{\pazocal{B}}
\newcommand*{\cC}{\pazocal{C}}
\newcommand*{\cD}{\pazocal{D}}
\newcommand*{\cE}{\pazocal{E}}
\newcommand*{\cF}{\pazocal{F}}
\newcommand*{\cG}{\pazocal{G}}
\newcommand*{\cH}{\pazocal{H}}
\newcommand*{\cI}{\pazocal{I}}
\newcommand*{\cJ}{\pazocal{J}}
\newcommand*{\cK}{\pazocal{K}}
\renewcommand*{\cL}{\pazocal{L}} % defined in complexity
\newcommand*{\cM}{\pazocal{M}}
\newcommand*{\cN}{\pazocal{N}}
\newcommand*{\cO}{\pazocal{O}}
\renewcommand*{\cP}{\pazocal{P}} % defined in complexity
\newcommand*{\cQ}{\pazocal{Q}}
\newcommand*{\cR}{\pazocal{R}}
\renewcommand*{\cS}{\pazocal{S}} % defined in complexity
\newcommand*{\cT}{\pazocal{T}}
\newcommand*{\cU}{\pazocal{U}}
\newcommand*{\cV}{\pazocal{V}}
\newcommand*{\cW}{\pazocal{W}}
\newcommand*{\cX}{\pazocal{X}}
\newcommand*{\cY}{\pazocal{Y}}
\newcommand*{\cZ}{\pazocal{Z}}

% Notation for languages. Make sure not to collide with L(H) for linear operators on H
%\DeclareMathAlphabet{\pazocal}{OMS}{zplm}{m}{n}
\newcommand{\linearOp}{\mathcal{L}} % Defines the opearator L(H)
\renewcommand*{\lang}{\pazocal{L}} % Defines a new Turing/Quantum Language

%https://tex.stackexchange.com/questions/195025/
\newcommand*{\rightSend}{\rightsquigarrow} % snake "a --> b" to say "I send a to b".

% For some reasons, proof-at-the-end do not like # (problem with detokenize? To fix...). So we use a macro:
\newcommand*{\diese}{\#}

\newcommand*{\eps}[0]{\varepsilon}
\renewcommand*{\D}[0]{\mathbb{D}}
\renewcommand*{\E}{\mathbb{E}}
\newcommand*{\N}[0]{\mathbb{N}}
\renewcommand*{\R}[0]{\mathbb{R}}
\renewcommand*{\C}[0]{\mathbb{C}}
\newcommand*{\Z}[0]{\mathbb{Z}}
\newcommand*{\T}[0]{\mathbb{T}}

\newcommand*{\CPA}[0]{\textsf{CPA}}
\newcommand*{\LWE}[0]{\textsf{LWE}}
\newcommand*{\GapSVP}[0]{\textsf{GapSVP}}
\newcommand*{\SIVP}[0]{\textsf{SIVP}}

% first param is alpha, second is q (optional)
\newcommand{\contGauss}[1]{\cD_{#1}}
\newcommand{\disGauss}[2]{\cD_{\Z,#1 #2}}
\newcommand{\disGaussCont}[1]{\cD_{\Z,#1}}
\newcommand{\disGaussAQ}{\cD_{\Z,\alpha q}}
\newcommand{\rsafe}{r_{\text{safe}}}
% domain of the function.
\newcommand{\ccX}{\cX}
\newcommand{\ccXSphere}{\cX_\bullet}
\newcommand{\ccXCube}[1][]{\cX_{\scalebox{.4}{$\blacksquare$}#1}}

\newcommand{\BBHeightyFor}{\textsc{BB84}}

% "Correct" spacing around cdot as wildcard
% https://tex.stackexchange.com/questions/78165/spacing-around-cdot-when-used-as-a-wildcard
\newcommand\cdotwild{{\mkern 2mu\cdot\mkern 2mu}}

\newcommand*{\id}{\mathrm{id}}
\DeclareMathOperator{\Tr}{Tr}
\DeclareMathOperator{\supp}{supp}
\DeclareMathOperator{\Det}{Det}
\DeclareMathOperator{\ERR}{ERR}
\DeclareMathOperator{\Round}{Round}

\newcommand{\Ver}{ {\tt Ver} } %% Verify public key
\newcommand{\Gen}{ {\tt Gen} }
\newcommand{\GenLocal}{\ensuremath{ {\tt Gen}_{\tt Loc}} }
\newcommand{\Enc}{ {\tt Enc} }
\newcommand{\Dec}{ {\tt Invert} }
\newcommand{\Eval}{ {\tt Eval} }

\newcommand{\HGen}{ {\tt H.Gen} }
\newcommand{\HEnc}{ {\tt H.Enc} }
\newcommand{\HDec}{ {\tt H.Invert} }
\newcommand{\HEval}{ {\tt H.Eval} }
\newcommand{\Hh}{\ensuremath{{\tt H.}h} }
\newcommand{\HCheckHonestTrapdoor}{\ensuremath{{\tt H.CheckTrapdoor}}}

\newcommand{\MPGen}{ {\tt MP.Gen} }
\newcommand{\MPDec}{ {\tt MP.Invert} }
\newcommand{\InvertSmallGadget}{ \ensuremath{{\texttt{InvertSmallGadget}}} }
\newcommand{\InvertGadget}{ {\tt InvertGadget} }
% \newcommand{\MPEval}{ $g$ }


\DeclareMathOperator{\RoundMod}{RoundMod}


% https://tex.stackexchange.com/a/253746/116348
\usepackage{mleftright}
\newcommand{\BlockMatrix}[2]{%
  \mleft[%
  \begin{array}{@{}#1@{}}%
    #2
  \end{array}%
  \mright]%
}
\newcommand{\SmallBlockMatrix}[2]{
  \mleft[\begin{array}{c}
    #1\tabularnewline % https://tex.stackexchange.com/questions/328745/matrices-with-cryptocode-package
    \hline
    #2
  \end{array}\mright]
}
\newcommand{\HorizBlockMatrix}[2]{
  \mleft[\begin{array}{@{}c|c@{}}
    #1 & #2
  \end{array}\mright]
}

\newcommand*{\vect}[1]{\ensuremath{\mathbf{#1}}}


%% Already defined in *recent* cryptocode.
% \DeclareMathOperator*{\argmax}{arg \, max}
% \DeclareMathOperator*{\argmin}{arg \, min}

\usepackage{xspace}
\xspaceaddexceptions{]\}}
\newcommand{\RSP}{\ensuremath{\sf{RSP}}\xspace}%

\newcommand*{\RSPenc}{\mathrm{RSP}_{\mathrm{enc}}}
\newcommand*{\filter}{\vdash}
\newcommand*{\channelBB}{\cS_{\Z\frac{\pi}{2}}}
% To denote a protocol between two parties
% \newcommand*{\protocol}{\boldsymbol{\pi}}
\newcommand\compRSP{composable $\sf{RSP}_{CC}$}
\newcommand\RSPlike{$\sf{RSP}_{CC}$}
\newcommand\compUBQC{composable $\sf{UBQC}_{CC}$}
\newcommand\UBQClike{$\sf{UBQC}_{CC}$}
% GHZ
\newcommand*{\eqdef}{\coloneqq}
\newcommand*{\PERM}{\ensuremath{}{\sf PERM}}
\newcommand*{\PPT}{\ensuremath{}{\sf PPT}}
\newcommand*{\QPT}{\ensuremath{}{\sf QPT}}
\newcommand*{\Auth}{\ensuremath{}{\sf Auth}}
% Prover, verifier, extractor
\renewcommand*{\P}{\ensuremath{}{\sf P}}
\newcommand*{\V}{\ensuremath{}{\sf V}}
\renewcommand*{\K}{\ensuremath{}{\sf K}}


% Here are the different protocols we use
% Initial protocol
\newcommand{\blind}{\ensuremath{ {\tt BLIND} }}
\newcommand{\blindZK}{\ensuremath{ {\tt BLIND-ZK} }}
% Reveals to people if they are part of the support
\newcommand{\blindSup}{\ensuremath{ {\tt BLIND}^{\tt sup} }}
% Make sure that people in the support share a canonical GHZ
\newcommand{\blindCan}{\ensuremath{ {\tt BLIND}_{\tt can} }}
% Make sure that people in the support share a canonical GHZ and that they know they are supported
\newcommand{\blindCanSup}{\ensuremath{ {\tt BLIND}^{\tt sup}_{\tt can} }}
\newcommand{\authBlindCanDist}{\ensuremath{ {\tt AUTH-BLIND}^{\tt dist}_{\tt can} }}
\newcommand{\verifCanSup}{\ensuremath{ {\tt VERIF}^{\tt sup}_{\tt can} }}
\newcommand{\blindPoly}{\ensuremath{ {\tt BLIND-poly} }}
\newcommand{\INDFCT}{\ensuremath{ {\tt IND-D0} }}
\newcommand{\INDBLIND}{\refGame*{INDBLIND}}
\newcommand{\INDBLINDtxt}{\ensuremath{ {\tt IND-\blind{}} }}
\newcommand{\INDBLINDSUP}{\refGame*{INDBLINDSUP}}
\newcommand{\INDBLINDSUPtxt}{\ensuremath{ {\tt IND-\blindSup{}} }}
\newcommand{\INDBLINDCANSUP}{\ensuremath{ {\tt IND-\blindCanSup{}} }}
\newcommand{\INDBLINDCANAUTH}{\ensuremath{ {\tt IND-\authBlindCanDist{}} }}
\newcommand{\INDPARTIAL}{\ensuremath{ {\tt IND-PARTIAL} }}
\newcommand*\GHZ{\ensuremath{}{\sf GHZ}}
\newcommand*\GHZCan{\ensuremath{{\sf GHZ^{\sf can}}}}
% Generalized GGHZ
\newcommand*\GGHZ{\ensuremath{ {\sf GHZ^G} }}
\newcommand*\HGHZ{\ensuremath{ {\sf GHZ^H} }}
\newcommand{\AssumpFctNoDelta}{\HGHZ{} capable}
\newcommand{\AssumpFctCanNoDelta}{\GHZCan{} capable}
\newcommand{\AssumpFct}{$\delta$-\HGHZ{} capable}
\newcommand{\AssumpFctNegl}{$\negl[\lambda]$-\HGHZ{} capable}
\newcommand{\AssumpFctCan}{$\delta$-\GHZCan{} capable}
\newcommand{\AssumpFctCanPrime}{$\delta^\prime$-\GHZCan{} capable}
\newcommand{\AssumpFctDist}{distributable}
\newcommand{\partialInfo}{\ensuremath{{\tt PartInfo}}}
\newcommand{\partialInfoLocal}{\ensuremath{{\tt PartInfo_{\tt Loc}}}}
\newcommand{\partialAlpha}{\ensuremath{{\tt PartAlpha_{\tt Loc}}}}
\newcommand{\CombineAlpha}{\ensuremath{{\tt CombineAlpha}}}
\newcommand{\CombineRandom}{\ensuremath{{\tt CombineRandom}}}
\newcommand{\CheckHonestTrapdoor}{\ensuremath{{\tt CheckTrapdoor}}}
\newcommand{\ZKGenCheck}{\ensuremath{{\tt ZK-GenCheck}}}

\newcommand{\highlightChanges}[1]{\textcolor{green!50!black}{#1}}

\newcommand{\Real}{\ensuremath{\mathsf{REAL}}}
\newcommand{\Sim}{\ensuremath{\mathsf{Sim}}}
\newcommand{\Ideal}{\ensuremath{\mathsf{IDEAL}}}
\newcommand{\OUT}{\ensuremath{\mathsf{OUT}}}

\usepackage{pifont}% http://ctan.org/pkg/pifont
% To denote that a user is not in the final GHZ
\newcommand{\cross}{\ensuremath{\text{\ding{55}}}}

\usepackage{verbatim}
\newcommand{\leftI}{{\normalfont\ttfamily left}}
\newcommand{\rightI}{{\normalfont\ttfamily right}}
\newcommand*{\mybar}[1]{\overline{#1}}

\newcommand*{\quout}{\mathtt{out}}

\DeclareMathOperator{\Image}{Image}
\DeclareMathOperator{\Ima}{Im}
\DeclareMathOperator{\Dom}{Dom}

\newcommand*{\Pub}{\mathtt{Pub}}
\newcommand*{\accept}{\mathtt{accept}}
\newcommand*{\abort}{\mathtt{abort}}
\newcommand*{\gadxor}{\mathtt{Gad}_{\oplus}}
\newcommand*{\quin}{\mathtt{in}}

%%% Provide a "subproof" environment (pretty simple for now,
%%% just indent the sub-proofs)
% I'd love to have left border... https://tex.stackexchange.com/questions/532948/robustly-add-a-border-to-the-left-of-a-text-spanning-several-pages
\usepackage{changepage}% http://ctan.org/pkg/changepage
\newenvironment{subproof}{\begin{adjustwidth}{2em}{0pt}}{\end{adjustwidth}}


%%% To have nice probabilities
% https://tex.stackexchange.com/questions/198771/align-in-substack
\makeatletter
\newcommand{\subalign}[1]{%
  \vcenter{%
    \Let@ \restore@math@cr \default@tag
    \baselineskip\fontdimen10 \scriptfont\tw@
    \advance\baselineskip\fontdimen12 \scriptfont\tw@
    \lineskip\thr@@\fontdimen8 \scriptfont\thr@@
    \lineskiplimit\lineskip
    \ialign{\hfil$\m@th\scriptstyle##$&$\m@th\scriptstyle{}##$\hfil\crcr
      #1\crcr
    }%
  }%
}
\makeatother
\usepackage{xparse}
% https://tex.stackexchange.com/questions/431794/same-spacing-around-middle-as-around-mid
\let\originalmiddle=\middle
\def\middle#1{\mathrel{}\originalmiddle#1\mathrel{}}
\NewDocumentCommand{\pr}{som}{\Pr\IfValueT{#2}{_{\substack{#2}}}\left[\,#3\,\right]}
\NewDocumentCommand{\esp}{som}{\IfValueTF{#2}{\underset{\subalign{#2}}{\mathbb{E}}}{\mathbb{E}}[\,#3\,]}
\NewDocumentEnvironment{pral}{soO{}}
 {%
  \Pr\IfValueT{#2}{_{\IfBooleanTF{#1}{\substack{#2}}{#2}}}
\begin{alignedat}[t]{2}
  [\, } {\,]
 #3
 \end{alignedat}
 }
\NewDocumentEnvironment{espal}{soO{}}
 {%
   \IfValueTF{#2}{\underset{\subalign{#2}}{\mathbb{E}}}{\mathbb{E}}
   \begin{alignedat}[t]{2}[\,
 }
 {\,]#3\end{alignedat}
 }
% To draw |A><B|, the second argument is facultative and gives |A><A|
\NewDocumentCommand{\ketbra}{mg}{
  | #1 \rangle \langle \IfValueTF{#2}{#2}{#1} |
}
% https://tex.stackexchange.com/a/188364/116348
\DeclareRobustCommand{\minwidthbox}[2]{%
  \ifmmode
    \expandafter\mathmakebox
  \else
    \expandafter\makebox
  \fi
  [\ifdim#2<\width\width\else#2\fi]{#1}%f
}

\NewDocumentCommand{\constructs}{mg}{
  \xrightarrow[\IfValueT{#2}{#2}]{\minwidthbox{#1}{5mm}}
}

%%%%%%%%%% This part is specific to llncs-based papers
% To enable this part of the configuration, add
%  \def\enableLlncsCode{}
% before the %% Let's put in these file all the definitions that we
%% require:

%% Setup language, encoding...
\RequirePackage{etex} 

\usepackage{standalone}
\usepackage{lmodern}
% Language
\usepackage[english]{babel}
% Input encoding
\usepackage[utf8]{inputenc}
% Output encoding https://tex.stackexchange.com/a/677. Important to copy accents
\usepackage[T1]{fontenc}
\usepackage{subfiles} % To quickly load this include file other files
\usepackage{microtype}
\usepackage[shortcuts,nospacearound]{extdash}    % Better endling of em dash (---) to cut sentences\===like 
%%% https://tex.stackexchange.com/questions/2773/how-do-i-make-latex-push-long-citations-to-a-new-line
\usepackage{breakcites}
\usepackage{booktabs}
% Makes sure floats don't fly accross sections
\usepackage{placeins}
\usepackage{adjustbox}
\usepackage{varwidth} % Like minipage, but with automatic width discovery.
\usepackage{complexity} % get \QMA{}...

\usepackage[normalem]{ulem}
%\usepackage[math-style=ISO]{unicode-math}
\usepackage{upgreek} % non italic greak letters

%%%% Bibliography
% I want to have separate references for appendix and main body
% https://tex.stackexchange.com/questions/98660/
\usepackage[style=trad-alpha,
  sortcites=true,
  doi=false,
  url=false,
  giveninits=true, % Bob Foo --> B. Foo
  isbn=false,
  url=false,
  eprint=false,
  sortcites=false, % \cite{B,A,C}: [A,B,C] --> [B,A,C]
]{biblatex}
\defbibfilter{appendixOnlyFilter}{
  segment=1 % Segment 1 will be chosen to be the one in appendix
  and not segment=0 % Default segment is 0
}
\renewcommand{\multicitedelim}{, } % [ABC96; DEF12] -> [ABC96, DEF12]
% [unknown_ref] => [??]
% https://tex.stackexchange.com/a/352573
\makeatletter
\protected\def\abx@missing#1{\textbf{??}}
\makeatother
\renewcommand*{\bibfont}{\normalfont\small} % BibLatex font seems bigger than Bibtex?

\usepackage{xcolor,cancel}
\usepackage{proof-at-the-end}
\NewDocumentEnvironment{theoremE}{O{}O{}+b}{%
  \begin{theoremEnd}[normal,#2]{theorem}[#1]%
    #3%
  \end{theoremEnd}
  %
}{}% Do not forget the second parameter or you might get Missing \begin{document} error
\NewDocumentEnvironment{lemmaE}{O{}O{}+b}{%
  \begin{theoremEnd}[normal,#2]{lemma}[#1]%
    #3%
  \end{theoremEnd}
  %
}{}% Do not forget the second parameter or you might get Missing \begin{document} error
\NewDocumentEnvironment{corollaryE}{O{}O{}+b}{%
  \begin{theoremEnd}[normal,#2]{corollary}[#1]%
    #3%
  \end{theoremEnd}
  %
}{}% Do not forget the second parameter or you might get Missing \begin{document} error
\NewDocumentEnvironment{proofE}{O{}+b}{%
  \begin{proofEnd}[#1]%
    #2
  \end{proofEnd}%
}{}%
\pgfkeys{/prAtEnd/global custom defaults/.style={
    text link={\noindent See \hyperref[proof:prAtEnd\pratendcountercurrent]{proof} in \cref{proofsection:prAtEnd\pratendcountercurrent}.}
  }
}

\ifdefined\displayArxivTexts%
  \pgfkeys{/prAtEnd/end if published/.style={no restate}}
  \pgfkeys{/prAtEnd/all end if published/.style={no restate}}
\else%
  \pgfkeys{/prAtEnd/end if published/.style={end}}
  \pgfkeys{/prAtEnd/all end if published/.style={all end}}
\fi

% https://tex.stackexchange.com/a/20613/116348
% https://tex.stackexchange.com/a/583244/116348
\usepackage{scalerel}
\newcommand\hcancel[2][black]{%
  \ThisStyle{%
    \setbox0=\hbox{$\SavedStyle #2$}%
    \rlap{\raisebox{.25\ht0}{\textcolor{#1}{\rule{\wd0}{1pt}}}}\SavedStyle #2}%
}

%% We provide 3 commands to add comments/text:
%% - \yournameAddText{your text to add}: proposition to add text
%% - \yournameRemoveText{text to remove}: proposition to remove text
%% - \yournameComment{your comments}: to add general comment to discuss. Should not contain proposition to add or remove text, but can be added before/after to explain the reason of the modification.
%% The interest it to provide some quick ways to see what is the final number of pages:
%% just add before the import a line:
%%   \def\removeCommentsAcceptPropositions{}
%% and the proposition will be "accepted" (i.e. the removed text is removed,
%% and the added text will be marked as black), and the comment will be removed, that way it is possible
%% to determine the final number of pages.
%% Example:
%% This text is normal\leoRemove{, this text should be removed}\leoAddText{, this text should be added}\leoComment{This is a comment, explaining why I made that changes.}.
\newcommand{\addEditor}[2]{%
  \expandafter\newcommand\csname #1AddText\endcsname[2][]{%
    \ifdefined\removeCommentsAcceptPropositions %
      ##2%
    \else%
      \textcolor{#2}{\sout{##1}}\textcolor{#2!60!black}{##2}%
    \fi%
  }%
  \expandafter\newcommand\csname #1Comment\endcsname[1]{%
    \ifdefined\removeCommentsAcceptPropositions %
    \else%
      \textcolor{#2!60!white}{\textbf{##1}}%
    \fi%
  }%
  \expandafter\newcommand\csname #1Remove\endcsname[1]{%
    \ifdefined\removeCommentsAcceptPropositions %
    \else%
      \textcolor{#2}{\sout{##1}}%
    \fi%
  }%
}
\definecolor{darkgreen}{rgb}{0., 0.7, 0}
\addEditor{leo}{darkgreen}
\addEditor{fred}{red}
\addEditor{elham}{blue}

%% We also provide an arxivOnly environment, in order to include text that are too long for the submitted version
%% but that would be good to have in the arxiv version.
% Use the \begin{arxivOnly} ... \end{arxivOnly} environment, by putting the opening and ending command
%% on there own line.
%% For shorter texts, you can use the inline version \arxivOnlyShort{your text}.
%% If you want to exclude text from the arxiv, use \begin{publishedOnly} ... \end{publishedOnly} and \publishedOnlyShort{your text}
%% "Comment" package is annoying, it does not allow block indentation, we use 'versions' instead
% \usepackage{versions} %% Do not work inside theoremE, don't know why
\ifdefined\displayArxivTexts%
  % \includeversion{arxivOnly}
  % \excludeversion{publishedOnly}
  \newcommand{\arxivOnly}[1]{#1}
  \newcommand{\arxivOnlyNoDiff}[1]{#1} % Like arxivOnly, but useful in diff mode (when it does not make sense to add colors, for instance inside a cite or when using arxivonly to create lists.
  \newcommand{\publishedOnly}[1]{}
  \newcommand{\publishedVsArxiv}[2]{#2}
  \newcommand{\inAppendixIfPublished}[1]{#1}
\else%
  \ifdefined\displayDiffTexts%
    \colorlet{arxivDiff}{green}
    \colorlet{publishedDiff}{red}
    \newcommand{\arxivOnly}[1]{{\color{arxivDiff}#1}}
    \newcommand{\arxivOnlyNoDiff}[1]{#1}
    \newcommand{\publishedOnly}[1]{{\color{publishedDiff}#1}}
    \newcommand{\publishedVsArxiv}[2]{{\color{publishedDiff}#1}{\color{arxivDiff}#2}}
    \newcommand{\inAppendixIfPublished}[1]{{\color{arxivDiff}\textbf{MOVED IN APPENDIX IN PUBLISHED}#1}\textEnd{{\color{publishedDiff}\textbf{MOVED IN MAIN BODY IN ARXIV}#1}}}
  \else
    % \excludeversion{arxivOnly}
    % \includeversion{publishedOnly}
    \newcommand{\arxivOnly}[1]{}
    \newcommand{\arxivOnlyNoDiff}[1]{}
    \newcommand{\publishedOnly}[1]{#1}
    \newcommand{\publishedVsArxiv}[2]{#1}
    \newcommand{\inAppendixIfPublished}[1]{\textEnd{#1}}
  \fi
\fi
% Alias, easier to remember  
\newcommand{\inAppendix}[1]{\textEnd{#1}}
  
% Symbol to display when an arxiv text is supposed to appear
\usepackage{todonotes}
\newcommand{\notifyArxiv}{\ifdefined\arxivDraftMode\todo{arXiv}\fi}

% Remove unwanted space before/after lists
\usepackage{enumitem}
\publishedOnly{\setlist[itemize]{noitemsep, topsep=0pt}}

% Usual packages:
\usepackage{amsmath}
\usepackage{amssymb}
% \usepackage{amsthm}
\usepackage{thmtools}
\usepackage{graphicx}
\graphicspath{{}{figures/}}
\usepackage{float}
\usepackage{subfig}
\usepackage{wrapfig}
\usepackage{array}
\usepackage{braket}
% Refer to footnotes:
% https://tex.stackexchange.com/a/167382/116348
\usepackage{footmisc}
% Useful to use the latest NewDocumentCommand:
%\usepackage{xparse}
%\usepackage[poster]{tcolorbox}

%% Space between pictures and text
%% https://mirrors.chevalier.io/CTAN/macros/latex/contrib/layouts/layman.pdf
%% \usepackage{layouts}
%% https://tex.stackexchange.com/questions/60477/remove-space-after-figure-and-before-text?rq=1
\setlength{\floatsep}{7pt plus 2pt minus 2pt}
\setlength{\textfloatsep}{7pt plus 2pt minus 2pt}
\setlength{\intextsep}{7pt plus 2pt minus 2pt}
% https://tex.stackexchange.com/questions/39017/how-to-influence-the-position-of-float-environments-like-figure-and-table-in-lat
\setcounter{topnumber}{8}
\setcounter{bottomnumber}{8}
\setcounter{totalnumber}{8}
%% https://tex.stackexchange.com/questions/45990/how-can-i-modify-vertical-space-between-figure-and-caption
%% plus/minus allows to stretch a bit around 15pt
% \setlength{\abovecaptionskip}{5pt}

% Remove number to subsubsection (in llncs it replaces \paragraph)
\setcounter{secnumdepth}{2}

%%% Some practical environment to define theorems, lemma...
% \declaretheorem[name=Theorem,numberwithin=section]{theorem}
% \declaretheorem[name=Definition,sibling=theorem]{definition}
% \declaretheorem[name=Lemma,sibling=theorem]{lemma}
% \declaretheorem[name=Corollary,sibling=theorem]{corollary}
% \declaretheorem[name=Remark,sibling=theorem]{remark}
%%% In LNCS we use
% https://tex.stackexchange.com/a/373125/116348
% \spnewtheorem{thm}{Theorem}[section]{\bfseries}{\itshape}

%%%
\usepackage[ruled,vlined]{algorithm2e}
\SetAlFnt{\normalsize}%
\setlength{\algomargin}{0em} % remove margin on left of algo
\SetAlgorithmName{Protocol}{protocol}{List of Protocols}
\newenvironment{protocol}[1][htb]{\begin{algorithm}[#1]}{\end{algorithm}}

%%% %%% Load algorithm package, and create a breakable version
%%% \usepackage{algorithm, algorithmicx, algpseudocode}
%%% % Decrease space between text:
%%% % https://tex.stackexchange.com/questions/25828/how-to-remove-change-the-vertical-spacing-before-and-after-an-algorithm-enviro/25832
%%% \setlength{\intextsep}{1\baselineskip}
%%% % https://tex.stackexchange.com/questions/387577/new-algorithm-environment-with-a-separate-counter
%%% \newcounter{protocol}
%%% \makeatletter
%%% \newenvironment{protocol}[1][htb]{%
%%%   \let\c@algorithm\c@protocol
%%%   \renewcommand{\ALG@name}{Protocol}% Update algorithm name
%%%   \begin{algorithm}[#1]%
%%%   }{\end{algorithm}
%%% }
%%% \makeatother
%%%
%%% \providecommand*\algorithmautorefname{Protocol}
%%% \makeatletter
%%% \newenvironment{breakablealgorithm}
%%%   {% \begin{breakablealgorithm}
%%%    % \begin{center}
%%%      \refstepcounter{algorithm}% New algorithm
%%%      \hrule height.8pt depth0pt \kern2pt% \@fs@pre for \@fs@ruled
%%%      \renewcommand{\caption}[2][\relax]{% Make a new \caption
%%%        % {\raggedright\textbf{\ALG@name~\thealgorithm} ##2\par}%
%%%        {\raggedright\textbf{Protocol~\thealgorithm} ##2\par}%
%%%        \ifx\relax##1\relax % #1 is \relax
%%%          \addcontentsline{loa}{algorithm}{\protect\numberline{\thealgorithm}##2}%
%%%        \else % #1 is not \relax
%%%          \addcontentsline{loa}{algorithm}{\protect\numberline{\thealgorithm}##1}%
%%%        \fi
%%%        \kern2pt\hrule\kern2pt
%%%        \small
%%%      }
%%%   }{% \end{breakablealgorithm}
%%%      \kern2pt\hrule\relax% \@fs@post for \@fs@ruled
%%%    % \end{center}
%%%   }
%%% \makeatother
%%%

\def\EQ#1{\begin{eqnarray}#1\end{eqnarray}} 


%%% Practical abreviations
\usepackage{calrsfs} % Replace \mathcal with different more "cursive" font
\DeclareMathAlphabet{\pazocal}{OMS}{zplm}{m}{n} % \pazocal is now the old \mathcal
\newcommand*{\cA}{\pazocal{A}}
\newcommand*{\cB}{\pazocal{B}}
\newcommand*{\cC}{\pazocal{C}}
\newcommand*{\cD}{\pazocal{D}}
\newcommand*{\cE}{\pazocal{E}}
\newcommand*{\cF}{\pazocal{F}}
\newcommand*{\cG}{\pazocal{G}}
\newcommand*{\cH}{\pazocal{H}}
\newcommand*{\cI}{\pazocal{I}}
\newcommand*{\cJ}{\pazocal{J}}
\newcommand*{\cK}{\pazocal{K}}
\renewcommand*{\cL}{\pazocal{L}} % defined in complexity
\newcommand*{\cM}{\pazocal{M}}
\newcommand*{\cN}{\pazocal{N}}
\newcommand*{\cO}{\pazocal{O}}
\renewcommand*{\cP}{\pazocal{P}} % defined in complexity
\newcommand*{\cQ}{\pazocal{Q}}
\newcommand*{\cR}{\pazocal{R}}
\renewcommand*{\cS}{\pazocal{S}} % defined in complexity
\newcommand*{\cT}{\pazocal{T}}
\newcommand*{\cU}{\pazocal{U}}
\newcommand*{\cV}{\pazocal{V}}
\newcommand*{\cW}{\pazocal{W}}
\newcommand*{\cX}{\pazocal{X}}
\newcommand*{\cY}{\pazocal{Y}}
\newcommand*{\cZ}{\pazocal{Z}}

% Notation for languages. Make sure not to collide with L(H) for linear operators on H
%\DeclareMathAlphabet{\pazocal}{OMS}{zplm}{m}{n}
\newcommand{\linearOp}{\mathcal{L}} % Defines the opearator L(H)
\renewcommand*{\lang}{\pazocal{L}} % Defines a new Turing/Quantum Language

%https://tex.stackexchange.com/questions/195025/
\newcommand*{\rightSend}{\rightsquigarrow} % snake "a --> b" to say "I send a to b".

% For some reasons, proof-at-the-end do not like # (problem with detokenize? To fix...). So we use a macro:
\newcommand*{\diese}{\#}

\newcommand*{\eps}[0]{\varepsilon}
\renewcommand*{\D}[0]{\mathbb{D}}
\renewcommand*{\E}{\mathbb{E}}
\newcommand*{\N}[0]{\mathbb{N}}
\renewcommand*{\R}[0]{\mathbb{R}}
\renewcommand*{\C}[0]{\mathbb{C}}
\newcommand*{\Z}[0]{\mathbb{Z}}
\newcommand*{\T}[0]{\mathbb{T}}

\newcommand*{\CPA}[0]{\textsf{CPA}}
\newcommand*{\LWE}[0]{\textsf{LWE}}
\newcommand*{\GapSVP}[0]{\textsf{GapSVP}}
\newcommand*{\SIVP}[0]{\textsf{SIVP}}

% first param is alpha, second is q (optional)
\newcommand{\contGauss}[1]{\cD_{#1}}
\newcommand{\disGauss}[2]{\cD_{\Z,#1 #2}}
\newcommand{\disGaussCont}[1]{\cD_{\Z,#1}}
\newcommand{\disGaussAQ}{\cD_{\Z,\alpha q}}
\newcommand{\rsafe}{r_{\text{safe}}}
% domain of the function.
\newcommand{\ccX}{\cX}
\newcommand{\ccXSphere}{\cX_\bullet}
\newcommand{\ccXCube}[1][]{\cX_{\scalebox{.4}{$\blacksquare$}#1}}

\newcommand{\BBHeightyFor}{\textsc{BB84}}

% "Correct" spacing around cdot as wildcard
% https://tex.stackexchange.com/questions/78165/spacing-around-cdot-when-used-as-a-wildcard
\newcommand\cdotwild{{\mkern 2mu\cdot\mkern 2mu}}

\newcommand*{\id}{\mathrm{id}}
\DeclareMathOperator{\Tr}{Tr}
\DeclareMathOperator{\supp}{supp}
\DeclareMathOperator{\Det}{Det}
\DeclareMathOperator{\ERR}{ERR}
\DeclareMathOperator{\Round}{Round}

\newcommand{\Ver}{ {\tt Ver} } %% Verify public key
\newcommand{\Gen}{ {\tt Gen} }
\newcommand{\GenLocal}{\ensuremath{ {\tt Gen}_{\tt Loc}} }
\newcommand{\Enc}{ {\tt Enc} }
\newcommand{\Dec}{ {\tt Invert} }
\newcommand{\Eval}{ {\tt Eval} }

\newcommand{\HGen}{ {\tt H.Gen} }
\newcommand{\HEnc}{ {\tt H.Enc} }
\newcommand{\HDec}{ {\tt H.Invert} }
\newcommand{\HEval}{ {\tt H.Eval} }
\newcommand{\Hh}{\ensuremath{{\tt H.}h} }
\newcommand{\HCheckHonestTrapdoor}{\ensuremath{{\tt H.CheckTrapdoor}}}

\newcommand{\MPGen}{ {\tt MP.Gen} }
\newcommand{\MPDec}{ {\tt MP.Invert} }
\newcommand{\InvertSmallGadget}{ \ensuremath{{\texttt{InvertSmallGadget}}} }
\newcommand{\InvertGadget}{ {\tt InvertGadget} }
% \newcommand{\MPEval}{ $g$ }


\DeclareMathOperator{\RoundMod}{RoundMod}


% https://tex.stackexchange.com/a/253746/116348
\usepackage{mleftright}
\newcommand{\BlockMatrix}[2]{%
  \mleft[%
  \begin{array}{@{}#1@{}}%
    #2
  \end{array}%
  \mright]%
}
\newcommand{\SmallBlockMatrix}[2]{
  \mleft[\begin{array}{c}
    #1\tabularnewline % https://tex.stackexchange.com/questions/328745/matrices-with-cryptocode-package
    \hline
    #2
  \end{array}\mright]
}
\newcommand{\HorizBlockMatrix}[2]{
  \mleft[\begin{array}{@{}c|c@{}}
    #1 & #2
  \end{array}\mright]
}

\newcommand*{\vect}[1]{\ensuremath{\mathbf{#1}}}


%% Already defined in *recent* cryptocode.
% \DeclareMathOperator*{\argmax}{arg \, max}
% \DeclareMathOperator*{\argmin}{arg \, min}

\usepackage{xspace}
\xspaceaddexceptions{]\}}
\newcommand{\RSP}{\ensuremath{\sf{RSP}}\xspace}%

\newcommand*{\RSPenc}{\mathrm{RSP}_{\mathrm{enc}}}
\newcommand*{\filter}{\vdash}
\newcommand*{\channelBB}{\cS_{\Z\frac{\pi}{2}}}
% To denote a protocol between two parties
% \newcommand*{\protocol}{\boldsymbol{\pi}}
\newcommand\compRSP{composable $\sf{RSP}_{CC}$}
\newcommand\RSPlike{$\sf{RSP}_{CC}$}
\newcommand\compUBQC{composable $\sf{UBQC}_{CC}$}
\newcommand\UBQClike{$\sf{UBQC}_{CC}$}
% GHZ
\newcommand*{\eqdef}{\coloneqq}
\newcommand*{\PERM}{\ensuremath{}{\sf PERM}}
\newcommand*{\PPT}{\ensuremath{}{\sf PPT}}
\newcommand*{\QPT}{\ensuremath{}{\sf QPT}}
\newcommand*{\Auth}{\ensuremath{}{\sf Auth}}
% Prover, verifier, extractor
\renewcommand*{\P}{\ensuremath{}{\sf P}}
\newcommand*{\V}{\ensuremath{}{\sf V}}
\renewcommand*{\K}{\ensuremath{}{\sf K}}


% Here are the different protocols we use
% Initial protocol
\newcommand{\blind}{\ensuremath{ {\tt BLIND} }}
\newcommand{\blindZK}{\ensuremath{ {\tt BLIND-ZK} }}
% Reveals to people if they are part of the support
\newcommand{\blindSup}{\ensuremath{ {\tt BLIND}^{\tt sup} }}
% Make sure that people in the support share a canonical GHZ
\newcommand{\blindCan}{\ensuremath{ {\tt BLIND}_{\tt can} }}
% Make sure that people in the support share a canonical GHZ and that they know they are supported
\newcommand{\blindCanSup}{\ensuremath{ {\tt BLIND}^{\tt sup}_{\tt can} }}
\newcommand{\authBlindCanDist}{\ensuremath{ {\tt AUTH-BLIND}^{\tt dist}_{\tt can} }}
\newcommand{\verifCanSup}{\ensuremath{ {\tt VERIF}^{\tt sup}_{\tt can} }}
\newcommand{\blindPoly}{\ensuremath{ {\tt BLIND-poly} }}
\newcommand{\INDFCT}{\ensuremath{ {\tt IND-D0} }}
\newcommand{\INDBLIND}{\refGame*{INDBLIND}}
\newcommand{\INDBLINDtxt}{\ensuremath{ {\tt IND-\blind{}} }}
\newcommand{\INDBLINDSUP}{\refGame*{INDBLINDSUP}}
\newcommand{\INDBLINDSUPtxt}{\ensuremath{ {\tt IND-\blindSup{}} }}
\newcommand{\INDBLINDCANSUP}{\ensuremath{ {\tt IND-\blindCanSup{}} }}
\newcommand{\INDBLINDCANAUTH}{\ensuremath{ {\tt IND-\authBlindCanDist{}} }}
\newcommand{\INDPARTIAL}{\ensuremath{ {\tt IND-PARTIAL} }}
\newcommand*\GHZ{\ensuremath{}{\sf GHZ}}
\newcommand*\GHZCan{\ensuremath{{\sf GHZ^{\sf can}}}}
% Generalized GGHZ
\newcommand*\GGHZ{\ensuremath{ {\sf GHZ^G} }}
\newcommand*\HGHZ{\ensuremath{ {\sf GHZ^H} }}
\newcommand{\AssumpFctNoDelta}{\HGHZ{} capable}
\newcommand{\AssumpFctCanNoDelta}{\GHZCan{} capable}
\newcommand{\AssumpFct}{$\delta$-\HGHZ{} capable}
\newcommand{\AssumpFctNegl}{$\negl[\lambda]$-\HGHZ{} capable}
\newcommand{\AssumpFctCan}{$\delta$-\GHZCan{} capable}
\newcommand{\AssumpFctCanPrime}{$\delta^\prime$-\GHZCan{} capable}
\newcommand{\AssumpFctDist}{distributable}
\newcommand{\partialInfo}{\ensuremath{{\tt PartInfo}}}
\newcommand{\partialInfoLocal}{\ensuremath{{\tt PartInfo_{\tt Loc}}}}
\newcommand{\partialAlpha}{\ensuremath{{\tt PartAlpha_{\tt Loc}}}}
\newcommand{\CombineAlpha}{\ensuremath{{\tt CombineAlpha}}}
\newcommand{\CombineRandom}{\ensuremath{{\tt CombineRandom}}}
\newcommand{\CheckHonestTrapdoor}{\ensuremath{{\tt CheckTrapdoor}}}
\newcommand{\ZKGenCheck}{\ensuremath{{\tt ZK-GenCheck}}}

\newcommand{\highlightChanges}[1]{\textcolor{green!50!black}{#1}}

\newcommand{\Real}{\ensuremath{\mathsf{REAL}}}
\newcommand{\Sim}{\ensuremath{\mathsf{Sim}}}
\newcommand{\Ideal}{\ensuremath{\mathsf{IDEAL}}}
\newcommand{\OUT}{\ensuremath{\mathsf{OUT}}}

\usepackage{pifont}% http://ctan.org/pkg/pifont
% To denote that a user is not in the final GHZ
\newcommand{\cross}{\ensuremath{\text{\ding{55}}}}

\usepackage{verbatim}
\newcommand{\leftI}{{\normalfont\ttfamily left}}
\newcommand{\rightI}{{\normalfont\ttfamily right}}
\newcommand*{\mybar}[1]{\overline{#1}}

\newcommand*{\quout}{\mathtt{out}}

\DeclareMathOperator{\Image}{Image}
\DeclareMathOperator{\Ima}{Im}
\DeclareMathOperator{\Dom}{Dom}

\newcommand*{\Pub}{\mathtt{Pub}}
\newcommand*{\accept}{\mathtt{accept}}
\newcommand*{\abort}{\mathtt{abort}}
\newcommand*{\gadxor}{\mathtt{Gad}_{\oplus}}
\newcommand*{\quin}{\mathtt{in}}

%%% Provide a "subproof" environment (pretty simple for now,
%%% just indent the sub-proofs)
% I'd love to have left border... https://tex.stackexchange.com/questions/532948/robustly-add-a-border-to-the-left-of-a-text-spanning-several-pages
\usepackage{changepage}% http://ctan.org/pkg/changepage
\newenvironment{subproof}{\begin{adjustwidth}{2em}{0pt}}{\end{adjustwidth}}


%%% To have nice probabilities
% https://tex.stackexchange.com/questions/198771/align-in-substack
\makeatletter
\newcommand{\subalign}[1]{%
  \vcenter{%
    \Let@ \restore@math@cr \default@tag
    \baselineskip\fontdimen10 \scriptfont\tw@
    \advance\baselineskip\fontdimen12 \scriptfont\tw@
    \lineskip\thr@@\fontdimen8 \scriptfont\thr@@
    \lineskiplimit\lineskip
    \ialign{\hfil$\m@th\scriptstyle##$&$\m@th\scriptstyle{}##$\hfil\crcr
      #1\crcr
    }%
  }%
}
\makeatother
\usepackage{xparse}
% https://tex.stackexchange.com/questions/431794/same-spacing-around-middle-as-around-mid
\let\originalmiddle=\middle
\def\middle#1{\mathrel{}\originalmiddle#1\mathrel{}}
\NewDocumentCommand{\pr}{som}{\Pr\IfValueT{#2}{_{\substack{#2}}}\left[\,#3\,\right]}
\NewDocumentCommand{\esp}{som}{\IfValueTF{#2}{\underset{\subalign{#2}}{\mathbb{E}}}{\mathbb{E}}[\,#3\,]}
\NewDocumentEnvironment{pral}{soO{}}
 {%
  \Pr\IfValueT{#2}{_{\IfBooleanTF{#1}{\substack{#2}}{#2}}}
\begin{alignedat}[t]{2}
  [\, } {\,]
 #3
 \end{alignedat}
 }
\NewDocumentEnvironment{espal}{soO{}}
 {%
   \IfValueTF{#2}{\underset{\subalign{#2}}{\mathbb{E}}}{\mathbb{E}}
   \begin{alignedat}[t]{2}[\,
 }
 {\,]#3\end{alignedat}
 }
% To draw |A><B|, the second argument is facultative and gives |A><A|
\NewDocumentCommand{\ketbra}{mg}{
  | #1 \rangle \langle \IfValueTF{#2}{#2}{#1} |
}
% https://tex.stackexchange.com/a/188364/116348
\DeclareRobustCommand{\minwidthbox}[2]{%
  \ifmmode
    \expandafter\mathmakebox
  \else
    \expandafter\makebox
  \fi
  [\ifdim#2<\width\width\else#2\fi]{#1}%f
}

\NewDocumentCommand{\constructs}{mg}{
  \xrightarrow[\IfValueT{#2}{#2}]{\minwidthbox{#1}{5mm}}
}

%%%%%%%%%% This part is specific to llncs-based papers
% To enable this part of the configuration, add
%  \def\enableLlncsCode{}
% before the \input{include}
\ifdefined\enableLlncsCode%
% go back to standard amsthm proof environments with Qed at the end
\let\proof\relax\let\endproof\relax
  \usepackage{amsthm}
  % To debug, it's a nightmare without page number... To disable
  % in the final version
  \pagestyle{plain}
\fi

% Define environments when loaded in standalone...
\ifdefined\enableNonLlncsCode%
\usepackage{thmtools}
\declaretheorem[name=Theorem,numberwithin=chapter]{theorem}
\declaretheorem[name=Definition,sibling=theorem]{definition}
\declaretheorem[name=Lemma,sibling=theorem]{lemma}
\declaretheorem[name=Corollary,sibling=theorem]{corollary}
\declaretheorem[name=Remark,sibling=theorem]{remark}
\fi

\usepackage{enumitem}
\newlist{compressedList}{itemize}{10}
\setlist[compressedList]{label=--,topsep=0pt,parsep=0pt,partopsep=0pt}
\usepackage{mathtools}
\DeclarePairedDelimiter\ceil{\lceil}{\rceil}
\DeclarePairedDelimiter\floor{\lfloor}{\rfloor}

% https://tex.stackexchange.com/questions/134278/overriding-the-paragraph-style-in-the-lncs-document-class
\usepackage{etoolbox}
\patchcmd{\paragraph}{\itshape}{\bfseries\boldmath}{}{}


% This part is specific to llncs-based papers with nicer output
% (page number, theorem labelled with sections...). This should
% be the version in the arxiv.
% To enable this part of the configuration, add
%  \def\nicerThanLlncs{}
% before the \input{include}
\ifdefined\nicerThanLlncs%
  % change margins, page layout
  \usepackage[a4paper, top=1.4in, bottom=1.4in, right=1in, left=1in]{geometry}
  % enable page numbering, suppressed by default by llncs
  \pagestyle{plain}
  \setcounter{tocdepth}{2}
  \renewcommand\small{\fontsize{10pt}{12pt}\selectfont}
\fi
\ifdefined\nicerThanLlncsAbstract%
  % change margins, page layout
  \usepackage[a4paper, margin=2.5cm]{geometry}
  % enable page numbering, suppressed by default by llncs
  \pagestyle{plain}
  \setcounter{tocdepth}{2}
\fi
\ifdefined\enableLlncsCodeSlightlyImproved%
  \pagestyle{plain}
  \setcounter{tocdepth}{2}
\fi


\ifdefined\disableHyperref%
\else%
% https://rcweb.dartmouth.edu/doc/texmf-dist/doc/latex/cryptocode/cryptocode.pdf (page 55)
%%%%%%%%%% This part should be loaded last
% Usually hyperref needs to be at the end
\definecolor{secondaryColor}{RGB}{206,149,0} %% darker orange, looks like gold. <3
\usepackage[colorlinks, allcolors=secondaryColor]{hyperref}
\fi
%% Hyperref should be loaded before cryptocode! Otherwise, errors with labels of lines for example.
%%% Cryptocode
\usepackage [
n,
advantage,
operators,
sets,
adversary,
landau,
probability,
notions,
logic,
ff,
mm,
primitives,
events,
complexity,
asymptotics,
keys
] {cryptocode}
% I don't like the way "samples" is defined in cryptocode with a big dollar in front.
% \renewcommand\sample{\mathrel{\overset{\vbox to.7ex{\hbox{\fontsize{6}{0}\selectfont\vspace{.5ex} \$}}}{\leftarrow}}}
%% https://tex.stackexchange.com/a/584533/116348
\makeatletter
\renewcommand{\sample}{\mathrel{\mathpalette\sample@\relax}}
\newcommand{\sample@}[2]{%
  \ooalign{%
    \hspace{\stretch{2}}\raisebox{0.7\height}{$\m@th\demotestyle{#1}\mathdollar$}\hspace{\stretch{1}}\cr
    $\m@th#1\leftarrow$\cr
  }%
}
\newcommand{\demotestyle}[1]{%
  \ifx#1\displaystyle\scriptstyle\else
  \ifx#1\textstyle\scriptstyle\else
  \scriptscriptstyle\fi\fi
}
\makeatother
% Create a new command to create games using \game[linenumbering]{Name of game}{Game definition}
\createprocedureblock{game}{center,boxed}{}{}{}
% Create a new environment (following what I did in the thesis) to provide named games
\newcommand{\gameFont}[1]{\texttt{#1}}
\renewcommand{\Game}[1]{\ensuremath{\gameFont{Game#1}}} % By default \Game produces a nice G (symetric), but it's not very standard.

\makeatletter
\createprocedureblock{gameProc}{center,boxed}{}{}{}
\createprocedureblock{interactGame}{center,boxed}{}{}{}
%%% Nicer way to refer to games.
%%% See also https://tex.stackexchange.com/questions/623163 which provides a cleaner way to make it work with cleverref.
% Usage: \begin{namedGame}*[label][short title]{title}*[game options like skipfirstln,lnstart=1] content \end{namedGame}
% first * = disable floating. WARNING: do not put labels with non-letter chars.
% second * = do not include the game in the TOC.
\NewDocumentEnvironment{namedGame}{soomsO{}b}{%
  \IfBooleanTF{#1}{}{%
    \par% without par, the figure will be put after the line arriving after the picture if there is no paragraph.
    \setlength{\intextsep}{5.0pt plus 2.0pt minus 2.0pt}% reduce the space when image is inserted inline.
    \begin{figure}[htbp]}% 
    \begin{pcimage}%
      %% Add to toc see command listofgames
      \phantomsection% Create a fake label, useful to ensure the link point to this part.
      \IfBooleanTF{#5}{}{% If no star is given, add to "table of contents" for games
        \IfValueTF{#3}{% If a short title is provided
          \addcontentsline{game}{section}{\ensuremath{#3}}%
        }{%
          \addcontentsline{game}{section}{\ensuremath{#4}}%
        }%
      }%
      {\normalfont\gameProc[linenumbering,#6]{\ensuremath{#4}}{\IfValueTF{#2}{%
            %% Add an anchor if a label is present
            \raisebox{1em}{\hypertarget{#2}{}}%
            %% Create a macro "mygametitle@nameoflabel" to store the title
            \IfValueTF{#3}{% If a short title is provided
              \expandafter\gdef\csname mygametitle@#2 \endcsname{\ensuremath{#3}}%%
              \write\@auxout{\gdef\string\mygametitle@#2{\ensuremath{#3}}}%
            }{%
              \expandafter\gdef\csname mygametitle@#2 \endcsname{\ensuremath{#4}}%%
              \write\@auxout{\gdef\string\mygametitle@#2{\ensuremath{#4}}}%
            }%
          }{} #7 }}%
    \end{pcimage}
    \IfBooleanTF{#1}{}{\end{figure}}%
}{}

% Usage: \refGame{label}. Use \refGame*{label} if you don't want to color the link (useful in proofs)
\NewDocumentCommand{\refGame}{sm}{%
  \IfBooleanTF{#1}{%
    \hyperlink{#2}{\textcolor{black}{\csname mygametitle@#2\endcsname}}% Do not put a white space after #1!
  }{%
    \hyperlink{#2}{\csname mygametitle@#2\endcsname}% Do not put a white space after #1!
  }%
}
\makeatother

\ifdefined\disableHyperref%
\else%
% Well... not as much as cleveref
\usepackage[capitalise,noabbrev]{cleveref}
% Don't abbrev general names... but still abbrev equations
\crefname{equation}{Eq.}{Eqs.}
\crefname{figure}{Fig.}{Figs.}
\crefname{algorithm}{Protocol}{Protocols}
\Crefname{equation}{Equation}{Equations}
\Crefname{figure}{Figure}{Figures}
\Crefname{algorithm}{Protocol}{Protocols}
\fi

% Put that package at the end of the file,
% or you will have some strange errors like
% "Only one # is allowed per tab"
\usetikzlibrary{quantikz}
% Bug in llncs https://tex.stackexchange.com/a/372108/116348
\def\theHlemma{\theHtheorem}
\def\theHdefinition{\theHtheorem}
\def\theHcorollary{\theHtheorem}

%%%%%% PLEASE DO NOT WRITE ANYTHING AFTER THIS LINE %%%%%%
%%%%%% Indeed, hyperref, cleveref and quantikz need %%%%%%
%%%%%% to be loaded at the very end.                %%%%%%

\ifdefined\enableLlncsCode%
% go back to standard amsthm proof environments with Qed at the end
\let\proof\relax\let\endproof\relax
  \usepackage{amsthm}
  % To debug, it's a nightmare without page number... To disable
  % in the final version
  \pagestyle{plain}
\fi

% Define environments when loaded in standalone...
\ifdefined\enableNonLlncsCode%
\usepackage{thmtools}
\declaretheorem[name=Theorem,numberwithin=chapter]{theorem}
\declaretheorem[name=Definition,sibling=theorem]{definition}
\declaretheorem[name=Lemma,sibling=theorem]{lemma}
\declaretheorem[name=Corollary,sibling=theorem]{corollary}
\declaretheorem[name=Remark,sibling=theorem]{remark}
\fi

\usepackage{enumitem}
\newlist{compressedList}{itemize}{10}
\setlist[compressedList]{label=--,topsep=0pt,parsep=0pt,partopsep=0pt}
\usepackage{mathtools}
\DeclarePairedDelimiter\ceil{\lceil}{\rceil}
\DeclarePairedDelimiter\floor{\lfloor}{\rfloor}

% https://tex.stackexchange.com/questions/134278/overriding-the-paragraph-style-in-the-lncs-document-class
\usepackage{etoolbox}
\patchcmd{\paragraph}{\itshape}{\bfseries\boldmath}{}{}


% This part is specific to llncs-based papers with nicer output
% (page number, theorem labelled with sections...). This should
% be the version in the arxiv.
% To enable this part of the configuration, add
%  \def\nicerThanLlncs{}
% before the %% Let's put in these file all the definitions that we
%% require:

%% Setup language, encoding...
\RequirePackage{etex} 

\usepackage{standalone}
\usepackage{lmodern}
% Language
\usepackage[english]{babel}
% Input encoding
\usepackage[utf8]{inputenc}
% Output encoding https://tex.stackexchange.com/a/677. Important to copy accents
\usepackage[T1]{fontenc}
\usepackage{subfiles} % To quickly load this include file other files
\usepackage{microtype}
\usepackage[shortcuts,nospacearound]{extdash}    % Better endling of em dash (---) to cut sentences\===like 
%%% https://tex.stackexchange.com/questions/2773/how-do-i-make-latex-push-long-citations-to-a-new-line
\usepackage{breakcites}
\usepackage{booktabs}
% Makes sure floats don't fly accross sections
\usepackage{placeins}
\usepackage{adjustbox}
\usepackage{varwidth} % Like minipage, but with automatic width discovery.
\usepackage{complexity} % get \QMA{}...

\usepackage[normalem]{ulem}
%\usepackage[math-style=ISO]{unicode-math}
\usepackage{upgreek} % non italic greak letters

%%%% Bibliography
% I want to have separate references for appendix and main body
% https://tex.stackexchange.com/questions/98660/
\usepackage[style=trad-alpha,
  sortcites=true,
  doi=false,
  url=false,
  giveninits=true, % Bob Foo --> B. Foo
  isbn=false,
  url=false,
  eprint=false,
  sortcites=false, % \cite{B,A,C}: [A,B,C] --> [B,A,C]
]{biblatex}
\defbibfilter{appendixOnlyFilter}{
  segment=1 % Segment 1 will be chosen to be the one in appendix
  and not segment=0 % Default segment is 0
}
\renewcommand{\multicitedelim}{, } % [ABC96; DEF12] -> [ABC96, DEF12]
% [unknown_ref] => [??]
% https://tex.stackexchange.com/a/352573
\makeatletter
\protected\def\abx@missing#1{\textbf{??}}
\makeatother
\renewcommand*{\bibfont}{\normalfont\small} % BibLatex font seems bigger than Bibtex?

\usepackage{xcolor,cancel}
\usepackage{proof-at-the-end}
\NewDocumentEnvironment{theoremE}{O{}O{}+b}{%
  \begin{theoremEnd}[normal,#2]{theorem}[#1]%
    #3%
  \end{theoremEnd}
  %
}{}% Do not forget the second parameter or you might get Missing \begin{document} error
\NewDocumentEnvironment{lemmaE}{O{}O{}+b}{%
  \begin{theoremEnd}[normal,#2]{lemma}[#1]%
    #3%
  \end{theoremEnd}
  %
}{}% Do not forget the second parameter or you might get Missing \begin{document} error
\NewDocumentEnvironment{corollaryE}{O{}O{}+b}{%
  \begin{theoremEnd}[normal,#2]{corollary}[#1]%
    #3%
  \end{theoremEnd}
  %
}{}% Do not forget the second parameter or you might get Missing \begin{document} error
\NewDocumentEnvironment{proofE}{O{}+b}{%
  \begin{proofEnd}[#1]%
    #2
  \end{proofEnd}%
}{}%
\pgfkeys{/prAtEnd/global custom defaults/.style={
    text link={\noindent See \hyperref[proof:prAtEnd\pratendcountercurrent]{proof} in \cref{proofsection:prAtEnd\pratendcountercurrent}.}
  }
}

\ifdefined\displayArxivTexts%
  \pgfkeys{/prAtEnd/end if published/.style={no restate}}
  \pgfkeys{/prAtEnd/all end if published/.style={no restate}}
\else%
  \pgfkeys{/prAtEnd/end if published/.style={end}}
  \pgfkeys{/prAtEnd/all end if published/.style={all end}}
\fi

% https://tex.stackexchange.com/a/20613/116348
% https://tex.stackexchange.com/a/583244/116348
\usepackage{scalerel}
\newcommand\hcancel[2][black]{%
  \ThisStyle{%
    \setbox0=\hbox{$\SavedStyle #2$}%
    \rlap{\raisebox{.25\ht0}{\textcolor{#1}{\rule{\wd0}{1pt}}}}\SavedStyle #2}%
}

%% We provide 3 commands to add comments/text:
%% - \yournameAddText{your text to add}: proposition to add text
%% - \yournameRemoveText{text to remove}: proposition to remove text
%% - \yournameComment{your comments}: to add general comment to discuss. Should not contain proposition to add or remove text, but can be added before/after to explain the reason of the modification.
%% The interest it to provide some quick ways to see what is the final number of pages:
%% just add before the import a line:
%%   \def\removeCommentsAcceptPropositions{}
%% and the proposition will be "accepted" (i.e. the removed text is removed,
%% and the added text will be marked as black), and the comment will be removed, that way it is possible
%% to determine the final number of pages.
%% Example:
%% This text is normal\leoRemove{, this text should be removed}\leoAddText{, this text should be added}\leoComment{This is a comment, explaining why I made that changes.}.
\newcommand{\addEditor}[2]{%
  \expandafter\newcommand\csname #1AddText\endcsname[2][]{%
    \ifdefined\removeCommentsAcceptPropositions %
      ##2%
    \else%
      \textcolor{#2}{\sout{##1}}\textcolor{#2!60!black}{##2}%
    \fi%
  }%
  \expandafter\newcommand\csname #1Comment\endcsname[1]{%
    \ifdefined\removeCommentsAcceptPropositions %
    \else%
      \textcolor{#2!60!white}{\textbf{##1}}%
    \fi%
  }%
  \expandafter\newcommand\csname #1Remove\endcsname[1]{%
    \ifdefined\removeCommentsAcceptPropositions %
    \else%
      \textcolor{#2}{\sout{##1}}%
    \fi%
  }%
}
\definecolor{darkgreen}{rgb}{0., 0.7, 0}
\addEditor{leo}{darkgreen}
\addEditor{fred}{red}
\addEditor{elham}{blue}

%% We also provide an arxivOnly environment, in order to include text that are too long for the submitted version
%% but that would be good to have in the arxiv version.
% Use the \begin{arxivOnly} ... \end{arxivOnly} environment, by putting the opening and ending command
%% on there own line.
%% For shorter texts, you can use the inline version \arxivOnlyShort{your text}.
%% If you want to exclude text from the arxiv, use \begin{publishedOnly} ... \end{publishedOnly} and \publishedOnlyShort{your text}
%% "Comment" package is annoying, it does not allow block indentation, we use 'versions' instead
% \usepackage{versions} %% Do not work inside theoremE, don't know why
\ifdefined\displayArxivTexts%
  % \includeversion{arxivOnly}
  % \excludeversion{publishedOnly}
  \newcommand{\arxivOnly}[1]{#1}
  \newcommand{\arxivOnlyNoDiff}[1]{#1} % Like arxivOnly, but useful in diff mode (when it does not make sense to add colors, for instance inside a cite or when using arxivonly to create lists.
  \newcommand{\publishedOnly}[1]{}
  \newcommand{\publishedVsArxiv}[2]{#2}
  \newcommand{\inAppendixIfPublished}[1]{#1}
\else%
  \ifdefined\displayDiffTexts%
    \colorlet{arxivDiff}{green}
    \colorlet{publishedDiff}{red}
    \newcommand{\arxivOnly}[1]{{\color{arxivDiff}#1}}
    \newcommand{\arxivOnlyNoDiff}[1]{#1}
    \newcommand{\publishedOnly}[1]{{\color{publishedDiff}#1}}
    \newcommand{\publishedVsArxiv}[2]{{\color{publishedDiff}#1}{\color{arxivDiff}#2}}
    \newcommand{\inAppendixIfPublished}[1]{{\color{arxivDiff}\textbf{MOVED IN APPENDIX IN PUBLISHED}#1}\textEnd{{\color{publishedDiff}\textbf{MOVED IN MAIN BODY IN ARXIV}#1}}}
  \else
    % \excludeversion{arxivOnly}
    % \includeversion{publishedOnly}
    \newcommand{\arxivOnly}[1]{}
    \newcommand{\arxivOnlyNoDiff}[1]{}
    \newcommand{\publishedOnly}[1]{#1}
    \newcommand{\publishedVsArxiv}[2]{#1}
    \newcommand{\inAppendixIfPublished}[1]{\textEnd{#1}}
  \fi
\fi
% Alias, easier to remember  
\newcommand{\inAppendix}[1]{\textEnd{#1}}
  
% Symbol to display when an arxiv text is supposed to appear
\usepackage{todonotes}
\newcommand{\notifyArxiv}{\ifdefined\arxivDraftMode\todo{arXiv}\fi}

% Remove unwanted space before/after lists
\usepackage{enumitem}
\publishedOnly{\setlist[itemize]{noitemsep, topsep=0pt}}

% Usual packages:
\usepackage{amsmath}
\usepackage{amssymb}
% \usepackage{amsthm}
\usepackage{thmtools}
\usepackage{graphicx}
\graphicspath{{}{figures/}}
\usepackage{float}
\usepackage{subfig}
\usepackage{wrapfig}
\usepackage{array}
\usepackage{braket}
% Refer to footnotes:
% https://tex.stackexchange.com/a/167382/116348
\usepackage{footmisc}
% Useful to use the latest NewDocumentCommand:
%\usepackage{xparse}
%\usepackage[poster]{tcolorbox}

%% Space between pictures and text
%% https://mirrors.chevalier.io/CTAN/macros/latex/contrib/layouts/layman.pdf
%% \usepackage{layouts}
%% https://tex.stackexchange.com/questions/60477/remove-space-after-figure-and-before-text?rq=1
\setlength{\floatsep}{7pt plus 2pt minus 2pt}
\setlength{\textfloatsep}{7pt plus 2pt minus 2pt}
\setlength{\intextsep}{7pt plus 2pt minus 2pt}
% https://tex.stackexchange.com/questions/39017/how-to-influence-the-position-of-float-environments-like-figure-and-table-in-lat
\setcounter{topnumber}{8}
\setcounter{bottomnumber}{8}
\setcounter{totalnumber}{8}
%% https://tex.stackexchange.com/questions/45990/how-can-i-modify-vertical-space-between-figure-and-caption
%% plus/minus allows to stretch a bit around 15pt
% \setlength{\abovecaptionskip}{5pt}

% Remove number to subsubsection (in llncs it replaces \paragraph)
\setcounter{secnumdepth}{2}

%%% Some practical environment to define theorems, lemma...
% \declaretheorem[name=Theorem,numberwithin=section]{theorem}
% \declaretheorem[name=Definition,sibling=theorem]{definition}
% \declaretheorem[name=Lemma,sibling=theorem]{lemma}
% \declaretheorem[name=Corollary,sibling=theorem]{corollary}
% \declaretheorem[name=Remark,sibling=theorem]{remark}
%%% In LNCS we use
% https://tex.stackexchange.com/a/373125/116348
% \spnewtheorem{thm}{Theorem}[section]{\bfseries}{\itshape}

%%%
\usepackage[ruled,vlined]{algorithm2e}
\SetAlFnt{\normalsize}%
\setlength{\algomargin}{0em} % remove margin on left of algo
\SetAlgorithmName{Protocol}{protocol}{List of Protocols}
\newenvironment{protocol}[1][htb]{\begin{algorithm}[#1]}{\end{algorithm}}

%%% %%% Load algorithm package, and create a breakable version
%%% \usepackage{algorithm, algorithmicx, algpseudocode}
%%% % Decrease space between text:
%%% % https://tex.stackexchange.com/questions/25828/how-to-remove-change-the-vertical-spacing-before-and-after-an-algorithm-enviro/25832
%%% \setlength{\intextsep}{1\baselineskip}
%%% % https://tex.stackexchange.com/questions/387577/new-algorithm-environment-with-a-separate-counter
%%% \newcounter{protocol}
%%% \makeatletter
%%% \newenvironment{protocol}[1][htb]{%
%%%   \let\c@algorithm\c@protocol
%%%   \renewcommand{\ALG@name}{Protocol}% Update algorithm name
%%%   \begin{algorithm}[#1]%
%%%   }{\end{algorithm}
%%% }
%%% \makeatother
%%%
%%% \providecommand*\algorithmautorefname{Protocol}
%%% \makeatletter
%%% \newenvironment{breakablealgorithm}
%%%   {% \begin{breakablealgorithm}
%%%    % \begin{center}
%%%      \refstepcounter{algorithm}% New algorithm
%%%      \hrule height.8pt depth0pt \kern2pt% \@fs@pre for \@fs@ruled
%%%      \renewcommand{\caption}[2][\relax]{% Make a new \caption
%%%        % {\raggedright\textbf{\ALG@name~\thealgorithm} ##2\par}%
%%%        {\raggedright\textbf{Protocol~\thealgorithm} ##2\par}%
%%%        \ifx\relax##1\relax % #1 is \relax
%%%          \addcontentsline{loa}{algorithm}{\protect\numberline{\thealgorithm}##2}%
%%%        \else % #1 is not \relax
%%%          \addcontentsline{loa}{algorithm}{\protect\numberline{\thealgorithm}##1}%
%%%        \fi
%%%        \kern2pt\hrule\kern2pt
%%%        \small
%%%      }
%%%   }{% \end{breakablealgorithm}
%%%      \kern2pt\hrule\relax% \@fs@post for \@fs@ruled
%%%    % \end{center}
%%%   }
%%% \makeatother
%%%

\def\EQ#1{\begin{eqnarray}#1\end{eqnarray}} 


%%% Practical abreviations
\usepackage{calrsfs} % Replace \mathcal with different more "cursive" font
\DeclareMathAlphabet{\pazocal}{OMS}{zplm}{m}{n} % \pazocal is now the old \mathcal
\newcommand*{\cA}{\pazocal{A}}
\newcommand*{\cB}{\pazocal{B}}
\newcommand*{\cC}{\pazocal{C}}
\newcommand*{\cD}{\pazocal{D}}
\newcommand*{\cE}{\pazocal{E}}
\newcommand*{\cF}{\pazocal{F}}
\newcommand*{\cG}{\pazocal{G}}
\newcommand*{\cH}{\pazocal{H}}
\newcommand*{\cI}{\pazocal{I}}
\newcommand*{\cJ}{\pazocal{J}}
\newcommand*{\cK}{\pazocal{K}}
\renewcommand*{\cL}{\pazocal{L}} % defined in complexity
\newcommand*{\cM}{\pazocal{M}}
\newcommand*{\cN}{\pazocal{N}}
\newcommand*{\cO}{\pazocal{O}}
\renewcommand*{\cP}{\pazocal{P}} % defined in complexity
\newcommand*{\cQ}{\pazocal{Q}}
\newcommand*{\cR}{\pazocal{R}}
\renewcommand*{\cS}{\pazocal{S}} % defined in complexity
\newcommand*{\cT}{\pazocal{T}}
\newcommand*{\cU}{\pazocal{U}}
\newcommand*{\cV}{\pazocal{V}}
\newcommand*{\cW}{\pazocal{W}}
\newcommand*{\cX}{\pazocal{X}}
\newcommand*{\cY}{\pazocal{Y}}
\newcommand*{\cZ}{\pazocal{Z}}

% Notation for languages. Make sure not to collide with L(H) for linear operators on H
%\DeclareMathAlphabet{\pazocal}{OMS}{zplm}{m}{n}
\newcommand{\linearOp}{\mathcal{L}} % Defines the opearator L(H)
\renewcommand*{\lang}{\pazocal{L}} % Defines a new Turing/Quantum Language

%https://tex.stackexchange.com/questions/195025/
\newcommand*{\rightSend}{\rightsquigarrow} % snake "a --> b" to say "I send a to b".

% For some reasons, proof-at-the-end do not like # (problem with detokenize? To fix...). So we use a macro:
\newcommand*{\diese}{\#}

\newcommand*{\eps}[0]{\varepsilon}
\renewcommand*{\D}[0]{\mathbb{D}}
\renewcommand*{\E}{\mathbb{E}}
\newcommand*{\N}[0]{\mathbb{N}}
\renewcommand*{\R}[0]{\mathbb{R}}
\renewcommand*{\C}[0]{\mathbb{C}}
\newcommand*{\Z}[0]{\mathbb{Z}}
\newcommand*{\T}[0]{\mathbb{T}}

\newcommand*{\CPA}[0]{\textsf{CPA}}
\newcommand*{\LWE}[0]{\textsf{LWE}}
\newcommand*{\GapSVP}[0]{\textsf{GapSVP}}
\newcommand*{\SIVP}[0]{\textsf{SIVP}}

% first param is alpha, second is q (optional)
\newcommand{\contGauss}[1]{\cD_{#1}}
\newcommand{\disGauss}[2]{\cD_{\Z,#1 #2}}
\newcommand{\disGaussCont}[1]{\cD_{\Z,#1}}
\newcommand{\disGaussAQ}{\cD_{\Z,\alpha q}}
\newcommand{\rsafe}{r_{\text{safe}}}
% domain of the function.
\newcommand{\ccX}{\cX}
\newcommand{\ccXSphere}{\cX_\bullet}
\newcommand{\ccXCube}[1][]{\cX_{\scalebox{.4}{$\blacksquare$}#1}}

\newcommand{\BBHeightyFor}{\textsc{BB84}}

% "Correct" spacing around cdot as wildcard
% https://tex.stackexchange.com/questions/78165/spacing-around-cdot-when-used-as-a-wildcard
\newcommand\cdotwild{{\mkern 2mu\cdot\mkern 2mu}}

\newcommand*{\id}{\mathrm{id}}
\DeclareMathOperator{\Tr}{Tr}
\DeclareMathOperator{\supp}{supp}
\DeclareMathOperator{\Det}{Det}
\DeclareMathOperator{\ERR}{ERR}
\DeclareMathOperator{\Round}{Round}

\newcommand{\Ver}{ {\tt Ver} } %% Verify public key
\newcommand{\Gen}{ {\tt Gen} }
\newcommand{\GenLocal}{\ensuremath{ {\tt Gen}_{\tt Loc}} }
\newcommand{\Enc}{ {\tt Enc} }
\newcommand{\Dec}{ {\tt Invert} }
\newcommand{\Eval}{ {\tt Eval} }

\newcommand{\HGen}{ {\tt H.Gen} }
\newcommand{\HEnc}{ {\tt H.Enc} }
\newcommand{\HDec}{ {\tt H.Invert} }
\newcommand{\HEval}{ {\tt H.Eval} }
\newcommand{\Hh}{\ensuremath{{\tt H.}h} }
\newcommand{\HCheckHonestTrapdoor}{\ensuremath{{\tt H.CheckTrapdoor}}}

\newcommand{\MPGen}{ {\tt MP.Gen} }
\newcommand{\MPDec}{ {\tt MP.Invert} }
\newcommand{\InvertSmallGadget}{ \ensuremath{{\texttt{InvertSmallGadget}}} }
\newcommand{\InvertGadget}{ {\tt InvertGadget} }
% \newcommand{\MPEval}{ $g$ }


\DeclareMathOperator{\RoundMod}{RoundMod}


% https://tex.stackexchange.com/a/253746/116348
\usepackage{mleftright}
\newcommand{\BlockMatrix}[2]{%
  \mleft[%
  \begin{array}{@{}#1@{}}%
    #2
  \end{array}%
  \mright]%
}
\newcommand{\SmallBlockMatrix}[2]{
  \mleft[\begin{array}{c}
    #1\tabularnewline % https://tex.stackexchange.com/questions/328745/matrices-with-cryptocode-package
    \hline
    #2
  \end{array}\mright]
}
\newcommand{\HorizBlockMatrix}[2]{
  \mleft[\begin{array}{@{}c|c@{}}
    #1 & #2
  \end{array}\mright]
}

\newcommand*{\vect}[1]{\ensuremath{\mathbf{#1}}}


%% Already defined in *recent* cryptocode.
% \DeclareMathOperator*{\argmax}{arg \, max}
% \DeclareMathOperator*{\argmin}{arg \, min}

\usepackage{xspace}
\xspaceaddexceptions{]\}}
\newcommand{\RSP}{\ensuremath{\sf{RSP}}\xspace}%

\newcommand*{\RSPenc}{\mathrm{RSP}_{\mathrm{enc}}}
\newcommand*{\filter}{\vdash}
\newcommand*{\channelBB}{\cS_{\Z\frac{\pi}{2}}}
% To denote a protocol between two parties
% \newcommand*{\protocol}{\boldsymbol{\pi}}
\newcommand\compRSP{composable $\sf{RSP}_{CC}$}
\newcommand\RSPlike{$\sf{RSP}_{CC}$}
\newcommand\compUBQC{composable $\sf{UBQC}_{CC}$}
\newcommand\UBQClike{$\sf{UBQC}_{CC}$}
% GHZ
\newcommand*{\eqdef}{\coloneqq}
\newcommand*{\PERM}{\ensuremath{}{\sf PERM}}
\newcommand*{\PPT}{\ensuremath{}{\sf PPT}}
\newcommand*{\QPT}{\ensuremath{}{\sf QPT}}
\newcommand*{\Auth}{\ensuremath{}{\sf Auth}}
% Prover, verifier, extractor
\renewcommand*{\P}{\ensuremath{}{\sf P}}
\newcommand*{\V}{\ensuremath{}{\sf V}}
\renewcommand*{\K}{\ensuremath{}{\sf K}}


% Here are the different protocols we use
% Initial protocol
\newcommand{\blind}{\ensuremath{ {\tt BLIND} }}
\newcommand{\blindZK}{\ensuremath{ {\tt BLIND-ZK} }}
% Reveals to people if they are part of the support
\newcommand{\blindSup}{\ensuremath{ {\tt BLIND}^{\tt sup} }}
% Make sure that people in the support share a canonical GHZ
\newcommand{\blindCan}{\ensuremath{ {\tt BLIND}_{\tt can} }}
% Make sure that people in the support share a canonical GHZ and that they know they are supported
\newcommand{\blindCanSup}{\ensuremath{ {\tt BLIND}^{\tt sup}_{\tt can} }}
\newcommand{\authBlindCanDist}{\ensuremath{ {\tt AUTH-BLIND}^{\tt dist}_{\tt can} }}
\newcommand{\verifCanSup}{\ensuremath{ {\tt VERIF}^{\tt sup}_{\tt can} }}
\newcommand{\blindPoly}{\ensuremath{ {\tt BLIND-poly} }}
\newcommand{\INDFCT}{\ensuremath{ {\tt IND-D0} }}
\newcommand{\INDBLIND}{\refGame*{INDBLIND}}
\newcommand{\INDBLINDtxt}{\ensuremath{ {\tt IND-\blind{}} }}
\newcommand{\INDBLINDSUP}{\refGame*{INDBLINDSUP}}
\newcommand{\INDBLINDSUPtxt}{\ensuremath{ {\tt IND-\blindSup{}} }}
\newcommand{\INDBLINDCANSUP}{\ensuremath{ {\tt IND-\blindCanSup{}} }}
\newcommand{\INDBLINDCANAUTH}{\ensuremath{ {\tt IND-\authBlindCanDist{}} }}
\newcommand{\INDPARTIAL}{\ensuremath{ {\tt IND-PARTIAL} }}
\newcommand*\GHZ{\ensuremath{}{\sf GHZ}}
\newcommand*\GHZCan{\ensuremath{{\sf GHZ^{\sf can}}}}
% Generalized GGHZ
\newcommand*\GGHZ{\ensuremath{ {\sf GHZ^G} }}
\newcommand*\HGHZ{\ensuremath{ {\sf GHZ^H} }}
\newcommand{\AssumpFctNoDelta}{\HGHZ{} capable}
\newcommand{\AssumpFctCanNoDelta}{\GHZCan{} capable}
\newcommand{\AssumpFct}{$\delta$-\HGHZ{} capable}
\newcommand{\AssumpFctNegl}{$\negl[\lambda]$-\HGHZ{} capable}
\newcommand{\AssumpFctCan}{$\delta$-\GHZCan{} capable}
\newcommand{\AssumpFctCanPrime}{$\delta^\prime$-\GHZCan{} capable}
\newcommand{\AssumpFctDist}{distributable}
\newcommand{\partialInfo}{\ensuremath{{\tt PartInfo}}}
\newcommand{\partialInfoLocal}{\ensuremath{{\tt PartInfo_{\tt Loc}}}}
\newcommand{\partialAlpha}{\ensuremath{{\tt PartAlpha_{\tt Loc}}}}
\newcommand{\CombineAlpha}{\ensuremath{{\tt CombineAlpha}}}
\newcommand{\CombineRandom}{\ensuremath{{\tt CombineRandom}}}
\newcommand{\CheckHonestTrapdoor}{\ensuremath{{\tt CheckTrapdoor}}}
\newcommand{\ZKGenCheck}{\ensuremath{{\tt ZK-GenCheck}}}

\newcommand{\highlightChanges}[1]{\textcolor{green!50!black}{#1}}

\newcommand{\Real}{\ensuremath{\mathsf{REAL}}}
\newcommand{\Sim}{\ensuremath{\mathsf{Sim}}}
\newcommand{\Ideal}{\ensuremath{\mathsf{IDEAL}}}
\newcommand{\OUT}{\ensuremath{\mathsf{OUT}}}

\usepackage{pifont}% http://ctan.org/pkg/pifont
% To denote that a user is not in the final GHZ
\newcommand{\cross}{\ensuremath{\text{\ding{55}}}}

\usepackage{verbatim}
\newcommand{\leftI}{{\normalfont\ttfamily left}}
\newcommand{\rightI}{{\normalfont\ttfamily right}}
\newcommand*{\mybar}[1]{\overline{#1}}

\newcommand*{\quout}{\mathtt{out}}

\DeclareMathOperator{\Image}{Image}
\DeclareMathOperator{\Ima}{Im}
\DeclareMathOperator{\Dom}{Dom}

\newcommand*{\Pub}{\mathtt{Pub}}
\newcommand*{\accept}{\mathtt{accept}}
\newcommand*{\abort}{\mathtt{abort}}
\newcommand*{\gadxor}{\mathtt{Gad}_{\oplus}}
\newcommand*{\quin}{\mathtt{in}}

%%% Provide a "subproof" environment (pretty simple for now,
%%% just indent the sub-proofs)
% I'd love to have left border... https://tex.stackexchange.com/questions/532948/robustly-add-a-border-to-the-left-of-a-text-spanning-several-pages
\usepackage{changepage}% http://ctan.org/pkg/changepage
\newenvironment{subproof}{\begin{adjustwidth}{2em}{0pt}}{\end{adjustwidth}}


%%% To have nice probabilities
% https://tex.stackexchange.com/questions/198771/align-in-substack
\makeatletter
\newcommand{\subalign}[1]{%
  \vcenter{%
    \Let@ \restore@math@cr \default@tag
    \baselineskip\fontdimen10 \scriptfont\tw@
    \advance\baselineskip\fontdimen12 \scriptfont\tw@
    \lineskip\thr@@\fontdimen8 \scriptfont\thr@@
    \lineskiplimit\lineskip
    \ialign{\hfil$\m@th\scriptstyle##$&$\m@th\scriptstyle{}##$\hfil\crcr
      #1\crcr
    }%
  }%
}
\makeatother
\usepackage{xparse}
% https://tex.stackexchange.com/questions/431794/same-spacing-around-middle-as-around-mid
\let\originalmiddle=\middle
\def\middle#1{\mathrel{}\originalmiddle#1\mathrel{}}
\NewDocumentCommand{\pr}{som}{\Pr\IfValueT{#2}{_{\substack{#2}}}\left[\,#3\,\right]}
\NewDocumentCommand{\esp}{som}{\IfValueTF{#2}{\underset{\subalign{#2}}{\mathbb{E}}}{\mathbb{E}}[\,#3\,]}
\NewDocumentEnvironment{pral}{soO{}}
 {%
  \Pr\IfValueT{#2}{_{\IfBooleanTF{#1}{\substack{#2}}{#2}}}
\begin{alignedat}[t]{2}
  [\, } {\,]
 #3
 \end{alignedat}
 }
\NewDocumentEnvironment{espal}{soO{}}
 {%
   \IfValueTF{#2}{\underset{\subalign{#2}}{\mathbb{E}}}{\mathbb{E}}
   \begin{alignedat}[t]{2}[\,
 }
 {\,]#3\end{alignedat}
 }
% To draw |A><B|, the second argument is facultative and gives |A><A|
\NewDocumentCommand{\ketbra}{mg}{
  | #1 \rangle \langle \IfValueTF{#2}{#2}{#1} |
}
% https://tex.stackexchange.com/a/188364/116348
\DeclareRobustCommand{\minwidthbox}[2]{%
  \ifmmode
    \expandafter\mathmakebox
  \else
    \expandafter\makebox
  \fi
  [\ifdim#2<\width\width\else#2\fi]{#1}%f
}

\NewDocumentCommand{\constructs}{mg}{
  \xrightarrow[\IfValueT{#2}{#2}]{\minwidthbox{#1}{5mm}}
}

%%%%%%%%%% This part is specific to llncs-based papers
% To enable this part of the configuration, add
%  \def\enableLlncsCode{}
% before the \input{include}
\ifdefined\enableLlncsCode%
% go back to standard amsthm proof environments with Qed at the end
\let\proof\relax\let\endproof\relax
  \usepackage{amsthm}
  % To debug, it's a nightmare without page number... To disable
  % in the final version
  \pagestyle{plain}
\fi

% Define environments when loaded in standalone...
\ifdefined\enableNonLlncsCode%
\usepackage{thmtools}
\declaretheorem[name=Theorem,numberwithin=chapter]{theorem}
\declaretheorem[name=Definition,sibling=theorem]{definition}
\declaretheorem[name=Lemma,sibling=theorem]{lemma}
\declaretheorem[name=Corollary,sibling=theorem]{corollary}
\declaretheorem[name=Remark,sibling=theorem]{remark}
\fi

\usepackage{enumitem}
\newlist{compressedList}{itemize}{10}
\setlist[compressedList]{label=--,topsep=0pt,parsep=0pt,partopsep=0pt}
\usepackage{mathtools}
\DeclarePairedDelimiter\ceil{\lceil}{\rceil}
\DeclarePairedDelimiter\floor{\lfloor}{\rfloor}

% https://tex.stackexchange.com/questions/134278/overriding-the-paragraph-style-in-the-lncs-document-class
\usepackage{etoolbox}
\patchcmd{\paragraph}{\itshape}{\bfseries\boldmath}{}{}


% This part is specific to llncs-based papers with nicer output
% (page number, theorem labelled with sections...). This should
% be the version in the arxiv.
% To enable this part of the configuration, add
%  \def\nicerThanLlncs{}
% before the \input{include}
\ifdefined\nicerThanLlncs%
  % change margins, page layout
  \usepackage[a4paper, top=1.4in, bottom=1.4in, right=1in, left=1in]{geometry}
  % enable page numbering, suppressed by default by llncs
  \pagestyle{plain}
  \setcounter{tocdepth}{2}
  \renewcommand\small{\fontsize{10pt}{12pt}\selectfont}
\fi
\ifdefined\nicerThanLlncsAbstract%
  % change margins, page layout
  \usepackage[a4paper, margin=2.5cm]{geometry}
  % enable page numbering, suppressed by default by llncs
  \pagestyle{plain}
  \setcounter{tocdepth}{2}
\fi
\ifdefined\enableLlncsCodeSlightlyImproved%
  \pagestyle{plain}
  \setcounter{tocdepth}{2}
\fi


\ifdefined\disableHyperref%
\else%
% https://rcweb.dartmouth.edu/doc/texmf-dist/doc/latex/cryptocode/cryptocode.pdf (page 55)
%%%%%%%%%% This part should be loaded last
% Usually hyperref needs to be at the end
\definecolor{secondaryColor}{RGB}{206,149,0} %% darker orange, looks like gold. <3
\usepackage[colorlinks, allcolors=secondaryColor]{hyperref}
\fi
%% Hyperref should be loaded before cryptocode! Otherwise, errors with labels of lines for example.
%%% Cryptocode
\usepackage [
n,
advantage,
operators,
sets,
adversary,
landau,
probability,
notions,
logic,
ff,
mm,
primitives,
events,
complexity,
asymptotics,
keys
] {cryptocode}
% I don't like the way "samples" is defined in cryptocode with a big dollar in front.
% \renewcommand\sample{\mathrel{\overset{\vbox to.7ex{\hbox{\fontsize{6}{0}\selectfont\vspace{.5ex} \$}}}{\leftarrow}}}
%% https://tex.stackexchange.com/a/584533/116348
\makeatletter
\renewcommand{\sample}{\mathrel{\mathpalette\sample@\relax}}
\newcommand{\sample@}[2]{%
  \ooalign{%
    \hspace{\stretch{2}}\raisebox{0.7\height}{$\m@th\demotestyle{#1}\mathdollar$}\hspace{\stretch{1}}\cr
    $\m@th#1\leftarrow$\cr
  }%
}
\newcommand{\demotestyle}[1]{%
  \ifx#1\displaystyle\scriptstyle\else
  \ifx#1\textstyle\scriptstyle\else
  \scriptscriptstyle\fi\fi
}
\makeatother
% Create a new command to create games using \game[linenumbering]{Name of game}{Game definition}
\createprocedureblock{game}{center,boxed}{}{}{}
% Create a new environment (following what I did in the thesis) to provide named games
\newcommand{\gameFont}[1]{\texttt{#1}}
\renewcommand{\Game}[1]{\ensuremath{\gameFont{Game#1}}} % By default \Game produces a nice G (symetric), but it's not very standard.

\makeatletter
\createprocedureblock{gameProc}{center,boxed}{}{}{}
\createprocedureblock{interactGame}{center,boxed}{}{}{}
%%% Nicer way to refer to games.
%%% See also https://tex.stackexchange.com/questions/623163 which provides a cleaner way to make it work with cleverref.
% Usage: \begin{namedGame}*[label][short title]{title}*[game options like skipfirstln,lnstart=1] content \end{namedGame}
% first * = disable floating. WARNING: do not put labels with non-letter chars.
% second * = do not include the game in the TOC.
\NewDocumentEnvironment{namedGame}{soomsO{}b}{%
  \IfBooleanTF{#1}{}{%
    \par% without par, the figure will be put after the line arriving after the picture if there is no paragraph.
    \setlength{\intextsep}{5.0pt plus 2.0pt minus 2.0pt}% reduce the space when image is inserted inline.
    \begin{figure}[htbp]}% 
    \begin{pcimage}%
      %% Add to toc see command listofgames
      \phantomsection% Create a fake label, useful to ensure the link point to this part.
      \IfBooleanTF{#5}{}{% If no star is given, add to "table of contents" for games
        \IfValueTF{#3}{% If a short title is provided
          \addcontentsline{game}{section}{\ensuremath{#3}}%
        }{%
          \addcontentsline{game}{section}{\ensuremath{#4}}%
        }%
      }%
      {\normalfont\gameProc[linenumbering,#6]{\ensuremath{#4}}{\IfValueTF{#2}{%
            %% Add an anchor if a label is present
            \raisebox{1em}{\hypertarget{#2}{}}%
            %% Create a macro "mygametitle@nameoflabel" to store the title
            \IfValueTF{#3}{% If a short title is provided
              \expandafter\gdef\csname mygametitle@#2 \endcsname{\ensuremath{#3}}%%
              \write\@auxout{\gdef\string\mygametitle@#2{\ensuremath{#3}}}%
            }{%
              \expandafter\gdef\csname mygametitle@#2 \endcsname{\ensuremath{#4}}%%
              \write\@auxout{\gdef\string\mygametitle@#2{\ensuremath{#4}}}%
            }%
          }{} #7 }}%
    \end{pcimage}
    \IfBooleanTF{#1}{}{\end{figure}}%
}{}

% Usage: \refGame{label}. Use \refGame*{label} if you don't want to color the link (useful in proofs)
\NewDocumentCommand{\refGame}{sm}{%
  \IfBooleanTF{#1}{%
    \hyperlink{#2}{\textcolor{black}{\csname mygametitle@#2\endcsname}}% Do not put a white space after #1!
  }{%
    \hyperlink{#2}{\csname mygametitle@#2\endcsname}% Do not put a white space after #1!
  }%
}
\makeatother

\ifdefined\disableHyperref%
\else%
% Well... not as much as cleveref
\usepackage[capitalise,noabbrev]{cleveref}
% Don't abbrev general names... but still abbrev equations
\crefname{equation}{Eq.}{Eqs.}
\crefname{figure}{Fig.}{Figs.}
\crefname{algorithm}{Protocol}{Protocols}
\Crefname{equation}{Equation}{Equations}
\Crefname{figure}{Figure}{Figures}
\Crefname{algorithm}{Protocol}{Protocols}
\fi

% Put that package at the end of the file,
% or you will have some strange errors like
% "Only one # is allowed per tab"
\usetikzlibrary{quantikz}
% Bug in llncs https://tex.stackexchange.com/a/372108/116348
\def\theHlemma{\theHtheorem}
\def\theHdefinition{\theHtheorem}
\def\theHcorollary{\theHtheorem}

%%%%%% PLEASE DO NOT WRITE ANYTHING AFTER THIS LINE %%%%%%
%%%%%% Indeed, hyperref, cleveref and quantikz need %%%%%%
%%%%%% to be loaded at the very end.                %%%%%%

\ifdefined\nicerThanLlncs%
  % change margins, page layout
  \usepackage[a4paper, top=1.4in, bottom=1.4in, right=1in, left=1in]{geometry}
  % enable page numbering, suppressed by default by llncs
  \pagestyle{plain}
  \setcounter{tocdepth}{2}
  \renewcommand\small{\fontsize{10pt}{12pt}\selectfont}
\fi
\ifdefined\nicerThanLlncsAbstract%
  % change margins, page layout
  \usepackage[a4paper, margin=2.5cm]{geometry}
  % enable page numbering, suppressed by default by llncs
  \pagestyle{plain}
  \setcounter{tocdepth}{2}
\fi
\ifdefined\enableLlncsCodeSlightlyImproved%
  \pagestyle{plain}
  \setcounter{tocdepth}{2}
\fi


\ifdefined\disableHyperref%
\else%
% https://rcweb.dartmouth.edu/doc/texmf-dist/doc/latex/cryptocode/cryptocode.pdf (page 55)
%%%%%%%%%% This part should be loaded last
% Usually hyperref needs to be at the end
\definecolor{secondaryColor}{RGB}{206,149,0} %% darker orange, looks like gold. <3
\usepackage[colorlinks, allcolors=secondaryColor]{hyperref}
\fi
%% Hyperref should be loaded before cryptocode! Otherwise, errors with labels of lines for example.
%%% Cryptocode
\usepackage [
n,
advantage,
operators,
sets,
adversary,
landau,
probability,
notions,
logic,
ff,
mm,
primitives,
events,
complexity,
asymptotics,
keys
] {cryptocode}
% I don't like the way "samples" is defined in cryptocode with a big dollar in front.
% \renewcommand\sample{\mathrel{\overset{\vbox to.7ex{\hbox{\fontsize{6}{0}\selectfont\vspace{.5ex} \$}}}{\leftarrow}}}
%% https://tex.stackexchange.com/a/584533/116348
\makeatletter
\renewcommand{\sample}{\mathrel{\mathpalette\sample@\relax}}
\newcommand{\sample@}[2]{%
  \ooalign{%
    \hspace{\stretch{2}}\raisebox{0.7\height}{$\m@th\demotestyle{#1}\mathdollar$}\hspace{\stretch{1}}\cr
    $\m@th#1\leftarrow$\cr
  }%
}
\newcommand{\demotestyle}[1]{%
  \ifx#1\displaystyle\scriptstyle\else
  \ifx#1\textstyle\scriptstyle\else
  \scriptscriptstyle\fi\fi
}
\makeatother
% Create a new command to create games using \game[linenumbering]{Name of game}{Game definition}
\createprocedureblock{game}{center,boxed}{}{}{}
% Create a new environment (following what I did in the thesis) to provide named games
\newcommand{\gameFont}[1]{\texttt{#1}}
\renewcommand{\Game}[1]{\ensuremath{\gameFont{Game#1}}} % By default \Game produces a nice G (symetric), but it's not very standard.

\makeatletter
\createprocedureblock{gameProc}{center,boxed}{}{}{}
\createprocedureblock{interactGame}{center,boxed}{}{}{}
%%% Nicer way to refer to games.
%%% See also https://tex.stackexchange.com/questions/623163 which provides a cleaner way to make it work with cleverref.
% Usage: \begin{namedGame}*[label][short title]{title}*[game options like skipfirstln,lnstart=1] content \end{namedGame}
% first * = disable floating. WARNING: do not put labels with non-letter chars.
% second * = do not include the game in the TOC.
\NewDocumentEnvironment{namedGame}{soomsO{}b}{%
  \IfBooleanTF{#1}{}{%
    \par% without par, the figure will be put after the line arriving after the picture if there is no paragraph.
    \setlength{\intextsep}{5.0pt plus 2.0pt minus 2.0pt}% reduce the space when image is inserted inline.
    \begin{figure}[htbp]}% 
    \begin{pcimage}%
      %% Add to toc see command listofgames
      \phantomsection% Create a fake label, useful to ensure the link point to this part.
      \IfBooleanTF{#5}{}{% If no star is given, add to "table of contents" for games
        \IfValueTF{#3}{% If a short title is provided
          \addcontentsline{game}{section}{\ensuremath{#3}}%
        }{%
          \addcontentsline{game}{section}{\ensuremath{#4}}%
        }%
      }%
      {\normalfont\gameProc[linenumbering,#6]{\ensuremath{#4}}{\IfValueTF{#2}{%
            %% Add an anchor if a label is present
            \raisebox{1em}{\hypertarget{#2}{}}%
            %% Create a macro "mygametitle@nameoflabel" to store the title
            \IfValueTF{#3}{% If a short title is provided
              \expandafter\gdef\csname mygametitle@#2 \endcsname{\ensuremath{#3}}%%
              \write\@auxout{\gdef\string\mygametitle@#2{\ensuremath{#3}}}%
            }{%
              \expandafter\gdef\csname mygametitle@#2 \endcsname{\ensuremath{#4}}%%
              \write\@auxout{\gdef\string\mygametitle@#2{\ensuremath{#4}}}%
            }%
          }{} #7 }}%
    \end{pcimage}
    \IfBooleanTF{#1}{}{\end{figure}}%
}{}

% Usage: \refGame{label}. Use \refGame*{label} if you don't want to color the link (useful in proofs)
\NewDocumentCommand{\refGame}{sm}{%
  \IfBooleanTF{#1}{%
    \hyperlink{#2}{\textcolor{black}{\csname mygametitle@#2\endcsname}}% Do not put a white space after #1!
  }{%
    \hyperlink{#2}{\csname mygametitle@#2\endcsname}% Do not put a white space after #1!
  }%
}
\makeatother

\ifdefined\disableHyperref%
\else%
% Well... not as much as cleveref
\usepackage[capitalise,noabbrev]{cleveref}
% Don't abbrev general names... but still abbrev equations
\crefname{equation}{Eq.}{Eqs.}
\crefname{figure}{Fig.}{Figs.}
\crefname{algorithm}{Protocol}{Protocols}
\Crefname{equation}{Equation}{Equations}
\Crefname{figure}{Figure}{Figures}
\Crefname{algorithm}{Protocol}{Protocols}
\fi

% Put that package at the end of the file,
% or you will have some strange errors like
% "Only one # is allowed per tab"
\usetikzlibrary{quantikz}
% Bug in llncs https://tex.stackexchange.com/a/372108/116348
\def\theHlemma{\theHtheorem}
\def\theHdefinition{\theHtheorem}
\def\theHcorollary{\theHtheorem}

%%%%%% PLEASE DO NOT WRITE ANYTHING AFTER THIS LINE %%%%%%
%%%%%% Indeed, hyperref, cleveref and quantikz need %%%%%%
%%%%%% to be loaded at the very end.                %%%%%%

\ifdefined\enableLlncsCode%
% go back to standard amsthm proof environments with Qed at the end
\let\proof\relax\let\endproof\relax
  \usepackage{amsthm}
  % To debug, it's a nightmare without page number... To disable
  % in the final version
  \pagestyle{plain}
\fi

% Define environments when loaded in standalone...
\ifdefined\enableNonLlncsCode%
\usepackage{thmtools}
\declaretheorem[name=Theorem,numberwithin=chapter]{theorem}
\declaretheorem[name=Definition,sibling=theorem]{definition}
\declaretheorem[name=Lemma,sibling=theorem]{lemma}
\declaretheorem[name=Corollary,sibling=theorem]{corollary}
\declaretheorem[name=Remark,sibling=theorem]{remark}
\fi

\usepackage{enumitem}
\newlist{compressedList}{itemize}{10}
\setlist[compressedList]{label=--,topsep=0pt,parsep=0pt,partopsep=0pt}
\usepackage{mathtools}
\DeclarePairedDelimiter\ceil{\lceil}{\rceil}
\DeclarePairedDelimiter\floor{\lfloor}{\rfloor}

% https://tex.stackexchange.com/questions/134278/overriding-the-paragraph-style-in-the-lncs-document-class
\usepackage{etoolbox}
\patchcmd{\paragraph}{\itshape}{\bfseries\boldmath}{}{}


% This part is specific to llncs-based papers with nicer output
% (page number, theorem labelled with sections...). This should
% be the version in the arxiv.
% To enable this part of the configuration, add
%  \def\nicerThanLlncs{}
% before the %% Let's put in these file all the definitions that we
%% require:

%% Setup language, encoding...
\RequirePackage{etex} 

\usepackage{standalone}
\usepackage{lmodern}
% Language
\usepackage[english]{babel}
% Input encoding
\usepackage[utf8]{inputenc}
% Output encoding https://tex.stackexchange.com/a/677. Important to copy accents
\usepackage[T1]{fontenc}
\usepackage{subfiles} % To quickly load this include file other files
\usepackage{microtype}
\usepackage[shortcuts,nospacearound]{extdash}    % Better endling of em dash (---) to cut sentences\===like 
%%% https://tex.stackexchange.com/questions/2773/how-do-i-make-latex-push-long-citations-to-a-new-line
\usepackage{breakcites}
\usepackage{booktabs}
% Makes sure floats don't fly accross sections
\usepackage{placeins}
\usepackage{adjustbox}
\usepackage{varwidth} % Like minipage, but with automatic width discovery.
\usepackage{complexity} % get \QMA{}...

\usepackage[normalem]{ulem}
%\usepackage[math-style=ISO]{unicode-math}
\usepackage{upgreek} % non italic greak letters

%%%% Bibliography
% I want to have separate references for appendix and main body
% https://tex.stackexchange.com/questions/98660/
\usepackage[style=trad-alpha,
  sortcites=true,
  doi=false,
  url=false,
  giveninits=true, % Bob Foo --> B. Foo
  isbn=false,
  url=false,
  eprint=false,
  sortcites=false, % \cite{B,A,C}: [A,B,C] --> [B,A,C]
]{biblatex}
\defbibfilter{appendixOnlyFilter}{
  segment=1 % Segment 1 will be chosen to be the one in appendix
  and not segment=0 % Default segment is 0
}
\renewcommand{\multicitedelim}{, } % [ABC96; DEF12] -> [ABC96, DEF12]
% [unknown_ref] => [??]
% https://tex.stackexchange.com/a/352573
\makeatletter
\protected\def\abx@missing#1{\textbf{??}}
\makeatother
\renewcommand*{\bibfont}{\normalfont\small} % BibLatex font seems bigger than Bibtex?

\usepackage{xcolor,cancel}
\usepackage{proof-at-the-end}
\NewDocumentEnvironment{theoremE}{O{}O{}+b}{%
  \begin{theoremEnd}[normal,#2]{theorem}[#1]%
    #3%
  \end{theoremEnd}
  %
}{}% Do not forget the second parameter or you might get Missing \begin{document} error
\NewDocumentEnvironment{lemmaE}{O{}O{}+b}{%
  \begin{theoremEnd}[normal,#2]{lemma}[#1]%
    #3%
  \end{theoremEnd}
  %
}{}% Do not forget the second parameter or you might get Missing \begin{document} error
\NewDocumentEnvironment{corollaryE}{O{}O{}+b}{%
  \begin{theoremEnd}[normal,#2]{corollary}[#1]%
    #3%
  \end{theoremEnd}
  %
}{}% Do not forget the second parameter or you might get Missing \begin{document} error
\NewDocumentEnvironment{proofE}{O{}+b}{%
  \begin{proofEnd}[#1]%
    #2
  \end{proofEnd}%
}{}%
\pgfkeys{/prAtEnd/global custom defaults/.style={
    text link={\noindent See \hyperref[proof:prAtEnd\pratendcountercurrent]{proof} in \cref{proofsection:prAtEnd\pratendcountercurrent}.}
  }
}

\ifdefined\displayArxivTexts%
  \pgfkeys{/prAtEnd/end if published/.style={no restate}}
  \pgfkeys{/prAtEnd/all end if published/.style={no restate}}
\else%
  \pgfkeys{/prAtEnd/end if published/.style={end}}
  \pgfkeys{/prAtEnd/all end if published/.style={all end}}
\fi

% https://tex.stackexchange.com/a/20613/116348
% https://tex.stackexchange.com/a/583244/116348
\usepackage{scalerel}
\newcommand\hcancel[2][black]{%
  \ThisStyle{%
    \setbox0=\hbox{$\SavedStyle #2$}%
    \rlap{\raisebox{.25\ht0}{\textcolor{#1}{\rule{\wd0}{1pt}}}}\SavedStyle #2}%
}

%% We provide 3 commands to add comments/text:
%% - \yournameAddText{your text to add}: proposition to add text
%% - \yournameRemoveText{text to remove}: proposition to remove text
%% - \yournameComment{your comments}: to add general comment to discuss. Should not contain proposition to add or remove text, but can be added before/after to explain the reason of the modification.
%% The interest it to provide some quick ways to see what is the final number of pages:
%% just add before the import a line:
%%   \def\removeCommentsAcceptPropositions{}
%% and the proposition will be "accepted" (i.e. the removed text is removed,
%% and the added text will be marked as black), and the comment will be removed, that way it is possible
%% to determine the final number of pages.
%% Example:
%% This text is normal\leoRemove{, this text should be removed}\leoAddText{, this text should be added}\leoComment{This is a comment, explaining why I made that changes.}.
\newcommand{\addEditor}[2]{%
  \expandafter\newcommand\csname #1AddText\endcsname[2][]{%
    \ifdefined\removeCommentsAcceptPropositions %
      ##2%
    \else%
      \textcolor{#2}{\sout{##1}}\textcolor{#2!60!black}{##2}%
    \fi%
  }%
  \expandafter\newcommand\csname #1Comment\endcsname[1]{%
    \ifdefined\removeCommentsAcceptPropositions %
    \else%
      \textcolor{#2!60!white}{\textbf{##1}}%
    \fi%
  }%
  \expandafter\newcommand\csname #1Remove\endcsname[1]{%
    \ifdefined\removeCommentsAcceptPropositions %
    \else%
      \textcolor{#2}{\sout{##1}}%
    \fi%
  }%
}
\definecolor{darkgreen}{rgb}{0., 0.7, 0}
\addEditor{leo}{darkgreen}
\addEditor{fred}{red}
\addEditor{elham}{blue}

%% We also provide an arxivOnly environment, in order to include text that are too long for the submitted version
%% but that would be good to have in the arxiv version.
% Use the \begin{arxivOnly} ... \end{arxivOnly} environment, by putting the opening and ending command
%% on there own line.
%% For shorter texts, you can use the inline version \arxivOnlyShort{your text}.
%% If you want to exclude text from the arxiv, use \begin{publishedOnly} ... \end{publishedOnly} and \publishedOnlyShort{your text}
%% "Comment" package is annoying, it does not allow block indentation, we use 'versions' instead
% \usepackage{versions} %% Do not work inside theoremE, don't know why
\ifdefined\displayArxivTexts%
  % \includeversion{arxivOnly}
  % \excludeversion{publishedOnly}
  \newcommand{\arxivOnly}[1]{#1}
  \newcommand{\arxivOnlyNoDiff}[1]{#1} % Like arxivOnly, but useful in diff mode (when it does not make sense to add colors, for instance inside a cite or when using arxivonly to create lists.
  \newcommand{\publishedOnly}[1]{}
  \newcommand{\publishedVsArxiv}[2]{#2}
  \newcommand{\inAppendixIfPublished}[1]{#1}
\else%
  \ifdefined\displayDiffTexts%
    \colorlet{arxivDiff}{green}
    \colorlet{publishedDiff}{red}
    \newcommand{\arxivOnly}[1]{{\color{arxivDiff}#1}}
    \newcommand{\arxivOnlyNoDiff}[1]{#1}
    \newcommand{\publishedOnly}[1]{{\color{publishedDiff}#1}}
    \newcommand{\publishedVsArxiv}[2]{{\color{publishedDiff}#1}{\color{arxivDiff}#2}}
    \newcommand{\inAppendixIfPublished}[1]{{\color{arxivDiff}\textbf{MOVED IN APPENDIX IN PUBLISHED}#1}\textEnd{{\color{publishedDiff}\textbf{MOVED IN MAIN BODY IN ARXIV}#1}}}
  \else
    % \excludeversion{arxivOnly}
    % \includeversion{publishedOnly}
    \newcommand{\arxivOnly}[1]{}
    \newcommand{\arxivOnlyNoDiff}[1]{}
    \newcommand{\publishedOnly}[1]{#1}
    \newcommand{\publishedVsArxiv}[2]{#1}
    \newcommand{\inAppendixIfPublished}[1]{\textEnd{#1}}
  \fi
\fi
% Alias, easier to remember  
\newcommand{\inAppendix}[1]{\textEnd{#1}}
  
% Symbol to display when an arxiv text is supposed to appear
\usepackage{todonotes}
\newcommand{\notifyArxiv}{\ifdefined\arxivDraftMode\todo{arXiv}\fi}

% Remove unwanted space before/after lists
\usepackage{enumitem}
\publishedOnly{\setlist[itemize]{noitemsep, topsep=0pt}}

% Usual packages:
\usepackage{amsmath}
\usepackage{amssymb}
% \usepackage{amsthm}
\usepackage{thmtools}
\usepackage{graphicx}
\graphicspath{{}{figures/}}
\usepackage{float}
\usepackage{subfig}
\usepackage{wrapfig}
\usepackage{array}
\usepackage{braket}
% Refer to footnotes:
% https://tex.stackexchange.com/a/167382/116348
\usepackage{footmisc}
% Useful to use the latest NewDocumentCommand:
%\usepackage{xparse}
%\usepackage[poster]{tcolorbox}

%% Space between pictures and text
%% https://mirrors.chevalier.io/CTAN/macros/latex/contrib/layouts/layman.pdf
%% \usepackage{layouts}
%% https://tex.stackexchange.com/questions/60477/remove-space-after-figure-and-before-text?rq=1
\setlength{\floatsep}{7pt plus 2pt minus 2pt}
\setlength{\textfloatsep}{7pt plus 2pt minus 2pt}
\setlength{\intextsep}{7pt plus 2pt minus 2pt}
% https://tex.stackexchange.com/questions/39017/how-to-influence-the-position-of-float-environments-like-figure-and-table-in-lat
\setcounter{topnumber}{8}
\setcounter{bottomnumber}{8}
\setcounter{totalnumber}{8}
%% https://tex.stackexchange.com/questions/45990/how-can-i-modify-vertical-space-between-figure-and-caption
%% plus/minus allows to stretch a bit around 15pt
% \setlength{\abovecaptionskip}{5pt}

% Remove number to subsubsection (in llncs it replaces \paragraph)
\setcounter{secnumdepth}{2}

%%% Some practical environment to define theorems, lemma...
% \declaretheorem[name=Theorem,numberwithin=section]{theorem}
% \declaretheorem[name=Definition,sibling=theorem]{definition}
% \declaretheorem[name=Lemma,sibling=theorem]{lemma}
% \declaretheorem[name=Corollary,sibling=theorem]{corollary}
% \declaretheorem[name=Remark,sibling=theorem]{remark}
%%% In LNCS we use
% https://tex.stackexchange.com/a/373125/116348
% \spnewtheorem{thm}{Theorem}[section]{\bfseries}{\itshape}

%%%
\usepackage[ruled,vlined]{algorithm2e}
\SetAlFnt{\normalsize}%
\setlength{\algomargin}{0em} % remove margin on left of algo
\SetAlgorithmName{Protocol}{protocol}{List of Protocols}
\newenvironment{protocol}[1][htb]{\begin{algorithm}[#1]}{\end{algorithm}}

%%% %%% Load algorithm package, and create a breakable version
%%% \usepackage{algorithm, algorithmicx, algpseudocode}
%%% % Decrease space between text:
%%% % https://tex.stackexchange.com/questions/25828/how-to-remove-change-the-vertical-spacing-before-and-after-an-algorithm-enviro/25832
%%% \setlength{\intextsep}{1\baselineskip}
%%% % https://tex.stackexchange.com/questions/387577/new-algorithm-environment-with-a-separate-counter
%%% \newcounter{protocol}
%%% \makeatletter
%%% \newenvironment{protocol}[1][htb]{%
%%%   \let\c@algorithm\c@protocol
%%%   \renewcommand{\ALG@name}{Protocol}% Update algorithm name
%%%   \begin{algorithm}[#1]%
%%%   }{\end{algorithm}
%%% }
%%% \makeatother
%%%
%%% \providecommand*\algorithmautorefname{Protocol}
%%% \makeatletter
%%% \newenvironment{breakablealgorithm}
%%%   {% \begin{breakablealgorithm}
%%%    % \begin{center}
%%%      \refstepcounter{algorithm}% New algorithm
%%%      \hrule height.8pt depth0pt \kern2pt% \@fs@pre for \@fs@ruled
%%%      \renewcommand{\caption}[2][\relax]{% Make a new \caption
%%%        % {\raggedright\textbf{\ALG@name~\thealgorithm} ##2\par}%
%%%        {\raggedright\textbf{Protocol~\thealgorithm} ##2\par}%
%%%        \ifx\relax##1\relax % #1 is \relax
%%%          \addcontentsline{loa}{algorithm}{\protect\numberline{\thealgorithm}##2}%
%%%        \else % #1 is not \relax
%%%          \addcontentsline{loa}{algorithm}{\protect\numberline{\thealgorithm}##1}%
%%%        \fi
%%%        \kern2pt\hrule\kern2pt
%%%        \small
%%%      }
%%%   }{% \end{breakablealgorithm}
%%%      \kern2pt\hrule\relax% \@fs@post for \@fs@ruled
%%%    % \end{center}
%%%   }
%%% \makeatother
%%%

\def\EQ#1{\begin{eqnarray}#1\end{eqnarray}} 


%%% Practical abreviations
\usepackage{calrsfs} % Replace \mathcal with different more "cursive" font
\DeclareMathAlphabet{\pazocal}{OMS}{zplm}{m}{n} % \pazocal is now the old \mathcal
\newcommand*{\cA}{\pazocal{A}}
\newcommand*{\cB}{\pazocal{B}}
\newcommand*{\cC}{\pazocal{C}}
\newcommand*{\cD}{\pazocal{D}}
\newcommand*{\cE}{\pazocal{E}}
\newcommand*{\cF}{\pazocal{F}}
\newcommand*{\cG}{\pazocal{G}}
\newcommand*{\cH}{\pazocal{H}}
\newcommand*{\cI}{\pazocal{I}}
\newcommand*{\cJ}{\pazocal{J}}
\newcommand*{\cK}{\pazocal{K}}
\renewcommand*{\cL}{\pazocal{L}} % defined in complexity
\newcommand*{\cM}{\pazocal{M}}
\newcommand*{\cN}{\pazocal{N}}
\newcommand*{\cO}{\pazocal{O}}
\renewcommand*{\cP}{\pazocal{P}} % defined in complexity
\newcommand*{\cQ}{\pazocal{Q}}
\newcommand*{\cR}{\pazocal{R}}
\renewcommand*{\cS}{\pazocal{S}} % defined in complexity
\newcommand*{\cT}{\pazocal{T}}
\newcommand*{\cU}{\pazocal{U}}
\newcommand*{\cV}{\pazocal{V}}
\newcommand*{\cW}{\pazocal{W}}
\newcommand*{\cX}{\pazocal{X}}
\newcommand*{\cY}{\pazocal{Y}}
\newcommand*{\cZ}{\pazocal{Z}}

% Notation for languages. Make sure not to collide with L(H) for linear operators on H
%\DeclareMathAlphabet{\pazocal}{OMS}{zplm}{m}{n}
\newcommand{\linearOp}{\mathcal{L}} % Defines the opearator L(H)
\renewcommand*{\lang}{\pazocal{L}} % Defines a new Turing/Quantum Language

%https://tex.stackexchange.com/questions/195025/
\newcommand*{\rightSend}{\rightsquigarrow} % snake "a --> b" to say "I send a to b".

% For some reasons, proof-at-the-end do not like # (problem with detokenize? To fix...). So we use a macro:
\newcommand*{\diese}{\#}

\newcommand*{\eps}[0]{\varepsilon}
\renewcommand*{\D}[0]{\mathbb{D}}
\renewcommand*{\E}{\mathbb{E}}
\newcommand*{\N}[0]{\mathbb{N}}
\renewcommand*{\R}[0]{\mathbb{R}}
\renewcommand*{\C}[0]{\mathbb{C}}
\newcommand*{\Z}[0]{\mathbb{Z}}
\newcommand*{\T}[0]{\mathbb{T}}

\newcommand*{\CPA}[0]{\textsf{CPA}}
\newcommand*{\LWE}[0]{\textsf{LWE}}
\newcommand*{\GapSVP}[0]{\textsf{GapSVP}}
\newcommand*{\SIVP}[0]{\textsf{SIVP}}

% first param is alpha, second is q (optional)
\newcommand{\contGauss}[1]{\cD_{#1}}
\newcommand{\disGauss}[2]{\cD_{\Z,#1 #2}}
\newcommand{\disGaussCont}[1]{\cD_{\Z,#1}}
\newcommand{\disGaussAQ}{\cD_{\Z,\alpha q}}
\newcommand{\rsafe}{r_{\text{safe}}}
% domain of the function.
\newcommand{\ccX}{\cX}
\newcommand{\ccXSphere}{\cX_\bullet}
\newcommand{\ccXCube}[1][]{\cX_{\scalebox{.4}{$\blacksquare$}#1}}

\newcommand{\BBHeightyFor}{\textsc{BB84}}

% "Correct" spacing around cdot as wildcard
% https://tex.stackexchange.com/questions/78165/spacing-around-cdot-when-used-as-a-wildcard
\newcommand\cdotwild{{\mkern 2mu\cdot\mkern 2mu}}

\newcommand*{\id}{\mathrm{id}}
\DeclareMathOperator{\Tr}{Tr}
\DeclareMathOperator{\supp}{supp}
\DeclareMathOperator{\Det}{Det}
\DeclareMathOperator{\ERR}{ERR}
\DeclareMathOperator{\Round}{Round}

\newcommand{\Ver}{ {\tt Ver} } %% Verify public key
\newcommand{\Gen}{ {\tt Gen} }
\newcommand{\GenLocal}{\ensuremath{ {\tt Gen}_{\tt Loc}} }
\newcommand{\Enc}{ {\tt Enc} }
\newcommand{\Dec}{ {\tt Invert} }
\newcommand{\Eval}{ {\tt Eval} }

\newcommand{\HGen}{ {\tt H.Gen} }
\newcommand{\HEnc}{ {\tt H.Enc} }
\newcommand{\HDec}{ {\tt H.Invert} }
\newcommand{\HEval}{ {\tt H.Eval} }
\newcommand{\Hh}{\ensuremath{{\tt H.}h} }
\newcommand{\HCheckHonestTrapdoor}{\ensuremath{{\tt H.CheckTrapdoor}}}

\newcommand{\MPGen}{ {\tt MP.Gen} }
\newcommand{\MPDec}{ {\tt MP.Invert} }
\newcommand{\InvertSmallGadget}{ \ensuremath{{\texttt{InvertSmallGadget}}} }
\newcommand{\InvertGadget}{ {\tt InvertGadget} }
% \newcommand{\MPEval}{ $g$ }


\DeclareMathOperator{\RoundMod}{RoundMod}


% https://tex.stackexchange.com/a/253746/116348
\usepackage{mleftright}
\newcommand{\BlockMatrix}[2]{%
  \mleft[%
  \begin{array}{@{}#1@{}}%
    #2
  \end{array}%
  \mright]%
}
\newcommand{\SmallBlockMatrix}[2]{
  \mleft[\begin{array}{c}
    #1\tabularnewline % https://tex.stackexchange.com/questions/328745/matrices-with-cryptocode-package
    \hline
    #2
  \end{array}\mright]
}
\newcommand{\HorizBlockMatrix}[2]{
  \mleft[\begin{array}{@{}c|c@{}}
    #1 & #2
  \end{array}\mright]
}

\newcommand*{\vect}[1]{\ensuremath{\mathbf{#1}}}


%% Already defined in *recent* cryptocode.
% \DeclareMathOperator*{\argmax}{arg \, max}
% \DeclareMathOperator*{\argmin}{arg \, min}

\usepackage{xspace}
\xspaceaddexceptions{]\}}
\newcommand{\RSP}{\ensuremath{\sf{RSP}}\xspace}%

\newcommand*{\RSPenc}{\mathrm{RSP}_{\mathrm{enc}}}
\newcommand*{\filter}{\vdash}
\newcommand*{\channelBB}{\cS_{\Z\frac{\pi}{2}}}
% To denote a protocol between two parties
% \newcommand*{\protocol}{\boldsymbol{\pi}}
\newcommand\compRSP{composable $\sf{RSP}_{CC}$}
\newcommand\RSPlike{$\sf{RSP}_{CC}$}
\newcommand\compUBQC{composable $\sf{UBQC}_{CC}$}
\newcommand\UBQClike{$\sf{UBQC}_{CC}$}
% GHZ
\newcommand*{\eqdef}{\coloneqq}
\newcommand*{\PERM}{\ensuremath{}{\sf PERM}}
\newcommand*{\PPT}{\ensuremath{}{\sf PPT}}
\newcommand*{\QPT}{\ensuremath{}{\sf QPT}}
\newcommand*{\Auth}{\ensuremath{}{\sf Auth}}
% Prover, verifier, extractor
\renewcommand*{\P}{\ensuremath{}{\sf P}}
\newcommand*{\V}{\ensuremath{}{\sf V}}
\renewcommand*{\K}{\ensuremath{}{\sf K}}


% Here are the different protocols we use
% Initial protocol
\newcommand{\blind}{\ensuremath{ {\tt BLIND} }}
\newcommand{\blindZK}{\ensuremath{ {\tt BLIND-ZK} }}
% Reveals to people if they are part of the support
\newcommand{\blindSup}{\ensuremath{ {\tt BLIND}^{\tt sup} }}
% Make sure that people in the support share a canonical GHZ
\newcommand{\blindCan}{\ensuremath{ {\tt BLIND}_{\tt can} }}
% Make sure that people in the support share a canonical GHZ and that they know they are supported
\newcommand{\blindCanSup}{\ensuremath{ {\tt BLIND}^{\tt sup}_{\tt can} }}
\newcommand{\authBlindCanDist}{\ensuremath{ {\tt AUTH-BLIND}^{\tt dist}_{\tt can} }}
\newcommand{\verifCanSup}{\ensuremath{ {\tt VERIF}^{\tt sup}_{\tt can} }}
\newcommand{\blindPoly}{\ensuremath{ {\tt BLIND-poly} }}
\newcommand{\INDFCT}{\ensuremath{ {\tt IND-D0} }}
\newcommand{\INDBLIND}{\refGame*{INDBLIND}}
\newcommand{\INDBLINDtxt}{\ensuremath{ {\tt IND-\blind{}} }}
\newcommand{\INDBLINDSUP}{\refGame*{INDBLINDSUP}}
\newcommand{\INDBLINDSUPtxt}{\ensuremath{ {\tt IND-\blindSup{}} }}
\newcommand{\INDBLINDCANSUP}{\ensuremath{ {\tt IND-\blindCanSup{}} }}
\newcommand{\INDBLINDCANAUTH}{\ensuremath{ {\tt IND-\authBlindCanDist{}} }}
\newcommand{\INDPARTIAL}{\ensuremath{ {\tt IND-PARTIAL} }}
\newcommand*\GHZ{\ensuremath{}{\sf GHZ}}
\newcommand*\GHZCan{\ensuremath{{\sf GHZ^{\sf can}}}}
% Generalized GGHZ
\newcommand*\GGHZ{\ensuremath{ {\sf GHZ^G} }}
\newcommand*\HGHZ{\ensuremath{ {\sf GHZ^H} }}
\newcommand{\AssumpFctNoDelta}{\HGHZ{} capable}
\newcommand{\AssumpFctCanNoDelta}{\GHZCan{} capable}
\newcommand{\AssumpFct}{$\delta$-\HGHZ{} capable}
\newcommand{\AssumpFctNegl}{$\negl[\lambda]$-\HGHZ{} capable}
\newcommand{\AssumpFctCan}{$\delta$-\GHZCan{} capable}
\newcommand{\AssumpFctCanPrime}{$\delta^\prime$-\GHZCan{} capable}
\newcommand{\AssumpFctDist}{distributable}
\newcommand{\partialInfo}{\ensuremath{{\tt PartInfo}}}
\newcommand{\partialInfoLocal}{\ensuremath{{\tt PartInfo_{\tt Loc}}}}
\newcommand{\partialAlpha}{\ensuremath{{\tt PartAlpha_{\tt Loc}}}}
\newcommand{\CombineAlpha}{\ensuremath{{\tt CombineAlpha}}}
\newcommand{\CombineRandom}{\ensuremath{{\tt CombineRandom}}}
\newcommand{\CheckHonestTrapdoor}{\ensuremath{{\tt CheckTrapdoor}}}
\newcommand{\ZKGenCheck}{\ensuremath{{\tt ZK-GenCheck}}}

\newcommand{\highlightChanges}[1]{\textcolor{green!50!black}{#1}}

\newcommand{\Real}{\ensuremath{\mathsf{REAL}}}
\newcommand{\Sim}{\ensuremath{\mathsf{Sim}}}
\newcommand{\Ideal}{\ensuremath{\mathsf{IDEAL}}}
\newcommand{\OUT}{\ensuremath{\mathsf{OUT}}}

\usepackage{pifont}% http://ctan.org/pkg/pifont
% To denote that a user is not in the final GHZ
\newcommand{\cross}{\ensuremath{\text{\ding{55}}}}

\usepackage{verbatim}
\newcommand{\leftI}{{\normalfont\ttfamily left}}
\newcommand{\rightI}{{\normalfont\ttfamily right}}
\newcommand*{\mybar}[1]{\overline{#1}}

\newcommand*{\quout}{\mathtt{out}}

\DeclareMathOperator{\Image}{Image}
\DeclareMathOperator{\Ima}{Im}
\DeclareMathOperator{\Dom}{Dom}

\newcommand*{\Pub}{\mathtt{Pub}}
\newcommand*{\accept}{\mathtt{accept}}
\newcommand*{\abort}{\mathtt{abort}}
\newcommand*{\gadxor}{\mathtt{Gad}_{\oplus}}
\newcommand*{\quin}{\mathtt{in}}

%%% Provide a "subproof" environment (pretty simple for now,
%%% just indent the sub-proofs)
% I'd love to have left border... https://tex.stackexchange.com/questions/532948/robustly-add-a-border-to-the-left-of-a-text-spanning-several-pages
\usepackage{changepage}% http://ctan.org/pkg/changepage
\newenvironment{subproof}{\begin{adjustwidth}{2em}{0pt}}{\end{adjustwidth}}


%%% To have nice probabilities
% https://tex.stackexchange.com/questions/198771/align-in-substack
\makeatletter
\newcommand{\subalign}[1]{%
  \vcenter{%
    \Let@ \restore@math@cr \default@tag
    \baselineskip\fontdimen10 \scriptfont\tw@
    \advance\baselineskip\fontdimen12 \scriptfont\tw@
    \lineskip\thr@@\fontdimen8 \scriptfont\thr@@
    \lineskiplimit\lineskip
    \ialign{\hfil$\m@th\scriptstyle##$&$\m@th\scriptstyle{}##$\hfil\crcr
      #1\crcr
    }%
  }%
}
\makeatother
\usepackage{xparse}
% https://tex.stackexchange.com/questions/431794/same-spacing-around-middle-as-around-mid
\let\originalmiddle=\middle
\def\middle#1{\mathrel{}\originalmiddle#1\mathrel{}}
\NewDocumentCommand{\pr}{som}{\Pr\IfValueT{#2}{_{\substack{#2}}}\left[\,#3\,\right]}
\NewDocumentCommand{\esp}{som}{\IfValueTF{#2}{\underset{\subalign{#2}}{\mathbb{E}}}{\mathbb{E}}[\,#3\,]}
\NewDocumentEnvironment{pral}{soO{}}
 {%
  \Pr\IfValueT{#2}{_{\IfBooleanTF{#1}{\substack{#2}}{#2}}}
\begin{alignedat}[t]{2}
  [\, } {\,]
 #3
 \end{alignedat}
 }
\NewDocumentEnvironment{espal}{soO{}}
 {%
   \IfValueTF{#2}{\underset{\subalign{#2}}{\mathbb{E}}}{\mathbb{E}}
   \begin{alignedat}[t]{2}[\,
 }
 {\,]#3\end{alignedat}
 }
% To draw |A><B|, the second argument is facultative and gives |A><A|
\NewDocumentCommand{\ketbra}{mg}{
  | #1 \rangle \langle \IfValueTF{#2}{#2}{#1} |
}
% https://tex.stackexchange.com/a/188364/116348
\DeclareRobustCommand{\minwidthbox}[2]{%
  \ifmmode
    \expandafter\mathmakebox
  \else
    \expandafter\makebox
  \fi
  [\ifdim#2<\width\width\else#2\fi]{#1}%f
}

\NewDocumentCommand{\constructs}{mg}{
  \xrightarrow[\IfValueT{#2}{#2}]{\minwidthbox{#1}{5mm}}
}

%%%%%%%%%% This part is specific to llncs-based papers
% To enable this part of the configuration, add
%  \def\enableLlncsCode{}
% before the %% Let's put in these file all the definitions that we
%% require:

%% Setup language, encoding...
\RequirePackage{etex} 

\usepackage{standalone}
\usepackage{lmodern}
% Language
\usepackage[english]{babel}
% Input encoding
\usepackage[utf8]{inputenc}
% Output encoding https://tex.stackexchange.com/a/677. Important to copy accents
\usepackage[T1]{fontenc}
\usepackage{subfiles} % To quickly load this include file other files
\usepackage{microtype}
\usepackage[shortcuts,nospacearound]{extdash}    % Better endling of em dash (---) to cut sentences\===like 
%%% https://tex.stackexchange.com/questions/2773/how-do-i-make-latex-push-long-citations-to-a-new-line
\usepackage{breakcites}
\usepackage{booktabs}
% Makes sure floats don't fly accross sections
\usepackage{placeins}
\usepackage{adjustbox}
\usepackage{varwidth} % Like minipage, but with automatic width discovery.
\usepackage{complexity} % get \QMA{}...

\usepackage[normalem]{ulem}
%\usepackage[math-style=ISO]{unicode-math}
\usepackage{upgreek} % non italic greak letters

%%%% Bibliography
% I want to have separate references for appendix and main body
% https://tex.stackexchange.com/questions/98660/
\usepackage[style=trad-alpha,
  sortcites=true,
  doi=false,
  url=false,
  giveninits=true, % Bob Foo --> B. Foo
  isbn=false,
  url=false,
  eprint=false,
  sortcites=false, % \cite{B,A,C}: [A,B,C] --> [B,A,C]
]{biblatex}
\defbibfilter{appendixOnlyFilter}{
  segment=1 % Segment 1 will be chosen to be the one in appendix
  and not segment=0 % Default segment is 0
}
\renewcommand{\multicitedelim}{, } % [ABC96; DEF12] -> [ABC96, DEF12]
% [unknown_ref] => [??]
% https://tex.stackexchange.com/a/352573
\makeatletter
\protected\def\abx@missing#1{\textbf{??}}
\makeatother
\renewcommand*{\bibfont}{\normalfont\small} % BibLatex font seems bigger than Bibtex?

\usepackage{xcolor,cancel}
\usepackage{proof-at-the-end}
\NewDocumentEnvironment{theoremE}{O{}O{}+b}{%
  \begin{theoremEnd}[normal,#2]{theorem}[#1]%
    #3%
  \end{theoremEnd}
  %
}{}% Do not forget the second parameter or you might get Missing \begin{document} error
\NewDocumentEnvironment{lemmaE}{O{}O{}+b}{%
  \begin{theoremEnd}[normal,#2]{lemma}[#1]%
    #3%
  \end{theoremEnd}
  %
}{}% Do not forget the second parameter or you might get Missing \begin{document} error
\NewDocumentEnvironment{corollaryE}{O{}O{}+b}{%
  \begin{theoremEnd}[normal,#2]{corollary}[#1]%
    #3%
  \end{theoremEnd}
  %
}{}% Do not forget the second parameter or you might get Missing \begin{document} error
\NewDocumentEnvironment{proofE}{O{}+b}{%
  \begin{proofEnd}[#1]%
    #2
  \end{proofEnd}%
}{}%
\pgfkeys{/prAtEnd/global custom defaults/.style={
    text link={\noindent See \hyperref[proof:prAtEnd\pratendcountercurrent]{proof} in \cref{proofsection:prAtEnd\pratendcountercurrent}.}
  }
}

\ifdefined\displayArxivTexts%
  \pgfkeys{/prAtEnd/end if published/.style={no restate}}
  \pgfkeys{/prAtEnd/all end if published/.style={no restate}}
\else%
  \pgfkeys{/prAtEnd/end if published/.style={end}}
  \pgfkeys{/prAtEnd/all end if published/.style={all end}}
\fi

% https://tex.stackexchange.com/a/20613/116348
% https://tex.stackexchange.com/a/583244/116348
\usepackage{scalerel}
\newcommand\hcancel[2][black]{%
  \ThisStyle{%
    \setbox0=\hbox{$\SavedStyle #2$}%
    \rlap{\raisebox{.25\ht0}{\textcolor{#1}{\rule{\wd0}{1pt}}}}\SavedStyle #2}%
}

%% We provide 3 commands to add comments/text:
%% - \yournameAddText{your text to add}: proposition to add text
%% - \yournameRemoveText{text to remove}: proposition to remove text
%% - \yournameComment{your comments}: to add general comment to discuss. Should not contain proposition to add or remove text, but can be added before/after to explain the reason of the modification.
%% The interest it to provide some quick ways to see what is the final number of pages:
%% just add before the import a line:
%%   \def\removeCommentsAcceptPropositions{}
%% and the proposition will be "accepted" (i.e. the removed text is removed,
%% and the added text will be marked as black), and the comment will be removed, that way it is possible
%% to determine the final number of pages.
%% Example:
%% This text is normal\leoRemove{, this text should be removed}\leoAddText{, this text should be added}\leoComment{This is a comment, explaining why I made that changes.}.
\newcommand{\addEditor}[2]{%
  \expandafter\newcommand\csname #1AddText\endcsname[2][]{%
    \ifdefined\removeCommentsAcceptPropositions %
      ##2%
    \else%
      \textcolor{#2}{\sout{##1}}\textcolor{#2!60!black}{##2}%
    \fi%
  }%
  \expandafter\newcommand\csname #1Comment\endcsname[1]{%
    \ifdefined\removeCommentsAcceptPropositions %
    \else%
      \textcolor{#2!60!white}{\textbf{##1}}%
    \fi%
  }%
  \expandafter\newcommand\csname #1Remove\endcsname[1]{%
    \ifdefined\removeCommentsAcceptPropositions %
    \else%
      \textcolor{#2}{\sout{##1}}%
    \fi%
  }%
}
\definecolor{darkgreen}{rgb}{0., 0.7, 0}
\addEditor{leo}{darkgreen}
\addEditor{fred}{red}
\addEditor{elham}{blue}

%% We also provide an arxivOnly environment, in order to include text that are too long for the submitted version
%% but that would be good to have in the arxiv version.
% Use the \begin{arxivOnly} ... \end{arxivOnly} environment, by putting the opening and ending command
%% on there own line.
%% For shorter texts, you can use the inline version \arxivOnlyShort{your text}.
%% If you want to exclude text from the arxiv, use \begin{publishedOnly} ... \end{publishedOnly} and \publishedOnlyShort{your text}
%% "Comment" package is annoying, it does not allow block indentation, we use 'versions' instead
% \usepackage{versions} %% Do not work inside theoremE, don't know why
\ifdefined\displayArxivTexts%
  % \includeversion{arxivOnly}
  % \excludeversion{publishedOnly}
  \newcommand{\arxivOnly}[1]{#1}
  \newcommand{\arxivOnlyNoDiff}[1]{#1} % Like arxivOnly, but useful in diff mode (when it does not make sense to add colors, for instance inside a cite or when using arxivonly to create lists.
  \newcommand{\publishedOnly}[1]{}
  \newcommand{\publishedVsArxiv}[2]{#2}
  \newcommand{\inAppendixIfPublished}[1]{#1}
\else%
  \ifdefined\displayDiffTexts%
    \colorlet{arxivDiff}{green}
    \colorlet{publishedDiff}{red}
    \newcommand{\arxivOnly}[1]{{\color{arxivDiff}#1}}
    \newcommand{\arxivOnlyNoDiff}[1]{#1}
    \newcommand{\publishedOnly}[1]{{\color{publishedDiff}#1}}
    \newcommand{\publishedVsArxiv}[2]{{\color{publishedDiff}#1}{\color{arxivDiff}#2}}
    \newcommand{\inAppendixIfPublished}[1]{{\color{arxivDiff}\textbf{MOVED IN APPENDIX IN PUBLISHED}#1}\textEnd{{\color{publishedDiff}\textbf{MOVED IN MAIN BODY IN ARXIV}#1}}}
  \else
    % \excludeversion{arxivOnly}
    % \includeversion{publishedOnly}
    \newcommand{\arxivOnly}[1]{}
    \newcommand{\arxivOnlyNoDiff}[1]{}
    \newcommand{\publishedOnly}[1]{#1}
    \newcommand{\publishedVsArxiv}[2]{#1}
    \newcommand{\inAppendixIfPublished}[1]{\textEnd{#1}}
  \fi
\fi
% Alias, easier to remember  
\newcommand{\inAppendix}[1]{\textEnd{#1}}
  
% Symbol to display when an arxiv text is supposed to appear
\usepackage{todonotes}
\newcommand{\notifyArxiv}{\ifdefined\arxivDraftMode\todo{arXiv}\fi}

% Remove unwanted space before/after lists
\usepackage{enumitem}
\publishedOnly{\setlist[itemize]{noitemsep, topsep=0pt}}

% Usual packages:
\usepackage{amsmath}
\usepackage{amssymb}
% \usepackage{amsthm}
\usepackage{thmtools}
\usepackage{graphicx}
\graphicspath{{}{figures/}}
\usepackage{float}
\usepackage{subfig}
\usepackage{wrapfig}
\usepackage{array}
\usepackage{braket}
% Refer to footnotes:
% https://tex.stackexchange.com/a/167382/116348
\usepackage{footmisc}
% Useful to use the latest NewDocumentCommand:
%\usepackage{xparse}
%\usepackage[poster]{tcolorbox}

%% Space between pictures and text
%% https://mirrors.chevalier.io/CTAN/macros/latex/contrib/layouts/layman.pdf
%% \usepackage{layouts}
%% https://tex.stackexchange.com/questions/60477/remove-space-after-figure-and-before-text?rq=1
\setlength{\floatsep}{7pt plus 2pt minus 2pt}
\setlength{\textfloatsep}{7pt plus 2pt minus 2pt}
\setlength{\intextsep}{7pt plus 2pt minus 2pt}
% https://tex.stackexchange.com/questions/39017/how-to-influence-the-position-of-float-environments-like-figure-and-table-in-lat
\setcounter{topnumber}{8}
\setcounter{bottomnumber}{8}
\setcounter{totalnumber}{8}
%% https://tex.stackexchange.com/questions/45990/how-can-i-modify-vertical-space-between-figure-and-caption
%% plus/minus allows to stretch a bit around 15pt
% \setlength{\abovecaptionskip}{5pt}

% Remove number to subsubsection (in llncs it replaces \paragraph)
\setcounter{secnumdepth}{2}

%%% Some practical environment to define theorems, lemma...
% \declaretheorem[name=Theorem,numberwithin=section]{theorem}
% \declaretheorem[name=Definition,sibling=theorem]{definition}
% \declaretheorem[name=Lemma,sibling=theorem]{lemma}
% \declaretheorem[name=Corollary,sibling=theorem]{corollary}
% \declaretheorem[name=Remark,sibling=theorem]{remark}
%%% In LNCS we use
% https://tex.stackexchange.com/a/373125/116348
% \spnewtheorem{thm}{Theorem}[section]{\bfseries}{\itshape}

%%%
\usepackage[ruled,vlined]{algorithm2e}
\SetAlFnt{\normalsize}%
\setlength{\algomargin}{0em} % remove margin on left of algo
\SetAlgorithmName{Protocol}{protocol}{List of Protocols}
\newenvironment{protocol}[1][htb]{\begin{algorithm}[#1]}{\end{algorithm}}

%%% %%% Load algorithm package, and create a breakable version
%%% \usepackage{algorithm, algorithmicx, algpseudocode}
%%% % Decrease space between text:
%%% % https://tex.stackexchange.com/questions/25828/how-to-remove-change-the-vertical-spacing-before-and-after-an-algorithm-enviro/25832
%%% \setlength{\intextsep}{1\baselineskip}
%%% % https://tex.stackexchange.com/questions/387577/new-algorithm-environment-with-a-separate-counter
%%% \newcounter{protocol}
%%% \makeatletter
%%% \newenvironment{protocol}[1][htb]{%
%%%   \let\c@algorithm\c@protocol
%%%   \renewcommand{\ALG@name}{Protocol}% Update algorithm name
%%%   \begin{algorithm}[#1]%
%%%   }{\end{algorithm}
%%% }
%%% \makeatother
%%%
%%% \providecommand*\algorithmautorefname{Protocol}
%%% \makeatletter
%%% \newenvironment{breakablealgorithm}
%%%   {% \begin{breakablealgorithm}
%%%    % \begin{center}
%%%      \refstepcounter{algorithm}% New algorithm
%%%      \hrule height.8pt depth0pt \kern2pt% \@fs@pre for \@fs@ruled
%%%      \renewcommand{\caption}[2][\relax]{% Make a new \caption
%%%        % {\raggedright\textbf{\ALG@name~\thealgorithm} ##2\par}%
%%%        {\raggedright\textbf{Protocol~\thealgorithm} ##2\par}%
%%%        \ifx\relax##1\relax % #1 is \relax
%%%          \addcontentsline{loa}{algorithm}{\protect\numberline{\thealgorithm}##2}%
%%%        \else % #1 is not \relax
%%%          \addcontentsline{loa}{algorithm}{\protect\numberline{\thealgorithm}##1}%
%%%        \fi
%%%        \kern2pt\hrule\kern2pt
%%%        \small
%%%      }
%%%   }{% \end{breakablealgorithm}
%%%      \kern2pt\hrule\relax% \@fs@post for \@fs@ruled
%%%    % \end{center}
%%%   }
%%% \makeatother
%%%

\def\EQ#1{\begin{eqnarray}#1\end{eqnarray}} 


%%% Practical abreviations
\usepackage{calrsfs} % Replace \mathcal with different more "cursive" font
\DeclareMathAlphabet{\pazocal}{OMS}{zplm}{m}{n} % \pazocal is now the old \mathcal
\newcommand*{\cA}{\pazocal{A}}
\newcommand*{\cB}{\pazocal{B}}
\newcommand*{\cC}{\pazocal{C}}
\newcommand*{\cD}{\pazocal{D}}
\newcommand*{\cE}{\pazocal{E}}
\newcommand*{\cF}{\pazocal{F}}
\newcommand*{\cG}{\pazocal{G}}
\newcommand*{\cH}{\pazocal{H}}
\newcommand*{\cI}{\pazocal{I}}
\newcommand*{\cJ}{\pazocal{J}}
\newcommand*{\cK}{\pazocal{K}}
\renewcommand*{\cL}{\pazocal{L}} % defined in complexity
\newcommand*{\cM}{\pazocal{M}}
\newcommand*{\cN}{\pazocal{N}}
\newcommand*{\cO}{\pazocal{O}}
\renewcommand*{\cP}{\pazocal{P}} % defined in complexity
\newcommand*{\cQ}{\pazocal{Q}}
\newcommand*{\cR}{\pazocal{R}}
\renewcommand*{\cS}{\pazocal{S}} % defined in complexity
\newcommand*{\cT}{\pazocal{T}}
\newcommand*{\cU}{\pazocal{U}}
\newcommand*{\cV}{\pazocal{V}}
\newcommand*{\cW}{\pazocal{W}}
\newcommand*{\cX}{\pazocal{X}}
\newcommand*{\cY}{\pazocal{Y}}
\newcommand*{\cZ}{\pazocal{Z}}

% Notation for languages. Make sure not to collide with L(H) for linear operators on H
%\DeclareMathAlphabet{\pazocal}{OMS}{zplm}{m}{n}
\newcommand{\linearOp}{\mathcal{L}} % Defines the opearator L(H)
\renewcommand*{\lang}{\pazocal{L}} % Defines a new Turing/Quantum Language

%https://tex.stackexchange.com/questions/195025/
\newcommand*{\rightSend}{\rightsquigarrow} % snake "a --> b" to say "I send a to b".

% For some reasons, proof-at-the-end do not like # (problem with detokenize? To fix...). So we use a macro:
\newcommand*{\diese}{\#}

\newcommand*{\eps}[0]{\varepsilon}
\renewcommand*{\D}[0]{\mathbb{D}}
\renewcommand*{\E}{\mathbb{E}}
\newcommand*{\N}[0]{\mathbb{N}}
\renewcommand*{\R}[0]{\mathbb{R}}
\renewcommand*{\C}[0]{\mathbb{C}}
\newcommand*{\Z}[0]{\mathbb{Z}}
\newcommand*{\T}[0]{\mathbb{T}}

\newcommand*{\CPA}[0]{\textsf{CPA}}
\newcommand*{\LWE}[0]{\textsf{LWE}}
\newcommand*{\GapSVP}[0]{\textsf{GapSVP}}
\newcommand*{\SIVP}[0]{\textsf{SIVP}}

% first param is alpha, second is q (optional)
\newcommand{\contGauss}[1]{\cD_{#1}}
\newcommand{\disGauss}[2]{\cD_{\Z,#1 #2}}
\newcommand{\disGaussCont}[1]{\cD_{\Z,#1}}
\newcommand{\disGaussAQ}{\cD_{\Z,\alpha q}}
\newcommand{\rsafe}{r_{\text{safe}}}
% domain of the function.
\newcommand{\ccX}{\cX}
\newcommand{\ccXSphere}{\cX_\bullet}
\newcommand{\ccXCube}[1][]{\cX_{\scalebox{.4}{$\blacksquare$}#1}}

\newcommand{\BBHeightyFor}{\textsc{BB84}}

% "Correct" spacing around cdot as wildcard
% https://tex.stackexchange.com/questions/78165/spacing-around-cdot-when-used-as-a-wildcard
\newcommand\cdotwild{{\mkern 2mu\cdot\mkern 2mu}}

\newcommand*{\id}{\mathrm{id}}
\DeclareMathOperator{\Tr}{Tr}
\DeclareMathOperator{\supp}{supp}
\DeclareMathOperator{\Det}{Det}
\DeclareMathOperator{\ERR}{ERR}
\DeclareMathOperator{\Round}{Round}

\newcommand{\Ver}{ {\tt Ver} } %% Verify public key
\newcommand{\Gen}{ {\tt Gen} }
\newcommand{\GenLocal}{\ensuremath{ {\tt Gen}_{\tt Loc}} }
\newcommand{\Enc}{ {\tt Enc} }
\newcommand{\Dec}{ {\tt Invert} }
\newcommand{\Eval}{ {\tt Eval} }

\newcommand{\HGen}{ {\tt H.Gen} }
\newcommand{\HEnc}{ {\tt H.Enc} }
\newcommand{\HDec}{ {\tt H.Invert} }
\newcommand{\HEval}{ {\tt H.Eval} }
\newcommand{\Hh}{\ensuremath{{\tt H.}h} }
\newcommand{\HCheckHonestTrapdoor}{\ensuremath{{\tt H.CheckTrapdoor}}}

\newcommand{\MPGen}{ {\tt MP.Gen} }
\newcommand{\MPDec}{ {\tt MP.Invert} }
\newcommand{\InvertSmallGadget}{ \ensuremath{{\texttt{InvertSmallGadget}}} }
\newcommand{\InvertGadget}{ {\tt InvertGadget} }
% \newcommand{\MPEval}{ $g$ }


\DeclareMathOperator{\RoundMod}{RoundMod}


% https://tex.stackexchange.com/a/253746/116348
\usepackage{mleftright}
\newcommand{\BlockMatrix}[2]{%
  \mleft[%
  \begin{array}{@{}#1@{}}%
    #2
  \end{array}%
  \mright]%
}
\newcommand{\SmallBlockMatrix}[2]{
  \mleft[\begin{array}{c}
    #1\tabularnewline % https://tex.stackexchange.com/questions/328745/matrices-with-cryptocode-package
    \hline
    #2
  \end{array}\mright]
}
\newcommand{\HorizBlockMatrix}[2]{
  \mleft[\begin{array}{@{}c|c@{}}
    #1 & #2
  \end{array}\mright]
}

\newcommand*{\vect}[1]{\ensuremath{\mathbf{#1}}}


%% Already defined in *recent* cryptocode.
% \DeclareMathOperator*{\argmax}{arg \, max}
% \DeclareMathOperator*{\argmin}{arg \, min}

\usepackage{xspace}
\xspaceaddexceptions{]\}}
\newcommand{\RSP}{\ensuremath{\sf{RSP}}\xspace}%

\newcommand*{\RSPenc}{\mathrm{RSP}_{\mathrm{enc}}}
\newcommand*{\filter}{\vdash}
\newcommand*{\channelBB}{\cS_{\Z\frac{\pi}{2}}}
% To denote a protocol between two parties
% \newcommand*{\protocol}{\boldsymbol{\pi}}
\newcommand\compRSP{composable $\sf{RSP}_{CC}$}
\newcommand\RSPlike{$\sf{RSP}_{CC}$}
\newcommand\compUBQC{composable $\sf{UBQC}_{CC}$}
\newcommand\UBQClike{$\sf{UBQC}_{CC}$}
% GHZ
\newcommand*{\eqdef}{\coloneqq}
\newcommand*{\PERM}{\ensuremath{}{\sf PERM}}
\newcommand*{\PPT}{\ensuremath{}{\sf PPT}}
\newcommand*{\QPT}{\ensuremath{}{\sf QPT}}
\newcommand*{\Auth}{\ensuremath{}{\sf Auth}}
% Prover, verifier, extractor
\renewcommand*{\P}{\ensuremath{}{\sf P}}
\newcommand*{\V}{\ensuremath{}{\sf V}}
\renewcommand*{\K}{\ensuremath{}{\sf K}}


% Here are the different protocols we use
% Initial protocol
\newcommand{\blind}{\ensuremath{ {\tt BLIND} }}
\newcommand{\blindZK}{\ensuremath{ {\tt BLIND-ZK} }}
% Reveals to people if they are part of the support
\newcommand{\blindSup}{\ensuremath{ {\tt BLIND}^{\tt sup} }}
% Make sure that people in the support share a canonical GHZ
\newcommand{\blindCan}{\ensuremath{ {\tt BLIND}_{\tt can} }}
% Make sure that people in the support share a canonical GHZ and that they know they are supported
\newcommand{\blindCanSup}{\ensuremath{ {\tt BLIND}^{\tt sup}_{\tt can} }}
\newcommand{\authBlindCanDist}{\ensuremath{ {\tt AUTH-BLIND}^{\tt dist}_{\tt can} }}
\newcommand{\verifCanSup}{\ensuremath{ {\tt VERIF}^{\tt sup}_{\tt can} }}
\newcommand{\blindPoly}{\ensuremath{ {\tt BLIND-poly} }}
\newcommand{\INDFCT}{\ensuremath{ {\tt IND-D0} }}
\newcommand{\INDBLIND}{\refGame*{INDBLIND}}
\newcommand{\INDBLINDtxt}{\ensuremath{ {\tt IND-\blind{}} }}
\newcommand{\INDBLINDSUP}{\refGame*{INDBLINDSUP}}
\newcommand{\INDBLINDSUPtxt}{\ensuremath{ {\tt IND-\blindSup{}} }}
\newcommand{\INDBLINDCANSUP}{\ensuremath{ {\tt IND-\blindCanSup{}} }}
\newcommand{\INDBLINDCANAUTH}{\ensuremath{ {\tt IND-\authBlindCanDist{}} }}
\newcommand{\INDPARTIAL}{\ensuremath{ {\tt IND-PARTIAL} }}
\newcommand*\GHZ{\ensuremath{}{\sf GHZ}}
\newcommand*\GHZCan{\ensuremath{{\sf GHZ^{\sf can}}}}
% Generalized GGHZ
\newcommand*\GGHZ{\ensuremath{ {\sf GHZ^G} }}
\newcommand*\HGHZ{\ensuremath{ {\sf GHZ^H} }}
\newcommand{\AssumpFctNoDelta}{\HGHZ{} capable}
\newcommand{\AssumpFctCanNoDelta}{\GHZCan{} capable}
\newcommand{\AssumpFct}{$\delta$-\HGHZ{} capable}
\newcommand{\AssumpFctNegl}{$\negl[\lambda]$-\HGHZ{} capable}
\newcommand{\AssumpFctCan}{$\delta$-\GHZCan{} capable}
\newcommand{\AssumpFctCanPrime}{$\delta^\prime$-\GHZCan{} capable}
\newcommand{\AssumpFctDist}{distributable}
\newcommand{\partialInfo}{\ensuremath{{\tt PartInfo}}}
\newcommand{\partialInfoLocal}{\ensuremath{{\tt PartInfo_{\tt Loc}}}}
\newcommand{\partialAlpha}{\ensuremath{{\tt PartAlpha_{\tt Loc}}}}
\newcommand{\CombineAlpha}{\ensuremath{{\tt CombineAlpha}}}
\newcommand{\CombineRandom}{\ensuremath{{\tt CombineRandom}}}
\newcommand{\CheckHonestTrapdoor}{\ensuremath{{\tt CheckTrapdoor}}}
\newcommand{\ZKGenCheck}{\ensuremath{{\tt ZK-GenCheck}}}

\newcommand{\highlightChanges}[1]{\textcolor{green!50!black}{#1}}

\newcommand{\Real}{\ensuremath{\mathsf{REAL}}}
\newcommand{\Sim}{\ensuremath{\mathsf{Sim}}}
\newcommand{\Ideal}{\ensuremath{\mathsf{IDEAL}}}
\newcommand{\OUT}{\ensuremath{\mathsf{OUT}}}

\usepackage{pifont}% http://ctan.org/pkg/pifont
% To denote that a user is not in the final GHZ
\newcommand{\cross}{\ensuremath{\text{\ding{55}}}}

\usepackage{verbatim}
\newcommand{\leftI}{{\normalfont\ttfamily left}}
\newcommand{\rightI}{{\normalfont\ttfamily right}}
\newcommand*{\mybar}[1]{\overline{#1}}

\newcommand*{\quout}{\mathtt{out}}

\DeclareMathOperator{\Image}{Image}
\DeclareMathOperator{\Ima}{Im}
\DeclareMathOperator{\Dom}{Dom}

\newcommand*{\Pub}{\mathtt{Pub}}
\newcommand*{\accept}{\mathtt{accept}}
\newcommand*{\abort}{\mathtt{abort}}
\newcommand*{\gadxor}{\mathtt{Gad}_{\oplus}}
\newcommand*{\quin}{\mathtt{in}}

%%% Provide a "subproof" environment (pretty simple for now,
%%% just indent the sub-proofs)
% I'd love to have left border... https://tex.stackexchange.com/questions/532948/robustly-add-a-border-to-the-left-of-a-text-spanning-several-pages
\usepackage{changepage}% http://ctan.org/pkg/changepage
\newenvironment{subproof}{\begin{adjustwidth}{2em}{0pt}}{\end{adjustwidth}}


%%% To have nice probabilities
% https://tex.stackexchange.com/questions/198771/align-in-substack
\makeatletter
\newcommand{\subalign}[1]{%
  \vcenter{%
    \Let@ \restore@math@cr \default@tag
    \baselineskip\fontdimen10 \scriptfont\tw@
    \advance\baselineskip\fontdimen12 \scriptfont\tw@
    \lineskip\thr@@\fontdimen8 \scriptfont\thr@@
    \lineskiplimit\lineskip
    \ialign{\hfil$\m@th\scriptstyle##$&$\m@th\scriptstyle{}##$\hfil\crcr
      #1\crcr
    }%
  }%
}
\makeatother
\usepackage{xparse}
% https://tex.stackexchange.com/questions/431794/same-spacing-around-middle-as-around-mid
\let\originalmiddle=\middle
\def\middle#1{\mathrel{}\originalmiddle#1\mathrel{}}
\NewDocumentCommand{\pr}{som}{\Pr\IfValueT{#2}{_{\substack{#2}}}\left[\,#3\,\right]}
\NewDocumentCommand{\esp}{som}{\IfValueTF{#2}{\underset{\subalign{#2}}{\mathbb{E}}}{\mathbb{E}}[\,#3\,]}
\NewDocumentEnvironment{pral}{soO{}}
 {%
  \Pr\IfValueT{#2}{_{\IfBooleanTF{#1}{\substack{#2}}{#2}}}
\begin{alignedat}[t]{2}
  [\, } {\,]
 #3
 \end{alignedat}
 }
\NewDocumentEnvironment{espal}{soO{}}
 {%
   \IfValueTF{#2}{\underset{\subalign{#2}}{\mathbb{E}}}{\mathbb{E}}
   \begin{alignedat}[t]{2}[\,
 }
 {\,]#3\end{alignedat}
 }
% To draw |A><B|, the second argument is facultative and gives |A><A|
\NewDocumentCommand{\ketbra}{mg}{
  | #1 \rangle \langle \IfValueTF{#2}{#2}{#1} |
}
% https://tex.stackexchange.com/a/188364/116348
\DeclareRobustCommand{\minwidthbox}[2]{%
  \ifmmode
    \expandafter\mathmakebox
  \else
    \expandafter\makebox
  \fi
  [\ifdim#2<\width\width\else#2\fi]{#1}%f
}

\NewDocumentCommand{\constructs}{mg}{
  \xrightarrow[\IfValueT{#2}{#2}]{\minwidthbox{#1}{5mm}}
}

%%%%%%%%%% This part is specific to llncs-based papers
% To enable this part of the configuration, add
%  \def\enableLlncsCode{}
% before the \input{include}
\ifdefined\enableLlncsCode%
% go back to standard amsthm proof environments with Qed at the end
\let\proof\relax\let\endproof\relax
  \usepackage{amsthm}
  % To debug, it's a nightmare without page number... To disable
  % in the final version
  \pagestyle{plain}
\fi

% Define environments when loaded in standalone...
\ifdefined\enableNonLlncsCode%
\usepackage{thmtools}
\declaretheorem[name=Theorem,numberwithin=chapter]{theorem}
\declaretheorem[name=Definition,sibling=theorem]{definition}
\declaretheorem[name=Lemma,sibling=theorem]{lemma}
\declaretheorem[name=Corollary,sibling=theorem]{corollary}
\declaretheorem[name=Remark,sibling=theorem]{remark}
\fi

\usepackage{enumitem}
\newlist{compressedList}{itemize}{10}
\setlist[compressedList]{label=--,topsep=0pt,parsep=0pt,partopsep=0pt}
\usepackage{mathtools}
\DeclarePairedDelimiter\ceil{\lceil}{\rceil}
\DeclarePairedDelimiter\floor{\lfloor}{\rfloor}

% https://tex.stackexchange.com/questions/134278/overriding-the-paragraph-style-in-the-lncs-document-class
\usepackage{etoolbox}
\patchcmd{\paragraph}{\itshape}{\bfseries\boldmath}{}{}


% This part is specific to llncs-based papers with nicer output
% (page number, theorem labelled with sections...). This should
% be the version in the arxiv.
% To enable this part of the configuration, add
%  \def\nicerThanLlncs{}
% before the \input{include}
\ifdefined\nicerThanLlncs%
  % change margins, page layout
  \usepackage[a4paper, top=1.4in, bottom=1.4in, right=1in, left=1in]{geometry}
  % enable page numbering, suppressed by default by llncs
  \pagestyle{plain}
  \setcounter{tocdepth}{2}
  \renewcommand\small{\fontsize{10pt}{12pt}\selectfont}
\fi
\ifdefined\nicerThanLlncsAbstract%
  % change margins, page layout
  \usepackage[a4paper, margin=2.5cm]{geometry}
  % enable page numbering, suppressed by default by llncs
  \pagestyle{plain}
  \setcounter{tocdepth}{2}
\fi
\ifdefined\enableLlncsCodeSlightlyImproved%
  \pagestyle{plain}
  \setcounter{tocdepth}{2}
\fi


\ifdefined\disableHyperref%
\else%
% https://rcweb.dartmouth.edu/doc/texmf-dist/doc/latex/cryptocode/cryptocode.pdf (page 55)
%%%%%%%%%% This part should be loaded last
% Usually hyperref needs to be at the end
\definecolor{secondaryColor}{RGB}{206,149,0} %% darker orange, looks like gold. <3
\usepackage[colorlinks, allcolors=secondaryColor]{hyperref}
\fi
%% Hyperref should be loaded before cryptocode! Otherwise, errors with labels of lines for example.
%%% Cryptocode
\usepackage [
n,
advantage,
operators,
sets,
adversary,
landau,
probability,
notions,
logic,
ff,
mm,
primitives,
events,
complexity,
asymptotics,
keys
] {cryptocode}
% I don't like the way "samples" is defined in cryptocode with a big dollar in front.
% \renewcommand\sample{\mathrel{\overset{\vbox to.7ex{\hbox{\fontsize{6}{0}\selectfont\vspace{.5ex} \$}}}{\leftarrow}}}
%% https://tex.stackexchange.com/a/584533/116348
\makeatletter
\renewcommand{\sample}{\mathrel{\mathpalette\sample@\relax}}
\newcommand{\sample@}[2]{%
  \ooalign{%
    \hspace{\stretch{2}}\raisebox{0.7\height}{$\m@th\demotestyle{#1}\mathdollar$}\hspace{\stretch{1}}\cr
    $\m@th#1\leftarrow$\cr
  }%
}
\newcommand{\demotestyle}[1]{%
  \ifx#1\displaystyle\scriptstyle\else
  \ifx#1\textstyle\scriptstyle\else
  \scriptscriptstyle\fi\fi
}
\makeatother
% Create a new command to create games using \game[linenumbering]{Name of game}{Game definition}
\createprocedureblock{game}{center,boxed}{}{}{}
% Create a new environment (following what I did in the thesis) to provide named games
\newcommand{\gameFont}[1]{\texttt{#1}}
\renewcommand{\Game}[1]{\ensuremath{\gameFont{Game#1}}} % By default \Game produces a nice G (symetric), but it's not very standard.

\makeatletter
\createprocedureblock{gameProc}{center,boxed}{}{}{}
\createprocedureblock{interactGame}{center,boxed}{}{}{}
%%% Nicer way to refer to games.
%%% See also https://tex.stackexchange.com/questions/623163 which provides a cleaner way to make it work with cleverref.
% Usage: \begin{namedGame}*[label][short title]{title}*[game options like skipfirstln,lnstart=1] content \end{namedGame}
% first * = disable floating. WARNING: do not put labels with non-letter chars.
% second * = do not include the game in the TOC.
\NewDocumentEnvironment{namedGame}{soomsO{}b}{%
  \IfBooleanTF{#1}{}{%
    \par% without par, the figure will be put after the line arriving after the picture if there is no paragraph.
    \setlength{\intextsep}{5.0pt plus 2.0pt minus 2.0pt}% reduce the space when image is inserted inline.
    \begin{figure}[htbp]}% 
    \begin{pcimage}%
      %% Add to toc see command listofgames
      \phantomsection% Create a fake label, useful to ensure the link point to this part.
      \IfBooleanTF{#5}{}{% If no star is given, add to "table of contents" for games
        \IfValueTF{#3}{% If a short title is provided
          \addcontentsline{game}{section}{\ensuremath{#3}}%
        }{%
          \addcontentsline{game}{section}{\ensuremath{#4}}%
        }%
      }%
      {\normalfont\gameProc[linenumbering,#6]{\ensuremath{#4}}{\IfValueTF{#2}{%
            %% Add an anchor if a label is present
            \raisebox{1em}{\hypertarget{#2}{}}%
            %% Create a macro "mygametitle@nameoflabel" to store the title
            \IfValueTF{#3}{% If a short title is provided
              \expandafter\gdef\csname mygametitle@#2 \endcsname{\ensuremath{#3}}%%
              \write\@auxout{\gdef\string\mygametitle@#2{\ensuremath{#3}}}%
            }{%
              \expandafter\gdef\csname mygametitle@#2 \endcsname{\ensuremath{#4}}%%
              \write\@auxout{\gdef\string\mygametitle@#2{\ensuremath{#4}}}%
            }%
          }{} #7 }}%
    \end{pcimage}
    \IfBooleanTF{#1}{}{\end{figure}}%
}{}

% Usage: \refGame{label}. Use \refGame*{label} if you don't want to color the link (useful in proofs)
\NewDocumentCommand{\refGame}{sm}{%
  \IfBooleanTF{#1}{%
    \hyperlink{#2}{\textcolor{black}{\csname mygametitle@#2\endcsname}}% Do not put a white space after #1!
  }{%
    \hyperlink{#2}{\csname mygametitle@#2\endcsname}% Do not put a white space after #1!
  }%
}
\makeatother

\ifdefined\disableHyperref%
\else%
% Well... not as much as cleveref
\usepackage[capitalise,noabbrev]{cleveref}
% Don't abbrev general names... but still abbrev equations
\crefname{equation}{Eq.}{Eqs.}
\crefname{figure}{Fig.}{Figs.}
\crefname{algorithm}{Protocol}{Protocols}
\Crefname{equation}{Equation}{Equations}
\Crefname{figure}{Figure}{Figures}
\Crefname{algorithm}{Protocol}{Protocols}
\fi

% Put that package at the end of the file,
% or you will have some strange errors like
% "Only one # is allowed per tab"
\usetikzlibrary{quantikz}
% Bug in llncs https://tex.stackexchange.com/a/372108/116348
\def\theHlemma{\theHtheorem}
\def\theHdefinition{\theHtheorem}
\def\theHcorollary{\theHtheorem}

%%%%%% PLEASE DO NOT WRITE ANYTHING AFTER THIS LINE %%%%%%
%%%%%% Indeed, hyperref, cleveref and quantikz need %%%%%%
%%%%%% to be loaded at the very end.                %%%%%%

\ifdefined\enableLlncsCode%
% go back to standard amsthm proof environments with Qed at the end
\let\proof\relax\let\endproof\relax
  \usepackage{amsthm}
  % To debug, it's a nightmare without page number... To disable
  % in the final version
  \pagestyle{plain}
\fi

% Define environments when loaded in standalone...
\ifdefined\enableNonLlncsCode%
\usepackage{thmtools}
\declaretheorem[name=Theorem,numberwithin=chapter]{theorem}
\declaretheorem[name=Definition,sibling=theorem]{definition}
\declaretheorem[name=Lemma,sibling=theorem]{lemma}
\declaretheorem[name=Corollary,sibling=theorem]{corollary}
\declaretheorem[name=Remark,sibling=theorem]{remark}
\fi

\usepackage{enumitem}
\newlist{compressedList}{itemize}{10}
\setlist[compressedList]{label=--,topsep=0pt,parsep=0pt,partopsep=0pt}
\usepackage{mathtools}
\DeclarePairedDelimiter\ceil{\lceil}{\rceil}
\DeclarePairedDelimiter\floor{\lfloor}{\rfloor}

% https://tex.stackexchange.com/questions/134278/overriding-the-paragraph-style-in-the-lncs-document-class
\usepackage{etoolbox}
\patchcmd{\paragraph}{\itshape}{\bfseries\boldmath}{}{}


% This part is specific to llncs-based papers with nicer output
% (page number, theorem labelled with sections...). This should
% be the version in the arxiv.
% To enable this part of the configuration, add
%  \def\nicerThanLlncs{}
% before the %% Let's put in these file all the definitions that we
%% require:

%% Setup language, encoding...
\RequirePackage{etex} 

\usepackage{standalone}
\usepackage{lmodern}
% Language
\usepackage[english]{babel}
% Input encoding
\usepackage[utf8]{inputenc}
% Output encoding https://tex.stackexchange.com/a/677. Important to copy accents
\usepackage[T1]{fontenc}
\usepackage{subfiles} % To quickly load this include file other files
\usepackage{microtype}
\usepackage[shortcuts,nospacearound]{extdash}    % Better endling of em dash (---) to cut sentences\===like 
%%% https://tex.stackexchange.com/questions/2773/how-do-i-make-latex-push-long-citations-to-a-new-line
\usepackage{breakcites}
\usepackage{booktabs}
% Makes sure floats don't fly accross sections
\usepackage{placeins}
\usepackage{adjustbox}
\usepackage{varwidth} % Like minipage, but with automatic width discovery.
\usepackage{complexity} % get \QMA{}...

\usepackage[normalem]{ulem}
%\usepackage[math-style=ISO]{unicode-math}
\usepackage{upgreek} % non italic greak letters

%%%% Bibliography
% I want to have separate references for appendix and main body
% https://tex.stackexchange.com/questions/98660/
\usepackage[style=trad-alpha,
  sortcites=true,
  doi=false,
  url=false,
  giveninits=true, % Bob Foo --> B. Foo
  isbn=false,
  url=false,
  eprint=false,
  sortcites=false, % \cite{B,A,C}: [A,B,C] --> [B,A,C]
]{biblatex}
\defbibfilter{appendixOnlyFilter}{
  segment=1 % Segment 1 will be chosen to be the one in appendix
  and not segment=0 % Default segment is 0
}
\renewcommand{\multicitedelim}{, } % [ABC96; DEF12] -> [ABC96, DEF12]
% [unknown_ref] => [??]
% https://tex.stackexchange.com/a/352573
\makeatletter
\protected\def\abx@missing#1{\textbf{??}}
\makeatother
\renewcommand*{\bibfont}{\normalfont\small} % BibLatex font seems bigger than Bibtex?

\usepackage{xcolor,cancel}
\usepackage{proof-at-the-end}
\NewDocumentEnvironment{theoremE}{O{}O{}+b}{%
  \begin{theoremEnd}[normal,#2]{theorem}[#1]%
    #3%
  \end{theoremEnd}
  %
}{}% Do not forget the second parameter or you might get Missing \begin{document} error
\NewDocumentEnvironment{lemmaE}{O{}O{}+b}{%
  \begin{theoremEnd}[normal,#2]{lemma}[#1]%
    #3%
  \end{theoremEnd}
  %
}{}% Do not forget the second parameter or you might get Missing \begin{document} error
\NewDocumentEnvironment{corollaryE}{O{}O{}+b}{%
  \begin{theoremEnd}[normal,#2]{corollary}[#1]%
    #3%
  \end{theoremEnd}
  %
}{}% Do not forget the second parameter or you might get Missing \begin{document} error
\NewDocumentEnvironment{proofE}{O{}+b}{%
  \begin{proofEnd}[#1]%
    #2
  \end{proofEnd}%
}{}%
\pgfkeys{/prAtEnd/global custom defaults/.style={
    text link={\noindent See \hyperref[proof:prAtEnd\pratendcountercurrent]{proof} in \cref{proofsection:prAtEnd\pratendcountercurrent}.}
  }
}

\ifdefined\displayArxivTexts%
  \pgfkeys{/prAtEnd/end if published/.style={no restate}}
  \pgfkeys{/prAtEnd/all end if published/.style={no restate}}
\else%
  \pgfkeys{/prAtEnd/end if published/.style={end}}
  \pgfkeys{/prAtEnd/all end if published/.style={all end}}
\fi

% https://tex.stackexchange.com/a/20613/116348
% https://tex.stackexchange.com/a/583244/116348
\usepackage{scalerel}
\newcommand\hcancel[2][black]{%
  \ThisStyle{%
    \setbox0=\hbox{$\SavedStyle #2$}%
    \rlap{\raisebox{.25\ht0}{\textcolor{#1}{\rule{\wd0}{1pt}}}}\SavedStyle #2}%
}

%% We provide 3 commands to add comments/text:
%% - \yournameAddText{your text to add}: proposition to add text
%% - \yournameRemoveText{text to remove}: proposition to remove text
%% - \yournameComment{your comments}: to add general comment to discuss. Should not contain proposition to add or remove text, but can be added before/after to explain the reason of the modification.
%% The interest it to provide some quick ways to see what is the final number of pages:
%% just add before the import a line:
%%   \def\removeCommentsAcceptPropositions{}
%% and the proposition will be "accepted" (i.e. the removed text is removed,
%% and the added text will be marked as black), and the comment will be removed, that way it is possible
%% to determine the final number of pages.
%% Example:
%% This text is normal\leoRemove{, this text should be removed}\leoAddText{, this text should be added}\leoComment{This is a comment, explaining why I made that changes.}.
\newcommand{\addEditor}[2]{%
  \expandafter\newcommand\csname #1AddText\endcsname[2][]{%
    \ifdefined\removeCommentsAcceptPropositions %
      ##2%
    \else%
      \textcolor{#2}{\sout{##1}}\textcolor{#2!60!black}{##2}%
    \fi%
  }%
  \expandafter\newcommand\csname #1Comment\endcsname[1]{%
    \ifdefined\removeCommentsAcceptPropositions %
    \else%
      \textcolor{#2!60!white}{\textbf{##1}}%
    \fi%
  }%
  \expandafter\newcommand\csname #1Remove\endcsname[1]{%
    \ifdefined\removeCommentsAcceptPropositions %
    \else%
      \textcolor{#2}{\sout{##1}}%
    \fi%
  }%
}
\definecolor{darkgreen}{rgb}{0., 0.7, 0}
\addEditor{leo}{darkgreen}
\addEditor{fred}{red}
\addEditor{elham}{blue}

%% We also provide an arxivOnly environment, in order to include text that are too long for the submitted version
%% but that would be good to have in the arxiv version.
% Use the \begin{arxivOnly} ... \end{arxivOnly} environment, by putting the opening and ending command
%% on there own line.
%% For shorter texts, you can use the inline version \arxivOnlyShort{your text}.
%% If you want to exclude text from the arxiv, use \begin{publishedOnly} ... \end{publishedOnly} and \publishedOnlyShort{your text}
%% "Comment" package is annoying, it does not allow block indentation, we use 'versions' instead
% \usepackage{versions} %% Do not work inside theoremE, don't know why
\ifdefined\displayArxivTexts%
  % \includeversion{arxivOnly}
  % \excludeversion{publishedOnly}
  \newcommand{\arxivOnly}[1]{#1}
  \newcommand{\arxivOnlyNoDiff}[1]{#1} % Like arxivOnly, but useful in diff mode (when it does not make sense to add colors, for instance inside a cite or when using arxivonly to create lists.
  \newcommand{\publishedOnly}[1]{}
  \newcommand{\publishedVsArxiv}[2]{#2}
  \newcommand{\inAppendixIfPublished}[1]{#1}
\else%
  \ifdefined\displayDiffTexts%
    \colorlet{arxivDiff}{green}
    \colorlet{publishedDiff}{red}
    \newcommand{\arxivOnly}[1]{{\color{arxivDiff}#1}}
    \newcommand{\arxivOnlyNoDiff}[1]{#1}
    \newcommand{\publishedOnly}[1]{{\color{publishedDiff}#1}}
    \newcommand{\publishedVsArxiv}[2]{{\color{publishedDiff}#1}{\color{arxivDiff}#2}}
    \newcommand{\inAppendixIfPublished}[1]{{\color{arxivDiff}\textbf{MOVED IN APPENDIX IN PUBLISHED}#1}\textEnd{{\color{publishedDiff}\textbf{MOVED IN MAIN BODY IN ARXIV}#1}}}
  \else
    % \excludeversion{arxivOnly}
    % \includeversion{publishedOnly}
    \newcommand{\arxivOnly}[1]{}
    \newcommand{\arxivOnlyNoDiff}[1]{}
    \newcommand{\publishedOnly}[1]{#1}
    \newcommand{\publishedVsArxiv}[2]{#1}
    \newcommand{\inAppendixIfPublished}[1]{\textEnd{#1}}
  \fi
\fi
% Alias, easier to remember  
\newcommand{\inAppendix}[1]{\textEnd{#1}}
  
% Symbol to display when an arxiv text is supposed to appear
\usepackage{todonotes}
\newcommand{\notifyArxiv}{\ifdefined\arxivDraftMode\todo{arXiv}\fi}

% Remove unwanted space before/after lists
\usepackage{enumitem}
\publishedOnly{\setlist[itemize]{noitemsep, topsep=0pt}}

% Usual packages:
\usepackage{amsmath}
\usepackage{amssymb}
% \usepackage{amsthm}
\usepackage{thmtools}
\usepackage{graphicx}
\graphicspath{{}{figures/}}
\usepackage{float}
\usepackage{subfig}
\usepackage{wrapfig}
\usepackage{array}
\usepackage{braket}
% Refer to footnotes:
% https://tex.stackexchange.com/a/167382/116348
\usepackage{footmisc}
% Useful to use the latest NewDocumentCommand:
%\usepackage{xparse}
%\usepackage[poster]{tcolorbox}

%% Space between pictures and text
%% https://mirrors.chevalier.io/CTAN/macros/latex/contrib/layouts/layman.pdf
%% \usepackage{layouts}
%% https://tex.stackexchange.com/questions/60477/remove-space-after-figure-and-before-text?rq=1
\setlength{\floatsep}{7pt plus 2pt minus 2pt}
\setlength{\textfloatsep}{7pt plus 2pt minus 2pt}
\setlength{\intextsep}{7pt plus 2pt minus 2pt}
% https://tex.stackexchange.com/questions/39017/how-to-influence-the-position-of-float-environments-like-figure-and-table-in-lat
\setcounter{topnumber}{8}
\setcounter{bottomnumber}{8}
\setcounter{totalnumber}{8}
%% https://tex.stackexchange.com/questions/45990/how-can-i-modify-vertical-space-between-figure-and-caption
%% plus/minus allows to stretch a bit around 15pt
% \setlength{\abovecaptionskip}{5pt}

% Remove number to subsubsection (in llncs it replaces \paragraph)
\setcounter{secnumdepth}{2}

%%% Some practical environment to define theorems, lemma...
% \declaretheorem[name=Theorem,numberwithin=section]{theorem}
% \declaretheorem[name=Definition,sibling=theorem]{definition}
% \declaretheorem[name=Lemma,sibling=theorem]{lemma}
% \declaretheorem[name=Corollary,sibling=theorem]{corollary}
% \declaretheorem[name=Remark,sibling=theorem]{remark}
%%% In LNCS we use
% https://tex.stackexchange.com/a/373125/116348
% \spnewtheorem{thm}{Theorem}[section]{\bfseries}{\itshape}

%%%
\usepackage[ruled,vlined]{algorithm2e}
\SetAlFnt{\normalsize}%
\setlength{\algomargin}{0em} % remove margin on left of algo
\SetAlgorithmName{Protocol}{protocol}{List of Protocols}
\newenvironment{protocol}[1][htb]{\begin{algorithm}[#1]}{\end{algorithm}}

%%% %%% Load algorithm package, and create a breakable version
%%% \usepackage{algorithm, algorithmicx, algpseudocode}
%%% % Decrease space between text:
%%% % https://tex.stackexchange.com/questions/25828/how-to-remove-change-the-vertical-spacing-before-and-after-an-algorithm-enviro/25832
%%% \setlength{\intextsep}{1\baselineskip}
%%% % https://tex.stackexchange.com/questions/387577/new-algorithm-environment-with-a-separate-counter
%%% \newcounter{protocol}
%%% \makeatletter
%%% \newenvironment{protocol}[1][htb]{%
%%%   \let\c@algorithm\c@protocol
%%%   \renewcommand{\ALG@name}{Protocol}% Update algorithm name
%%%   \begin{algorithm}[#1]%
%%%   }{\end{algorithm}
%%% }
%%% \makeatother
%%%
%%% \providecommand*\algorithmautorefname{Protocol}
%%% \makeatletter
%%% \newenvironment{breakablealgorithm}
%%%   {% \begin{breakablealgorithm}
%%%    % \begin{center}
%%%      \refstepcounter{algorithm}% New algorithm
%%%      \hrule height.8pt depth0pt \kern2pt% \@fs@pre for \@fs@ruled
%%%      \renewcommand{\caption}[2][\relax]{% Make a new \caption
%%%        % {\raggedright\textbf{\ALG@name~\thealgorithm} ##2\par}%
%%%        {\raggedright\textbf{Protocol~\thealgorithm} ##2\par}%
%%%        \ifx\relax##1\relax % #1 is \relax
%%%          \addcontentsline{loa}{algorithm}{\protect\numberline{\thealgorithm}##2}%
%%%        \else % #1 is not \relax
%%%          \addcontentsline{loa}{algorithm}{\protect\numberline{\thealgorithm}##1}%
%%%        \fi
%%%        \kern2pt\hrule\kern2pt
%%%        \small
%%%      }
%%%   }{% \end{breakablealgorithm}
%%%      \kern2pt\hrule\relax% \@fs@post for \@fs@ruled
%%%    % \end{center}
%%%   }
%%% \makeatother
%%%

\def\EQ#1{\begin{eqnarray}#1\end{eqnarray}} 


%%% Practical abreviations
\usepackage{calrsfs} % Replace \mathcal with different more "cursive" font
\DeclareMathAlphabet{\pazocal}{OMS}{zplm}{m}{n} % \pazocal is now the old \mathcal
\newcommand*{\cA}{\pazocal{A}}
\newcommand*{\cB}{\pazocal{B}}
\newcommand*{\cC}{\pazocal{C}}
\newcommand*{\cD}{\pazocal{D}}
\newcommand*{\cE}{\pazocal{E}}
\newcommand*{\cF}{\pazocal{F}}
\newcommand*{\cG}{\pazocal{G}}
\newcommand*{\cH}{\pazocal{H}}
\newcommand*{\cI}{\pazocal{I}}
\newcommand*{\cJ}{\pazocal{J}}
\newcommand*{\cK}{\pazocal{K}}
\renewcommand*{\cL}{\pazocal{L}} % defined in complexity
\newcommand*{\cM}{\pazocal{M}}
\newcommand*{\cN}{\pazocal{N}}
\newcommand*{\cO}{\pazocal{O}}
\renewcommand*{\cP}{\pazocal{P}} % defined in complexity
\newcommand*{\cQ}{\pazocal{Q}}
\newcommand*{\cR}{\pazocal{R}}
\renewcommand*{\cS}{\pazocal{S}} % defined in complexity
\newcommand*{\cT}{\pazocal{T}}
\newcommand*{\cU}{\pazocal{U}}
\newcommand*{\cV}{\pazocal{V}}
\newcommand*{\cW}{\pazocal{W}}
\newcommand*{\cX}{\pazocal{X}}
\newcommand*{\cY}{\pazocal{Y}}
\newcommand*{\cZ}{\pazocal{Z}}

% Notation for languages. Make sure not to collide with L(H) for linear operators on H
%\DeclareMathAlphabet{\pazocal}{OMS}{zplm}{m}{n}
\newcommand{\linearOp}{\mathcal{L}} % Defines the opearator L(H)
\renewcommand*{\lang}{\pazocal{L}} % Defines a new Turing/Quantum Language

%https://tex.stackexchange.com/questions/195025/
\newcommand*{\rightSend}{\rightsquigarrow} % snake "a --> b" to say "I send a to b".

% For some reasons, proof-at-the-end do not like # (problem with detokenize? To fix...). So we use a macro:
\newcommand*{\diese}{\#}

\newcommand*{\eps}[0]{\varepsilon}
\renewcommand*{\D}[0]{\mathbb{D}}
\renewcommand*{\E}{\mathbb{E}}
\newcommand*{\N}[0]{\mathbb{N}}
\renewcommand*{\R}[0]{\mathbb{R}}
\renewcommand*{\C}[0]{\mathbb{C}}
\newcommand*{\Z}[0]{\mathbb{Z}}
\newcommand*{\T}[0]{\mathbb{T}}

\newcommand*{\CPA}[0]{\textsf{CPA}}
\newcommand*{\LWE}[0]{\textsf{LWE}}
\newcommand*{\GapSVP}[0]{\textsf{GapSVP}}
\newcommand*{\SIVP}[0]{\textsf{SIVP}}

% first param is alpha, second is q (optional)
\newcommand{\contGauss}[1]{\cD_{#1}}
\newcommand{\disGauss}[2]{\cD_{\Z,#1 #2}}
\newcommand{\disGaussCont}[1]{\cD_{\Z,#1}}
\newcommand{\disGaussAQ}{\cD_{\Z,\alpha q}}
\newcommand{\rsafe}{r_{\text{safe}}}
% domain of the function.
\newcommand{\ccX}{\cX}
\newcommand{\ccXSphere}{\cX_\bullet}
\newcommand{\ccXCube}[1][]{\cX_{\scalebox{.4}{$\blacksquare$}#1}}

\newcommand{\BBHeightyFor}{\textsc{BB84}}

% "Correct" spacing around cdot as wildcard
% https://tex.stackexchange.com/questions/78165/spacing-around-cdot-when-used-as-a-wildcard
\newcommand\cdotwild{{\mkern 2mu\cdot\mkern 2mu}}

\newcommand*{\id}{\mathrm{id}}
\DeclareMathOperator{\Tr}{Tr}
\DeclareMathOperator{\supp}{supp}
\DeclareMathOperator{\Det}{Det}
\DeclareMathOperator{\ERR}{ERR}
\DeclareMathOperator{\Round}{Round}

\newcommand{\Ver}{ {\tt Ver} } %% Verify public key
\newcommand{\Gen}{ {\tt Gen} }
\newcommand{\GenLocal}{\ensuremath{ {\tt Gen}_{\tt Loc}} }
\newcommand{\Enc}{ {\tt Enc} }
\newcommand{\Dec}{ {\tt Invert} }
\newcommand{\Eval}{ {\tt Eval} }

\newcommand{\HGen}{ {\tt H.Gen} }
\newcommand{\HEnc}{ {\tt H.Enc} }
\newcommand{\HDec}{ {\tt H.Invert} }
\newcommand{\HEval}{ {\tt H.Eval} }
\newcommand{\Hh}{\ensuremath{{\tt H.}h} }
\newcommand{\HCheckHonestTrapdoor}{\ensuremath{{\tt H.CheckTrapdoor}}}

\newcommand{\MPGen}{ {\tt MP.Gen} }
\newcommand{\MPDec}{ {\tt MP.Invert} }
\newcommand{\InvertSmallGadget}{ \ensuremath{{\texttt{InvertSmallGadget}}} }
\newcommand{\InvertGadget}{ {\tt InvertGadget} }
% \newcommand{\MPEval}{ $g$ }


\DeclareMathOperator{\RoundMod}{RoundMod}


% https://tex.stackexchange.com/a/253746/116348
\usepackage{mleftright}
\newcommand{\BlockMatrix}[2]{%
  \mleft[%
  \begin{array}{@{}#1@{}}%
    #2
  \end{array}%
  \mright]%
}
\newcommand{\SmallBlockMatrix}[2]{
  \mleft[\begin{array}{c}
    #1\tabularnewline % https://tex.stackexchange.com/questions/328745/matrices-with-cryptocode-package
    \hline
    #2
  \end{array}\mright]
}
\newcommand{\HorizBlockMatrix}[2]{
  \mleft[\begin{array}{@{}c|c@{}}
    #1 & #2
  \end{array}\mright]
}

\newcommand*{\vect}[1]{\ensuremath{\mathbf{#1}}}


%% Already defined in *recent* cryptocode.
% \DeclareMathOperator*{\argmax}{arg \, max}
% \DeclareMathOperator*{\argmin}{arg \, min}

\usepackage{xspace}
\xspaceaddexceptions{]\}}
\newcommand{\RSP}{\ensuremath{\sf{RSP}}\xspace}%

\newcommand*{\RSPenc}{\mathrm{RSP}_{\mathrm{enc}}}
\newcommand*{\filter}{\vdash}
\newcommand*{\channelBB}{\cS_{\Z\frac{\pi}{2}}}
% To denote a protocol between two parties
% \newcommand*{\protocol}{\boldsymbol{\pi}}
\newcommand\compRSP{composable $\sf{RSP}_{CC}$}
\newcommand\RSPlike{$\sf{RSP}_{CC}$}
\newcommand\compUBQC{composable $\sf{UBQC}_{CC}$}
\newcommand\UBQClike{$\sf{UBQC}_{CC}$}
% GHZ
\newcommand*{\eqdef}{\coloneqq}
\newcommand*{\PERM}{\ensuremath{}{\sf PERM}}
\newcommand*{\PPT}{\ensuremath{}{\sf PPT}}
\newcommand*{\QPT}{\ensuremath{}{\sf QPT}}
\newcommand*{\Auth}{\ensuremath{}{\sf Auth}}
% Prover, verifier, extractor
\renewcommand*{\P}{\ensuremath{}{\sf P}}
\newcommand*{\V}{\ensuremath{}{\sf V}}
\renewcommand*{\K}{\ensuremath{}{\sf K}}


% Here are the different protocols we use
% Initial protocol
\newcommand{\blind}{\ensuremath{ {\tt BLIND} }}
\newcommand{\blindZK}{\ensuremath{ {\tt BLIND-ZK} }}
% Reveals to people if they are part of the support
\newcommand{\blindSup}{\ensuremath{ {\tt BLIND}^{\tt sup} }}
% Make sure that people in the support share a canonical GHZ
\newcommand{\blindCan}{\ensuremath{ {\tt BLIND}_{\tt can} }}
% Make sure that people in the support share a canonical GHZ and that they know they are supported
\newcommand{\blindCanSup}{\ensuremath{ {\tt BLIND}^{\tt sup}_{\tt can} }}
\newcommand{\authBlindCanDist}{\ensuremath{ {\tt AUTH-BLIND}^{\tt dist}_{\tt can} }}
\newcommand{\verifCanSup}{\ensuremath{ {\tt VERIF}^{\tt sup}_{\tt can} }}
\newcommand{\blindPoly}{\ensuremath{ {\tt BLIND-poly} }}
\newcommand{\INDFCT}{\ensuremath{ {\tt IND-D0} }}
\newcommand{\INDBLIND}{\refGame*{INDBLIND}}
\newcommand{\INDBLINDtxt}{\ensuremath{ {\tt IND-\blind{}} }}
\newcommand{\INDBLINDSUP}{\refGame*{INDBLINDSUP}}
\newcommand{\INDBLINDSUPtxt}{\ensuremath{ {\tt IND-\blindSup{}} }}
\newcommand{\INDBLINDCANSUP}{\ensuremath{ {\tt IND-\blindCanSup{}} }}
\newcommand{\INDBLINDCANAUTH}{\ensuremath{ {\tt IND-\authBlindCanDist{}} }}
\newcommand{\INDPARTIAL}{\ensuremath{ {\tt IND-PARTIAL} }}
\newcommand*\GHZ{\ensuremath{}{\sf GHZ}}
\newcommand*\GHZCan{\ensuremath{{\sf GHZ^{\sf can}}}}
% Generalized GGHZ
\newcommand*\GGHZ{\ensuremath{ {\sf GHZ^G} }}
\newcommand*\HGHZ{\ensuremath{ {\sf GHZ^H} }}
\newcommand{\AssumpFctNoDelta}{\HGHZ{} capable}
\newcommand{\AssumpFctCanNoDelta}{\GHZCan{} capable}
\newcommand{\AssumpFct}{$\delta$-\HGHZ{} capable}
\newcommand{\AssumpFctNegl}{$\negl[\lambda]$-\HGHZ{} capable}
\newcommand{\AssumpFctCan}{$\delta$-\GHZCan{} capable}
\newcommand{\AssumpFctCanPrime}{$\delta^\prime$-\GHZCan{} capable}
\newcommand{\AssumpFctDist}{distributable}
\newcommand{\partialInfo}{\ensuremath{{\tt PartInfo}}}
\newcommand{\partialInfoLocal}{\ensuremath{{\tt PartInfo_{\tt Loc}}}}
\newcommand{\partialAlpha}{\ensuremath{{\tt PartAlpha_{\tt Loc}}}}
\newcommand{\CombineAlpha}{\ensuremath{{\tt CombineAlpha}}}
\newcommand{\CombineRandom}{\ensuremath{{\tt CombineRandom}}}
\newcommand{\CheckHonestTrapdoor}{\ensuremath{{\tt CheckTrapdoor}}}
\newcommand{\ZKGenCheck}{\ensuremath{{\tt ZK-GenCheck}}}

\newcommand{\highlightChanges}[1]{\textcolor{green!50!black}{#1}}

\newcommand{\Real}{\ensuremath{\mathsf{REAL}}}
\newcommand{\Sim}{\ensuremath{\mathsf{Sim}}}
\newcommand{\Ideal}{\ensuremath{\mathsf{IDEAL}}}
\newcommand{\OUT}{\ensuremath{\mathsf{OUT}}}

\usepackage{pifont}% http://ctan.org/pkg/pifont
% To denote that a user is not in the final GHZ
\newcommand{\cross}{\ensuremath{\text{\ding{55}}}}

\usepackage{verbatim}
\newcommand{\leftI}{{\normalfont\ttfamily left}}
\newcommand{\rightI}{{\normalfont\ttfamily right}}
\newcommand*{\mybar}[1]{\overline{#1}}

\newcommand*{\quout}{\mathtt{out}}

\DeclareMathOperator{\Image}{Image}
\DeclareMathOperator{\Ima}{Im}
\DeclareMathOperator{\Dom}{Dom}

\newcommand*{\Pub}{\mathtt{Pub}}
\newcommand*{\accept}{\mathtt{accept}}
\newcommand*{\abort}{\mathtt{abort}}
\newcommand*{\gadxor}{\mathtt{Gad}_{\oplus}}
\newcommand*{\quin}{\mathtt{in}}

%%% Provide a "subproof" environment (pretty simple for now,
%%% just indent the sub-proofs)
% I'd love to have left border... https://tex.stackexchange.com/questions/532948/robustly-add-a-border-to-the-left-of-a-text-spanning-several-pages
\usepackage{changepage}% http://ctan.org/pkg/changepage
\newenvironment{subproof}{\begin{adjustwidth}{2em}{0pt}}{\end{adjustwidth}}


%%% To have nice probabilities
% https://tex.stackexchange.com/questions/198771/align-in-substack
\makeatletter
\newcommand{\subalign}[1]{%
  \vcenter{%
    \Let@ \restore@math@cr \default@tag
    \baselineskip\fontdimen10 \scriptfont\tw@
    \advance\baselineskip\fontdimen12 \scriptfont\tw@
    \lineskip\thr@@\fontdimen8 \scriptfont\thr@@
    \lineskiplimit\lineskip
    \ialign{\hfil$\m@th\scriptstyle##$&$\m@th\scriptstyle{}##$\hfil\crcr
      #1\crcr
    }%
  }%
}
\makeatother
\usepackage{xparse}
% https://tex.stackexchange.com/questions/431794/same-spacing-around-middle-as-around-mid
\let\originalmiddle=\middle
\def\middle#1{\mathrel{}\originalmiddle#1\mathrel{}}
\NewDocumentCommand{\pr}{som}{\Pr\IfValueT{#2}{_{\substack{#2}}}\left[\,#3\,\right]}
\NewDocumentCommand{\esp}{som}{\IfValueTF{#2}{\underset{\subalign{#2}}{\mathbb{E}}}{\mathbb{E}}[\,#3\,]}
\NewDocumentEnvironment{pral}{soO{}}
 {%
  \Pr\IfValueT{#2}{_{\IfBooleanTF{#1}{\substack{#2}}{#2}}}
\begin{alignedat}[t]{2}
  [\, } {\,]
 #3
 \end{alignedat}
 }
\NewDocumentEnvironment{espal}{soO{}}
 {%
   \IfValueTF{#2}{\underset{\subalign{#2}}{\mathbb{E}}}{\mathbb{E}}
   \begin{alignedat}[t]{2}[\,
 }
 {\,]#3\end{alignedat}
 }
% To draw |A><B|, the second argument is facultative and gives |A><A|
\NewDocumentCommand{\ketbra}{mg}{
  | #1 \rangle \langle \IfValueTF{#2}{#2}{#1} |
}
% https://tex.stackexchange.com/a/188364/116348
\DeclareRobustCommand{\minwidthbox}[2]{%
  \ifmmode
    \expandafter\mathmakebox
  \else
    \expandafter\makebox
  \fi
  [\ifdim#2<\width\width\else#2\fi]{#1}%f
}

\NewDocumentCommand{\constructs}{mg}{
  \xrightarrow[\IfValueT{#2}{#2}]{\minwidthbox{#1}{5mm}}
}

%%%%%%%%%% This part is specific to llncs-based papers
% To enable this part of the configuration, add
%  \def\enableLlncsCode{}
% before the \input{include}
\ifdefined\enableLlncsCode%
% go back to standard amsthm proof environments with Qed at the end
\let\proof\relax\let\endproof\relax
  \usepackage{amsthm}
  % To debug, it's a nightmare without page number... To disable
  % in the final version
  \pagestyle{plain}
\fi

% Define environments when loaded in standalone...
\ifdefined\enableNonLlncsCode%
\usepackage{thmtools}
\declaretheorem[name=Theorem,numberwithin=chapter]{theorem}
\declaretheorem[name=Definition,sibling=theorem]{definition}
\declaretheorem[name=Lemma,sibling=theorem]{lemma}
\declaretheorem[name=Corollary,sibling=theorem]{corollary}
\declaretheorem[name=Remark,sibling=theorem]{remark}
\fi

\usepackage{enumitem}
\newlist{compressedList}{itemize}{10}
\setlist[compressedList]{label=--,topsep=0pt,parsep=0pt,partopsep=0pt}
\usepackage{mathtools}
\DeclarePairedDelimiter\ceil{\lceil}{\rceil}
\DeclarePairedDelimiter\floor{\lfloor}{\rfloor}

% https://tex.stackexchange.com/questions/134278/overriding-the-paragraph-style-in-the-lncs-document-class
\usepackage{etoolbox}
\patchcmd{\paragraph}{\itshape}{\bfseries\boldmath}{}{}


% This part is specific to llncs-based papers with nicer output
% (page number, theorem labelled with sections...). This should
% be the version in the arxiv.
% To enable this part of the configuration, add
%  \def\nicerThanLlncs{}
% before the \input{include}
\ifdefined\nicerThanLlncs%
  % change margins, page layout
  \usepackage[a4paper, top=1.4in, bottom=1.4in, right=1in, left=1in]{geometry}
  % enable page numbering, suppressed by default by llncs
  \pagestyle{plain}
  \setcounter{tocdepth}{2}
  \renewcommand\small{\fontsize{10pt}{12pt}\selectfont}
\fi
\ifdefined\nicerThanLlncsAbstract%
  % change margins, page layout
  \usepackage[a4paper, margin=2.5cm]{geometry}
  % enable page numbering, suppressed by default by llncs
  \pagestyle{plain}
  \setcounter{tocdepth}{2}
\fi
\ifdefined\enableLlncsCodeSlightlyImproved%
  \pagestyle{plain}
  \setcounter{tocdepth}{2}
\fi


\ifdefined\disableHyperref%
\else%
% https://rcweb.dartmouth.edu/doc/texmf-dist/doc/latex/cryptocode/cryptocode.pdf (page 55)
%%%%%%%%%% This part should be loaded last
% Usually hyperref needs to be at the end
\definecolor{secondaryColor}{RGB}{206,149,0} %% darker orange, looks like gold. <3
\usepackage[colorlinks, allcolors=secondaryColor]{hyperref}
\fi
%% Hyperref should be loaded before cryptocode! Otherwise, errors with labels of lines for example.
%%% Cryptocode
\usepackage [
n,
advantage,
operators,
sets,
adversary,
landau,
probability,
notions,
logic,
ff,
mm,
primitives,
events,
complexity,
asymptotics,
keys
] {cryptocode}
% I don't like the way "samples" is defined in cryptocode with a big dollar in front.
% \renewcommand\sample{\mathrel{\overset{\vbox to.7ex{\hbox{\fontsize{6}{0}\selectfont\vspace{.5ex} \$}}}{\leftarrow}}}
%% https://tex.stackexchange.com/a/584533/116348
\makeatletter
\renewcommand{\sample}{\mathrel{\mathpalette\sample@\relax}}
\newcommand{\sample@}[2]{%
  \ooalign{%
    \hspace{\stretch{2}}\raisebox{0.7\height}{$\m@th\demotestyle{#1}\mathdollar$}\hspace{\stretch{1}}\cr
    $\m@th#1\leftarrow$\cr
  }%
}
\newcommand{\demotestyle}[1]{%
  \ifx#1\displaystyle\scriptstyle\else
  \ifx#1\textstyle\scriptstyle\else
  \scriptscriptstyle\fi\fi
}
\makeatother
% Create a new command to create games using \game[linenumbering]{Name of game}{Game definition}
\createprocedureblock{game}{center,boxed}{}{}{}
% Create a new environment (following what I did in the thesis) to provide named games
\newcommand{\gameFont}[1]{\texttt{#1}}
\renewcommand{\Game}[1]{\ensuremath{\gameFont{Game#1}}} % By default \Game produces a nice G (symetric), but it's not very standard.

\makeatletter
\createprocedureblock{gameProc}{center,boxed}{}{}{}
\createprocedureblock{interactGame}{center,boxed}{}{}{}
%%% Nicer way to refer to games.
%%% See also https://tex.stackexchange.com/questions/623163 which provides a cleaner way to make it work with cleverref.
% Usage: \begin{namedGame}*[label][short title]{title}*[game options like skipfirstln,lnstart=1] content \end{namedGame}
% first * = disable floating. WARNING: do not put labels with non-letter chars.
% second * = do not include the game in the TOC.
\NewDocumentEnvironment{namedGame}{soomsO{}b}{%
  \IfBooleanTF{#1}{}{%
    \par% without par, the figure will be put after the line arriving after the picture if there is no paragraph.
    \setlength{\intextsep}{5.0pt plus 2.0pt minus 2.0pt}% reduce the space when image is inserted inline.
    \begin{figure}[htbp]}% 
    \begin{pcimage}%
      %% Add to toc see command listofgames
      \phantomsection% Create a fake label, useful to ensure the link point to this part.
      \IfBooleanTF{#5}{}{% If no star is given, add to "table of contents" for games
        \IfValueTF{#3}{% If a short title is provided
          \addcontentsline{game}{section}{\ensuremath{#3}}%
        }{%
          \addcontentsline{game}{section}{\ensuremath{#4}}%
        }%
      }%
      {\normalfont\gameProc[linenumbering,#6]{\ensuremath{#4}}{\IfValueTF{#2}{%
            %% Add an anchor if a label is present
            \raisebox{1em}{\hypertarget{#2}{}}%
            %% Create a macro "mygametitle@nameoflabel" to store the title
            \IfValueTF{#3}{% If a short title is provided
              \expandafter\gdef\csname mygametitle@#2 \endcsname{\ensuremath{#3}}%%
              \write\@auxout{\gdef\string\mygametitle@#2{\ensuremath{#3}}}%
            }{%
              \expandafter\gdef\csname mygametitle@#2 \endcsname{\ensuremath{#4}}%%
              \write\@auxout{\gdef\string\mygametitle@#2{\ensuremath{#4}}}%
            }%
          }{} #7 }}%
    \end{pcimage}
    \IfBooleanTF{#1}{}{\end{figure}}%
}{}

% Usage: \refGame{label}. Use \refGame*{label} if you don't want to color the link (useful in proofs)
\NewDocumentCommand{\refGame}{sm}{%
  \IfBooleanTF{#1}{%
    \hyperlink{#2}{\textcolor{black}{\csname mygametitle@#2\endcsname}}% Do not put a white space after #1!
  }{%
    \hyperlink{#2}{\csname mygametitle@#2\endcsname}% Do not put a white space after #1!
  }%
}
\makeatother

\ifdefined\disableHyperref%
\else%
% Well... not as much as cleveref
\usepackage[capitalise,noabbrev]{cleveref}
% Don't abbrev general names... but still abbrev equations
\crefname{equation}{Eq.}{Eqs.}
\crefname{figure}{Fig.}{Figs.}
\crefname{algorithm}{Protocol}{Protocols}
\Crefname{equation}{Equation}{Equations}
\Crefname{figure}{Figure}{Figures}
\Crefname{algorithm}{Protocol}{Protocols}
\fi

% Put that package at the end of the file,
% or you will have some strange errors like
% "Only one # is allowed per tab"
\usetikzlibrary{quantikz}
% Bug in llncs https://tex.stackexchange.com/a/372108/116348
\def\theHlemma{\theHtheorem}
\def\theHdefinition{\theHtheorem}
\def\theHcorollary{\theHtheorem}

%%%%%% PLEASE DO NOT WRITE ANYTHING AFTER THIS LINE %%%%%%
%%%%%% Indeed, hyperref, cleveref and quantikz need %%%%%%
%%%%%% to be loaded at the very end.                %%%%%%

\ifdefined\nicerThanLlncs%
  % change margins, page layout
  \usepackage[a4paper, top=1.4in, bottom=1.4in, right=1in, left=1in]{geometry}
  % enable page numbering, suppressed by default by llncs
  \pagestyle{plain}
  \setcounter{tocdepth}{2}
  \renewcommand\small{\fontsize{10pt}{12pt}\selectfont}
\fi
\ifdefined\nicerThanLlncsAbstract%
  % change margins, page layout
  \usepackage[a4paper, margin=2.5cm]{geometry}
  % enable page numbering, suppressed by default by llncs
  \pagestyle{plain}
  \setcounter{tocdepth}{2}
\fi
\ifdefined\enableLlncsCodeSlightlyImproved%
  \pagestyle{plain}
  \setcounter{tocdepth}{2}
\fi


\ifdefined\disableHyperref%
\else%
% https://rcweb.dartmouth.edu/doc/texmf-dist/doc/latex/cryptocode/cryptocode.pdf (page 55)
%%%%%%%%%% This part should be loaded last
% Usually hyperref needs to be at the end
\definecolor{secondaryColor}{RGB}{206,149,0} %% darker orange, looks like gold. <3
\usepackage[colorlinks, allcolors=secondaryColor]{hyperref}
\fi
%% Hyperref should be loaded before cryptocode! Otherwise, errors with labels of lines for example.
%%% Cryptocode
\usepackage [
n,
advantage,
operators,
sets,
adversary,
landau,
probability,
notions,
logic,
ff,
mm,
primitives,
events,
complexity,
asymptotics,
keys
] {cryptocode}
% I don't like the way "samples" is defined in cryptocode with a big dollar in front.
% \renewcommand\sample{\mathrel{\overset{\vbox to.7ex{\hbox{\fontsize{6}{0}\selectfont\vspace{.5ex} \$}}}{\leftarrow}}}
%% https://tex.stackexchange.com/a/584533/116348
\makeatletter
\renewcommand{\sample}{\mathrel{\mathpalette\sample@\relax}}
\newcommand{\sample@}[2]{%
  \ooalign{%
    \hspace{\stretch{2}}\raisebox{0.7\height}{$\m@th\demotestyle{#1}\mathdollar$}\hspace{\stretch{1}}\cr
    $\m@th#1\leftarrow$\cr
  }%
}
\newcommand{\demotestyle}[1]{%
  \ifx#1\displaystyle\scriptstyle\else
  \ifx#1\textstyle\scriptstyle\else
  \scriptscriptstyle\fi\fi
}
\makeatother
% Create a new command to create games using \game[linenumbering]{Name of game}{Game definition}
\createprocedureblock{game}{center,boxed}{}{}{}
% Create a new environment (following what I did in the thesis) to provide named games
\newcommand{\gameFont}[1]{\texttt{#1}}
\renewcommand{\Game}[1]{\ensuremath{\gameFont{Game#1}}} % By default \Game produces a nice G (symetric), but it's not very standard.

\makeatletter
\createprocedureblock{gameProc}{center,boxed}{}{}{}
\createprocedureblock{interactGame}{center,boxed}{}{}{}
%%% Nicer way to refer to games.
%%% See also https://tex.stackexchange.com/questions/623163 which provides a cleaner way to make it work with cleverref.
% Usage: \begin{namedGame}*[label][short title]{title}*[game options like skipfirstln,lnstart=1] content \end{namedGame}
% first * = disable floating. WARNING: do not put labels with non-letter chars.
% second * = do not include the game in the TOC.
\NewDocumentEnvironment{namedGame}{soomsO{}b}{%
  \IfBooleanTF{#1}{}{%
    \par% without par, the figure will be put after the line arriving after the picture if there is no paragraph.
    \setlength{\intextsep}{5.0pt plus 2.0pt minus 2.0pt}% reduce the space when image is inserted inline.
    \begin{figure}[htbp]}% 
    \begin{pcimage}%
      %% Add to toc see command listofgames
      \phantomsection% Create a fake label, useful to ensure the link point to this part.
      \IfBooleanTF{#5}{}{% If no star is given, add to "table of contents" for games
        \IfValueTF{#3}{% If a short title is provided
          \addcontentsline{game}{section}{\ensuremath{#3}}%
        }{%
          \addcontentsline{game}{section}{\ensuremath{#4}}%
        }%
      }%
      {\normalfont\gameProc[linenumbering,#6]{\ensuremath{#4}}{\IfValueTF{#2}{%
            %% Add an anchor if a label is present
            \raisebox{1em}{\hypertarget{#2}{}}%
            %% Create a macro "mygametitle@nameoflabel" to store the title
            \IfValueTF{#3}{% If a short title is provided
              \expandafter\gdef\csname mygametitle@#2 \endcsname{\ensuremath{#3}}%%
              \write\@auxout{\gdef\string\mygametitle@#2{\ensuremath{#3}}}%
            }{%
              \expandafter\gdef\csname mygametitle@#2 \endcsname{\ensuremath{#4}}%%
              \write\@auxout{\gdef\string\mygametitle@#2{\ensuremath{#4}}}%
            }%
          }{} #7 }}%
    \end{pcimage}
    \IfBooleanTF{#1}{}{\end{figure}}%
}{}

% Usage: \refGame{label}. Use \refGame*{label} if you don't want to color the link (useful in proofs)
\NewDocumentCommand{\refGame}{sm}{%
  \IfBooleanTF{#1}{%
    \hyperlink{#2}{\textcolor{black}{\csname mygametitle@#2\endcsname}}% Do not put a white space after #1!
  }{%
    \hyperlink{#2}{\csname mygametitle@#2\endcsname}% Do not put a white space after #1!
  }%
}
\makeatother

\ifdefined\disableHyperref%
\else%
% Well... not as much as cleveref
\usepackage[capitalise,noabbrev]{cleveref}
% Don't abbrev general names... but still abbrev equations
\crefname{equation}{Eq.}{Eqs.}
\crefname{figure}{Fig.}{Figs.}
\crefname{algorithm}{Protocol}{Protocols}
\Crefname{equation}{Equation}{Equations}
\Crefname{figure}{Figure}{Figures}
\Crefname{algorithm}{Protocol}{Protocols}
\fi

% Put that package at the end of the file,
% or you will have some strange errors like
% "Only one # is allowed per tab"
\usetikzlibrary{quantikz}
% Bug in llncs https://tex.stackexchange.com/a/372108/116348
\def\theHlemma{\theHtheorem}
\def\theHdefinition{\theHtheorem}
\def\theHcorollary{\theHtheorem}

%%%%%% PLEASE DO NOT WRITE ANYTHING AFTER THIS LINE %%%%%%
%%%%%% Indeed, hyperref, cleveref and quantikz need %%%%%%
%%%%%% to be loaded at the very end.                %%%%%%

\ifdefined\nicerThanLlncs%
  % change margins, page layout
  \usepackage[a4paper, top=1.4in, bottom=1.4in, right=1in, left=1in]{geometry}
  % enable page numbering, suppressed by default by llncs
  \pagestyle{plain}
  \setcounter{tocdepth}{2}
  \renewcommand\small{\fontsize{10pt}{12pt}\selectfont}
\fi
\ifdefined\nicerThanLlncsAbstract%
  % change margins, page layout
  \usepackage[a4paper, margin=2.5cm]{geometry}
  % enable page numbering, suppressed by default by llncs
  \pagestyle{plain}
  \setcounter{tocdepth}{2}
\fi
\ifdefined\enableLlncsCodeSlightlyImproved%
  \pagestyle{plain}
  \setcounter{tocdepth}{2}
\fi


\ifdefined\disableHyperref%
\else%
% https://rcweb.dartmouth.edu/doc/texmf-dist/doc/latex/cryptocode/cryptocode.pdf (page 55)
%%%%%%%%%% This part should be loaded last
% Usually hyperref needs to be at the end
\definecolor{secondaryColor}{RGB}{206,149,0} %% darker orange, looks like gold. <3
\usepackage[colorlinks, allcolors=secondaryColor]{hyperref}
\fi
%% Hyperref should be loaded before cryptocode! Otherwise, errors with labels of lines for example.
%%% Cryptocode
\usepackage [
n,
advantage,
operators,
sets,
adversary,
landau,
probability,
notions,
logic,
ff,
mm,
primitives,
events,
complexity,
asymptotics,
keys
] {cryptocode}
% I don't like the way "samples" is defined in cryptocode with a big dollar in front.
% \renewcommand\sample{\mathrel{\overset{\vbox to.7ex{\hbox{\fontsize{6}{0}\selectfont\vspace{.5ex} \$}}}{\leftarrow}}}
%% https://tex.stackexchange.com/a/584533/116348
\makeatletter
\renewcommand{\sample}{\mathrel{\mathpalette\sample@\relax}}
\newcommand{\sample@}[2]{%
  \ooalign{%
    \hspace{\stretch{2}}\raisebox{0.7\height}{$\m@th\demotestyle{#1}\mathdollar$}\hspace{\stretch{1}}\cr
    $\m@th#1\leftarrow$\cr
  }%
}
\newcommand{\demotestyle}[1]{%
  \ifx#1\displaystyle\scriptstyle\else
  \ifx#1\textstyle\scriptstyle\else
  \scriptscriptstyle\fi\fi
}
\makeatother
% Create a new command to create games using \game[linenumbering]{Name of game}{Game definition}
\createprocedureblock{game}{center,boxed}{}{}{}
% Create a new environment (following what I did in the thesis) to provide named games
\newcommand{\gameFont}[1]{\texttt{#1}}
\renewcommand{\Game}[1]{\ensuremath{\gameFont{Game#1}}} % By default \Game produces a nice G (symetric), but it's not very standard.

\makeatletter
\createprocedureblock{gameProc}{center,boxed}{}{}{}
\createprocedureblock{interactGame}{center,boxed}{}{}{}
%%% Nicer way to refer to games.
%%% See also https://tex.stackexchange.com/questions/623163 which provides a cleaner way to make it work with cleverref.
% Usage: \begin{namedGame}*[label][short title]{title}*[game options like skipfirstln,lnstart=1] content \end{namedGame}
% first * = disable floating. WARNING: do not put labels with non-letter chars.
% second * = do not include the game in the TOC.
\NewDocumentEnvironment{namedGame}{soomsO{}b}{%
  \IfBooleanTF{#1}{}{%
    \par% without par, the figure will be put after the line arriving after the picture if there is no paragraph.
    \setlength{\intextsep}{5.0pt plus 2.0pt minus 2.0pt}% reduce the space when image is inserted inline.
    \begin{figure}[htbp]}% 
    \begin{pcimage}%
      %% Add to toc see command listofgames
      \phantomsection% Create a fake label, useful to ensure the link point to this part.
      \IfBooleanTF{#5}{}{% If no star is given, add to "table of contents" for games
        \IfValueTF{#3}{% If a short title is provided
          \addcontentsline{game}{section}{\ensuremath{#3}}%
        }{%
          \addcontentsline{game}{section}{\ensuremath{#4}}%
        }%
      }%
      {\normalfont\gameProc[linenumbering,#6]{\ensuremath{#4}}{\IfValueTF{#2}{%
            %% Add an anchor if a label is present
            \raisebox{1em}{\hypertarget{#2}{}}%
            %% Create a macro "mygametitle@nameoflabel" to store the title
            \IfValueTF{#3}{% If a short title is provided
              \expandafter\gdef\csname mygametitle@#2 \endcsname{\ensuremath{#3}}%%
              \write\@auxout{\gdef\string\mygametitle@#2{\ensuremath{#3}}}%
            }{%
              \expandafter\gdef\csname mygametitle@#2 \endcsname{\ensuremath{#4}}%%
              \write\@auxout{\gdef\string\mygametitle@#2{\ensuremath{#4}}}%
            }%
          }{} #7 }}%
    \end{pcimage}
    \IfBooleanTF{#1}{}{\end{figure}}%
}{}

% Usage: \refGame{label}. Use \refGame*{label} if you don't want to color the link (useful in proofs)
\NewDocumentCommand{\refGame}{sm}{%
  \IfBooleanTF{#1}{%
    \hyperlink{#2}{\textcolor{black}{\csname mygametitle@#2\endcsname}}% Do not put a white space after #1!
  }{%
    \hyperlink{#2}{\csname mygametitle@#2\endcsname}% Do not put a white space after #1!
  }%
}
\makeatother

\ifdefined\disableHyperref%
\else%
% Well... not as much as cleveref
\usepackage[capitalise,noabbrev]{cleveref}
% Don't abbrev general names... but still abbrev equations
\crefname{equation}{Eq.}{Eqs.}
\crefname{figure}{Fig.}{Figs.}
\crefname{algorithm}{Protocol}{Protocols}
\Crefname{equation}{Equation}{Equations}
\Crefname{figure}{Figure}{Figures}
\Crefname{algorithm}{Protocol}{Protocols}
\fi

% Put that package at the end of the file,
% or you will have some strange errors like
% "Only one # is allowed per tab"
\usetikzlibrary{quantikz}
% Bug in llncs https://tex.stackexchange.com/a/372108/116348
\def\theHlemma{\theHtheorem}
\def\theHdefinition{\theHtheorem}
\def\theHcorollary{\theHtheorem}

%%%%%% PLEASE DO NOT WRITE ANYTHING AFTER THIS LINE %%%%%%
%%%%%% Indeed, hyperref, cleveref and quantikz need %%%%%%
%%%%%% to be loaded at the very end.                %%%%%%

\ifdefined\nicerThanLlncs%
  % change margins, page layout
  \usepackage[a4paper, top=1.4in, bottom=1.4in, right=1in, left=1in]{geometry}
  % enable page numbering, suppressed by default by llncs
  \pagestyle{plain}
  \setcounter{tocdepth}{2}
  \renewcommand\small{\fontsize{10pt}{12pt}\selectfont}
\fi
\ifdefined\nicerThanLlncsAbstract%
  % change margins, page layout
  \usepackage[a4paper, margin=2.5cm]{geometry}
  % enable page numbering, suppressed by default by llncs
  \pagestyle{plain}
  \setcounter{tocdepth}{2}
\fi
\ifdefined\enableLlncsCodeSlightlyImproved%
  \pagestyle{plain}
  \setcounter{tocdepth}{2}
\fi


\ifdefined\disableHyperref%
\else%
% https://rcweb.dartmouth.edu/doc/texmf-dist/doc/latex/cryptocode/cryptocode.pdf (page 55)
%%%%%%%%%% This part should be loaded last
% Usually hyperref needs to be at the end
\definecolor{secondaryColor}{RGB}{206,149,0} %% darker orange, looks like gold. <3
\usepackage[colorlinks, allcolors=secondaryColor]{hyperref}
\fi
%% Hyperref should be loaded before cryptocode! Otherwise, errors with labels of lines for example.
%%% Cryptocode
\usepackage [
n,
advantage,
operators,
sets,
adversary,
landau,
probability,
notions,
logic,
ff,
mm,
primitives,
events,
complexity,
asymptotics,
keys
] {cryptocode}
% I don't like the way "samples" is defined in cryptocode with a big dollar in front.
% \renewcommand\sample{\mathrel{\overset{\vbox to.7ex{\hbox{\fontsize{6}{0}\selectfont\vspace{.5ex} \$}}}{\leftarrow}}}
%% https://tex.stackexchange.com/a/584533/116348
\makeatletter
\renewcommand{\sample}{\mathrel{\mathpalette\sample@\relax}}
\newcommand{\sample@}[2]{%
  \ooalign{%
    \hspace{\stretch{2}}\raisebox{0.7\height}{$\m@th\demotestyle{#1}\mathdollar$}\hspace{\stretch{1}}\cr
    $\m@th#1\leftarrow$\cr
  }%
}
\newcommand{\demotestyle}[1]{%
  \ifx#1\displaystyle\scriptstyle\else
  \ifx#1\textstyle\scriptstyle\else
  \scriptscriptstyle\fi\fi
}
\makeatother
% Create a new command to create games using \game[linenumbering]{Name of game}{Game definition}
\createprocedureblock{game}{center,boxed}{}{}{}
% Create a new environment (following what I did in the thesis) to provide named games
\newcommand{\gameFont}[1]{\texttt{#1}}
\renewcommand{\Game}[1]{\ensuremath{\gameFont{Game#1}}} % By default \Game produces a nice G (symetric), but it's not very standard.

\makeatletter
\createprocedureblock{gameProc}{center,boxed}{}{}{}
\createprocedureblock{interactGame}{center,boxed}{}{}{}
%%% Nicer way to refer to games.
%%% See also https://tex.stackexchange.com/questions/623163 which provides a cleaner way to make it work with cleverref.
% Usage: \begin{namedGame}*[label][short title]{title}*[game options like skipfirstln,lnstart=1] content \end{namedGame}
% first * = disable floating. WARNING: do not put labels with non-letter chars.
% second * = do not include the game in the TOC.
\NewDocumentEnvironment{namedGame}{soomsO{}b}{%
  \IfBooleanTF{#1}{}{%
    \par% without par, the figure will be put after the line arriving after the picture if there is no paragraph.
    \setlength{\intextsep}{5.0pt plus 2.0pt minus 2.0pt}% reduce the space when image is inserted inline.
    \begin{figure}[htbp]}% 
    \begin{pcimage}%
      %% Add to toc see command listofgames
      \phantomsection% Create a fake label, useful to ensure the link point to this part.
      \IfBooleanTF{#5}{}{% If no star is given, add to "table of contents" for games
        \IfValueTF{#3}{% If a short title is provided
          \addcontentsline{game}{section}{\ensuremath{#3}}%
        }{%
          \addcontentsline{game}{section}{\ensuremath{#4}}%
        }%
      }%
      {\normalfont\gameProc[linenumbering,#6]{\ensuremath{#4}}{\IfValueTF{#2}{%
            %% Add an anchor if a label is present
            \raisebox{1em}{\hypertarget{#2}{}}%
            %% Create a macro "mygametitle@nameoflabel" to store the title
            \IfValueTF{#3}{% If a short title is provided
              \expandafter\gdef\csname mygametitle@#2 \endcsname{\ensuremath{#3}}%%
              \write\@auxout{\gdef\string\mygametitle@#2{\ensuremath{#3}}}%
            }{%
              \expandafter\gdef\csname mygametitle@#2 \endcsname{\ensuremath{#4}}%%
              \write\@auxout{\gdef\string\mygametitle@#2{\ensuremath{#4}}}%
            }%
          }{} #7 }}%
    \end{pcimage}
    \IfBooleanTF{#1}{}{\end{figure}}%
}{}

% Usage: \refGame{label}. Use \refGame*{label} if you don't want to color the link (useful in proofs)
\NewDocumentCommand{\refGame}{sm}{%
  \IfBooleanTF{#1}{%
    \hyperlink{#2}{\textcolor{black}{\csname mygametitle@#2\endcsname}}% Do not put a white space after #1!
  }{%
    \hyperlink{#2}{\csname mygametitle@#2\endcsname}% Do not put a white space after #1!
  }%
}
\makeatother

\ifdefined\disableHyperref%
\else%
% Well... not as much as cleveref
\usepackage[capitalise,noabbrev]{cleveref}
% Don't abbrev general names... but still abbrev equations
\crefname{equation}{Eq.}{Eqs.}
\crefname{figure}{Fig.}{Figs.}
\crefname{algorithm}{Protocol}{Protocols}
\Crefname{equation}{Equation}{Equations}
\Crefname{figure}{Figure}{Figures}
\Crefname{algorithm}{Protocol}{Protocols}
\fi

% Put that package at the end of the file,
% or you will have some strange errors like
% "Only one # is allowed per tab"
\usetikzlibrary{quantikz}
% Bug in llncs https://tex.stackexchange.com/a/372108/116348
\def\theHlemma{\theHtheorem}
\def\theHdefinition{\theHtheorem}
\def\theHcorollary{\theHtheorem}

%%%%%% PLEASE DO NOT WRITE ANYTHING AFTER THIS LINE %%%%%%
%%%%%% Indeed, hyperref, cleveref and quantikz need %%%%%%
%%%%%% to be loaded at the very end.                %%%%%%
